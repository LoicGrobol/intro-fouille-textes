% LTeX: language=fr
% Copyright © 2019, Loïc Grobol <loic.grobol@gmail.com>
% This document is available under the terms of the Creative Commons Attribution 4.0 International License (CC BY 4.0) (https://creativecommons.org/licenses/by/4.0/)

\documentclass[../allslides.tex]{subfiles}
\renewcommand\titlepagesubtitle{Cours 6 : TF⋅IDF et représentations vectorielles}
\renewcommand\docdate{2021-02-18}  % chktex 8

\begin{document}
\part{\titlepagesubtitle{}}

\begin{frame}[plain]
	\partpage % chktex 1
\end{frame}

% TODO: mieux choisir les exemples pour avoir un tableau moins creux et plus de trucs intéressants à dire sur les différentes représentations
\begin{frame}{Exemples de travail}
	Donner trois représentations tabulaires (booléennes, occurrences, fréquences) fondées sur le vocabulaire du corpus suivant et ne prenant en compte que les mots \emph{France}, \emph{art}, \emph{cinéma} et \emph{économie}.
	\begin{enumerate}
		\item « Le cinéma est un art, c’est aussi une industrie. »
		\item « Personne, quand il est petit, ne veut être critique de cinéma. Mais ensuite, en France, tout le monde a un deuxième métier : critique de cinéma ! »
		\item « Tout le monde a des rêves de Hollywood. »
		\item « C’est la crise, l’économie de la France est menacée par la mondialisation. »
		\item « En temps de crise, reconstruire l’industrie : tout un art ! »
		\item « Quand une usine ferme, c’est que l’économie va mal. »
	\end{enumerate}
\end{frame}


\begin{frame}[fragile]{Exemples de travail}
    \begin{figure}
        \caption{Représentation tabulaire par \emph{occurences}}
        \begin{tabu} to \linewidth {l*{4}{X[c]}}
            \toprule
            \# & France & cinéma & art & économie\\
            \midrule
            1 & 0 & 1 & 1 & 0\\
            2 & 1 & 2 & 0 & 0\\
            3 & 0 & 0 & 0 & 0\\
            4 & 1 & 0 & 0 & 1\\
            5 & 0 & 0 & 1 & 0\\
            6 & 0 & 0 & 0 & 1\\
            \hline
            \bottomrule
        \end{tabu}
    \end{figure}
\end{frame}

\begin{frame}<-2>[fragile]{Exemples de travail}
    \begin{figure}
        \caption{Représentation tabulaire \emph{booléenne}}
        \begin{tabu} to \linewidth {l*{4}{X[c]}}
            \toprule
            \# & France & cinéma & art & économie\\
            \midrule
            1 & 0 & 1 & 1 & 0\\
            2 & 1 & 1 & 0 & 0\\
            3 & 0 & 0 & 0 & 0\\
            4 & 1 & 0 & 0 & 1\\
            5 & 0 & 0 & 1 & 0\\
            6 & 0 & 0 & 0 & 1\\
            \bottomrule
        \end{tabu}
    \end{figure}
\end{frame}

\begin{frame}[fragile]{Exemples de travail}
    \begin{figure}
        \caption{Représentation tabulaire par \emph{fréquences relatives}}
        \begin{tabu} to \linewidth {l*{4}{X[c]}}
            \toprule
            \# & France & cinéma & art & économie\\
            \midrule
            1 & 0 & 0.11 & 0.11 & 0\\
            2 & 0.04 & 0.08 & 0 & 0\\
            3 & 0 & 0 & 0 & 0\\
            4 & 0.07 & 0 & 0 & 0.07\\
            5 & 0 & 0 & 0.1 & 0\\
            6 & 0 & 0 & 0 & 0.09\\
            \bottomrule
        \end{tabu}
    \end{figure}
\end{frame}

\section{TF⋅IDF}

\begin{frame}{TF⋅IDF}
    \only<-2>{On veut pouvoir \alert{pondérer} l'importance d'un terme (mot, caractère, $n$-gramme…) en fonction de sa \alert{spécificité} (au sens large) pour un document.}

    \begin{itemize}
        \item \emph{\textbf{T}erm \textbf{F}requency $×$ \textbf{I}nverse \textbf{D}ocument \textbf{F}requency}
        \item $\TF(i,j) = n_{i,j}$ ou $\TF(i,j) = \frac{n_{i,j}}{\sum_k n_{i,j}}$
        \item
            \only<1>{
                \begin{equation}
                    \IDF(j) = \log\mleft(\frac{m}{\#\compset{i}{n_{i,j}>0}}\mright)
                \end{equation}
            }
            \only<2>{
                \begin{equation}
                    \IDF(j) = \log\frac{\overbrace{m}^{\mathclap{\text{nombre total de documents}}}}{\underbrace{\#\compset{i}{n_{i,j}>0}}_{\mathclap{\text{nombre de documents contenant $a_j$}}}}
                \end{equation}
            }
            \only<3->{$\IDF(j) = \log\frac{m}{\#\compset{i}{n_{i,j}>0}}$}
        \only<3->{
            \item Intuitivement $\IDF$ mesure la rareté d'un attribut, donc (on espère !) son informativité
            \item « Si $a_j$ est présent dans tous les textes, savoir qu'il est présent dans $t_i$ ne donne pas d'information sur $t_i$ »
            \item Permet un filtrage moins arbitraire des \emph{stop words}
            \item \alt<4>{$\log$ donne à $\IDF$ la forme d'une \emph{surprise} au sens de Shannon}{$\log$ est une façon pratique de modérer les effets de cette pondération : on devrait parler de $\log$-$\IDF$}
        }
    \end{itemize}
\end{frame}

\begin{frame}{IDF, probabilités et surprise}
	Dans un corpus de \num{1000} textes, \num{55} contiennent le mot \emph{kiwi}.
    \pause

	On choisit un texte au hasard sans en favoriser aucun. Serait-il surprenant que le texte choisi contienne le mot \emph{kiwi} ?
\end{frame}

\begin{frame}{Les maths en cinq secondes : probabilités}
    Objectif : on veut une mesure \alert{quantitative} du fait qu'un évènement dépendant du hasard (aléatoire) ait des chances de se produire. On appelle cette mesure \alert{probabilité} de l'évènement.
    
	\pause
	On impose que
	\begin{itemize}
		\item la probabilité d'un évènement soit un nombre compris entre \num{0} (pour un évènement impossible) et \num{1} (pour un évènement certain)
		\item Si deux évènements sont \alert{incompatibles}, la probabilité qu'un des deux se produise est la somme de leurs probabilités respectives
	\end{itemize}
\end{frame}

\begin{frame}{IDF, probabilités et surprise}
	Dans un corpus de \num{1000} textes, \num{55} contiennent le mot \emph{kiwi}.
    \pause
    
	On considère les évènements $Aᵢ$ « avoir choisi le texte $i$ ».
	\begin{enumerate}
		\item Combien y a-t-il de $Aᵢ$ ?  \only<2>{→ \num{1000}}
		\item Quelle est la probabilité qu'au moins un des $Aᵢ$ se réalise ?  \only<2>{→ \num{1}}
		\item Pour un $i$ donné, quelle est la probabilité (notée $P(Aᵢ)$) de $Aᵢ$ ?  \only<2>{→ \(\frac{1}{1000}\)}
		\item En déduire la probabilité que le texte choisi contienne le mot \emph{kiwi}  \only<2>{→ \(\frac{55}{1000}\)}
	\end{enumerate}
\end{frame}

\begin{frame}{IDF, probabilités et surprise}
	La notion de \alert{surprise} vient de la théorie de l'information. La surprise d'un évènement $A$ mesure le caractère inattendu de la réalisation d'un évènement.

	On la définit par
	\begin{equation}
		I(A) = -\log P(A)
	\end{equation}

	Intuitivement, plus la probabilité d'un évènement est faible, plus sa surprise est grande.
\end{frame}

\begin{frame}{Surprise}
    \begin{figure}
        \tikzset{external/export=true}
        \begin{tikzpicture}
            \begin{axis}[
				title={Surprise : $y=-\log x$},
				xlabel={$P(A)$},
				ylabel={$I(A)$},
				xmin=0, xmax=1, ymin=0,
                width=0.9\textwidth,
                height=0.8\textheight,
			]
                \addplot[highlighta, domain=0:1, samples=1001] {-ln(x)};
            \end{axis}
        \end{tikzpicture}
        \caption{Suprise d'un évènement en fonction de sa probabilité}
    \end{figure}
\end{frame}

\begin{frame}{IDF}
	On a pour le mot $m_j$
	\begin{equation}
		\IDF(m_j) = \log\mleft(\frac{m}{\#\compset{i}{n_{i,j}>0}}\mright) = -\log\mleft(\frac{\#\compset{i}{n_{i,j}>0}}{m}\mright) = I(…)
	\end{equation}

	Autrement dit : l'\(\IDF\) du mot \emph{kiwi} est la surprise de l'événement \enquote{un texte choisi au hasard dans le corpus contient le mot \emph{kiwi}}.
\end{frame}

% \begin{frame}{Exercice}
% 	Donner une représentation tabulaire du corpus d'exemple qui utilise les TF⋅IDF.
% \end{frame}

\section{Représentations vectorielles et similarités}
% TODO: toute cette partie est à refaire
%  - VIrer la partie somme et scale de vecteurs : RAF
%  - Virer la partie formelle, prendre un exemple et le filer pour chaque distance, laisser les formules pour le poly
%  - Pour les distances, voir les textes comme des points pas comme des vecteurs
%  - Pour la simil, garder les vecteurs, mais commencer par parler d'angle et introduire le produit scalaire comme outil de calcul (et définir la norme)
% ! CECI EST VRAIMENT IMPORTANT !

\begin{frame}{Des sacs de mots aux vecteurs}
    Dans tout ce qui précède, on ne tient pas compte des positions relatives des termes, mais seulement de leurs nombres d'occurrences.
    \begin{itemize}
        \item Approche \emph{bag-of-words} (sac de mots)
        \item « Siobhan applaudit Eóghan » vs. « Eóghan applaudit Siobhan »…
        \begin{itemize}
            \item[→] On compte (comme d'habitude) sur la masse de données pour passer outre ce genre de problèmes
        \end{itemize}
    \end{itemize}
    En pratique on fait même plus fort
    \begin{itemize}
        \item Hypothèse d'\alert{indépendance} : si $k≠ℓ$, $a_k$ et $a_ℓ$ sont indépendants
        \item On peut alors considérer chaque ligne comme un vecteur (ou un point…) d'un espace vectoriel
        \item Et accéder à la puissance de l'algèbre linéaire et de la géométrie vectorielle !
    \end{itemize}
\end{frame}

\subsection{Les maths en cinq secondes : vecteurs}

\begin{frame}{Les maths en cinq secondes : vecteurs}
    On appelle \emph{vecteur de dimension $n$} tout $n$-uplet de nombres
    \begin{equation}
        v = \vcord{v_1, ⋮, v_n}
    \end{equation}
    Si $n=2$, on peut les représenter simplement dans le plan
    \begin{figure}
        \tikzset{external/export=true}
        \begin{tikzpicture}
            \draw [help lines, xstep=1cm, ystep=1cm] (0, 0) grid (5.25, 4.25);

            \draw[->] (-0.5,0) -- (5.25, 0);
            \foreach \x in {1,...,5}
                \draw[shift={(\x, 0)}] (0pt,2pt) -- (0pt,-2pt) node[below] {\footnotesize $\x$};
            \draw[->] (0, -0.5) -- (0, 4.25);

            \foreach \y in {1,...,4}
                \draw[shift={(0,\y)}] (2pt,0pt) -- (-2pt,0pt) node[left] {\footnotesize $\y$};

            \draw (0, 0) node[below left] {\footnotesize $0$};

            \draw [->, highlighta, line width=2pt] (0,0) -- (2,3);

            \draw (1.5, 1.5) node[anchor=west] {$\textcolor{highlighta}{v}=\vcord{2,3}$};
        \end{tikzpicture}
        \caption{Vecteur en dimension $2$}
    \end{figure}
\end{frame}

\begin{frame}{Les maths en cinq secondes : vecteurs}
    On définit la \emph{somme} et le \emph{produit par un nombre} par
    \begin{equation}
        \vcord{a_1, ⋮, a_n} + \vcord{b_1, ⋮, b_n} = \vcord{a_1+b_1, ⋮, a_n+b_n}
    \end{equation}
    et
    \begin{equation}
        λ⋅\vcord{a_1, ⋮, a_n} = \vcord{λa_1, ⋮, λa_n}
    \end{equation}
    par exemple
    \begin{equation}
        \vcord{2, 7} + \vcord{1, 3} = \vcord{3, 10}
    \end{equation}
    \begin{equation}
        5⋅\vcord{3.5, -3} = \vcord{17.5, -15}
    \end{equation}
\end{frame}

\begin{frame}[fragile]{Les maths en cinq secondes : vecteurs}
    Géométriquement
    \only<1-4>{
        \begin{figure}
            \tikzset{external/export=true}
            \begin{tikzpicture}
                \draw [help lines, xstep=1cm, ystep=1cm] (0, 0) grid (5.25, 4.25);

                \draw[->] (-0.5,0) -- (5.25, 0);
                \foreach \x in {1,...,5}
                    \draw[shift={(\x, 0)}] (0pt,2pt) -- (0pt,-2pt) node[below] {\footnotesize $\x$};
                \draw[->] (0, -0.5) -- (0, 4.25);

                \foreach \y in {1,...,4}
                    \draw[shift={(0,\y)}] (2pt,0pt) -- (-2pt,0pt) node[left] {\footnotesize $\y$};

                \draw (0, 0) node[below left] {\footnotesize $0$};

                \draw [->, highlighta, line width=2pt, alt=<2-3>{invisible}{}] (0,0) -- (3,1) node[midway, above] {$v$};
                \draw [->, highlighta, line width=2pt, alt=<2-3>{}{invisible}] (1,2) -- (4,3) node[midway, above] {$v$};

                \draw [->, highlight3, line width=2pt] (0,0) -- (1,2) node[midway, above left] {$u$};

                \draw[->, highlight6, line width=2pt, alt=<3->{}{invisible}] (0,0) -- (4,3) node[midway, right] {$u+v$};
            \end{tikzpicture}
            \caption{Somme de deux vecteurs}
        \end{figure}
    }
    \only<5-7>{
        \begin{figure}
            \tikzset{external/export=true}
            \begin{tikzpicture}
                \draw [help lines, xstep=1cm, ystep=1cm] (0, 0) grid (5.25, 4.25);

                \draw[->] (-0.5,0) -- (5.25, 0);
                \foreach \x in {1,...,5}
                    \draw[shift={(\x, 0)}] (0pt,2pt) -- (0pt,-2pt) node[below] {\footnotesize $\x$};
                \draw[->] (0, -0.5) -- (0, 4.25);

                \foreach \y in {1,...,4}
                    \draw[shift={(0,\y)}] (2pt,0pt) -- (-2pt,0pt) node[left] {\footnotesize $\y$};

                \draw (0, 0) node[below left] {\footnotesize $0$};

                \draw [->, highlighta, line width=2pt, alt=<5>{}{invisible}] (0,0) -- (2,1) node[midway, above] {$v$};

                \draw[->, highlighta, line width=2pt, alt=<6>{}{invisible}] (0,0) -- (4,2) node[midway, above] {$2v$};
                \draw[->, highlighta, line width=2pt, alt=<7>{}{invisible}] (0,0) -- (1,0.5) node[above] {$0.5v$};
            \end{tikzpicture}
            \caption{produit d'un vecteur par un nombre}
        \end{figure}
    }
\end{frame}

\begin{frame}{Exercice}
	Calculer
	\begin{enumerate}
		\item
			\begin{equation}
				\vcord{2, 1} + \vcord{3, 7}
			\end{equation}
		\item
			\begin{equation}
				3⋅\vcord{2, 1, -2, 0}
			\end{equation}
		\item
			\begin{equation}
				\frac{1}{2}⋅\left(\vcord{2, 4} + \vcord{3,6}\right)
			\end{equation}
		\item
			\begin{equation}
				0.2⋅\vcord{2, 4} + 0.8⋅\vcord{3,6}
			\end{equation}
	\end{enumerate}
\end{frame}

\begin{frame}{Interprétation}
    Bien entendu, l'hypothèse d'indépendance est fausse en général
    \begin{itemize}
        \item[→] Si un texte contient \emph{main}, il est plus probable qu'il contienne également \emph{doigt}
    \end{itemize}
    Et représenter les documents par des vecteurs a ses limites
    \begin{itemize}
        \item[→] Que signifie le fait des sommer les attributs de deux documents ?
    \end{itemize}
    Mais ces approximations permettent d'utiliser une machinerie mathématique qui marche suffisamment bien en pratique.

    On est bien dans la philosophie de la fouille de textes !
\end{frame}

\subsection{Distances vectorielles}

\begin{frame}{Similarité : exemple de travail}
    En pratique, les documents considérés auront beaucoup de colonnes
    \begin{itemize}
        \item De quelques-unes à plusieurs centaines
        \item Donc les vecteurs considérés seront de grande dimension
        \item Mais pour l'intuition on va se retreindre à la dimension $2$ : le plan
    \end{itemize}
\end{frame}

\begin{frame}[standout]
    To deal with a $\mathbf{14}$-dimensional space, visualize a $3$-D space and say `fourteen' to yourself very loudly. Everyone does it.
	\vspace{\bigskipamount}

    Geoffrey Hinton
\end{frame}

\begin{frame}[fragile=singleslide]{Similarités : exemple de travail}
    On considère deux vecteurs $t_1=(v_{1,1}, … v_{1,n})$ et $t_2=(v_{2,1}, …, v_{2,n})$ représentant des documents

    Par exemple, si on utilise comme traits les fréquences de \emph{orange} et \emph{kiwi}, étant donné le fichier tabulaire suivant
    \begin{columns}
    \column{0.5\textwidth}
        \begin{figure}
            \begin{tabu} to .9\linewidth {*{3}{X[c]}}
                \texttt{t}  & \texttt{orange} & \texttt{kiwi}\\
                \hline
                \texttt{1}  & \texttt{3}      & \texttt{1}\\
                \texttt{2}  & \texttt{1}      & \texttt{2}\\
            \end{tabu}
        \end{figure}

        On aura $t_1=\tvcord{3, 1}$ et $t_2=\tvcord{1, 2}$
    \column{0.5\textwidth}
        \begin{figure}
            \tikzset{external/export=true}
            \begin{tikzpicture}[true scale=0.8]
                \draw [help lines, xstep=1cm, ystep=1cm] (0, 0) grid (5.25, 4.25);

                \draw[->] (-0.5,0) -- (5.25, 0);
                \foreach \x in {1,...,5}
                    \draw[shift={(\x, 0)}] (0pt,2pt) -- (0pt,-2pt) node[below] {\footnotesize $\x$};
                \draw[->] (0, -0.5) -- (0, 4.25);

                \foreach \y in {1,...,4}
                    \draw[shift={(0,\y)}] (2pt,0pt) -- (-2pt,0pt) node[left] {\footnotesize $\y$};

                \draw (0, 0) node[below left] {\footnotesize $0$};

                \draw [->, highlighta, line width=2pt] (0,0) -- (3,1) node[midway, above] {$t_1$};
                \draw [->, highlight3, line width=2pt] (0,0) -- (1,2) node[midway, above left] {$t_2$};
            \end{tikzpicture}
            \caption{Exemple de travail}
    \end{figure}
\end{columns}
\end{frame}

\begin{frame}{Exemple}
	\begin{enumerate}
		\item Représenter le corpus d'exemple sous forme tabulaire en utilisant comme attributs les fréquences des mots ayant trait au \emph{cinéma} et à l'économie \emph{économie}
		\item En déduire une représentation du corpus comme un ensemble de vecteurs du plan
	\end{enumerate}
	Pour la suite, choisir deux de ces vecteurs et calculer leur distance pour chacune des distances proposées.
\end{frame}

% TODO: overlay equation uncover ?
% TODO: distances comparisons (particularly minkovsky) : show distorted diagrams ?
% TODO: for all examples → connect to real-world examples with word frequences
\begin{frame}[fragile]{Distance de Manhattan}
    \begin{block}{Définition (distance de Manhattan)}
        On appelle \emph{distance de Manhattan} entre $t_1$ et $t_2$ le nombre
        \begin{equation}
            \norm{t_2-t_1}_1 = ∑_{k=1}^n\abs{v_{2,k}-v_{1,k}} \left(=\abs{v_{2,1}-v_{1,1}}+…+\abs{v_{2,n}-v_{1,n}}\right)
        \end{equation}
    \end{block}

    \only<1>{
    \begin{columns}
        \column{0.5\textwidth}
        On l'appelle aussi \emph{taxicab distance} : c'est la distance du chemin le plus court pour aller d'un point à un autre en se déplaçant sur une grille

        \column{0.5\textwidth}
        \begin{figure}
            \tikzset{external/export=true}
            \begin{tikzpicture}[background rectangle/.style={fill=gray}, show background rectangle, true scale=0.7]
                \draw [xstep=1, ystep=1, line width=5pt, white] (0, 0) grid (5.1, 5.1);
                \draw[sronbrightpurple, line width=2pt] (0, 0) -| (5, 5);
                \draw[sronbrightgrey, line width=2pt] (0, 0) |- (5, 5);
                \draw[sronbrightred, line width=2pt] (0, 0) -| (1, 1) -| (2, 2) -| (3, 3) -| (4, 4) -| (5, 5);
                \draw[highlight4, line width=2pt] (0, 0) |- (2, 3) -| (5, 5);
                \fill[highlight3] (0, 0) circle [radius=2pt];
                \fill[highlight3] (5, 5) circle [radius=2pt];
            \end{tikzpicture}
            \caption{\emph{Taxicab distance}}
        \end{figure}
    \end{columns}
    }
    \only<2>{
        \vspace{-1\bigskipamount}
        Pour notre exemple
        \vspace{-1\smallskipamount}
        \begin{columns}
        \column{.6\textwidth}
            \begin{align}
                \norm{\textcolor{highlight3}{t₂}-\textcolor{highlighta}{t₁}}
                    &= ∑_{k=1}^n\abs{v_{2,k}-v_{1,k}}\\
                    &= \textcolor{highlight2}{\abs{v_{2,1}-v_{1,1}}}+\textcolor{highlight5}{\abs{v_{2,2}-v_{1,2}}}\\
                    &= \textcolor{highlight2}{\abs{1-3}}+\textcolor{highlight5}{\abs{2-1}}\\
                    &= 2+1\\
                    &= \textcolor{highlight4}{3}
            \end{align}
        \column{.45\textwidth}
            \vspace{-1.5\bigskipamount}
            \begin{figure}
                \tikzset{external/export=true}
                \begin{tikzpicture}[true scale=0.8]
                    \draw [help lines, xstep=1cm, ystep=1cm] (0, 0) grid (5, 4);

                    \draw[->] (-0.5,0) -- (5.25, 0);
                    \foreach \x in {1,...,5}
                        \draw[shift={(\x, 0)}] (0pt,2pt) -- (0pt,-2pt) node[below] {\footnotesize $\x$};
                    \draw[->] (0, -0.5) -- (0, 4.25);

                    \foreach \y in {1,...,4}
                        \draw[shift={(0,\y)}] (2pt,0pt) -- (-2pt,0pt) node[left] {\footnotesize $\y$};

                    \draw (0, 0) node[below left] {\footnotesize $0$};

                    \draw [->, highlighta, line width=2pt] (0,0) -- (3,1) node[midway, below right] {$t₁$};
                    \draw [->, highlight3, line width=2pt] (0,0) -- (1,2) node[midway, above left] {$t₂$};
                    \draw [<->, highlight2] (3, 2) -- (1, 2) node [midway, above] {$\abs{1-3}$};
                    \draw[<->, highlight5] (3, 2) -- (3, 1)  node [midway, right] {$\abs{2-1}$};
                    \draw [<->, highlight4] (1, 2) --  (3, 1)  node [midway, below left] {$3$};
                \end{tikzpicture}
                \caption{Distance de Manhattan}
            \end{figure}
        \end{columns}
    }
\end{frame}

\begin{frame}{Distance euclidienne}
    \begin{block}{Définition (distance euclidienne)}
        On appelle \emph{distance euclidienne\textsuperscript{1}} entre $t_1$ et $t_2$ le nombre
        \begin{equation}
            \norm{t₂-t₁}₂ = √{∑_{k=1}^n(v_{2,k}-v_{1,k})²}
        \end{equation}
    \end{block}
    \vspace{-1.5\bigskipamount}
    C'est la distance usuelle dans le plan. Pour notre exemple :
    \vspace{-1\bigskipamount}
    \begin{columns}
    \column{.65\textwidth}
        \begin{align}
            \norm{\textcolor{highlight3}{t₂}-\textcolor{highlighta}{t₁}}₂
                &= √{∑_{k=1}^n(v_{2,k}-v_{1,k})²}\\
                &= √{\textcolor{highlight2}{(v_{2,1}-v_{1,1})}² +
                     \textcolor{highlight5}{(v_{2,2}-v_{1,2})}²}\\
                &= √{\textcolor{highlight2}{(1-3)}²+\textcolor{highlight5}{(2-1)}²}\\
                &= √{2²+1²}\\
                &= \textcolor{highlight4}{√{5}}
        \end{align}
    \column{.5\textwidth}
        \begin{figure}
            \tikzset{external/export=true}
            \begin{tikzpicture}[true scale=0.85]
                \draw [help lines, xstep=1cm, ystep=1cm] (0, 0) grid (5.25, 4.25);

                \draw[->] (-0.5,0) -- (5.25, 0);
                \foreach \x in {1,...,5}
                    \draw[shift={(\x, 0)}] (0pt,2pt) -- (0pt,-2pt) node[below] {\footnotesize $\x$};
                \draw[->] (0, -0.5) -- (0, 4.25);

                \foreach \y in {1,...,4}
                    \draw[shift={(0,\y)}] (2pt,0pt) -- (-2pt,0pt) node[left] {\footnotesize $\y$};

                \draw (0, 0) node[below left] {\footnotesize $0$};

                \draw [->, highlighta, line width=2pt] (0,0) -- (3,1) node[midway, below right] {$t₁$};
                \draw [->, highlight3, line width=2pt] (0,0) -- (1,2) node[midway, above left] {$t₂$};
                \draw [->, highlight2] (3, 2) -- (1, 2)  node [midway, above] {$(1-3)$};
                \draw[->, highlight5] (3, 2) -- (3, 1)  node [midway, right] {$(2-1)$};
                \draw [<->, highlight4] (1, 2) --  (3, 1)  node [midway, below left] {$√{5}$};
            \end{tikzpicture}
            \caption{Distance euclidienne}
        \end{figure}
    \end{columns}
    \vspace{-1\bigskipamount}
    {\tiny 1. Euclide d'Alexandrie, IIIè s AEC}
\end{frame}

\begin{frame}{Distance de Tchebychev}
    \begin{block}{Définition (distance de Tchebychev)}
        On appelle \emph{distance de Tchebychev\textsuperscript{1}} entre $t_1$ et $t_2$ le nombre
        \begin{equation}
            \norm{t₂-t₁}_∞ = \max_{1⩽k⩽n}\abs{v_{2,k}-v_{1,k}}
        \end{equation}
    \end{block}
    \vspace{-1\smallskipamount}
    \begin{columns}
    \column{.65\textwidth}
        \begin{align}
            \norm{\textcolor{highlight3}{t₂}-\textcolor{highlighta}{t₁}}_∞
                &= \max_{1⩽k⩽n}\abs{v_{2,k}-v_{1,k}}\\
                &= \max_{1⩽k⩽n}\mleft(
                    \textcolor{highlight2}{\abs{v_{2,1}-v_{1,1}}},
                    \textcolor{highlight5}{\abs{v_{2,2}-v_{1,2}}}\mright)\\
                &= \max_{1⩽k⩽n}\mleft(
                    \textcolor{highlight2}{\abs{1-3}},
                    \textcolor{highlight5}{\abs{2-1}}\mright)\\
                &= \max_{1⩽k⩽n}(2, 1)\\
                &= \textcolor{highlight4}{2}
        \end{align}
    \column{.5\textwidth}
        \begin{figure}
            \tikzset{external/export=true}
            \begin{tikzpicture}[true scale=0.85]
                \draw [help lines, xstep=1cm, ystep=1cm] (0, 0) grid (5.25, 4.25);

                \draw[->] (-0.5,0) -- (5.25, 0);
                \foreach \x in {1,...,5}
                    \draw[shift={(\x, 0)}] (0pt,2pt) -- (0pt,-2pt) node[below] {\footnotesize $\x$};
                \draw[->] (0, -0.5) -- (0, 4.25);

                \foreach \y in {1,...,4}
                    \draw[shift={(0,\y)}] (2pt,0pt) -- (-2pt,0pt) node[left] {\footnotesize $\y$};

                \draw (0, 0) node[below left] {\footnotesize $0$};

                \draw [->, highlighta, line width=2pt] (0,0) -- (3,1) node[midway, below right] {$t₁$};
                \draw [->, highlight3, line width=2pt] (0,0) -- (1,2) node[midway, above left] {$t₂$};
                \draw [->, highlight2] (3, 2) -- (1, 2)  node [midway, above] {$\abs{1-3}$};
                \draw[->, highlight5] (3, 2) -- (3, 1)  node [midway, right] {$\abs{2-1}$};
                \draw [<->, highlight4] (1, 2) --  (3, 1)  node [midway, below left] {$2$};
            \end{tikzpicture}
            \caption{Distance de Tchebychev}
        \end{figure}
    \end{columns}
    \vspace{-1\smallskipamount}
    {\tiny 1. Пафну́тий Льво́вич Чебышёв, 1821–1894}
\end{frame}

\begin{frame}{Distances de Minkowski}
    \begin{block}{Définition (distance de Minkowski)}
        Pour tout $p⩾1$, on appelle \emph{distance de Minkowski\textsuperscript{1} de paramètre $p$} entre $t_1$ et $t_2$ le nombre
        \begin{equation}
            \norm{t₂-t₁}_p ={\left(∑_{k=1}^n\abs{v_{2,k}-v_{1,k}}^p\right)}^{\frac{1}{p}}
        \end{equation}
    \end{block}
    \vspace{-1\bigskipamount}
    Pour $p=1$ on retrouve la distance de Manhattan, pour $p=2$ la distance euclidienne, et pour $p→+∞$, la distance de Minkowski tend vers la distance de Tchebychev.

    Intuitivement, pour $p=1$, elle traite également tous les écarts entre coordonnées, pour $p=∞$, elle ne conserve que le plus grand écart, et les autres $p$ donnent une interpolation entre ces deux extrêmes.
    \vskip0pt plus 1fill
    {\tiny 1. Hermann Minkowski, 1864–1909}
\end{frame}

\begin{frame}{Distances de Minkowski}
    \begin{figure}
        \tikzset{external/export=true}
        \begin{tikzpicture}
            \begin{axis}[xmin=1, xmax=6, ymin=0,
                         width=0.9\textwidth,
                         height=0.8\textheight,
                         xlabel={$p$},
                         ylabel={$\norm{t₂-t₁}_p$},
                         title={Distances de Minkowski : $y=(2^x+1^x)^{\frac{1}{x}}$},
                         scaled ticks=false,
                         ticklabel style={/pgf/number format/.cd, 1000 sep={\,}}]
                \addplot[highlighta, domain=1:6, samples=1001] {(2^x+1)^(1/x)};
            \end{axis}
        \end{tikzpicture}
        \caption{Distances de Minkowski pour notre exemple}
    \end{figure}
\end{frame}

\subsection{Similarités}

\begin{frame}{Les maths en cinq secondes : produit scalaire}
    Intuitivement, le produit scalaire mesure la similarité des directions des vecteurs
    \begin{columns}
    \column{.5\textwidth}
        \begin{figure}
            \tikzset{external/export=true}
            \begin{tikzpicture}[true scale=0.8]
                \draw [help lines, xstep=1cm, ystep=1cm] (0, 0) grid (4, 3);

                \draw[->] (-0.5,0) -- (4.25, 0);
                \foreach \x in {1,...,4}
                    \draw[shift={(\x, 0)}] (0pt,2pt) -- (0pt,-2pt) node[below] {\footnotesize $\x$};
                \draw[->] (0, -0.5) -- (0, 3.25);

                \foreach \y in {1,...,3}
                    \draw[shift={(0,\y)}] (2pt,0pt) -- (-2pt,0pt) node[left] {\footnotesize $\y$};

                \draw (0, 0) node[below left] {\footnotesize $0$};

                \draw [->, highlighta, line width=2pt] (0,0) -- (3,0) node[midway, below right] {$t₁$};
                \draw [->, highlight3, line width=2pt] (0,0) -- (0,2) node[midway, above left] {$t₂$};

                \draw [domain=0:90] plot ({cos(\x)}, {sin(\x)}) (0.9, 0.9) node {$90$};
            \end{tikzpicture}
        \end{figure}
        \vspace{-\bigskipamount}
        Nul pour des vecteurs orthogonaux
        \begin{equation}
            \innprod{t₁}{t₂} =\norm{t₁}₂×\norm{t₂}₂×\underbrace{\cos 90}_{0} = 0
        \end{equation}
    \column{.5\textwidth}
        \begin{figure}
            \tikzset{external/export=true}
            \begin{tikzpicture}[true scale=0.8]
                \draw [help lines, xstep=1cm, ystep=1cm] (0, 0) grid (4, 3);

                \draw[->] (-0.5,0) -- (4.25, 0);
                \foreach \x in {1,...,4}
                    \draw[shift={(\x, 0)}] (0pt,2pt) -- (0pt,-2pt) node[below] {\footnotesize $\x$};
                \draw[->] (0, -0.5) -- (0, 3.25);

                \foreach \y in {1,...,3}
                    \draw[shift={(0,\y)}] (2pt,0pt) -- (-2pt,0pt) node[left] {\footnotesize $\y$};

                \draw (0, 0) node[below left] {\footnotesize $0$};

                \draw [->, highlighta, line width=2pt] (0,0) -- (3,2) node[midway, below right] {$t₁$};
                \draw [->, highlight3, line width=2pt] (0,0) -- (1.5,1) node[midway, above left] {$t₂$};
            \end{tikzpicture}
        \end{figure}
        \vspace{-\bigskipamount}
        Maximal pour des vecteurs colinéaires
        \begin{equation}
            \innprod{t₁}{t₂} = \norm{t₁}₂×\norm{t₂}₂×\underbrace{\cos 0}_{1} = \norm{t₁}₂×\norm{t}₂
        \end{equation}
    \end{columns}
    Il est négatif pour des vecteurs de sens opposés,  mais ça ne peut pas être le cas avec nos sacs de mots.
\end{frame}

\begin{frame}{Similarités cosinus}
    On est tenté d'utiliser le produit scalaire de deux vecteurs comme mesure de similarité
        \begin{itemize}
            \item Si deux textes n'ont aucun mot en commun, les sacs de mots correspondants seront des vecteurs orthogonaux → similarité nulle
        \end{itemize}
    Mais le terme $\norm{t₁}₂×\norm{t}₂$ induit un bias
        \begin{itemize}
            \item Utiliser des fréquences ne garantit pas que la norme euclidienne vaille $1$ !
        \end{itemize}
    On utilise donc plutôt des normalisations du produit scalaire.
\end{frame}

\begin{frame}{Similarité cosinus}
    \begin{block}{Définition (Similarité cosinus)}
        On appelle \emph{similarité cosinus des vecteur $t₁$ et $t₂$} le cosinus de l'écart angulaire entre $t₁$ et $t₂$.
    \end{block}
    Autrement dit $\cos θ$. Ce qui revient à calculer
    \begin{equation}
        \cos θ = \frac{\innprod{t₁}{t₂}}{\norm{t₁}₂×\norm{t₂}₂}
    \end{equation}
    Soit pour notre exemple
    \begin{equation}
        \cos θ = \frac{5}{√{5} + √{10}} = \frac{√2}{2}
    \end{equation}

    C'est la plus simple des normalisations du produit scalaire, et celle qu'on utilise en général.
\end{frame}

\begin{frame}{Autres normalisations}
    \begin{itemize}
        \item Mesure de Sørensen–Dice
            \begin{equation}
                2×\frac{\innprod{t₁}{t₂}}{\norm{t₁}₂+\norm{t₂}₂}
            \end{equation}
        \item Mesure de Jaccard
            \begin{equation}
                \frac{\innprod{t₁}{t₂}}{\norm{t₁+t₂}₂}
            \end{equation}
        \item Coefficient de recouvrement
            \begin{equation}
                \frac{\innprod{t₁}{t₂}}{\min\mleft(\norm{t₁}₂, \norm{t₂}₂\mright)}
            \end{equation}
    \end{itemize}

Attention : en général, ces mesures ne seront pas nécessairement inférieures à $1$.
\end{frame}

\begin{frame}{Autres normalisations : versions booléennes}
    Si les coordonnées des vecteurs indiquent la présence d'un mot plutôt que sa fréquence, on a
    \begin{itemize}
        \item Mesure de Sørensen–Dice
            \begin{equation}
                2×\frac{\#(t₁∩t₂)}{\#t₁+\#t₂}
            \end{equation}
        \item Mesure de Jaccard
            \begin{equation}
                \frac{\#(t₁∩t₂)}{\#(t₁∪t₂)}
            \end{equation}
        \item Coefficient de recouvrement
            \begin{equation}
                \frac{\#(t₁∩t₂)}{\min(\#t₁, \#t₂)}
            \end{equation}
    \end{itemize}
Dans ce cas, ces mesures sont bien comprises entre $0$ et $1$.
\end{frame}

%  █████  ██████  ██████  ███████ ███    ██ ██████  ██ ██   ██
% ██   ██ ██   ██ ██   ██ ██      ████   ██ ██   ██ ██  ██ ██
% ███████ ██████  ██████  █████   ██ ██  ██ ██   ██ ██   ███
% ██   ██ ██      ██      ██      ██  ██ ██ ██   ██ ██  ██ ██
% ██   ██ ██      ██      ███████ ██   ████ ██████  ██ ██   ██

\ifSubfilesClassLoaded{
	\appendix
	\pgfkeys{/metropolis/inner/sectionpage=simple}  % Avoid random errors with section page progressbar
	\section{Annexes}
	\pdfbookmark[2]{Remerciements}{acknowledgements}
	\begin{frame}{Remerciements}
		Ce cours a été construit à partir du polycopié de cours \citetitle{tellier2017IntroductionFouilleTextes} \parencite{tellier2017IntroductionFouilleTextes} et des précieux conseils d'Isabelle Tellier que je ne saurais trop remercier pour sa confiance et son dévouement.
	\end{frame}

	\pdfbookmark[2]{Références}{references}
	\begin{frame}[allowframebreaks]{References}
		\printbibliography[heading=none]
	\end{frame}

	\pdfbookmark[2]{Licence}{licence}
	\begin{frame}{Licence}
		\begin{center}
			{\huge \ccby}
			\vfill
			This document is available under the terms of the Creative Commons Attribution 4.0 International License (CC BY 4.0) (\shorturl{creativecommons.org/licenses/by/4.0})

			Exceptions to the above statement are listed at {\small\shorturl{loicgrobol.github.io/intro-fouille-textes\#licences}}
			\vfill
			© 2019, Loïc Grobol <\shorturl[mailto][:]{loic.grobol@gmail.com}>

			\shorturl[http]{lattice.cnrs.fr/Grobol-Loic}
		\end{center}
	\end{frame}
}{}

\end{document}
