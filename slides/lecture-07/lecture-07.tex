% LTeX: language=fr
% Copyright © 2019, Loïc Grobol <loic.grobol@gmail.com>
% This document is available under the terms of the Creative Commons Attribution 4.0 International License (CC BY 4.0) (https://creativecommons.org/licenses/by/4.0/)

\documentclass[../allslides.tex]{subfiles}

\begin{document}

\renewcommand\docdate{2021-02-18}  % chktex 8

\lecture{Classification et apprentissage artificiel}{Cours 7}


\begin{frame}{Exemples de travail}
	Mettre le corpus suivant sous forme tabulaire, en utilisant comme attributs les occurrences de noms ayant trait au \emph{cinéma} ou à l'\emph{économie} puis le représenter dans le plan
	{\small
		\begin{enumerate}
			\item « Le cinéma est un art, c’est aussi une industrie. »
			\item « Personne, quand il est petit, ne veut être critique de cinéma. Mais ensuite, en France, tout le monde a un deuxième métier : critique de cinéma ! »
			\item « Tout le monde a des rêves de Hollywood. »
			\item « Pendant la crise, l’usine à rêves Hollywood critique le cynisme de l’industrie. »
			\item « C’est la crise, l’économie de la France est menacée par la mondialisation. »
			\item « En temps de crise, reconstruire l’industrie : tout un art ! »
			\item « Quand une usine ferme, c’est que l’économie va mal. »
		\end{enumerate}
	}
\end{frame}

\begin{frame}{Exemples de travail}
	\begin{enumerate}
		\item<+-> Identifier dans la représentation précédente deux classes
		\item<+-> Dans la suite du cours on considérera que les documents 1 à 4 appartiennent à la classe \emph{culture} et les documents de 5 à 7 à la classe \emph{société}.
	\end{enumerate}
\end{frame}

\section{Classification}

\begin{frame}[fragile=singleslide]{Classification}
    La \emph{classification} est la tâche consistant à \alert{répartir} des données dans un ensemble de classes.
    \begin{itemize}
        \item « Détecter les \textcolor{highlight8}{spams} dans un ensemble d'\textcolor{highlight4}{emails} »
        \item « Repérer dans un ensemble de \textcolor{highlight4}{tweets} ceux qui sont \textcolor{highlight8}{sarcastiques} ou \textcolor{highlight8}{ironiques} »
        \item « Classer un ensemble de \textcolor{highlight4}{textes} par \textcolor{highlight8}{genre littéraire} »
    \end{itemize}
    \begin{figure}
        \tikzset{external/export=true}
        \begin{tikzpicture}[true scale=0.80]
            \node[data, highlight=4] (in) {Donnée};
            \node[program, right=1cm of in, text width=15ex] (prog) {Programme de classification};
            \node[data, right=1cm of prog, highlight=1] (out) {Classe};
            \node[data, below=.6cm of prog, highlight=8] (res) {Classes possibles};
            \draw[->] (in) -- (prog);
            \draw[->] (prog) -- (out);
            \draw[->] (res) -- (prog);
        \end{tikzpicture}
    \end{figure}
    \begin{itemize}
        \item Les \textcolor{highlight8}{classes} sont connus à l'avance
        \item Chaque \textcolor{highlight4}{donnée} appartient à une et une seule \textcolor{highlight1}{classe}.
    \end{itemize}
\end{frame}

\subsection{Classification par règles expertes}
\begin{frame}{Classification par règles expertes}
    La façon traditionnelle de faire de la classification
    \begin{itemize}
        \item<+-> Faire écrire par des experts du domaine un ensemble de règles associant un exemple à une classe
        \item<+-> Exemple : classification du sentiment
			\begin{itemize}
                \item Établir un lexique affectif mot → \(±1\)
                \item Associer à chaque texte la somme des polarités de ses termes
                \item Un texte est positif si la somme est positive, négatif sinon
            \end{itemize}
        \item<+-> Qu'en pensez-vous ?
        \item<+-> Pour notre exemple de classification culture/société, comment faire ?
    \end{itemize}
\end{frame}

\begin{frame}{Classification par apprentissage artificiel}
    \begin{block}{Rappel}
        Un \textcolor{highlight0}{programme d'apprentissage artificiel} est un programme générant à partir d'\textcolor{highlight2}{exemples} traités par des humains un \textcolor{highlight4}{modèle} qui sert de ressource à un \textcolor{highlight5}{programme réalisant une tâche}.
    \end{block}
    \begin{figure}
        \tikzset{external/export=true}
        \begin{tikzpicture}[scale=0.9, every node/.style={transform shape}]
            \node[data] (in) {Entrée};
            \node[program, highlight=5, right=.75cm of in, text width=10ex] (prog) {Programme réalisant la tâche};
            \node[data, right=.75cm of prog] (out) {Résultat};
            \node[program, highlight=0, above=1cm of prog, text width=13ex] (progapp) {Programme d'apprentissage};
            \node[data, highlight=4, right=.75cm of progapp] (mod) {Modèle};
            \node[data, highlight=2, left=.75cm of progapp] (train) {Exemples};
            \draw[->] (in) -- (prog);
            \draw[->] (prog) -- (out);
            \draw[->] (mod) |- ($(prog.north)+(0, 0.5cm)$) -| (prog.north);
            \draw[->] (train) -- (progapp);
            \draw[->] (progapp) -- (mod);
        \end{tikzpicture}
        \caption{Schéma d'une approche par apprentissage artificiel}
    \end{figure}
\end{frame}

\begin{frame}{Exemple : régression}
    « Sachant que ma maison fait \(s\,\si{m\square}\), en fonction du marché, quel prix puis-je en attendre ? »
    \begin{description}
        \item[Hypothèse] Il existe une fonction \(f\) qui à la surface \(x\) d'une maison associe son prix de vente
        \item[Données] Le prix de vente de \num{8} maisons et leurs surfaces respectives
        \item[Objectif] Déterminer l'expression de \(f\) et en déduire \(f(s)\), le prix de vente de ma maison
    \end{description}
    \begin{itemize}
        \item[→] Problème : \(f\) pourrait avoir n'importe quelle forme !
    \end{itemize}
\end{frame}

\begin{frame}{Choix d'une famille de fonctions}
    Problème : \(f\) pourrait avoir n'importe quelle forme !
    \begin{itemize}
        \item[→] On choisit une famille de fonctions et on y cherche celle qui semble le mieux correspondre à nos exemples
        \item Par exemple avec des polynômes :
            \begin{equation}
                f(x) = a₀ + a₁x + a₂x²+ … + a_nx^n
            \end{equation}
        \item En général « correspondre » au sens des moindres carrés
    \end{itemize}
    \textbf{Vocabulaire} Le degré (fixé manuellement) est un \alert{hyperparamètre}, les coefficients \(a_i\) appris automatiquement sont des \alert{paramètres}.
    Déterminer les paramètres qui minimisent le coût (ici la distance entre le modèle est les exemple) est une tâche d'\alert{optimisation}.

    Étant donné une surface, on peut alors donner le prix correspondant : on donc a appris à prédire les prix !
\end{frame}

\begin{frame}{Régression linéaire}
    \vspace{-1\bigskipamount}
    \begin{figure}
        \tikzset{external/export=true}
        \begin{tikzpicture}
            \begin{axis}[
				xmin=0, xmax=800,
				ymin=0, ymax=600000,
                width=\textwidth,
                height=0.8\textheight,
                xlabel={Surface},
                ylabel={Prix},
                title={Régression linéaire : \(y=ax+b\)},
                scaled ticks=false,
                ticklabel style={/pgf/number format/.cd, 1000 sep={\,}}
			]
                \addplot[
                    only marks
                ] table [x index=0, y index=1, header=false, row sep=crcr]{
                    150	200000\\
                    155	210000\\
                    160	250000\\
                    170	300000\\
                    200	310000\\
                    260	350000\\
                    300	400000\\
                    600	405000\\
                };
                \addplot[highlighta, domain=0:800, samples=1001] {402.28*x+202807.17};
            \end{axis}
        \end{tikzpicture}
        \caption{Polynôme de degré \num{1} (linéaire)}
    \end{figure}

    Voir aussi \shorturl{ggbm.at/TagmZdxN}
\end{frame}

\begin{frame}{Régression quadratique}
    \vspace{-1\bigskipamount}
    \begin{figure}
        \tikzset{external/export=true}
        \begin{tikzpicture}
            \begin{axis}[
				xmin=0, xmax=800,
				ymin=0, ymax=600000,
                width=\textwidth,
                height=0.8\textheight,
                xlabel={Surface},
                ylabel={Prix},
                title={Régression quadratique : \(y=ax²+bx+c\)},
                scaled ticks=false,
                ticklabel style={/pgf/number format/.cd, 1000 sep={\,}}
			]
                \addplot[
                    only marks
                ] table [x index=0, y index=1, header=false, row sep=crcr]{
                    150	200000\\
                    155	210000\\
                    160	250000\\
                    170	300000\\
                    200	310000\\
                    260	350000\\
                    300	400000\\
                    600	405000\\
                };
                \addplot[highlighta, domain=0:800, samples=1001] {-2.56*x^2+2319.29*x-64113.23};
            \end{axis}
        \end{tikzpicture}
        \caption{Polynôme de degré \num{2} (quadratique)}
    \end{figure}
\end{frame}

\begin{frame}{Régression polynomiale}
    \vspace{-1\bigskipamount}
    \begin{figure}
        \tikzset{external/export=true}
        \begin{tikzpicture}
            \begin{axis}[
				xmin=0, xmax=800,
				ymin=0, ymax=600000,
				restrict y to domain=-1000:700000,
				width=\textwidth,
				height=0.8\textheight,
				xlabel={Surface},
				ylabel={Prix},
				title={\(y=a₀ + a₁x + a₂x²+ … + a₆x⁶\)},
				scaled ticks=false,
				ticklabel style={/pgf/number format/.cd, 1000 sep={\,}}
            ]
                \addplot[
                    only marks
                ] table [x index=0, y index=1, header=false, row sep=crcr]{
                    150	200000\\
                    155	210000\\
                    160	250000\\
                    170	300000\\
                    200	310000\\
                    260	350000\\
                    300	400000\\
                    600	405000\\
                };
                \addplot[highlighta, domain=100:310, samples at={100,101,...,311,311.1,311.15,311.18970915,311.19,311.2,...,312}] {9.133693238227375E-7*x^6-0.0015210072640020665*x^5+0.9901617434548338*x^4-326.9727250825201*x^3+58224.97189714596*x^2-5327933.125949258*x+196687335.202412};
                \addplot[highlighta, domain=590:610, samples at={599.001,599.002,...,599.945,599.9455,599.945526073,599.9456,599.95,599.96,...,600,600.00005,...,610}] {9.133693238227375E-7*x^6-0.0015210072640020665*x^5+0.9901617434548338*x^4-326.9727250825201*x^3+58224.97189714596*x^2-5327933.125949258*x+196687335.202412};
            \end{axis}
        \end{tikzpicture}
        \caption{Polynôme de degré \num{6}}\label{fig|overfit}
    \end{figure}
\end{frame}

\begin{frame}{Surapprentissage}
    La figure \ref{fig|overfit} est un exemple de cas de \alert{surapprentisage} (\emph{overfitting})
    \begin{itemize}
        \item Avoir choisi le degré \num{6} permet de passer exactement par tous les points d'entraînement
        \item Mais le modèle est très mauvais !
    \end{itemize}
    Comment trouver un modèle \emph{vraiment} bon ?
    \begin{itemize}
        \item En limitant le nombre de paramètres
        \item En limitant la précision des paramètres
        \item Par le choix des hyperparamètres
        \item En utilisant un ensemble de développement
    \end{itemize}
\end{frame}

\begin{frame}{« Apprentissage » ?}
    Ce qu'on appelle \emph{apprendre} pour une machine consiste à transformer des exemples \((x_i, y_i)\) en règle \(f: x⟼y\). Soit
    \begin{itemize}
        \item Étant donné un espace de recherche \(F\)
        \item Étant donnée une mesure de qualité \(q\) sur \(F\)
        \item Trouver \(f∈F\) qui maximise \(q\)
    \end{itemize}
    En général \(F\) est paramétré par des nombre \(a₁,…,a_n\) et on se ramène ainsi à chercher leurs valeurs pour que \(q(f_{a₁,…,a_n})\) soit maximal.
\end{frame}

\begin{frame}{« Automatique » ?}
    Il reste encore à l'humain à
    \begin{enumerate}[<alert@+>]
        \item\label{item|searchspace} Choisir un espace de recherche adapté
        \item\label{item|loss} Choisir une mesure de qualité adaptée
        \item\label{item|optim} Choisir une procédure d'optimisation adaptée
    \end{enumerate}
    \only<1-3>{En tenant compte}

    \begin{overprint}
    \onslide<1>
        Pour \ref{item|searchspace},
        \begin{itemize}
            \item De l'expressivité du modèle et de son adéquation au problème
            \item De la lisibilité du résultat
        \end{itemize}
    \onslide<2>
        Pour \ref{item|loss},
        \begin{itemize}
            \item De l'adéquation aux données d'apprentissage
            \item La généralisabilité (éviter le surapprentissage)
            \item La simplicité : rasoir d'Ockham
        \end{itemize}
    \onslide<3>
        Pour \ref{item|optim},
        \begin{itemize}
            \item De l'incrémentalité
            \item Du temps de calcul
        \end{itemize}
        On verra dans la suite que ces trois choix sont souvent dépendants.
    \onslide<4>
        \begin{itemize}
            \item Il n'y a pas de recette miracle qui marche à chaque fois : « \emph{There is no free lunch} »
            \item Pour cette raison, je préfère parler d'apprentissage \emph{artificiel} plutôt qu'\emph{automatique} cf \emph{Machine Learning}
        \end{itemize}
    \end{overprint}
\end{frame}

\begin{frame}{Plan des réjouissances}
    On présente dans la suite plusieurs techniques classiques de classification par apprentissage artificiel suivant avec pour chacune
    \begin{itemize}
        \item Espace de recherche
        \item Hyperparamètres
        \item Technique de recherche
        \item Propriétés et usages
    \end{itemize}
\end{frame}

\begin{frame}{Think fast}
    Dans un problème de classification, dans l'ensemble d'entraînement, il y a \num{1798} exemples de la classe A et \num{2} de la classe B.

    On vous donne un nouvel exemple dont vous ne connaissez pas la classe sans plus d'information. Dans quelle classe le rangez-vous ?
\end{frame}

% ███████ ███████ ██████   ██████  ██████
%    ███  ██      ██   ██ ██    ██ ██   ██
%   ███   █████   ██████  ██    ██ ██████
%  ███    ██      ██   ██ ██    ██ ██   ██
% ███████ ███████ ██   ██  ██████  ██   ██

\subsection{Algorithme de la classe majoritaire}
\begin{frame}{Algorithme de la classe majoritaire}
    \textbf{Dans Weka} classifiers > rules > ZeroR

    \textbf{Espace de recherche} Ensemble des fonctions constantes

    Autrement dit : on renvoie toujours la même classe quelle que soit l'entrée.

    \textbf{Hyperparamètres} Aucun !

    \textbf{Technique de recherche} On choisit la fonction qui renvoie la classe majoritaire dans l'ensemble d'apprentissage.
\end{frame}

\begin{frame}{Propriétés}
    \begin{itemize}
        \item Très rapide en apprentissage comme en test
        \item Très peu performant
            \begin{itemize}
                \item[→] Dans le meilleur des cas (si les classes sont déséquilibrées) on fait un peu mieux que le hasard
            \end{itemize}
    \end{itemize}
    Il s'agit plutôt d'une \emph{baseline} : un système sérieux devrait en principe faire mieux.
\end{frame}

% ██   ██      ███    ██ ███    ██
% ██  ██       ████   ██ ████   ██
% █████  █████ ██ ██  ██ ██ ██  ██
% ██  ██       ██  ██ ██ ██  ██ ██
% ██   ██      ██   ████ ██   ████

\subsection{\(k\)-plus proches voisins}

\begin{frame}{\(k\)-plus proches voisins}
    \textbf{Dans Weka} classifiers > lazy > lBk

    \textbf{Espace de recherche} Le singleton \(\{f\}\), où \(f\) est la fonction qui à un exemple \(x\) associe la classe majoritaire dans l'ensemble des \(k\) données d'entraînement les plus proches de \(x\).

    En général on utilise la distance euclidienne ou le coefficient d'overlap (pour des vecteurs booléens)

    \textbf{Hyperparamètres} \(k\), un entier \only<2>{
        \begin{itemize}
            \item Plus petit que le nombre d'exemples d'apprentissages
            \item Idéalement non-multiple du nombre de classes
        \end{itemize}}

    \textbf{Technique de recherche} Aucune !
\end{frame}

% TODO: improve this
\begin{frame}[fragile]{Exemple}
    \begin{figure}
        \tikzset{external/export=true}
        \begin{tikzpicture}
            \draw [help lines, xstep=1cm, ystep=1cm] (0, 0) grid (5.25, 4.25);

            \draw[->] (-0.5,0) -- (5.25, 0);
            \foreach \x in {1,...,5}
                \draw[shift={(\x, 0)}] (0pt,2pt) -- (0pt,-2pt) node[below] {\footnotesize \(\x\)};

            \draw[->] (0, -0.5) -- (0, 4.25);
            \foreach \y in {1,...,4}
                \draw[shift={(0,\y)}] (2pt,0pt) -- (-2pt,0pt) node[left] {\footnotesize \(\y\)};

            \draw (0, 0) node[below left] {\footnotesize \num{0}};

            \foreach \x/\y/\c in {2/1/highlight4,4/0/highlight4,1/0/highlight4,2/3/highlight4,0/3/highlighta,1/2/highlighta,0/2/highlighta}{
                \path[fill=\c] (\x, \y) circle[radius=3pt];
            }

            \foreach \ex/\ey [count=\i from 1] in {1/3,1/1,3/2} {
                \path[fill=highlight2, visible on=\i] (\ex, \ey) circle[radius=3pt];
            }
        \end{tikzpicture}
        \caption{Exemple de classification : \(k\)-NN}
    \end{figure}
    \only<3>{}
\end{frame}

\begin{frame}{Propriétés}
    \begin{itemize}
        \item Aucune recherche → apprentissage immédiat
        \item Coût en opération : Taille de l'ensemble d'entraînement \(×\) coût d'un calcul de distance
        \item Pour être efficace : ensemble d'entraînement suffisamment dense
            \begin{itemize}
                \item[→] D'où sa faible utilisation en pratique en fouille de texte (à part comme baseline)
            \end{itemize}
        \item Fréquemment utilisé pour des moteurs de recommandation
    \end{itemize}
\end{frame}

%  █████  ██████  ██████  ███████ ███    ██ ██████  ██ ██   ██
% ██   ██ ██   ██ ██   ██ ██      ████   ██ ██   ██ ██  ██ ██
% ███████ ██████  ██████  █████   ██ ██  ██ ██   ██ ██   ███
% ██   ██ ██      ██      ██      ██  ██ ██ ██   ██ ██  ██ ██
% ██   ██ ██      ██      ███████ ██   ████ ██████  ██ ██   ██

\ifSubfilesClassLoaded{
	\appendix
	\pgfkeys{/metropolis/inner/sectionpage=simple}  % Avoid random errors with section page progressbar
	\section{Annexes}
	\pdfbookmark[2]{Remerciements}{acknowledgements}
	\begin{frame}{Remerciements}
		Ce cours a été construit à partir du polycopié de cours \citetitle{tellier2017IntroductionFouilleTextes} \parencite{tellier2017IntroductionFouilleTextes} et des précieux conseils d'Isabelle Tellier que je ne saurais trop remercier pour sa confiance et son dévouement.
	\end{frame}

	\pdfbookmark[2]{Références}{references}
	\begin{frame}[allowframebreaks]{References}
		\printbibliography[heading=none]
	\end{frame}

	\pdfbookmark[2]{Licence}{licence}
	\begin{frame}{Licence}
		\begin{center}
			{\huge \ccby}
			\vfill
			This document is available under the terms of the Creative Commons Attribution 4.0 International License (CC BY 4.0) (\shorturl{creativecommons.org/licenses/by/4.0})

			Exceptions to the above statement are listed at {\small\shorturl{loicgrobol.github.io/intro-fouille-textes\#licences}}
			\vfill
			© 2019, Loïc Grobol <\shorturl[mailto][:]{loic.grobol@gmail.com}>

			\shorturl[http]{lattice.cnrs.fr/Grobol-Loic}
		\end{center}
	\end{frame}
}{}
\end{document}
