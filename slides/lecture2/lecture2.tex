% Copyright © 2018, Loïc Grobol <loic.grobol@gmail.com>
% This document is available under the terms of the Creative Commons Attribution 4.0 International License (CC BY 4.0) (https://creativecommons.org/licenses/by/4.0/)

\RequirePackage{xparse}
\RequirePackage{shellesc}
% Settings
\NewDocumentCommand\myname{}{Loïc Grobol}
\NewDocumentCommand\mylab{}{Lattice / ALMAnaCH}
\NewDocumentCommand\pdftitle{}{Introduction à la fouille de textes}
\NewDocumentCommand\mymail{}{loic.grobol@gmail.com}
\NewDocumentCommand\titlepagetitle{}{\pdftitle}
\NewDocumentCommand\docdate{}{2019-01-28}
\NewDocumentCommand\conference{}{M1 Plurital}

\documentclass[hyperref={unicode}, xcolor={svgnames}, french]{beamer}
\usetheme[sectionpage=progressbar,
          subsectionpage=progressbar,
          progressbar=frametitle]{metropolis}
    \definecolor{accent}{RGB}{51, 34, 136}
    \setbeamercolor{alerted text}{fg=accent}
    \makeatletter
        \setlength{\metropolis@progressinheadfoot@linewidth}{1pt}
    \makeatother
    % FIXME: not pretty and footnotes are still too big
    \let\footnoterule\relax  % No footnote rule, push down footnote

% 9-Highlights colour palette from [Paul Tol's technical note](https://personal.sron.nl/~pault/data/colourschemes.pdf)
\definecolor{highlight0}{RGB}{51, 34, 136}    % Deep blue
\definecolor{highlight1}{RGB}{136, 204, 238}  % Clear blue
\definecolor{highlight2}{RGB}{68, 170, 153}   % Teal
\definecolor{highlight3}{RGB}{17, 119, 51}    % Forest green
\definecolor{highlight4}{RGB}{153, 153, 51}   % Khaki
\definecolor{highlight5}{RGB}{221, 204, 119}  % Beige
\definecolor{highlight6}{RGB}{204, 102, 119}  % Pink
\definecolor{highlight7}{RGB}{136, 34, 85}    % Purple
\definecolor{highlight8}{RGB}{170, 68, 153}   % Violet

% Alternative navy blue for ⩽4 palettes
\definecolor{highlighta}{RGB}{68, 118, 170}  % Navy blue

% Use non-standard fonts
\usefonttheme{professionalfonts}
\setsansfont[BoldFont={Fira Sans SemiBold}, ItalicFont={Fira Sans Book Italic}]{Fira Sans Book}
\setmonofont[Scale=0.9]{Fira Mono}

% Fix missing glyphs in Fira by delegating to polyglossia/babel
\usepackage{newunicodechar}
\newunicodechar{ }{~}   % U+202F NARROW NO-BREAK SPACE
\newunicodechar{ }{ }  % U+2009 THIN SPACE

% Notes on left screen
% \usepackage{pgfpages}
% \setbeameroption{show notes on second screen=left}


\usepackage[french, english]{babel}
\usepackage{amsfonts,amssymb}
\usepackage{amsmath,amsthm}
\usepackage{mathtools}	% AMS Maths service pack
	\newtagform{brackets}{[}{]}	% Pour des lignes d'équation numérotées entre crochets
	\mathtoolsset{showonlyrefs, showmanualtags, mathic}	% affiche les tags manuels (\tag et \tag*) et corrige le kerning des maths inline dans un bloc italique voir la doc de mathtools
	\usetagform{brackets}	% Utilise le style de tags défini plus haut
\usepackage{lualatex-math}

\usepackage[math-style=french]{unicode-math}
	\setmathfont{Libertinus Math}
\usepackage{newunicodechar}
	\newunicodechar{√}{\sqrt}
\usepackage{mleftright}


\usepackage{tabu}
\usepackage{booktabs}
\usepackage{siunitx}
\usepackage{multicol}
\usepackage{ccicons}
\usepackage{bookmark}
\usepackage{caption}
    \captionsetup{skip=1ex}
\usepackage[iso]{datetime}

\usepackage{csquotes}
\usepackage{tikz}
	\NewDocumentCommand{\textnode}{O{}mm}{\tikz[remember picture, baseline=(#2.base), inner sep=0pt]{\node[#1] (#2) {#3};}}
    \NewDocumentCommand{\mathnode}{O{}mm}{\tikz[remember picture, baseline=(#2.base), inner sep = 0pt]{\node[#1] (#2) {$\displaystyle #3$};}}
	\tikzset{
		alt/.code args={<#1>#2#3}{%
		\alt<#1>{\pgfkeysalso{#2}}{\pgfkeysalso{#3}} % \pgfkeysalso doesn't change the path
		},
        invisible/.style={opacity=0, fill opacity=0},
		visible on/.style={alt={<#1>{}{invisible}}}
	}
    \usepackage{forest}
    \usepackage{tkz-graph}
    \usepackage[beamer, markings]{hf-tikz}
    \usepackage{tikz-3dplot}
    \usepackage{pgfplots}
        % Due to pgfplots meddling with pgfkeys, we have to redefine alt here.
        \pgfplotsset{
    		alt/.code args={<#1>#2#3}{%
    		\alt<#1>{\pgfkeysalso{#2}}{\pgfkeysalso{#3}} % \pgfkeysalso doesn't change the path
    		},
    	}
        \pgfplotsset{compat=1.15}
        \pgfplotsset{colormap={SRON}{rgb255=(61,82,161) rgb255=(255,250,210) rgb255=(174,28,62)}}

    \usetikzlibrary{matrix}
    \usetikzlibrary{shapes, shapes.geometric}
    \usetikzlibrary{decorations.pathreplacing}
	\usetikzlibrary{positioning, calc, intersections}
    \usetikzlibrary{fit}
    \usetikzlibrary{backgrounds}

    % Do evil things with soft path
    % From <https://tex.stackexchange.com/a/301364/8547>
    \makeatletter
        \def\@appendnamedsoftpath#1{%
            \pgfsyssoftpath@getcurrentpath\@temppatha
            \expandafter\let\expandafter\@temppathb\csname tikz@intersect@path@name@#1\endcsname
            \expandafter\expandafter\expandafter\def\expandafter\expandafter\expandafter\@temppatha\expandafter\expandafter\expandafter{\expandafter\@temppatha\@temppathb}%
            \pgfsyssoftpath@setcurrentpath\@temppatha
        }
        \def\@appendnamedpathforactions#1{%
            \pgfsyssoftpath@getcurrentpath\@temppatha
            \expandafter\let\expandafter\@temppathb\csname tikz@intersect@path@name@#1\endcsname
            \expandafter\def\expandafter\@temppatha\expandafter{\csname @temppatha\expandafter\endcsname\@temppathb}%\usepackage{tikz-3dplot}

            \let\tikz@actions@path\@temppatha
        }

        \tikzset{
            use path for main/.code={%
                \tikz@addmode{%
                    \expandafter\pgfsyssoftpath@setcurrentpath\csname tikz@intersect@path@name@#1\endcsname
                }%
            },
            append path for main/.code={%
                \tikz@addmode{%
                    \@appendnamedsoftpath{#1}%
                }%
            },
            use path for actions/.code={%
                \expandafter\def\expandafter\tikz@preactions\expandafter{\tikz@preactions\expandafter\let\expandafter\tikz@actions@path\csname tikz@intersect@path@name@#1\endcsname}%
            },
            append path for actions/.code={%
                \expandafter\def\expandafter\tikz@preactions\expandafter{\tikz@preactions
                \@appendnamedpathforactions{#1}}%
            },
            use path/.style={%
                use path for main=#1,
                use path for actions=#1,
            },
            append path/.style={%
                append path for main=#1,
                append path for actions=#1
            }
        }
    \makeatother

    % TikZ externalisation
    \usetikzlibrary{external}
    % Create the `tikzpics/` folder if it does not exist
    \ShellEscape{mkdir -p tikzpics}
    % Only externalise pictures on demand (to avoid messing up with metropolis theme)
    \tikzset{
        external/export=false,
        external/prefix=tikzpics/
    }
    \tikzexternalize

\usepackage{minted}
	\usemintedstyle{lovelace}
	\setminted{autogobble, fontsize=\scriptsize, tabsize=2}
	\setmintedinline{fontsize=auto}

\usepackage[style=authoryear, block=ragged, doi=false, isbn=false]{biblatex}
    \AtEveryBibitem{
        \ifentrytype{online}
        {} {
            \iffieldequalstr{howpublished}{online}
            {
                \clearfield{howpublished}
            } {
                \clearfield{urlyear}\clearfield{urlmonth}\clearfield{urlday}
            }
        }
    }

	\addbibresource{biblio.bib}

% Compact bibliography style
\setbeamertemplate{bibliography item}[text]

\AtEveryBibitem{
    \clearfield{series}
    \clearfield{pages}
    \clearlist{publisher}
    \clearname{editor}
    \clearlist{location}
}
\renewcommand*{\bibfont}{\tiny}

\usepackage{hyperxmp}	% XMP metadata

\usepackage[type={CC},modifier={by},version={4.0}]{doclicense}

\usepackage{todonotes}
\let\todox\todo
\renewcommand\todo[1]{\todox[inline]{#1}}

\title{\titlepagetitle}
\subtitle{Cours 2 : \docdate}
\author{\textbf{\myname} (\mylab)}
\institute{}
\date{\footnotesize Version {\yyyymmdddate\today}T\currenttime}

\titlegraphic{\ccby}

% Tikz styles

% Schémas de tâches
\tikzset{
    >=stealth,
    hair lines/.style={line width = 0.05pt, lightgray},
    data/.style={draw, ellipse},
    program/.style={draw, rectangle},
    accent on/.style={alt={<#1>{draw=accent, text=accent, thick}{draw}}},
    true scale/.style={scale=#1, every node/.style={transform shape}},
    highlight/.style={color=highlight#1}
}

% Styles des heatmap pour les moyennes
\pgfplotsset{
    meanheatmap/.style={
        colorbar, colormap name=SRON,
        view={0}{90},
        samples=100,
        domain=0:1,
        min=0, max=1,
        xlabel={$P$},
        ylabel={$R$},
    }
}

% Commands spécifiques
\NewDocumentCommand\shorturl{ O{https} O{://} m }{%
    \href{#1#2#3}{\nolinkurl{#3}}%
}

\DeclarePairedDelimiter\norm{\lVert}{\rVert}
\DeclarePairedDelimiter\abs{\lvert}{\rvert}
\DeclarePairedDelimiterX\compset[2]{\lbrace}{\rbrace}{#1\,\delimsize|\,#2}
\DeclarePairedDelimiterX\innprod[2]{\langle}{\rangle}{#1\,\delimsize|\,#2}

\DeclareMathOperator*\argmax{argmax}

% Easy column vectors \vcord{a,b,c} ou \vcord[;]{a;b;c}
% Here be black magic
\ExplSyntaxOn
	\NewDocumentCommand{\vcord}{O{,}m}{\vector_main:nnnn{p}{\\}{#1}{#2}}
	\NewDocumentCommand{\tvcord}{O{,}m}{\vector_main:nnnn{psmall}{\\}{#1}{#2}}
	\seq_new:N\l__vector_arg_seq
	\cs_new_protected:Npn\vector_main:nnnn #1 #2 #3 #4{
		\seq_set_split:Nnn\l__vector_arg_seq{#3}{#4}
		\begin{#1matrix}
			\seq_use:Nnnn\l__vector_arg_seq{#2}{#2}{#2}
		\end{#1matrix}
	}
\ExplSyntaxOff

\DeclareMathOperator{\TF}{TF}
\DeclareMathOperator{\IDF}{IDF}

\ExplSyntaxOn
    \DeclareExpandableDocumentCommand\eval{m}{\fp_eval:n{#1}}
\ExplSyntaxOff


% ██████   ██████   ██████ ██    ██ ███    ███ ███████ ███    ██ ████████
% ██   ██ ██    ██ ██      ██    ██ ████  ████ ██      ████   ██    ██
% ██   ██ ██    ██ ██      ██    ██ ██ ████ ██ █████   ██ ██  ██    ██
% ██   ██ ██    ██ ██      ██    ██ ██  ██  ██ ██      ██  ██ ██    ██
% ██████   ██████   ██████  ██████  ██      ██ ███████ ██   ████    ██


\begin{document}
\pdfbookmark[2]{Title}{title}

\begin{frame}[plain]
	\titlepage
\end{frame}

% ███████ ██    ██  █████  ██      ██    ██  █████  ████████ ██  ██████  ███    ██
% ██      ██    ██ ██   ██ ██      ██    ██ ██   ██    ██    ██ ██    ██ ████   ██
% █████   ██    ██ ███████ ██      ██    ██ ███████    ██    ██ ██    ██ ██ ██  ██
% ██       ██  ██  ██   ██ ██      ██    ██ ██   ██    ██    ██ ██    ██ ██  ██ ██
% ███████   ████   ██   ██ ███████  ██████  ██   ██    ██    ██  ██████  ██   ████

\section{Évaluation}

\begin{frame}{Évaluation}
    Objectif : comparer \alert{objectivement} des programmes réalisant une même tâche
    \begin{itemize}
        \item Doit se faire avec exactement les mêmes données
            \begin{itemize}
                \item Ressources et entrées !
            \end{itemize}
        \item[→] Établissement de données de référence \alert{\textit{Gold Standard}}
            \begin{itemize}
                \item[→] MetaTreebank Universal Dependencies \shorturl[http]{universaldependencies.org}
                \item[→] OntoNotes \shorturl{catalog.ldc.upenn.edu/LDC2013T19}
            \end{itemize}
        \item On met ensuite en concurrence les programmes sur ces données → challenge/shared task
            \begin{itemize}
                \item[→] CoNLL 2018 \shorturl[http]{universaldependencies.org/conll18}
                \item[→] \alert{DEFT} \shorturl{deft.limsi.fr}
            \end{itemize}
    \end{itemize}
\end{frame}

\begin{frame}{\textit{Train} et \textit{test}}
    L'évaluation d'un système ne se fait pas sur le jeu de données utilisées pour l'entraînement
    \begin{itemize}
        \item<+-> Sinon une simple mémorisation suffit !
        \item<+-> On sépare le \textit{Gold Standard} en une partie \alert{\textit{train}} et une partie \alert{\textit{test}}
        \item<.-> Dans le cas d'une campagne d'évaluation, le \textit{test} est souvent tenu secret
            \begin{itemize}
                \item[→] Voir par exemple la \textit{shared task CoNLL 2017 pour un exemple sophistiqué}
            \end{itemize}
        \item<.->[→] Valable y compris pour les programmes non-appris !
        \item<.-> On ajoute aussi parfois un ensemble dit « de développement »
        \item<.-> Classiquement \SI{80}{\percent}/\SI{10}{\percent}/\SI{10}{\percent}
    \end{itemize}
\end{frame}

\begin{frame}[fragile]{Validation croisée}
    Le découpage train/dev/test ajoute de l'objectivité et est facile à mettre en œuvre mais
    \begin{itemize}
        \item Il prive d'une partie des données → sous-performance
        \item Il induit un \alert{biais} lié au découpage si celui-ci est fixe
    \end{itemize}

    Pour y limiter cet effet, on utilise parfois la méthode de \alert{validation croisée}
    \begin{itemize}
        \item<1-> On découpe le corpus en $n$ parties de taille égales
        \item<2-> On utilise $n-1$ parties comme \textcolor{highlighta}{\textit{train}} et la dernière comme \textcolor{highlight6}{\textit{test}}
        \item<3-> On répète l'opération pour les $n$ combinaisons possibles
        \item<12-> On retient comme évaluation globale une moyenne des résultats
    \end{itemize}
    \begin{center}
        \tikzset{external/export=true}
        \begin{tikzpicture}[scale=0.7]
            \foreach \x in {2, ..., 11}
                \draw[alt=<\x>{fill=highlight6}{{alt=<1,12>{}{fill=highlighta}}}]
                    (\x-2, 0) -- ++(1, 0) -- ++(0, 1) -- ++(-1, 0) -- ++(0, -1) -- cycle;
        \end{tikzpicture}
    \end{center}
\end{frame}

\begin{frame}{Tâches élémentaires}
    Parmi l'ensemble des tâches, possible, certaines paraissent plus naturellement \alert{élémentaires}
    \begin{itemize}
        \item Par leur relative \alert{simplicité} de description
        \item Par leur capacité, en se \alert{combinant}, à décrire une grande partie des autres tâches
    \end{itemize}

    Il s'agit de
        \begin{itemize}
            \item la recherche d'information (RI)
            \item la classification
            \item l'annotation
            \item l'extraction d'information (EI)
        \end{itemize}
\end{frame}

% ██████  ██
% ██   ██ ██
% ██████  ██
% ██   ██ ██
% ██   ██ ██


\section{Recherche d'information}

\begin{frame}[fragile=singleslide]{Recherche d'information}
    La \emph{Recherche d'Information} est la tâche consistant à trouver dans un corpus des documents \alert{pertinents} étant donnée une requête
    \begin{itemize}
        \item « Trouver dans le catalogue d'une bibliothèque de ouvrages qui traitent de l'art macabre au XVème siècle »
        \item « Trouver un site internet qui vend des ordinateurs portable sans système d'exploitation »
    \end{itemize}
    \begin{figure}
        \tikzset{external/export=true}
        \begin{tikzpicture}[true scale=0.80]
            \node[data, highlight=4] (in) {Requête};
            \node[program, right=1cm of in, text width=15ex] (prog) {Programme de RI};
            \node[data, right=1cm of prog] (out) {documents};
            \node[data, below=.6cm of prog, highlight=1] (res) {Corpus};
            \draw[->] (in) -- (prog);
            \draw[->] (prog) -- (out);
            \draw[->] (res) -- (prog);
        \end{tikzpicture}
    \end{figure}
    \vskip-1ex
    \begin{itemize}
        \item Le \textcolor{highlight1}{corpus} est une ressource obligatoire
        \item La \textcolor{highlight4}{requête} n'est \emph{pas} faite en langage fortement structuré (type SPARQL)
    \end{itemize}
\end{frame}

\begin{frame}{Domaines d'application}
    Globalement tous les \alert{moteurs de recherche}
    \begin{itemize}
        \item Web : Qwant, Duckduckgo, Framabee, Google…
        \item Site : tous les sites avec une fonction de recherche (e.g. sites marchands)
        \item Documentaires : en bibliothèques, archives…
        \item …
    \end{itemize}
    Mais aussi les détecteurs de similarité
    \begin{itemize}
        \item Détection de plagiat e.g.\ Compilatio (\shorturl{compilatio.net})
        \item Détection de doublons e.g.\ sur le réseau StackExchange
    \end{itemize}
    Note : avoir défini précisément ce que « pertinent » veut dire n'est pas forcément nécessaire ! Il suffit d'avoir défini un \textit{Gold}.
\end{frame}

% WARNING: Here be dragons
\begin{frame}[fragile]{Évaluation : préliminaires}
    \begin{figure}
        \tikzset{external/export=true}
        \begin{tikzpicture}[fill opacity=0.5, text opacity=1,
                            alt=<1>{true scale=0.9}{true scale=0.75}]
            \draw[name path=D]
                (-6,-2) rectangle (6,2)
                node [above left] {$D$ : ensemble de tous les documents};
            \draw[alt={<1,6>{fill=highlight2, text=highlight2}{}}, name path=P]
                (-1, 0) circle [x radius=2, y radius=1, rotate=30]
                node[below left=1.3 and .15] {$P$ : documents pertinents};
            \draw[alt={<1,6>{fill=highlight4, text=highlight4}{}}, name path=R]
                (1, 0) circle [x radius=2, y radius=1.5]
                node[above right=1.5 and 0.1] {$R$ : documents retournés};
            \begin{scope}[alt=<2-4>{even odd rule}{}]
                \clip[alt=<2-3>{use path=P}{use path=R},
                      alt=<3-5>{append path=D}{}];
                \only<5>{\clip[use path=P, append path=D];}
                \path[alt={<2,5>{fill=highlight3}{}},
                      alt={<3,4>{fill=highlight6}{}},
                      alt=<2-3>{use path=R}{use path=P},
                      alt=<5>{use path=D}];
            \end{scope}
        \end{tikzpicture}
        \caption{Représentation de l'ensemble des documents}
    \end{figure}
    \only<2->{
        On définit les indicateurs suivants
        \begin{tabu}{|*{3}{X[m,c]|}}
            \hline
                & \uncover<2,3,6>{\textcolor{highlight4}{Retournés}}   & \uncover<4,5,6>{Non-retournés}\\
            \hline
            \uncover<2,4,6>{\textcolor{highlight2}{Pertinents}}    & \uncover<2,6>{\textcolor{highlight3}{Vrais positifs $P∩R$}}    & \uncover<4,6>{\textcolor{highlight6}{Faux négatifs $P\smallsetminus R$}}\\
            \hline
            \uncover<3,5,6>{Non-pertinents}    & \uncover<3,6>{\textcolor{highlight6}{Faux positifs $R\smallsetminus P$}} & \uncover<5,6>{\textcolor{highlight3}{Vrais négatifs $D\smallsetminus(P∪R)$}}\\
            \hline
        \end{tabu}
    }
\end{frame}

\begin{frame}{Évaluation}
    Les mesures d'évaluations usuelles rendent compte de façon synthétique du tableau précédent pour une expérience de RI
    \begin{description}
        \item[La précision] $P=\frac{VP}{VP+FP}$ « combien de documents retournés sont réellement pertinents ? »
        \item[Le rappel] $R=\frac{VP}{VP+FN}$ « parmi les documents pertinents, combien ont été trouvés ? »
        \item[La F-mesure] $F=2×\frac{P×R}{P+R}$ est la moyenne harmonique\footnote{Exactement $F=\frac{2}{\frac{1}{P}+\frac{1}{R}}=2×\frac{P×R}{P+R}$} des deux précédentes
    \end{description}
\end{frame}

\begin{frame}[fragile]{Exemples}
    Déterminer $P$, $R$ et $F$ dans les cas suivants
    \begin{itemize}
        \item<1,2> $VP=100$, $VN=50$, $FP=20$, $FN=15$
        \item $VP=1$, $VN=0$, $FP=1$, $FN=1$
        \item \mbox{}\\
            {
                \tikzset{external/export=true}
                \begin{tikzpicture}[fill opacity=0.5, true scale=0.7, text opacity=1]
                    \draw[name path=D]
                        (-6,-2) rectangle (6,2)
                        node [above left] {$D$};
                    \draw[fill=highlight2, text=highlight2]
                        (-1.5, -0.25) node[text=black] {$15$} circle [x radius=2.5, y radius=1, rotate=20]
                        node[below left=1.1 and .15] {$P$ };
                    \draw[fill=highlight4, text=highlight4]
                        (1.5, 0) node[text=black] {$10$ }circle [x radius=2.5, y radius=1.7]
                        node[above right=1.5 and 0.5] {$R$};
                    \node at (0, 0) {$40$};
                    \node at (5.5, 1.5) {$2713$};
                \end{tikzpicture}
            }
    \end{itemize}
\end{frame}

\begin{frame}{Exemple}
    Pour la requête « achat en ligne de canards », un moteur de recherche renvoie \num{60} pages. Parmi celles-ci, on voit que \num{17} sont pertinentes.

    Quelle sont la précision, le rappel et la F-mesure du moteur de recherche pour cette requête ?
\end{frame}

\begin{frame}{Questions}
    \begin{enumerate}
        \item $P$, $R$ et $F$ dépendent-elles de la taille de $D$ ?
        \item Donner une méthode pour avoir à coup sûr $R=1$
        \item Donner une méthode pour avoir à coup sûr $P=1$
        \item Dans ces cas, que vaut $F$ ? Et que vaudrait la moyenne arithmétique de $P$ et $R$.
    \end{enumerate}
\end{frame}

\begin{frame}[fragile]{Aparté : moyennes}
    \begin{onlyenv}<1>
        \begin{figure}
            \tikzset{external/export=true}
            \begin{tikzpicture}
                \begin{axis}[meanheatmap]
                    \addplot3[surf, shader=interp]
                        {2*(x*y)/(x+y)};
                \end{axis}
            \end{tikzpicture}
        \caption{Moyenne harmonique de $P$ et $R$ : $2×\frac{P×R}{P+R}$}
        \end{figure}
    \end{onlyenv}
    \begin{onlyenv}<2>
        \begin{figure}
            \tikzset{external/export=true}
            \begin{tikzpicture}
                \begin{axis}[meanheatmap]
                    \addplot3[surf, shader=interp]
                        {(x+y)/2};
                \end{axis}
            \end{tikzpicture}
            \caption{Moyenne arithmétique de $P$ et $R$ : $\frac{P+R}{2}$}
        \end{figure}
    \end{onlyenv}
    \begin{onlyenv}<3>
        \begin{figure}
            \tikzset{external/export=true}
            \begin{tikzpicture}
                \begin{axis}[meanheatmap]
                    \addplot3[surf, shader=interp]
                        {(sqrt(x*y))};
                \end{axis}
            \end{tikzpicture}
        \caption{Moyenne géométrique de $P$ et $R$ : $√{P×R}$}
        \end{figure}
    \end{onlyenv}
    \begin{onlyenv}<4>
        \begin{figure}
            \tikzset{external/export=true}
            \begin{tikzpicture}
                \begin{axis}[meanheatmap]
                    \addplot3[surf, shader=interp]
                        {sqrt((x*x+y*y)/2)};
                \end{axis}
            \end{tikzpicture}
        \caption{Moyenne quadratique de $P$ et $R$ : $√{\frac{P²+R²}{2}}$}
        \end{figure}
    \end{onlyenv}
\end{frame}

\begin{frame}[fragile=singleslide]{F-scores généralisés}
    \begin{figure}
        \tikzset{external/export=true}
        \begin{tikzpicture}
            \begin{axis}[meanheatmap]
                \addplot3[surf, shader=interp]
                    {(x*y*(1+0.5^2))/(y+x*0.5^2)};
            \end{axis}
        \end{tikzpicture}
    \caption{$F_β$ ($β=0.5$)}
    \end{figure}
\end{frame}

\begin{frame}[fragile=singleslide]
    \begin{columns}
        \column{0.45\textwidth}
            \begin{figure}
                \tikzset{external/export=true}
                \begin{tikzpicture}
                    \begin{axis}[meanheatmap]
                        \addplot3[surf, shader=interp]
                            {(x*y*(1+0.5^2))/(y+x*0.5^2)};
                    \end{axis}
                \end{tikzpicture}
            \caption{$F_β$ ($β=0.5$)}
            \end{figure}
        \column{0.45\textwidth}
            L'idée est de pondérer la moyenne harmonique
            \begin{align}
                F_{λ, μ}(P, R)
                    &= \frac{λ+μ}{\frac{λ}{P}+\frac{μ}{R}}\\
                    &= (λ+μ) \frac{P×R}{λP + μR}\\
                    &= \frac{λ+μ}{μ} \frac{P×R}{\underbrace{\frac{λ}{μ}}_{β²}P + R}\\
                    &= (1+β²)\frac{P×R}{β²P + R}
            \end{align}
    \end{columns}
    La F-mesure est simplement le cas $β=1$.
\end{frame}

%  ██████ ██       █████  ███████ ███████ ██ ███████
% ██      ██      ██   ██ ██      ██      ██ ██
% ██      ██      ███████ ███████ ███████ ██ █████
% ██      ██      ██   ██      ██      ██ ██ ██
%  ██████ ███████ ██   ██ ███████ ███████ ██ ██


\section{Classification}

\begin{frame}[label=classif,fragile]{Classification}
    La \emph{classification} est la tâche consistant à \alert{répartir} des données dans un ensemble de classes.
    \begin{itemize}
        \item « Détecter les spams dans un ensemble d'emails »
        \item « Repérer dans un ensemble de tweets ceux qui sont sarcastiques ou ironique »
        \item « Classer un ensemble de textes par genre littéraire »
    \end{itemize}
    \begin{figure}
        \tikzset{external/export=true}
        \begin{tikzpicture}[true scale=0.80]
            \node[data, highlight=4] (in) {Donnée};
            \node[program, right=1cm of in, text width=15ex] (prog) {Programme de classification};
            \node[data, right=1cm of prog] (out) {Classe};
            \node[data, below=.6cm of prog, highlight=8] (res) {Classes};
            \draw[->] (in) -- (prog);
            \draw[->] (prog) -- (out);
            \draw[->] (res) -- (prog);
        \end{tikzpicture}
    \end{figure}
    \begin{itemize}
        \item Les \textcolor{highlight8}{classes} sont connus à l'avance
        \item Chaque \textcolor{highlight4}{donnée} appartient à une classe.
    \end{itemize}
\end{frame}

\begin{frame}{Domaines d'application}
    \begin{itemize}
        \item Classification par type de document
            \begin{itemize}
                \item[→] DEFT 2018 tâche 1 (\shorturl{deft.limsi.fr/2018})
            \end{itemize}
        \item Classification par opinion/sentiment
            \begin{itemize}
                \item[→] DEFT 2018 tâche 2
            \end{itemize}
        \item Classification par qualité
            \begin{itemize}
                \item[→] \shorturl{www.youtube.com/watch?v=anwy2MPT5RE}
            \end{itemize}
        \item[…]
    \end{itemize}
    Ne s'applique pas qu'aux textes, mais de façon générale à tous les types de données !
    \begin{itemize}
        \item[→] Le temps permet-il de faire un tennis ? (Weka \texttt{weater.numeric})
        \item[→] Cet empan de texte est-il un groupe nominal ?
        \item[…]
    \end{itemize}
\end{frame}

\begin{frame}[fragile]{Évaluation}
    \colorlet{VP}{highlight3}
    \colorlet{VN}{highlight2}
    \colorlet{FN}{highlight4}
    \colorlet{FP}{highlight8}
    \begin{figure}
        \begin{tikzpicture}[
            table/.style={
                matrix of nodes,
                row sep=-\pgflinewidth,
                column sep=-\pgflinewidth,
                nodes={draw, text width=3ex, align=center},
                text depth=0.25ex,
                text height=1em,
                nodes in empty cells
            },
            aa/.style={alt=<3>{fill=VP}{}, alt={<4,5>{fill=VN}{}}},
            bb/.style={alt=<4>{fill=VP}{}, alt={<3,5>{fill=VN}{}}},
            cc/.style={alt=<5>{fill=VP}{}, alt={<3,4>{fill=VN}{}}},
            ab/.style={alt=<3>{fill=FN}{}, alt=<4>{fill=FP}{}, alt=<5>{fill=VN}{}},
            ac/.style={alt=<3>{fill=FN}{}, alt=<4>{fill=VN}{}, alt=<5>{fill=FP}{}},
            ba/.style={alt=<3>{fill=FP}{}, alt=<4>{fill=FN}{}, alt=<5>{fill=VN}{}},
            bc/.style={alt=<3>{fill=VN}{}, alt=<4>{fill=FN}{}, alt=<5>{fill=FP}{}},
            ca/.style={alt=<3>{fill=FP}{}, alt=<4>{fill=VN}{}, alt=<5>{fill=FN}{}},
            cb/.style={alt=<3>{fill=VN}{}, alt=<4>{fill=FP}{}, alt=<5>{fill=FN}{}},
        ]
            \matrix[table]{
                    & |(apred)| A   & |(bpred)| B   & |(cpred)| C\\
                |(atrue)| A   & |[aa]| $16$  & |[ab]| $0$   & |[ac]| $0$\\
                |(btrue)| B   & |[ba]| $0$   & |[bb]| $19$  & |[bc]| $1$\\
                |(ctrue)| C   & |[ca]| $0$   & |[cb]| $2$   & |[cc]| $15$\\
            };

            \draw [decorate, decoration={brace, amplitude=1em}, yshift=-0.5ex] (apred.north west) -- (cpred.north east) node [midway, above=1.25ex] {Classes prédites};
            \draw [decorate, decoration={brace, amplitude=1em, mirror}, xshift=-0.5ex] (atrue.north west) -- (ctrue.south west) node [midway, left=1.25ex, rotate=90, anchor=south] {Vraies classes};
        \end{tikzpicture}
        \caption{Matrice de confusion pour un problème à trois classes}
    \end{figure}
    \vspace{-1em}
    \only<2>{On calcule pour chaque classe les mêmes indicateurs que pour la RI}
    \only<3>{Pour A}\only<4>{Pour B}\only<5>{Pour C}
    \only<2-5>{
        \begin{itemize}
            % TODO: dans la liste suivante ajouter les totaux pour chaque classe à chaque slide
            \item \textcolor{VP}{Vrais Positifs}
                \only<3>{$16$}\only<4>{$19$}\only<5>{$15$}
            \item \textcolor{VN}{Vrais Négatifs}
                \only<3>{$19+2+1+15=37$}
                \only<4>{$16+0+0+15=31$}
                \only<5>{$16+0+0+19=35$}
            \item \textcolor{FN}{Faux Négatifs}
                \only<3>{$0+0=0$}\only<4>{$0+1=1$}\only<5>{$0+2=2$}
            \item \textcolor{FP}{Faux Positifs}
                \only<3>{$0+0=0$}\only<4>{$0+2=2$}\only<5>{$0+1=1$}
        \end{itemize}
    }
    \only<6>{Puis pour chaque classe, $P$, $R$ et $F$ et la moyenne sur l'ensemble
        \begin{description}
            \item[macro-average] Sans pondération
            \item[micro-average] Pondérée par la taille des classes
        \end{description}
    }
\end{frame}

\begin{frame}{Exercice}
    Étant donnée la matrice de confusion précédente, déterminer précision, rappel et F-mesure pour chacune des classes, ainsi que la micro- et macro-average.
\end{frame}



%  █████  ██████  ██████  ███████ ███    ██ ██████  ██ ██   ██
% ██   ██ ██   ██ ██   ██ ██      ████   ██ ██   ██ ██  ██ ██
% ███████ ██████  ██████  █████   ██ ██  ██ ██   ██ ██   ███
% ██   ██ ██      ██      ██      ██  ██ ██ ██   ██ ██  ██ ██
% ██   ██ ██      ██      ███████ ██   ████ ██████  ██ ██   ██

\appendix
\pgfkeys{/metropolis/inner/sectionpage=simple}  % Avoid random errors with section page progressbar
\section{Annexes}
\pdfbookmark[2]{Remerciements}{acknowledgements}
\begin{frame}{Remerciements}
    Ce cours a été construit à partir du polycopié de cours \citetitle{tellier2017fouille} \parencite{tellier2017fouille} et des précieux conseils d'Isabelle Tellier que je ne saurais trop remercier pour sa confiance et son dévouement.
\end{frame}

\pdfbookmark[2]{Références}{references}
\begin{frame}[allowframebreaks]{References}
    \printbibliography[heading=none]
\end{frame}

\pdfbookmark[2]{Licence}{licence}
\begin{frame}{Licence}
    \begin{center}
        {\huge \ccby}
        \vfill
        This document is available under the terms of the Creative Commons Attribution 4.0 International License (CC BY 4.0) (\shorturl{creativecommons.org/licenses/by/4.0})

        Exceptions to the above statement are listed at {\small\shorturl{loicgrobol.github.io/intro-fouille-textes\#licences}}
        \vfill
        © 2019, Loïc Grobol <\shorturl[mailto][:]{loic.grobol@gmail.com}>

        \shorturl[http]{lattice.cnrs.fr/Grobol-Loic}
    \end{center}
\end{frame}

\end{document}
