% Copyright © 2018, Loïc Grobol <loic.grobol@gmail.com>
% This document is available under the terms of the Creative Commons Attribution 4.0 International License (CC BY 4.0) (https://creativecommons.org/licenses/by/4.0/)

\RequirePackage{xparse}
\RequirePackage{shellesc}
% Settings
\NewDocumentCommand\myname{}{Loïc Grobol}
\NewDocumentCommand\mylab{}{Lattice / ALMAnaCH}
\NewDocumentCommand\pdftitle{}{Introduction à la fouille de textes}
\NewDocumentCommand\mymail{}{loic.grobol@gmail.com}
\NewDocumentCommand\titlepagetitle{}{\pdftitle}
\NewDocumentCommand\titlepagesubtitle{}{Cours 9 : Naïve Bayes}
\NewDocumentCommand\docdate{}{2019-03-25}
\NewDocumentCommand\conference{}{M1 Plurital}

\documentclass[hyperref={unicode}, xcolor={svgnames}, french]{beamer}
% Alternative navy blue for ⩽4 palettes
\definecolor{highlighta}{RGB}{68, 118, 170}  % Navy blue

% Colour palette from [Paul Tol's technical note](https://personal.sron.nl/~pault/data/colourschemes.pdf) v3.1
% Bright scheme
\definecolor{sronbrighttblue}{RGB}{68, 119, 170}
\definecolor{sronbrightcyan}{RGB}{102, 204, 238}
\definecolor{sronbrightgreen}{RGB}{34, 136, 51}
\definecolor{sronbrightyellow}{RGB}{204, 187, 68}
\definecolor{sronbrightred}{RGB}{238, 102, 119}
\definecolor{sronbrightpurple}{RGB}{170, 51, 119}
\definecolor{sronbrightgrey}{RGB}{187, 187, 187}

\colorlet{highlight0}{sronbrighttblue}
\colorlet{highlight1}{sronbrightcyan}
\colorlet{highlight2}{sronbrightgreen}
\colorlet{highlight3}{sronbrightyellow}
\colorlet{highlight4}{sronbrightred}
\colorlet{highlight5}{sronbrightpurple}
\colorlet{highlight6}{sronbrightgrey}
% Legacy highlights
\definecolor{highlight7}{RGB}{136, 34, 85}    % Purple
\definecolor{highlight8}{RGB}{170, 68, 153}   % Violet


\usetheme[
	sectionpage=progressbar,
    subsectionpage=progressbar,
	progressbar=frametitle,
]{metropolis}
	\colorlet{accent}{sronbrightgreen}
	\setbeamercolor{frametitle}{bg=DarkSlateGrey}
	\setbeamercolor{alerted text}{fg=accent}
	\makeatletter
		\setlength{\metropolis@progressinheadfoot@linewidth}{2pt}
	\makeatother
	% Avoid ugly whitespace below figures
	% \makeatletter
	% 	\renewenvironment{figure}[1][]{%
	% 	\def\@captype{figure}%
	% 	\par\centering}%
	% 	{\par}
	% \makeatother
	% FIXME: not pretty and footnotes are still too big
	\let\footnoterule\relax  % No footnote rule, push down footnote


% Use non-standard fonts
\usefonttheme{professionalfonts}
\usefonttheme{professionalfonts}
	\setsansfont[
		BoldFont={Fira Sans SemiBold},
		ItalicFont={Fira Sans Italic},
		BoldItalicFont={Fira Sans Bold Italic},
	]{Fira Sans Book}
\setmonofont[Scale=0.9]{Fira Mono}

% Fix missing glyphs in Fira by delegating to polyglossia/babel
\usepackage{newunicodechar}
\newunicodechar{ }{~}   % U+202F NARROW NO-BREAK SPACE
\newunicodechar{ }{ }  % U+2009 THIN SPACE

% Notes on left screen
% \usepackage{pgfpages}
% \setbeameroption{show notes on second screen=left}


\usepackage[french, english]{babel}
\usepackage{amsfonts,amssymb}
\usepackage{amsmath,amsthm}
\usepackage{mathtools}	% AMS Maths service pack
	\newtagform{brackets}{[}{]}	% Pour des lignes d'équation numérotées entre crochets
	\mathtoolsset{showonlyrefs, showmanualtags, mathic}	% affiche les tags manuels (\tag et \tag*) et corrige le kerning des maths inline dans un bloc italique voir la doc de mathtools
	\usetagform{brackets}	% Utilise le style de tags défini plus haut
\usepackage{lualatex-math}

\usepackage[math-style=french]{unicode-math}
	\setmathfont{Libertinus Math}
\usepackage{newunicodechar}
	\newunicodechar{√}{\sqrt}
\usepackage{mleftright}

\usepackage{tabu}
\usepackage{booktabs}
\usepackage{siunitx}
\usepackage{multicol}
\usepackage{ccicons}
\usepackage{bookmark}
\usepackage{caption}
    \captionsetup{skip=1ex}
\usepackage[iso]{datetime}

\usepackage{tikz}
	\NewDocumentCommand{\textnode}{O{}mm}{\tikz[remember picture, baseline=(#2.base), inner sep=0pt]{\node[#1] (#2) {#3};}}
    \NewDocumentCommand{\mathnode}{O{}mm}{\tikz[remember picture, baseline=(#2.base), inner sep = 0pt]{\node[#1] (#2) {$\displaystyle #3$};}}
	\tikzset{
		alt/.code args={<#1>#2#3}{%
		\alt<#1>{\pgfkeysalso{#2}}{\pgfkeysalso{#3}} % \pgfkeysalso doesn't change the path
		},
        invisible/.style={opacity=0, fill opacity=0},
		visible on/.style={alt={<#1>{}{invisible}}}
	}
    \usepackage{forest}
    \usepackage{tkz-graph}
    \usepackage[beamer, markings]{hf-tikz}
    \usepackage{tikz-3dplot}
    \usepackage{pgfplots}
        % Due to pgfplots meddling with pgfkeys, we have to redefine alt here.
        \pgfplotsset{
    		alt/.code args={<#1>#2#3}{%
    		\alt<#1>{\pgfkeysalso{#2}}{\pgfkeysalso{#3}} % \pgfkeysalso doesn't change the path
    		},
    	}
        \pgfplotsset{compat=1.15}
        \pgfplotsset{colormap={SRON}{rgb255=(61,82,161) rgb255=(255,250,210) rgb255=(174,28,62)}}

    \usetikzlibrary{matrix}
    \usetikzlibrary{shapes, shapes.geometric}
    \usetikzlibrary{decorations.pathreplacing}
	\usetikzlibrary{positioning, calc, intersections}
    \usetikzlibrary{fit}
    \usetikzlibrary{backgrounds}

    % Do evil things with soft path
    % From <https://tex.stackexchange.com/a/301364/8547>
    \makeatletter
        \def\@appendnamedsoftpath#1{%
            \pgfsyssoftpath@getcurrentpath\@temppatha
            \expandafter\let\expandafter\@temppathb\csname tikz@intersect@path@name@#1\endcsname
            \expandafter\expandafter\expandafter\def\expandafter\expandafter\expandafter\@temppatha\expandafter\expandafter\expandafter{\expandafter\@temppatha\@temppathb}%
            \pgfsyssoftpath@setcurrentpath\@temppatha
        }
        \def\@appendnamedpathforactions#1{%
            \pgfsyssoftpath@getcurrentpath\@temppatha
            \expandafter\let\expandafter\@temppathb\csname tikz@intersect@path@name@#1\endcsname
            \expandafter\def\expandafter\@temppatha\expandafter{\csname @temppatha\expandafter\endcsname\@temppathb}%\usepackage{tikz-3dplot}

            \let\tikz@actions@path\@temppatha
        }

        \tikzset{
            use path for main/.code={%
                \tikz@addmode{%
                    \expandafter\pgfsyssoftpath@setcurrentpath\csname tikz@intersect@path@name@#1\endcsname
                }%
            },
            append path for main/.code={%
                \tikz@addmode{%
                    \@appendnamedsoftpath{#1}%
                }%
            },
            use path for actions/.code={%
                \expandafter\def\expandafter\tikz@preactions\expandafter{\tikz@preactions\expandafter\let\expandafter\tikz@actions@path\csname tikz@intersect@path@name@#1\endcsname}%
            },
            append path for actions/.code={%
                \expandafter\def\expandafter\tikz@preactions\expandafter{\tikz@preactions
                \@appendnamedpathforactions{#1}}%
            },
            use path/.style={%
                use path for main=#1,
                use path for actions=#1,
            },
            append path/.style={%
                append path for main=#1,
                append path for actions=#1
            }
        }
    \makeatother

    % TikZ externalisation
    \usetikzlibrary{external}
    % Create the `tikzpics/` folder if it does not exist
    \ShellEscape{mkdir -p tikzpics}
    % Only externalise pictures on demand (to avoid messing up with metropolis theme)
    \tikzset{
        external/export=false,
        external/prefix=tikzpics/
    }
    \tikzexternalize

\usepackage{minted}
	\usemintedstyle{lovelace}
	\setminted{autogobble, fontsize=\scriptsize, tabsize=2}
	\setmintedinline{fontsize=auto}

\usepackage{csquotes}

\usepackage[style=authoryear, block=ragged, doi=false, isbn=false]{biblatex}
    \AtEveryBibitem{
        \ifentrytype{online}
        {} {
            \iffieldequalstr{howpublished}{online}
            {
                \clearfield{howpublished}
            } {
                \clearfield{urlyear}\clearfield{urlmonth}\clearfield{urlday}
            }
        }
    }

	\addbibresource{biblio.bib}

% Compact bibliography style
\setbeamertemplate{bibliography item}[text]

\AtEveryBibitem{
    \clearfield{series}
    \clearfield{pages}
    \clearlist{publisher}
    \clearname{editor}
    \clearlist{location}
}
\renewcommand*{\bibfont}{\tiny}

\usepackage{hyperxmp}	% XMP metadata

\usepackage[type={CC}, modifier={by}, version={4.0}]{doclicense}

\usepackage{todonotes}
\let\todox\todo
\renewcommand\todo[1]{\todox[inline]{#1}}

\title{\titlepagetitle}
\subtitle{\titlepagesubtitle}
\author{\textbf{\myname} (\mylab)}
\institute{}
\date{\tiny Version {\yyyymmdddate\today}T\currenttime}

\titlegraphic{\ccby}

% Tikz styles

% Schémas de tâches
\tikzset{
    >=stealth,
    hair lines/.style={line width = 0.05pt, lightgray},
    data/.style={draw, ellipse},
    program/.style={draw, rectangle},
    accent on/.style={alt={<#1>{draw=accent, text=accent, thick}{draw}}},
    true scale/.style={scale=#1, every node/.style={transform shape}},
    highlight/.style={color=highlight#1}
}

% Styles des heatmap pour les moyennes
\pgfplotsset{
    meanheatmap/.style={
        colorbar, colormap name=SRON,
        view={0}{90},
        samples=100,
        domain=0:1,
        min=0, max=1,
        xlabel={$P$},
        ylabel={$R$},
    }
}

% Commands spécifiques
\NewDocumentCommand\shorturl{ O{https} O{://} m }{%
    \href{#1#2#3}{\nolinkurl{#3}}%
}

\DeclarePairedDelimiter\norm{\lVert}{\rVert}
\DeclarePairedDelimiter\abs{\lvert}{\rvert}
\DeclarePairedDelimiterX\compset[2]{\lbrace}{\rbrace}{#1\,\delimsize|\,#2}
\DeclarePairedDelimiterX\innprod[2]{\langle}{\rangle}{#1\,\delimsize|\,#2}

\DeclareMathOperator*\argmax{argmax}

% Easy column vectors \vcord{a,b,c} ou \vcord[;]{a;b;c}
% Here be black magic
\ExplSyntaxOn
	\NewDocumentCommand{\vcord}{O{,}m}{\vector_main:nnnn{p}{\\}{#1}{#2}}
	\NewDocumentCommand{\tvcord}{O{,}m}{\vector_main:nnnn{psmall}{\\}{#1}{#2}}
	\seq_new:N\l__vector_arg_seq
	\cs_new_protected:Npn\vector_main:nnnn #1 #2 #3 #4{
		\seq_set_split:Nnn\l__vector_arg_seq{#3}{#4}
		\begin{#1matrix}
			\seq_use:Nnnn\l__vector_arg_seq{#2}{#2}{#2}
		\end{#1matrix}
	}
\ExplSyntaxOff

\DeclareMathOperator{\TF}{TF}
\DeclareMathOperator{\IDF}{IDF}

\ExplSyntaxOn
    \DeclareExpandableDocumentCommand\eval{m}{\fp_eval:n{#1}}
\ExplSyntaxOff


% Fixes for pauses after an unveiled list
\newcommand{\itpause}{%
	\addtocounter{beamerpauses}{-1}%
	\pause
}


% ██████   ██████   ██████ ██    ██ ███    ███ ███████ ███    ██ ████████
% ██   ██ ██    ██ ██      ██    ██ ████  ████ ██      ████   ██    ██
% ██   ██ ██    ██ ██      ██    ██ ██ ████ ██ █████   ██ ██  ██    ██
% ██   ██ ██    ██ ██      ██    ██ ██  ██  ██ ██      ██  ██ ██    ██
% ██████   ██████   ██████  ██████  ██      ██ ███████ ██   ████    ██


\begin{document}
\pdfbookmark[2]{Title}{title}

\begin{frame}[plain]
	\titlepage
\end{frame}

\begin{frame}[fragile=singleslide]{Exercice}
	Étant donnée la représentation tabulaire suivante d'un corpus, attribuez les classes manquantes en construisant puis appliquant un arbre de décisions
	\begin{table}
		\begin{tabu} to .9\linewidth {*{4}{c}}
			\texttt{t}  & \texttt{wug} & \texttt{neurone} & \texttt{classe}\\
			\hline
			\texttt{1}  & \texttt{3}      & \texttt{2}	& linguistique\\
			\texttt{2}  & \texttt{1}      & \texttt{0}	& linguistique\\
			\texttt{3}  & \texttt{0}      & \texttt{2}	& linguistique\\
			\texttt{4}  & \texttt{2}      & \texttt{1}	& linguistique\\
			\texttt{5}  & \texttt{4}      & \texttt{4}	& informatique\\
			\texttt{6}  & \texttt{1}      & \texttt{3}	& informatique\\
			\texttt{7}  & \texttt{1}      & \texttt{2}	& informatique\\
			\texttt{8}  & \texttt{2}      & \texttt{2}	& ?\\
			\texttt{9}  & \texttt{1}      & \texttt{4}	& ?\\
			\texttt{10}  & \texttt{3}      & \texttt{1}	& ?\\
		\end{tabu}
	\end{table}
\end{frame}

%
% ██████   █████  ██    ██ ███████ ███████
% ██   ██ ██   ██  ██  ██  ██      ██
% ██████  ███████   ████   █████   ███████
% ██   ██ ██   ██    ██    ██           ██
% ██████  ██   ██    ██    ███████ ███████

\section{\emph{Naïve Bayes}}

% TODO: moyen fluide, peut-être un peu plus d'exmples et des exos à chaque étape
\begin{frame}[fragile]{Les maths en cinq secondes : probabilités conditionnelles}
    On lance un dé à six faces équilibré. On note
    \begin{description}
        \item[A] « le résultat est inférieur ou égal à $4$ »
        \item[B] « le résultat est supérieur ou égal à $3$ »
    \end{description}
    \begin{columns}
    \alt<1>{\column{\textwidth}}{\column{0.45\textwidth}}
        \begin{figure}
            \tikzset{external/export=true}
            \begin{tikzpicture}[
                    hfit/.style={ellipse, inner sep=2pt},
                    vfit/.style={ellipse, inner sep=2pt},
                ]
                \matrix[matrix of nodes, column sep=1em, row sep=1em] (m) {
                    $1$   & $2$\\
                    $3$   & $4$\\
                    $5$   & $6$\\
                };

                \node[fit=(m), inner sep=1em, draw, label=above right:$Ω$] (O) {};
                \node[fit=(m-1-2)(m-1-2)(m-2-1)(m-2-2), hfit, draw, fill=highlighta, fill opacity=0.5, label=above:$A$] (A) {};
                \node[fit=(m-2-1)(m-3-2), vfit, draw, fill=highlight6, fill opacity=0.5, label=below:$B$] (B) {};
            \end{tikzpicture}
            \caption{Univers}
        \end{figure}
    \only<2-3>{\column{0.45\textwidth}
        On a
        \begin{align}
            P(A) &= \alt<1-2>{\phantom{\frac{4}{6}=\frac{2}{3}}}{\frac{4}{6}=\frac{2}{3}}\\
            P(B) &= \alt<1-2>{\phantom{\frac{4}{6}=\frac{2}{3}}}{\frac{4}{6}=\frac{2}{3}}
        \end{align}
    }
    \end{columns}
\end{frame}

\begin{frame}[fragile]{Les maths en cinq secondes : probabilités conditionnelles}
    On note
    \vspace{-1\smallskipamount}
    \begin{itemize}
        \item $P(A|B)$ \alt<1>{la probabilité que $A$ se réalise sachant que $B$ se réalise}{la probabilité que le résultat soit inférieur ou égal à $4$ sachant qu'il est supérieur ou égal à $3$}
        \item $P(B|A)$ \alt<1>{la probabilité que $B$ se réalise sachant que $A$ se réalise}{la probabilité que le résultat soit supérieur ou égal à $3$ sachant qu'il est inférieur ou égal à $4$}
    \end{itemize}
    \vspace{-1\bigskipamount}
    \begin{columns}
    \alt<1-2>{\column{\textwidth}}{\column{0.45\textwidth}}
        \begin{figure}
            \tikzset{external/export=true}
            \begin{tikzpicture}[
                    hfit/.style={ellipse, inner sep=2pt},
                    vfit/.style={ellipse, inner sep=2pt},
                    true scale=0.8,
                ]
                \matrix[matrix of nodes, column sep=1em, row sep=1em] (m) {
                    $1$   & $2$\\
                    $3$   & $4$\\
                    $5$   & $6$\\
                };

                \node[fit=(m), inner sep=1em, draw, label=above right:$Ω$] (O) {};
                \node[fit=(m-1-2)(m-1-2)(m-2-1)(m-2-2), hfit, draw, fill=highlighta, fill opacity=0.5, label=above:$A$] (A) {};
                \node[fit=(m-2-1)(m-3-2), vfit, draw, fill=highlight6, fill opacity=0.5, label=below:$B$] (B) {};
            \end{tikzpicture}
            \caption{Univers}
        \end{figure}
    \only<3->{\column{0.45\textwidth}
        On a
        \begin{align}
            P(A|B) &= \alt<1-3>{\phantom{\frac{2}{3}}}{\frac{2}{4}=\frac{1}{2}}\\
            P(B|A) &= \alt<1-3>{\phantom{\frac{1}{2}}}{\frac{2}{4}=\frac{1}{2}}
        \end{align}
    }
    \end{columns}
    \only<5>{Dans le cas général, on a $P(X|Y) = \frac{P(X∩Y)}{P(Y)}$}
\end{frame}

\begin{frame}<1>[label=naivebayes]{\emph{Naïve Bayes}}
    Soit un document $d$ à catégoriser.
    On lui attribue la classe $C$ telle que $P(C|d)$ est maximale.

	\only<-2>{
		Problème: Comment estimer cette probabilité ?
	}

	\uncover<2->{
	    Pour une classe donnée $C$, on a par le théorème de Bayes
	    \begin{equation}
	        P(C|d) = \frac{P(d|C)×P(C)}{P(d)}
	    \end{equation}
	}

	\uncover<3->{
	    $P(d)$ ne dépend pas de $C$, il s'agit donc de maximiser $s(C,d)=P(d|C)×P(C)$.
	}

	\uncover<4->{
		Problème: Comment estimer ces probabilités ?
	}
\end{frame}

\begin{frame}{Les maths en cinq secondes : théorème de Bayes}
    On a
    \begin{align}
        P(X|Y)
            &= \frac{P(X∩Y)}{P(Y)}\\
            &= \frac{\frac{P(X∩Y)}{P(X)}×P(X)}{P(Y)}\\
            &= \frac{P(Y|X)×P(X)}{P(Y)}
    \end{align}
    \begin{block}{Théorème de Bayes}
        Pour tous évènements $X$ et $Y$ de probabilités non-nulles,
        \begin{equation}
            P(X|Y)=\frac{P(Y|X)×P(X)}{P(Y)}
        \end{equation}
    \end{block}
\end{frame}

\againframe<1-3>{naivebayes}

\begin{frame}{Probabilité d'une classe}
	Problème : estimer $P(C)$.

	\pause
	Facile : Si on a un échantillon (corpus) suffisamment grand, la probabilité d'une classe peut être estimée par sa fréquence empirique. Autrement dit

	\pause

	\begin{equation}
		P(C) ≈ \frac{\text{Nombre de documents de la classe $C$}}{\text{Nombre de documents du corpus}}
	\end{equation}

	\pause
	Il reste à estimer $P(d|C)$.
\end{frame}

\begin{frame}{Les maths en cinq secondes : indépendance}
	On lance deux dés à \num{6} faces l'un après l'autre

	\begin{itemize}
		\item<+-> Quelle est la probabilité que le premier dé donne un résultat pair ?
		\item<+-> Si je sais que le premier dé a donné un résultat pair, quelle est la probabilité que le deuxième dé donne \num{6} ?
	\end{itemize}
	\itpause

	\begin{block}{Indépendance (définition)}
		On dit que deux évènements $A$ et $B$ sont indépendants si le fait que $B$ se réalise n'a aucune influence sur la probabilité de $A$.

		Formellement: $P(A|B)=P(A)$.
	\end{block}
\end{frame}


\begin{frame}{Les maths en cinq secondes : indépendance}
	Les assertions suivantes sont équivalentes
	\begin{enumerate}[<+->]
		\item $A$ et $B$ sont indépendants
		\item $P(A|B)=P(A)$
		\item $P(B|A)=P(B)$
		\item $P(A∩B) = P(A)×P(B)$
	\end{enumerate}
\end{frame}


\begin{frame}{Probabilité d'un document}
    Si on représente les documents par des sacs de mots, $d$ est un vecteur $(x₁, …, x_n)$.
    Donc
    \begin{equation}
        P(d|C) = P((x₁, …, x_n)|C)
    \end{equation}

    Et en supposant que les $x_i$ sont indépendants (« Naïve »)
    \begin{equation}
        P(d|C) = P(x₁|C)×…×P(x_n|C)
    \end{equation}

    Il reste à choisir comment déterminer les $P(x_i|C)$.
\end{frame}

\begin{frame}{Probabilité d'un attribut}
	Problème : estimer $P(x_i|C)$.

	\pause
	Facile : Si on a un échantillon (classe) suffisamment grand, la probabilité d'un attribut peut être estimée par sa fréquence empirique. Autrement dit

	\pause

	\begin{equation}
		P(x_i|C) ≈ \frac{\text{Nombre de documents $C$ qui possèdent le trait $x_i$}}{\text{Nombre de documents de $C$}}
	\end{equation}

	\pause
	Et on a terminé !
\end{frame}

\begin{frame}{\emph{Naïve Bayes}}
    \textbf{Dans Weka} classifiers > bayes > NaiveBayes

    \textbf{Espace de recherche} Ensemble des fonctions de la forme
    \begin{equation}
        x ⟼ \argmax_k s(C_k | x)
    \end{equation}
    avec
    \begin{equation}
        s(C_k, x) = P(C_k) × P(x₁ | C_k) × … × P(x_n | C_k)
    \end{equation}

    \textbf{Technique de recherche} En générale : formule explicite étant donné l'ensemble d'apprentissage
\end{frame}

% TODO: exemples
% TODO: usages courants, avantages et inconvénients de NB


%  █████  ██████  ██████  ███████ ███    ██ ██████  ██ ██   ██
% ██   ██ ██   ██ ██   ██ ██      ████   ██ ██   ██ ██  ██ ██
% ███████ ██████  ██████  █████   ██ ██  ██ ██   ██ ██   ███
% ██   ██ ██      ██      ██      ██  ██ ██ ██   ██ ██  ██ ██
% ██   ██ ██      ██      ███████ ██   ████ ██████  ██ ██   ██

\appendix
\pgfkeys{/metropolis/inner/sectionpage=simple}  % Avoid random errors with section page progressbar
\section{Annexes}
\pdfbookmark[2]{Remerciements}{acknowledgements}
\begin{frame}{Remerciements}
    Ce cours a été construit à partir du polycopié de cours \citetitle{tellier2017IntroductionFouilleTextes} \parencite{tellier2017IntroductionFouilleTextes} et des précieux conseils d'Isabelle Tellier que je ne saurais trop remercier pour sa confiance et son dévouement.
\end{frame}

\pdfbookmark[2]{Références}{references}
\begin{frame}[allowframebreaks]{References}
    \printbibliography[heading=none]
\end{frame}

\pdfbookmark[2]{Licence}{licence}
\begin{frame}{Licence}
    \begin{center}
        {\huge \ccby}
        \vfill
        This document is available under the terms of the Creative Commons Attribution 4.0 International License (CC BY 4.0) (\shorturl{creativecommons.org/licenses/by/4.0})

        Exceptions to the above statement are listed at {\small\shorturl{loicgrobol.github.io/intro-fouille-textes\#licences}}
        \vfill
        © 2019, Loïc Grobol <\shorturl[mailto][:]{loic.grobol@gmail.com}>

        \shorturl[http]{lattice.cnrs.fr/Grobol-Loic}
    \end{center}
\end{frame}

\end{document}
