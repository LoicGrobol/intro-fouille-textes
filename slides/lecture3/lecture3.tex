% Copyright © 2018, Loïc Grobol <loic.grobol@gmail.com>
% This document is available under the terms of the Creative Commons Attribution 4.0 International License (CC BY 4.0) (https://creativecommons.org/licenses/by/4.0/)

\RequirePackage{xparse}
\RequirePackage{shellesc}
% Settings
\NewDocumentCommand\myname{}{Loïc Grobol}
\NewDocumentCommand\mylab{}{Lattice / ALMAnaCH}
\NewDocumentCommand\pdftitle{}{Introduction à la fouille de textes}
\NewDocumentCommand\mymail{}{loic.grobol@gmail.com}
\NewDocumentCommand\titlepagetitle{}{\pdftitle}
\NewDocumentCommand\docdate{}{2019-02-04}
\NewDocumentCommand\conference{}{M1 Plurital}

\documentclass[hyperref={unicode}, xcolor={svgnames}, french]{beamer}
\usetheme[sectionpage=progressbar,
          subsectionpage=progressbar,
          progressbar=frametitle]{metropolis}
    \definecolor{accent}{RGB}{51, 34, 136}
    \setbeamercolor{alerted text}{fg=accent}
    \makeatletter
        \setlength{\metropolis@progressinheadfoot@linewidth}{1pt}
    \makeatother
    % FIXME: not pretty and footnotes are still too big
    \let\footnoterule\relax  % No footnote rule, push down footnote

% 9-Highlights colour palette from [Paul Tol's technical note](https://personal.sron.nl/~pault/data/colourschemes.pdf)
\definecolor{highlight0}{RGB}{51, 34, 136}    % Deep blue
\definecolor{highlight1}{RGB}{136, 204, 238}  % Clear blue
\definecolor{highlight2}{RGB}{68, 170, 153}   % Teal
\definecolor{highlight3}{RGB}{17, 119, 51}    % Forest green
\definecolor{highlight4}{RGB}{153, 153, 51}   % Khaki
\definecolor{highlight5}{RGB}{221, 204, 119}  % Beige
\definecolor{highlight6}{RGB}{204, 102, 119}  % Pink
\definecolor{highlight7}{RGB}{136, 34, 85}    % Purple
\definecolor{highlight8}{RGB}{170, 68, 153}   % Violet

% Alternative navy blue for ⩽4 palettes
\definecolor{highlighta}{RGB}{68, 118, 170}  % Navy blue

% Use non-standard fonts
\usefonttheme{professionalfonts}
\setsansfont[BoldFont={Fira Sans SemiBold}, ItalicFont={Fira Sans Book Italic}]{Fira Sans Book}
\setmonofont[Scale=0.9]{Fira Mono}

% Fix missing glyphs in Fira by delegating to polyglossia/babel
\usepackage{newunicodechar}
\newunicodechar{ }{~}   % U+202F NARROW NO-BREAK SPACE
\newunicodechar{ }{ }  % U+2009 THIN SPACE

% Notes on left screen
% \usepackage{pgfpages}
% \setbeameroption{show notes on second screen=left}


\usepackage[french, english]{babel}
\usepackage{amsfonts,amssymb}
\usepackage{amsmath,amsthm}
\usepackage{mathtools}	% AMS Maths service pack
	\newtagform{brackets}{[}{]}	% Pour des lignes d'équation numérotées entre crochets
	\mathtoolsset{showonlyrefs, showmanualtags, mathic}	% affiche les tags manuels (\tag et \tag*) et corrige le kerning des maths inline dans un bloc italique voir la doc de mathtools
	\usetagform{brackets}	% Utilise le style de tags défini plus haut
\usepackage{lualatex-math}

\usepackage[math-style=french]{unicode-math}
	\setmathfont{Libertinus Math}
\usepackage{newunicodechar}
	\newunicodechar{√}{\sqrt}
\usepackage{mleftright}


\usepackage{tabu}
\usepackage{booktabs}
\usepackage{siunitx}
\usepackage{multicol}
\usepackage{ccicons}
\usepackage{bookmark}
\usepackage{caption}
    \captionsetup{skip=1ex}
\usepackage[iso]{datetime}

\usepackage{csquotes}
\usepackage{tikz}
	\NewDocumentCommand{\textnode}{O{}mm}{\tikz[remember picture, baseline=(#2.base), inner sep=0pt]{\node[#1] (#2) {#3};}}
    \NewDocumentCommand{\mathnode}{O{}mm}{\tikz[remember picture, baseline=(#2.base), inner sep = 0pt]{\node[#1] (#2) {$\displaystyle #3$};}}
	\tikzset{
		alt/.code args={<#1>#2#3}{%
		\alt<#1>{\pgfkeysalso{#2}}{\pgfkeysalso{#3}} % \pgfkeysalso doesn't change the path
		},
        invisible/.style={opacity=0, fill opacity=0},
		visible on/.style={alt={<#1>{}{invisible}}}
	}
    \usepackage{forest}
    \usepackage{tkz-graph}
    \usepackage[beamer, markings]{hf-tikz}
    \usepackage{tikz-3dplot}
    \usepackage{pgfplots}
        % Due to pgfplots meddling with pgfkeys, we have to redefine alt here.
        \pgfplotsset{
    		alt/.code args={<#1>#2#3}{%
    		\alt<#1>{\pgfkeysalso{#2}}{\pgfkeysalso{#3}} % \pgfkeysalso doesn't change the path
    		},
    	}
        \pgfplotsset{compat=1.15}
        \pgfplotsset{colormap={SRON}{rgb255=(61,82,161) rgb255=(255,250,210) rgb255=(174,28,62)}}

    \usetikzlibrary{matrix}
    \usetikzlibrary{shapes, shapes.geometric}
    \usetikzlibrary{decorations.pathreplacing}
	\usetikzlibrary{positioning, calc, intersections}
    \usetikzlibrary{fit}
    \usetikzlibrary{backgrounds}

    % Do evil things with soft path
    % From <https://tex.stackexchange.com/a/301364/8547>
    \makeatletter
        \def\@appendnamedsoftpath#1{%
            \pgfsyssoftpath@getcurrentpath\@temppatha
            \expandafter\let\expandafter\@temppathb\csname tikz@intersect@path@name@#1\endcsname
            \expandafter\expandafter\expandafter\def\expandafter\expandafter\expandafter\@temppatha\expandafter\expandafter\expandafter{\expandafter\@temppatha\@temppathb}%
            \pgfsyssoftpath@setcurrentpath\@temppatha
        }
        \def\@appendnamedpathforactions#1{%
            \pgfsyssoftpath@getcurrentpath\@temppatha
            \expandafter\let\expandafter\@temppathb\csname tikz@intersect@path@name@#1\endcsname
            \expandafter\def\expandafter\@temppatha\expandafter{\csname @temppatha\expandafter\endcsname\@temppathb}%\usepackage{tikz-3dplot}

            \let\tikz@actions@path\@temppatha
        }

        \tikzset{
            use path for main/.code={%
                \tikz@addmode{%
                    \expandafter\pgfsyssoftpath@setcurrentpath\csname tikz@intersect@path@name@#1\endcsname
                }%
            },
            append path for main/.code={%
                \tikz@addmode{%
                    \@appendnamedsoftpath{#1}%
                }%
            },
            use path for actions/.code={%
                \expandafter\def\expandafter\tikz@preactions\expandafter{\tikz@preactions\expandafter\let\expandafter\tikz@actions@path\csname tikz@intersect@path@name@#1\endcsname}%
            },
            append path for actions/.code={%
                \expandafter\def\expandafter\tikz@preactions\expandafter{\tikz@preactions
                \@appendnamedpathforactions{#1}}%
            },
            use path/.style={%
                use path for main=#1,
                use path for actions=#1,
            },
            append path/.style={%
                append path for main=#1,
                append path for actions=#1
            }
        }
    \makeatother

    % TikZ externalisation
    \usetikzlibrary{external}
    % Create the `tikzpics/` folder if it does not exist
    \ShellEscape{mkdir -p tikzpics}
    % Only externalise pictures on demand (to avoid messing up with metropolis theme)
    \tikzset{
        external/export=false,
        external/prefix=tikzpics/
    }
    \tikzexternalize

\usepackage{minted}
	\usemintedstyle{lovelace}
	\setminted{autogobble, fontsize=\scriptsize, tabsize=2}
	\setmintedinline{fontsize=auto}

\usepackage[style=authoryear, block=ragged, doi=false, isbn=false]{biblatex}
    \AtEveryBibitem{
        \ifentrytype{online}
        {} {
            \iffieldequalstr{howpublished}{online}
            {
                \clearfield{howpublished}
            } {
                \clearfield{urlyear}\clearfield{urlmonth}\clearfield{urlday}
            }
        }
    }

	\addbibresource{biblio.bib}

% Compact bibliography style
\setbeamertemplate{bibliography item}[text]

\AtEveryBibitem{
    \clearfield{series}
    \clearfield{pages}
    \clearlist{publisher}
    \clearname{editor}
    \clearlist{location}
}
\renewcommand*{\bibfont}{\tiny}

\usepackage{hyperxmp}	% XMP metadata

\usepackage[type={CC},modifier={by},version={4.0}]{doclicense}

\usepackage{todonotes}
\let\todox\todo
\renewcommand\todo[1]{\todox[inline]{#1}}

\title{\titlepagetitle}
\subtitle{Cours 3 : \docdate}
\author{\textbf{\myname} (\mylab)}
\institute{}
\date{\footnotesize Version {\yyyymmdddate\today}T\currenttime}

\titlegraphic{\ccby}

% Tikz styles

% Schémas de tâches
\tikzset{
    >=stealth,
    hair lines/.style={line width = 0.05pt, lightgray},
    data/.style={draw, ellipse},
    program/.style={draw, rectangle},
    accent on/.style={alt={<#1>{draw=accent, text=accent, thick}{draw}}},
    true scale/.style={scale=#1, every node/.style={transform shape}},
    highlight/.style={color=highlight#1}
}

% Styles des heatmap pour les moyennes
\pgfplotsset{
    meanheatmap/.style={
        colorbar, colormap name=SRON,
        view={0}{90},
        samples=100,
        domain=0:1,
        min=0, max=1,
        xlabel={$P$},
        ylabel={$R$},
    }
}

% Commands spécifiques
\NewDocumentCommand\shorturl{ O{https} O{://} m }{%
    \href{#1#2#3}{\nolinkurl{#3}}%
}

\DeclarePairedDelimiter\norm{\lVert}{\rVert}
\DeclarePairedDelimiter\abs{\lvert}{\rvert}
\DeclarePairedDelimiterX\compset[2]{\lbrace}{\rbrace}{#1\,\delimsize|\,#2}
\DeclarePairedDelimiterX\innprod[2]{\langle}{\rangle}{#1\,\delimsize|\,#2}

\DeclareMathOperator*\argmax{argmax}

% Easy column vectors \vcord{a,b,c} ou \vcord[;]{a;b;c}
% Here be black magic
\ExplSyntaxOn
	\NewDocumentCommand{\vcord}{O{,}m}{\vector_main:nnnn{p}{\\}{#1}{#2}}
	\NewDocumentCommand{\tvcord}{O{,}m}{\vector_main:nnnn{psmall}{\\}{#1}{#2}}
	\seq_new:N\l__vector_arg_seq
	\cs_new_protected:Npn\vector_main:nnnn #1 #2 #3 #4{
		\seq_set_split:Nnn\l__vector_arg_seq{#3}{#4}
		\begin{#1matrix}
			\seq_use:Nnnn\l__vector_arg_seq{#2}{#2}{#2}
		\end{#1matrix}
	}
\ExplSyntaxOff

\DeclareMathOperator{\TF}{TF}
\DeclareMathOperator{\IDF}{IDF}

\ExplSyntaxOn
    \DeclareExpandableDocumentCommand\eval{m}{\fp_eval:n{#1}}
\ExplSyntaxOff


% ██████   ██████   ██████ ██    ██ ███    ███ ███████ ███    ██ ████████
% ██   ██ ██    ██ ██      ██    ██ ████  ████ ██      ████   ██    ██
% ██   ██ ██    ██ ██      ██    ██ ██ ████ ██ █████   ██ ██  ██    ██
% ██   ██ ██    ██ ██      ██    ██ ██  ██  ██ ██      ██  ██ ██    ██
% ██████   ██████   ██████  ██████  ██      ██ ███████ ██   ████    ██


\begin{document}
\pdfbookmark[2]{Title}{title}

\begin{frame}[plain]
	\titlepage
\end{frame}

\begin{frame}{Exercice}
    \begin{itemize}
        \item Un moteur de recherche donné renvoie en général la plupart des documents pertinent pour une requête, mais aussi beaucoup de documents qui n'ont rien à voir. A-t-il une bonne précision ? Un bon rappel ?
        \item J'entre dans un moteur de recherche la requête « Chopin ». Il renvoie \num{10} documents qui concernent Frédéric Chopin, \num{5} qui concernent Florent Chopin, un qui concerne Paul-François Choppin, \num{2} qui concernent George Sand et \num{3} qui concernent la mission Apollo 13. Évaluer ce résultat.
    \end{itemize}
\end{frame}

\begin{frame}[fragile=singleslide]{Exercice}
    Étant donnée la matrice de confusion suivante (vraies classes en lignes)
    \begin{enumerate}
        \item Sans faire de calculs, que penser des résultats pour la classe « pony » ?
        \item Déterminer précision, rappel et F-mesure pour chacune des classes, ainsi que la micro- et macro-average.
    \end{enumerate}
    \begin{table}
        \begin{tabu}{|l|*{2}{c|}}
            \hline
                        &  pony    & metal band\\
            \hline
            pony        & \num{20} & \num{12}\\
            \hline
            metal band  & \num{4}  & \num{12}\\
            \hline
        \end{tabu}
        \caption{Matrice de confusion pour un problème à deux classes}
    \end{table}

    {\small\shorturl[http]{aiweirdness.com/post/174211306032/metal-band-or-my-little-pony}}
\end{frame}

% ██       █████  ██████  ███████ ██      ██      ██ ███    ██  ██████
% ██      ██   ██ ██   ██ ██      ██      ██      ██ ████   ██ ██
% ██      ███████ ██████  █████   ██      ██      ██ ██ ██  ██ ██   ███
% ██      ██   ██ ██   ██ ██      ██      ██      ██ ██  ██ ██ ██    ██
% ███████ ██   ██ ██████  ███████ ███████ ███████ ██ ██   ████  ██████

\section{Annotation}

\begin{frame}[fragile=singleslide]{Annotation}
    L'\emph{annotation} ou \emph{étiquetage} est la tâche consistant à attribuer une étiquette à chaque élément d'un ensemble \alert{structuré}.

    \begin{itemize}
        \item « Déterminer les catégories grammaticales des mots d'une phrase »
        \item « Découper un discours en parties thématiques »
    \end{itemize}

    \begin{figure}
        \tikzset{external/export=true}
        \begin{tikzpicture}[true scale=0.80]
            \node[data, highlight=4] (in) {Donnée};
            \node[program, right=1cm of in, text width=12ex] (prog) {Programme d'annotation};
            \node[data, right=1cm of prog] (out) {Donnée annotée};
            \node[data, below=.6cm of prog, highlight=8] (res) {Ensemble des étiquettes};
            \draw[->] (in) -- (prog);
            \draw[->] (prog) -- (out);
            \draw[->] (res) -- (prog);
        \end{tikzpicture}
    \end{figure}
    \vskip-1em
    \begin{itemize}
        \item La \textcolor{highlight4}{donnée} à annoter est composée de plusieurs unités et munie d'une structure d'\alert{ordre}
        \item Chacune des unités se voit attribuer une des \textcolor{highlight8}{étiquettes} possibles, qui sont connus à l'avance
    \end{itemize}
\end{frame}

\begin{frame}{Structure et ordre}
    L'annotation se distingue de la classification par la notion de \emph{structure}.
    \begin{itemize}
        \item Les unités à annoter sont liées par des relations
            \begin{itemize}
                \item Ordre total (séquence)
                \item Ordre partiel (arbre, DAG)
                \item Arbitraires (graphe)
            \end{itemize}
        \item Les étiquettes qui leur sont attribuées dépendent de ces relations
    \end{itemize}
\end{frame}

\begin{frame}[fragile=singleslide]{Exemples}
    \begin{figure}
        \tikzset{external/export=true}
        \begin{tikzpicture}
            \matrix[matrix of nodes, anchor=base, row sep=1.5em] (m) {
                Le  & petit & chat  & est & content\\
                DET & ADJ   & NC    & V   & ADJ\\
            };
            \foreach \x in {1, ..., 5}
                \draw[->] (m-1-\x) -- (m-2-\x);
        \end{tikzpicture}
        \caption{\emph{Part Of Speech Tagging}}
    \end{figure}
\end{frame}

\begin{frame}[fragile=singleslide]{Exemples}
    \begin{figure}
        \tikzset{external/export=true}
        \begin{tikzpicture}
            \matrix[matrix of nodes, anchor=base, row sep=1.5em] (m) {
                Le  & petit & chat  & est & content\\
                B & I   & I   & B & B\\
            };
            \foreach \x in {1, ..., 5}
                \draw[->] (m-1-\x) -- (m-2-\x);

            \matrix[matrix of nodes, anchor=base, below=2em of m] (s){
                (   & Le    & petit & chat & )  & ( & est   & ) & (  & mort & )\\
            };
            \foreach \x/\y in {4/5, 4/6, 5/8, 5/9}
                \draw[->] (m-2-\x) -- (s-1-\y);
        \end{tikzpicture}
        \caption{\emph{Chunking}}
    \end{figure}
\end{frame}

\begin{frame}[fragile=singleslide]{Exemples}
    \begin{figure}
        \tikzset{external/export=true}
        \begin{forest}
            [S
                [NP→\textcolor{highlighta}{sub}
                    [Det the]
                    [N cat]
                ]
                [VP→\textcolor{highlighta}{pred}
                    [V sat]
                    [PP→\textcolor{highlighta}{mod}
                        [P on]
                        [NP
                            [Det the]
                            [N mat]
                        ]
                    ]
                ]
            ]
        \end{forest}
        \caption{\textcolor{highlighta}{Étiquetage fonctionnel} d'un arbre syntaxique}
    \end{figure}
\end{frame}

\begin{frame}{Évaluation}
    Pour évaluer un étiquetage, on choisit en général de le considérer comme un problème de classification à plusieurs classes
    \begin{itemize}
        \item On considère chaque unité comme une donnée et chaque étiquette comme une classe
        \item On calcule comme précédemment $P$, $R$ et $F$
        \item On privilégie les micro-averages pour les scores globaux
    \end{itemize}
\end{frame}

\begin{frame}[fragile]{Aparté : micro-average et exactitude}
    \begin{theorem}
        Si chaque unité reçoit une unique étiquette, les micro-average de $P$, $R$ et $F$ sont égales à l'exactitude.
    \end{theorem}
    On considère la matrice de confusion suivante
    \begin{center}
        \tikzset{external/export=true}
        \begin{tikzpicture}[
            table/.style={
                matrix of nodes,
                row sep=-\pgflinewidth,
                column sep=-\pgflinewidth,
                nodes={draw, text width=3ex, align=center},
                text depth=0.25ex,
                text height=1em,
                nodes in empty cells
            },
            aa/.style={},
            bb/.style={},
            cc/.style={},
            ab/.style={},
            ac/.style={},
            ba/.style={},
            bc/.style={},
            ca/.style={},
            cb/.style={},
        ]
            \matrix[table]{
                    & |(apred)| A   & |(bpred)| B   & |(cpred)| C\\
                |(atrue)| A   & |[aa]| $aa$  & |[ab]| $ab$   & |[ac]| $ac$\\
                |(btrue)| B   & |[ba]| $ba$   & |[bb]| $bb$  & |[bc]| $bc$\\
                |(ctrue)| C   & |[ca]| $ca$   & |[cb]| $cb$   & |[cc]| $cc$\\
            };

            \draw [decorate, decoration={brace, amplitude=1em}, yshift=-0.5ex] (apred.north west) -- (cpred.north east) node [midway, above=1.25ex] {Classes prédites};
            \draw [decorate, decoration={brace, amplitude=1em, mirror}, xshift=-0.5ex] (atrue.north west) -- (ctrue.south west) node [midway, left=1.25ex, rotate=90, anchor=south] {Vraies classes};
        \end{tikzpicture}
    \end{center}
    Et on regarde le tableau !
\end{frame}

% TODO: add slides for the proof above

% ███████ ██
% ██      ██
% █████   ██
% ██      ██
% ███████ ██

\section{L'extraction d'informations}
\begin{frame}[fragile]{Extraction d'informations}
    L'\emph{extraction d'informations} est la tâche consistant à trouver dans un document les \alert{réponses} à des questions prédéfinies.

    \only<1>{
        \begin{itemize}
            \item « Déterminer le nom du héros et des principaux protagonistes d'un texte narratif »
            \item « Déterminer les symptômes présentés par un patient à partir d'un rapport clinique »
        \end{itemize}
    }

    \begin{figure}
        \tikzset{external/export=true}
        \begin{tikzpicture}[true scale=0.80]
            \node[data, highlight=4] (in) {Donnée};
            \node[program, right=1cm of in, text width=12ex] (prog) {Programme d'EI};
            \node[data, right=1cm of prog, highlight=6] (out) {Champs remplis};
            \node[data, below=.6cm of prog, highlight=8] (res) {Champs à remplir};
            \draw[->] (in) -- (prog);
            \draw[->] (prog) -- (out);
            \draw[->] (res) -- (prog);
        \end{tikzpicture}
    \end{figure}

    \only<2>{
        \begin{itemize}
            \item La \textcolor{highlight4}{donnée} est un texte brut ou un document semi-structuré
            \item La \textcolor{highlight6}{sortie} est un tableau qui donne la valeur de chacun des champs
            \item Les \textcolor{highlight8}{champs} \emph{peuvent} être multivalués et en nombre variable
        \end{itemize}
    }
\end{frame}

% TODO: add cites
\begin{frame}{Domaines d'application}
    Ils sont nombreux !
    \begin{itemize}
        \item Identification de protagonistes (littérature, militaire)
        \item Identification d'événement (biomédical, militaire)
        \item Fouille d'arguments
            \begin{itemize}
                \item Aide à la décision (politique)
                    \begin{itemize}
                        \item[→] IBM Watson (\textcite{slonim2014claims}, \shorturl{youtube.com/watch?v=sEf0GLvrP9U})
                    \end{itemize}
                \item Aide à la lecture (judiciaire, scientifique)
                    \begin{itemize}
                        \item[→] « Argument zoning » (\textcite{teufel2009argument}, \shorturl{www.cl.cam.ac.uk/~sht25/az.html})
                    \end{itemize}
            \end{itemize}
        \item …
    \end{itemize}
    À l'extrême, la tâche se confond avec la compréhension de document.
\end{frame}

\begin{frame}{Évaluation}
    L'idée est toujours de se comparer à une référence humaine.
    \begin{itemize}
        \item Basiquement : on compare les contenus des champs et on en déduit $P$, $R$ et $F$
        \item Plus subtilement : on peut utiliser des mesures d'adéquation entre les réponses
            \begin{itemize}
                \item[→] Une réponse peut être partiellement correcte : on pondère~!
            \end{itemize}
        \item Pour certaines tâches, on utilise des mesures spécifiques
            \begin{itemize}
                \item Pour les tâches comprenant une composante de localisation (NER, Argument Zoning), on peut appliquer des pondérations différentes à extraction et localisation
            \end{itemize}
    \end{itemize}
\end{frame}


%  █████  ██████  ██████  ███████ ███    ██ ██████  ██ ██   ██
% ██   ██ ██   ██ ██   ██ ██      ████   ██ ██   ██ ██  ██ ██
% ███████ ██████  ██████  █████   ██ ██  ██ ██   ██ ██   ███
% ██   ██ ██      ██      ██      ██  ██ ██ ██   ██ ██  ██ ██
% ██   ██ ██      ██      ███████ ██   ████ ██████  ██ ██   ██

\appendix
\pgfkeys{/metropolis/inner/sectionpage=simple}  % Avoid random errors with section page progressbar
\section{Annexes}
\pdfbookmark[2]{Remerciements}{acknowledgements}
\begin{frame}{Remerciements}
    Ce cours a été construit à partir du polycopié de cours \citetitle{tellier2017fouille} \parencite{tellier2017fouille} et des précieux conseils d'Isabelle Tellier que je ne saurais trop remercier pour sa confiance et son dévouement.
\end{frame}

\pdfbookmark[2]{Références}{references}
\begin{frame}[allowframebreaks]{References}
    \printbibliography[heading=none]
\end{frame}

\pdfbookmark[2]{Licence}{licence}
\begin{frame}{Licence}
    \begin{center}
        {\huge \ccby}
        \vfill
        This document is available under the terms of the Creative Commons Attribution 4.0 International License (CC BY 4.0) (\shorturl{creativecommons.org/licenses/by/4.0})

        Exceptions to the above statement are listed at {\small\shorturl{loicgrobol.github.io/intro-fouille-textes\#licences}}
        \vfill
        © 2019, Loïc Grobol <\shorturl[mailto][:]{loic.grobol@gmail.com}>

        \shorturl[http]{lattice.cnrs.fr/Grobol-Loic}
    \end{center}
\end{frame}

\end{document}
