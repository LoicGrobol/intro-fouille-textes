% Copyright © 2018, Loïc Grobol <loic.grobol@gmail.com>
% This document is available under the terms of the Creative Commons Attribution 4.0 International License (CC BY 4.0) (https://creativecommons.org/licenses/by/4.0/)

\RequirePackage{xparse}
\RequirePackage{shellesc}
% Settings
\NewDocumentCommand\myname{}{Loïc Grobol}
\NewDocumentCommand\mylab{}{Lattice / ALMAnaCH}
\NewDocumentCommand\pdftitle{}{Introduction à la fouille de textes}
\NewDocumentCommand\mymail{}{loic.grobol@gmail.com}
\NewDocumentCommand\titlepagetitle{}{\pdftitle}
\NewDocumentCommand\titlepagesubtitle{}{Cours 11 : Introduction aux réseaux de neurones}
\NewDocumentCommand\docdate{}{2019-04-15}
\NewDocumentCommand\conference{}{M1 Plurital}

\documentclass[hyperref={unicode}, xcolor={svgnames}, french]{beamer}
% Alternative navy blue for ⩽4 palettes
\definecolor{highlighta}{RGB}{68, 118, 170}  % Navy blue

% Colour palette from [Paul Tol's technical note](https://personal.sron.nl/~pault/data/colourschemes.pdf) v3.1
% Bright scheme
\definecolor{sronbrighttblue}{RGB}{68, 119, 170}
\definecolor{sronbrightcyan}{RGB}{102, 204, 238}
\definecolor{sronbrightgreen}{RGB}{34, 136, 51}
\definecolor{sronbrightyellow}{RGB}{204, 187, 68}
\definecolor{sronbrightred}{RGB}{238, 102, 119}
\definecolor{sronbrightpurple}{RGB}{170, 51, 119}
\definecolor{sronbrightgrey}{RGB}{187, 187, 187}

\colorlet{highlight0}{sronbrighttblue}
\colorlet{highlight1}{sronbrightred}
\colorlet{highlight2}{sronbrightgreen}
\colorlet{highlight3}{sronbrightyellow}
\colorlet{highlight4}{sronbrightcyan}
\colorlet{highlight5}{sronbrightpurple}
\colorlet{highlight6}{sronbrightgrey}
% Legacy highlights
\definecolor{highlight7}{RGB}{136, 34, 85}    % Purple
\definecolor{highlight8}{RGB}{170, 68, 153}   % Violet


\usetheme[
	sectionpage=progressbar,
    subsectionpage=progressbar,
	progressbar=frametitle,
]{metropolis}
	\colorlet{accent}{sronbrightgreen}
	\setbeamercolor{frametitle}{bg=DarkSlateGrey}
	\setbeamercolor{alerted text}{fg=accent}
	\makeatletter
		\setlength{\metropolis@progressinheadfoot@linewidth}{2pt}
	\makeatother
	% Avoid ugly whitespace below figures
	% \makeatletter
	% 	\renewenvironment{figure}[1][]{%
	% 	\def\@captype{figure}%
	% 	\par\centering}%
	% 	{\par}
	% \makeatother
	% FIXME: not pretty and footnotes are still too big
	\let\footnoterule\relax  % No footnote rule, push down footnote


% Use non-standard fonts
\usefonttheme{professionalfonts}
	\setsansfont[
		BoldFont={Fira Sans SemiBold},
		ItalicFont={Fira Sans Italic},
		BoldItalicFont={Fira Sans Bold Italic},
	]{Fira Sans Book}
\setmonofont[Scale=0.9]{Fira Mono}

% Fix missing glyphs in Fira by delegating to polyglossia/babel
\usepackage{newunicodechar}
\newunicodechar{ }{~}   % U+202F NARROW NO-BREAK SPACE
\newunicodechar{ }{ }  % U+2009 THIN SPACE

% Notes on left screen
% \usepackage{pgfpages}
% \setbeameroption{show notes on second screen=left}


\usepackage[french, english]{babel}
\usepackage{amsfonts,amssymb}
\usepackage{amsmath,amsthm}
\usepackage{mathtools}	% AMS Maths service pack
	\newtagform{brackets}{[}{]}	% Pour des lignes d'équation numérotées entre crochets
	\mathtoolsset{showonlyrefs, showmanualtags, mathic}	% affiche les tags manuels (\tag et \tag*) et corrige le kerning des maths inline dans un bloc italique voir la doc de mathtools
	\usetagform{brackets}	% Utilise le style de tags défini plus haut
\usepackage{lualatex-math}

\usepackage[math-style=french]{unicode-math}
	\setmathfont{Libertinus Math}
\usepackage{newunicodechar}
	\newunicodechar{√}{\sqrt}
\usepackage{mleftright}

\usepackage{tabu}
\usepackage{booktabs}
\usepackage{siunitx}
\usepackage{multicol}
\usepackage{ccicons}
\usepackage{bookmark}
\usepackage{caption}
    \captionsetup{skip=1ex}
\usepackage[iso]{datetime}

\usepackage{tikz}
	\NewDocumentCommand{\textnode}{O{}mm}{\tikz[remember picture, baseline=(#2.base), inner sep=0pt]{\node[#1] (#2) {#3};}}
    \NewDocumentCommand{\mathnode}{O{}mm}{\tikz[remember picture, baseline=(#2.base), inner sep = 0pt]{\node[#1] (#2) {$\displaystyle #3$};}}
	\tikzset{
		alt/.code args={<#1>#2#3}{%
		\alt<#1>{\pgfkeysalso{#2}}{\pgfkeysalso{#3}} % \pgfkeysalso doesn't change the path
		},
        invisible/.style={opacity=0, fill opacity=0},
		visible on/.style={alt={<#1>{}{invisible}}}
	}
    \usepackage{forest}
    \usepackage{tkz-graph}
    \usepackage[beamer, markings]{hf-tikz}
    \usepackage{tikz-3dplot}
    \usepackage{pgfplots}
        % Due to pgfplots meddling with pgfkeys, we have to redefine alt here.
        \pgfplotsset{
    		alt/.code args={<#1>#2#3}{%
    		\alt<#1>{\pgfkeysalso{#2}}{\pgfkeysalso{#3}} % \pgfkeysalso doesn't change the path
    		},
    	}
        \pgfplotsset{compat=1.15}
        \pgfplotsset{colormap={SRON}{rgb255=(61,82,161) rgb255=(255,250,210) rgb255=(174,28,62)}}

    \usetikzlibrary{matrix}
    \usetikzlibrary{shapes, shapes.geometric}
    \usetikzlibrary{decorations.pathreplacing}
	\usetikzlibrary{positioning, calc, intersections}
    \usetikzlibrary{fit}
    \usetikzlibrary{backgrounds}

    % Do evil things with soft path
    \makeatletter
        \def\@appendnamedsoftpath#1{%
            \pgfsyssoftpath@getcurrentpath\@temppatha
            \expandafter\let\expandafter\@temppathb\csname tikz@intersect@path@name@#1\endcsname
            \expandafter\expandafter\expandafter\def\expandafter\expandafter\expandafter\@temppatha\expandafter\expandafter\expandafter{\expandafter\@temppatha\@temppathb}%
            \pgfsyssoftpath@setcurrentpath\@temppatha
        }
        \def\@appendnamedpathforactions#1{%
            \pgfsyssoftpath@getcurrentpath\@temppatha
            \expandafter\let\expandafter\@temppathb\csname tikz@intersect@path@name@#1\endcsname
            \expandafter\def\expandafter\@temppatha\expandafter{\csname @temppatha\expandafter\endcsname\@temppathb}%\usepackage{tikz-3dplot}

            \let\tikz@actions@path\@temppatha
        }

        \tikzset{
            use path for main/.code={%
                \tikz@addmode{%
                    \expandafter\pgfsyssoftpath@setcurrentpath\csname tikz@intersect@path@name@#1\endcsname
                }%
            },
            append path for main/.code={%
                \tikz@addmode{%
                    \@appendnamedsoftpath{#1}%
                }%
            },
            use path for actions/.code={%
                \expandafter\def\expandafter\tikz@preactions\expandafter{\tikz@preactions\expandafter\let\expandafter\tikz@actions@path\csname tikz@intersect@path@name@#1\endcsname}%
            },
            append path for actions/.code={%
                \expandafter\def\expandafter\tikz@preactions\expandafter{\tikz@preactions
                \@appendnamedpathforactions{#1}}%
            },
            use path/.style={%
                use path for main=#1,
                use path for actions=#1,
            },
            append path/.style={%
                append path for main=#1,
                append path for actions=#1
            }
        }
    \makeatother

    % TikZ externalisation
    \usetikzlibrary{external}
    % Create the `tikzpics/` folder if it does not exist
    \ShellEscape{mkdir -p tikzpics}
    % Only externalise pictures on demand (to avoid messing up with metropolis theme)
    \tikzset{
        external/export=false,
        external/prefix=tikzpics/
    }
    \tikzexternalize

\usepackage{minted}
	\usemintedstyle{lovelace}
	\setminted{autogobble, fontsize=\scriptsize, tabsize=2}
	\setmintedinline{fontsize=auto}

\usepackage{csquotes}

\usepackage[style=authoryear, block=ragged, doi=false, isbn=false]{biblatex}
    \AtEveryBibitem{
        \ifentrytype{online}
        {} {
            \iffieldequalstr{howpublished}{online}
            {
                \clearfield{howpublished}
            } {
                \clearfield{urlyear}\clearfield{urlmonth}\clearfield{urlday}
            }
        }
    }

	\addbibresource{biblio.bib}

% Compact bibliography style
\setbeamertemplate{bibliography item}[text]

\AtEveryBibitem{
    \clearfield{series}
    \clearfield{pages}
    \clearlist{publisher}
    \clearname{editor}
    \clearlist{location}
}
\renewcommand*{\bibfont}{\tiny}

\usepackage{hyperxmp}	% XMP metadata

\usepackage[type={CC}, modifier={by}, version={4.0}]{doclicense}

\usepackage{todonotes}
\let\todox\todo
\renewcommand\todo[1]{\todox[inline]{#1}}

\title{\titlepagetitle}
\subtitle{\titlepagesubtitle}
\author{\textbf{\myname} (\mylab)}
\institute{}
\date{\tiny Version {\yyyymmdddate\today}T\currenttime}

\titlegraphic{\ccby}

% Tikz styles

% Schémas de tâches
\tikzset{
    >=stealth,
    hair lines/.style={line width = 0.05pt, lightgray},
    data/.style={draw, ellipse},
    program/.style={draw, rectangle},
    accent on/.style={alt={<#1>{draw=accent, text=accent, thick}{draw}}},
    true scale/.style={scale=#1, every node/.style={transform shape}},
    highlight/.style={color=highlight#1}
}

% Styles des heatmap pour les moyennes
\pgfplotsset{
    meanheatmap/.style={
        colorbar, colormap name=SRON,
        view={0}{90},
        samples=100,
        domain=0:1,
        min=0, max=1,
        xlabel={$P$},
        ylabel={$R$},
    }
}

% Commands spécifiques
\NewDocumentCommand\shorturl{ O{https} O{://} m }{%
    \href{#1#2#3}{\nolinkurl{#3}}%
}

\DeclarePairedDelimiter\norm{\lVert}{\rVert}
\DeclarePairedDelimiter\abs{\lvert}{\rvert}
\DeclarePairedDelimiterX\compset[2]{\lbrace}{\rbrace}{#1\,\delimsize|\,#2}
\DeclarePairedDelimiterX\innprod[2]{\langle}{\rangle}{#1\,\delimsize|\,#2}

\DeclareMathOperator*\argmax{argmax}

% Easy column vectors \vcord{a,b,c} ou \vcord[;]{a;b;c}
% Here be black magic
\ExplSyntaxOn
	\NewDocumentCommand{\vcord}{O{,}m}{\vector_main:nnnn{p}{\\}{#1}{#2}}
	\NewDocumentCommand{\tvcord}{O{,}m}{\vector_main:nnnn{psmall}{\\}{#1}{#2}}
	\seq_new:N\l__vector_arg_seq
	\cs_new_protected:Npn\vector_main:nnnn #1 #2 #3 #4{
		\seq_set_split:Nnn\l__vector_arg_seq{#3}{#4}
		\begin{#1matrix}
			\seq_use:Nnnn\l__vector_arg_seq{#2}{#2}{#2}
		\end{#1matrix}
	}
\ExplSyntaxOff

\DeclareMathOperator{\TF}{TF}
\DeclareMathOperator{\IDF}{IDF}

\ExplSyntaxOn
    \DeclareExpandableDocumentCommand\eval{m}{\fp_eval:n{#1}}
\ExplSyntaxOff


% Fixes for pauses after an unveiled list
\newcommand{\itpause}{%
	\addtocounter{beamerpauses}{-1}%
	\pause
}


% ██████   ██████   ██████ ██    ██ ███    ███ ███████ ███    ██ ████████
% ██   ██ ██    ██ ██      ██    ██ ████  ████ ██      ████   ██    ██
% ██   ██ ██    ██ ██      ██    ██ ██ ████ ██ █████   ██ ██  ██    ██
% ██   ██ ██    ██ ██      ██    ██ ██  ██  ██ ██      ██  ██ ██    ██
% ██████   ██████   ██████  ██████  ██      ██ ███████ ██   ████    ██


\begin{document}
\pdfbookmark[2]{Title}{title}

\begin{frame}[plain]
	\titlepage
\end{frame}


% ███    ██ ███████ ████████
% ████   ██ ██         ██
% ██ ██  ██ █████      ██
% ██  ██ ██ ██         ██
% ██   ████ ███████    ██


\section{Réseaux de neurones}

% TODO: cite
\begin{frame}{Réseaux de neurones}
    \textbf{Dans Weka} classifiers > functions > MultilayerPerceptron
    \pause

    \textbf{Espace de recherche} Ensemble des chaînes de fonctions linéaires et d'activations non-linéaires
    \pause

    \textbf{Technique de recherche} en général : descente de gradient stochastique par rétropropagation
\end{frame}

\begin{frame}{Le modèle du perceptron}
    Frank Rosenblatt (1957) d'après Warren McCulloch et Walter Pitts (1943).
	\vspace{-2\bigskipamount}
    \begin{figure}
        \tikzset{external/export=true}
        \begin{tikzpicture}[
            basic/.style={draw, minimum width=3em},
            input/.style={basic, circle},
            weights/.style={basic, circle, minimum width=2em},
            functions/.style={basic, circle},
            true scale=0.8,
        ]
            \node[functions] (sum) at (7,0) {$∑$};
            \foreach \x [count=\hi] in {n,1,2,0}{%
				\node[input] (f\hi) at (0,\hi*2cm-5 cm) {$x_{\x}$};
                \path (f\hi) -- node[weights] (w\hi) {$w_{\x}$} (sum);
                \draw[->] (f\hi) -- (w\hi);
                \draw[->] (w\hi) -- (sum) node[midway, above] {$w_{\x}x_{\x}$};
            }
            \node at ($(f2)!0.5!(f1)$) {$⋮$};
            \node at ($(w2)!0.5!(w1)$) {$⋮$};
            \node[functions] (step) at (11,0) {};
               \begin{scope}[xshift=11cm,scale=.75]
                    \draw[thick]
						(0.5em,0.5em) -- (0,0.5em) --
                        (0,-0.5em) --  (-0.5em,-0.5em)
                        (0em,0.75em) -- (0em,-0.75em)
                        (0.75em,0em) -- (-0.75em,0em);
               \end{scope}
            \draw[->] (sum) -- (step) node[midway, above] {$∑wᵢxᵢ$};
            \draw[->] (step) -- ++(1,0);
            \node[above right=0.5em and 3em of step] (class0) {$0$};
            \node[below right=0.5em and 3em of step] (class1) {$1$};
            \draw [decorate, decoration={brace, amplitude=1em}, xshift=-0.5ex] (class1.west) -- (class0.west);
            % Labels
            \node[above=1cm] at (f4) {Entrées};
            \node[above=1cm] at (w4) {Poids};
            \node[above=1cm of step] {Activation};
        \end{tikzpicture}
        \caption{Perceptron simple}
    \end{figure}
\end{frame}

\begin{frame}{Exercice}
	On considère un perceptron simple dont les poids sont
	\begin{equation}
		w = \vcord{2,-1,0,1}
	\end{equation}

	Donner la sortie $y$ du neurone pour l'entrée
	\begin{equation}
		x = \vcord{-2,-1,5,4}
	\end{equation}
\end{frame}


\begin{frame}[fragile]{Fonction de Heavyside}
    La fonction de Heavyside $H$ est l'activation classique pour faire du perceptron un classifieur binaire : elle transforme les nombres négatifs en $0$ et les nombres positifs en $1$.
    \vspace{-1\bigskipamount}
    \begin{figure}
        \tikzset{external/export=true}
        \begin{tikzpicture}
            \begin{axis}[
				xmin=-10, xmax=10, ymin=-0.5,
                width=0.9\textwidth,
                height=0.8\textheight,
                xlabel={$x$},
                ylabel={$H(x)$},
                title={Fonction de Heavyside : $y=\max(x, 0)$},
                scaled ticks=false,
			]
                \addplot[highlighta, domain=-10:0, samples=100] {greater(x,0)};
                \addplot[highlighta, domain=0:10, samples=100] {greater(x,0)};
            \end{axis}
        \end{tikzpicture}
        \caption{Courbe de la fonction de Heavyside}
    \end{figure}
\end{frame}

\begin{frame}{Exercice}
	On considère un perceptron simple dont les poids sont
	\begin{equation}
		w = \vcord{-2,-1,1,0}
	\end{equation}

	\begin{enumerate}
		\item Donner la classe obtenue pour l'entrée
			\begin{equation}
				x = \vcord{-1, 0, -3, 1}
			\end{equation}
		\item Donner une entrée pour laquelle on obtient la classe $1$
	\end{enumerate}
\end{frame}

\begin{frame}{Algorithme du perceptron}
    Le perceptron s'entraîne en parcourant l'ensemble d'entraînement point par point.
    On commence en initialisant tous les poids$ w_i$ à $0$ (ou à un petit nombre aléatoire).

    Pour chaque exemple $x$ dont la classe est $y$
    \begin{enumerate}
        \item Calculer la classe observée $̄\bar{y}=H\mleft(∑w_ix_i\mright)$
        \item
            \begin{itemize}
                \item Si $y=\bar{y}$, on ne change rien
                \item Sinon, on met à jour les poids $w_i$ selon la règle
                    \begin{equation}
                        w_i ← w_i + α(y-\bar{y})x_i
                    \end{equation}
            \end{itemize}
    \end{enumerate}
    Une fois l'ensemble d'entraînement parcouru, on a réalisé une  \emph{époque d'apprentissage}, on peut alors soit s'arrêter, soit recommencer du début.
    \begin{itemize}
        \item $α>0$ est un hyperparamètre appelé \emph{taux d'apprentissage}
    \end{itemize}
\end{frame}


\begin{frame}[fragile]{Algorithme du perceptron}
    En deux dimensions, avec les classes \textcolor{highlighta}{$1$} et \textcolor{highlight1}{$0$}, avec un taux d'apprentissage $α=0.5$.
    % TODO: this could be even more automated
    \begin{figure}
        % \tikzset{external/export=true}
        \begin{tikzpicture}
            \begin{scope}
                \path[clip] (-0.5, -0.5) rectangle (5.25, 4.25);
                \draw [help lines, xstep=1cm, ystep=1cm] (0, 0) grid (5.25, 4.25);

                \draw[->] (-0.5,0) -- (5.25, 0);
                \foreach \x in {1,...,5}
                    \draw[shift={(\x, 0)}] (0pt,2pt) -- (0pt,-2pt) node[below] {\footnotesize $\x$};

                \draw[->] (0, -0.5) -- (0, 4.25);
                \foreach \y in {1,...,4}
                    \draw[shift={(0,\y)}] (2pt,0pt) -- (-2pt,0pt) node[left] {\footnotesize $\y$};

                \draw (0, 0) node[below left] {\footnotesize $0$};

                \foreach \x/\y/\c [count=\i] in {2/1/1,4/0/1,1/0/1,2/1/1,0/1/a,1/1/a,0/1/a}{
                    \node at (\x, \y) (sample\i) {};
                    \path[fill=highlight\c] (sample\i) circle[radius=3pt];
                }

                \begin{scope}[domain=-0.5:5.25]
                    \begin{scope}
                        \pgfmathsetmacro{\a}{2}
                        \pgfmathsetmacro{\b}{1}
                        \pgfmathsetmacro{\c}{-4}
                        \pgfmathtruncatemacro{\toprightscore}{\a*5.25+\b*4.25+\c}
                        \path[
                            fill={\ifnum \toprightscore>0 highlighta\else highlight1\fi},
                            fill opacity=0.5, visible on={3-4}
                        ]
                            plot (\x, {(\a*\x+\c)/(-\b)}) -- (5.25,4.25) -- (-0.5,4.25) -- cycle;
                        \path[
                            fill={\ifnum \toprightscore>0 highlight1\else highlighta\fi},
                            fill opacity=0.5, visible on={3-4}
                        ]
                            plot (\x, {(\a*\x+\c)/(-\b)}) -- (5.25,-0.5) -- (-0.5,-0.5) -- cycle;

                        \draw[highlight1, thick, visible on={2-4}] plot (\x, {(\a*\x+\c)/(-\b)});

                        \draw[->, thick, visible on={2-5}] (1, {(\a+\c)/(-\b)}) -- +(\a, \b) node[coordinate, label=above:$w$] (w1) {};

                        \path[draw=highlight5, very thick, visible on={4-7}] (sample1) circle[radius=3pt];
                        \draw[->, thick, visible on={4-7}] (0, 0) -- (sample1) node[midway, above left] {$x$};

                        \draw[->, thick, visible on={5}] (w1) -- +(-1, -0.5) node[midway, below right] {$α×(-1)⋅x$};
                    \end{scope}
                    \begin{scope}
                        \pgfmathsetmacro{\a}{1}
                        \pgfmathsetmacro{\b}{0.5}
                        \pgfmathsetmacro{\c}{-4.5}
                        \pgfmathtruncatemacro{\toprightscore}{\a*5.25+\b*4.25+\c}
                        \path[
                            fill={\ifnum \toprightscore>0 highlighta\else highlight1\fi},
                            fill opacity=0.5, visible on=8,
                        ]
                            plot (\x, {(\a*\x+\c)/(-\b)}) -- (5.25,4.25) -- (-0.5,4.25) -- cycle;
                        \path[
                            fill={\ifnum \toprightscore>0 highlight1\else highlighta\fi},
                            fill opacity=0.5, visible on=8,
                        ] plot (\x, {(\a*\x+\c)/(-\b)}) -- (5.25,-0.5) -- (-0.5,-0.5) -- cycle;

                        \draw[highlight1, thick, visible on={7-8}] plot (\x, {(\a*\x+\c)/(-\b)});

                        \draw[->, thick, visible on={6-8}] (3.5, {(\a*3.5+\c)/(-\b)}) -- +(\a, \b) node[coordinate, label=above:$w$] (w2) {};

                        \path[draw=highlight3, very thick, visible on=8] (sample1) circle[radius=3pt];
                        \draw[->, thick, visible on=8] (0, 0) -- (sample1) node[midway, above left] {$x$};
                    \end{scope}
                \end{scope}
            \end{scope}
        \end{tikzpicture}
        \caption{Algorithme du perceptron}
    \end{figure}
    \only<8>{}
\end{frame}


%  █████  ██████  ██████  ███████ ███    ██ ██████  ██ ██   ██
% ██   ██ ██   ██ ██   ██ ██      ████   ██ ██   ██ ██  ██ ██
% ███████ ██████  ██████  █████   ██ ██  ██ ██   ██ ██   ███
% ██   ██ ██      ██      ██      ██  ██ ██ ██   ██ ██  ██ ██
% ██   ██ ██      ██      ███████ ██   ████ ██████  ██ ██   ██

\appendix
\pgfkeys{/metropolis/inner/sectionpage=simple}  % Avoid random errors with section page progressbar
\section{Annexes}
\pdfbookmark[2]{Remerciements}{acknowledgements}
\begin{frame}{Remerciements}
    Ce cours a été construit à partir du polycopié de cours \citetitle{tellier2017fouille} \parencite{tellier2017fouille} et des précieux conseils d'Isabelle Tellier que je ne saurais trop remercier pour sa confiance et son dévouement.
\end{frame}

\pdfbookmark[2]{Références}{references}
\begin{frame}[allowframebreaks]{References}
    \printbibliography[heading=none]
\end{frame}

\pdfbookmark[2]{Licence}{licence}
\begin{frame}{Licence}
    \begin{center}
        {\huge \ccby}
        \vfill
        This document is available under the terms of the Creative Commons Attribution 4.0 International License (CC BY 4.0) (\shorturl{creativecommons.org/licenses/by/4.0})

        Exceptions to the above statement are listed at {\small\shorturl{loicgrobol.github.io/intro-fouille-textes\#licences}}
        \vfill
        © 2019, Loïc Grobol <\shorturl[mailto][:]{loic.grobol@gmail.com}>

        \shorturl[http]{lattice.cnrs.fr/Grobol-Loic}
    \end{center}
\end{frame}

\end{document}
