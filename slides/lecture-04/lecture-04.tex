% LTeX: language=fr
% Copyright © 2019, Loïc Grobol <loic.grobol@gmail.com>
% This document is available under the terms of the Creative Commons Attribution 4.0 International License (CC BY 4.0) (https://creativecommons.org/licenses/by/4.0/)

\documentclass[../allslides.tex]{subfiles}
\renewcommand\titlepagesubtitle{Cours 4 : Décomposition en tâches élémentaires et représentations des données}
\renewcommand\docdate{2021-02-18}  % chktex 8

\begin{document}
\part{\titlepagesubtitle{}}

\begin{frame}[plain]
	\partpage % chktex 1
\end{frame}


% ██████  ███████  ██████  ██████  ███    ███ ██████
% ██   ██ ██      ██      ██    ██ ████  ████ ██   ██
% ██   ██ █████   ██      ██    ██ ██ ████ ██ ██████
% ██   ██ ██      ██      ██    ██ ██  ██  ██ ██
% ██████  ███████  ██████  ██████  ██      ██ ██

\section{Décomposition en tâches élémentaires}
\begin{frame}[fragile]{Décomposition en tâches élémentaires}
	Les tâches \emph{élémentaires} le sont parce qu'on peut y réduire des tâches plus complexes.

	\only<1>{
		\begin{figure}
			\tikzset{external/export=true}
			\begin{tikzpicture}[true scale=0.75]
				\node[data] (in) {Entrée};
				\node[right=1cm of in, text width=10ex, accent on=1] (prog) {Programme};
				\node[data, right=1cm of prog] (out) {Résultat};
				\draw[->] (in) -- (prog);
				\draw[->] (prog) -- (out);
			\end{tikzpicture}
			\caption{Schéma général d'une tâche}
		\end{figure}
		Le \alert{programme} réalisant la tâche sera ainsi composé de sous-programmes réalisant des tâches élémentaires.
	}

	\only<2>{
		\begin{figure}
			\tikzset{external/export=true}
			\begin{tikzpicture}[true scale=0.6]
				\node[data] (in1) {Entrée};
				\node[program, right=1cm of in1, text width=12ex] (prog1) {Programme élémentaire};
				\node[data, right=0.75cm of prog1] (out1) {Résultat};
				\draw[->] (in1) -- (prog1);
				\draw[->] (prog1) -- (out1);

				\node[program, right=0.75cm of out1, text width=12ex] (prog2) {Programme élémentaire};
				\node[data, right=1cm of prog2] (out2) {Résultat};
				\draw[->] (out1) -- (prog2);
				\draw[->] (prog2) -- (out2);

				\draw ($(prog1.north west)+(-0.5cm, 1cm)$) rectangle ($(prog2.south east)+(0.5cm, -1cm)$);
			\end{tikzpicture}
			\caption{Décomposition d'une tâche complexe}
		\end{figure}
	}

	\only<3>{
		\begin{figure}
			\tikzset{external/export=true}
			\begin{tikzpicture}[true scale=0.6]
				\node[data] (in1) {Entrée};
				\node[program, right=2cm of in1, text width=12ex] (prog1) {Programme élémentaire};
				\node[data, right=1cm of prog1] (out1) {Résultat};
				\draw[->] (in1) -- (prog1);
				\draw[->] (prog1) -- (out1);

				\node[program, above=1cm of out1, text width=12ex] (prog2) {Programme élémentaire};
				\node[data, right=2cm of prog2] (out2) {Résultat};
				\draw[->] (out1) -- (prog2);
				\draw[->] (prog2) -- (out2);
				\draw[->] (in1) -- ($(prog1.west)-(0.75cm, 0)$) |- (prog2.west);

				\draw ($(prog1.south west)+(-1cm, -1cm)$) rectangle ($(prog2.north east)+(1cm, 1cm)$);
			\end{tikzpicture}
			\caption{Décomposition d'une tâche complexe}
		\end{figure}
	}
\end{frame}

\begin{frame}{Exemple : les systèmes de Question/Réponse}
	Les systèmes de question/réponse fournissent des réponses factuelles à des requêtes données en langage naturel

	\begin{figure}
		\tikzset{external/export=true}
		\begin{tikzpicture}[true scale=0.75]
			\node[data, text width=15ex] (in) {« Quelle est la vitesse d'une hirondelle ? »};
			\node[program, right=1cm of in, text width=15ex] (prog) {Programme de Q/R};
			\node[data, right=1cm of prog] (out) {« \SI{60}{km/h} »};
			\node[data, below=.6cm of prog] (res) {Corpus};
			\draw[->] (in) -- (prog);
			\draw[->] (prog) -- (out);
			\draw[->] (res) -- (prog);
		\end{tikzpicture}
		\caption{Structure d'un système de Q/R}
	\end{figure}
\end{frame}

\begin{frame}[fragile]{Q/R via RI}
	% TODO: add progressive result feedback on the “Quelle est la vitesse d'une hirondelle sample question”
	% TODO: add arrow colours
	La première reformulation possible utilise un système de \alert<1>{RI}

	\begin{figure}
		\tikzset{external/export=true}
		\begin{tikzpicture}[true scale=0.55]
			\node[data, alt=<2>{highlight=6}{}] (in) {Question};
			\node[coordinate, right=1em of in] (split) {};
			\fill[alt=<2>{highlight=6}{}] (split) circle[radius=2pt];
			\draw[alt=<2>{highlight=6, thick}{}] (in) -- (split);

			\node[program, above right=0.5em and 1em of split, alt=<2>{highlight=a}{}] (term) {Terminologie};
			\node[data, right=1em of term, alt=<3>{highlight=a}{}] (words) {Mots};
			\node[program, right=1em of words, accent on=1, alt=<3>{highlight=3}{}] (ri) {RI};
			\node[data, right=1em of ri, alt=<3-4>{highlight=8}{}] (docs) {Docs};
			\node[program, right=1em of docs, alt=<4>{highlight=3}{}] (ei) {EI};
			\node[coordinate, below right=0.5em and 1em of ei] (gather) {};
			\node[data, right=1em of gather, alt=<4>{highlight=a}{}] (answer) {Réponse};
			\draw[->, alt=<2>{highlight=6, thick}{}] (split) |- (term);
			\draw[->] (term) -- (words);
			\draw[->] (words) -- (ri);
			\draw[->] (ri) -- (docs);
			\draw[->] (docs) -- (ei);
			\draw[->] (ei) -| (gather) -- (answer);

			\node[program, below right=0.5em and 1em of split, alt=<2>{highlight=3}{}] (classif) {Classifieur};
			\node[data, below=1em of ei, alt=<4>{highlight=2}{}] (field) {Champs};
			\draw[->, alt=<2>{highlight=6, thick}{}] (split) |- (classif);
			\draw[->] (classif) -- (field);
			\draw[->] (field) -- (ei);

			\draw ($(split)+(-0.5cm, 2cm)$) rectangle ($(field.south east)+(1cm, -0.5cm)$);

			\node[data, below=3em of ri, alt=<3>{highlight=2}{}] (corpus) {Corpus};
			\draw[->] (corpus) -- (ri);
		\end{tikzpicture}
		\caption{Système de Question/Réponse via Recherche d'Informations}
	\end{figure}
	\only<2->{
		\begin{itemize}
			\only<2>{
				\item \textcolor{highlight6}{La question en langage naturel} est envoyée en parallèle à \textcolor{highlighta}{un extracteur de mots-clés} et à un \textcolor{highlight3}{classifieur} qui détermine le type de la réponse attendue
			}
			\only<3>{
				\item \textcolor{highlighta}{Les mots-clés extraits} sont utilisés comme requête pour \textcolor{highlight3}{le programme de RI} qui sélectionne dans \textcolor{highlight2}{le corpus} \textcolor{highlight8}{les documents pertinents} pour répondre à la question
			}
			\only<4>{
				\item<4> \textcolor{highlight3}{Un programme d'EI} extrait des \textcolor{highlight8}{documents pertinents} le contenu des \textcolor{highlight2}{champs}, qui constituent la \textcolor{highlighta}{réponse} à la question posée
			}
		\end{itemize}
	}
\end{frame}

\begin{frame}<1>[fragile]{Q/R via base de connaissances}
	% TODO: add further overlay
	% TODO: add progressive result feedback on the “Quelle est la vitesse d'une hirondelle sample question”
	% TODO: add arrow colours
	Une autre stratégie consiste à reformuler la question en requête dans une \alert<1>{base de connaissances structurée}

	\begin{figure}
		\tikzset{external/export=true}
		\begin{tikzpicture}[true scale=0.6]
			\node[data, alt=<2>{highlight=6}{}] (in) {Question};
			\node[coordinate, right=1em of in] (split) {};
			\draw[alt=<2>{highlight=6, thick}{}] (in) -- (split);

			\node[program, above right=0.5em and 1em of split, alt=<2>{highlight=a}{}] (trad) {Traduction};
			\node[data, right=1em of trad, alt=<2-3>{highlight=3}{}] (req) {Requête};
			\node[program, right=1em of req, alt=<3>{highlight=3}{}] (int) {Interrogation};
			\node[coordinate, below right=0.5em and 1em of int] (gather) {};
			\node[data, right=1em of gather, alt=<4>{highlight=a}{}] (answer) {Réponse};
			\draw[->, alt=<2>{highlight=6, thick}{}] (split) |- (trad);
			\draw[->] (trad) -- (req);
			\draw[->] (req) -- (int);
			\draw[->] (int) -| (gather) -- (answer);

			\node[program, below=1em of req, alt=<2>{highlight=3}{}] (ei) {EI};
			\node[data, accent on=1, alt=<4>{highlight=2}{}] at (int|-ei) (base) {Base};
			\draw[->] (ei) -- (base);
			\draw[->] (base) -- (int);

			\draw ($(split|-trad.north)+(-0.5cm,0.5cm)$) rectangle ($(int.east|-ei.south)+(1.25cm, -0.5cm)$);

			\node[data, below=3em of ei, alt=<3>{highlight=2}{}] (corpus) {Corpus};
			\draw[->] (corpus.west) -- ++ (-0.25em, 0) |- (ei.west);
		\end{tikzpicture}
		\caption{Système de Question/Réponse utilisant une base de connaissances}
	\end{figure}
	\only<2->{
		\begin{itemize}
			\item<2|only@2> \textcolor{highlight6}{La question en langue naturelle} est \textcolor{highlighta}{traduite} en une \textcolor{highlight3}{requête} dans un langage formel (SQL, SPARQL…)
		\end{itemize}
	}
\end{frame}

% TODO: add links and refs for the two deompositions

\begin{frame}{Conclusion}
	On a donc vu qu'on pouvait
	\begin{itemize}
		\item Réaliser des tâches complexes en se ramenant des tâches élémentaires
		\item Reformuler les tâches élémentaires en terme de classification
	\end{itemize}

	En pratique, on peut donc presque toujours ramener n'importe quelle tâche de fouille de texte à de la classification, on parlera donc pour celle-ci de \alert{tâche pivot}.
	\begin{itemize}
		\item[→] Attention cependant à ne pas faire cuire les pâtes comme un mathématicien…
	\end{itemize}
\end{frame}

% ██████  ███████ ██████  ██████  ███████ ███████
% ██   ██ ██      ██   ██ ██   ██ ██      ██
% ██████  █████   ██████  ██████  █████   ███████
% ██   ██ ██      ██      ██   ██ ██           ██
% ██   ██ ███████ ██      ██   ██ ███████ ███████


\section{Représentation des données}

\begin{frame}{Représentation des données}
	Étant donné que
	\begin{itemize}
		\item La classification joue en fouille de texte le rôle de tâche pivot
		\item On privilégie pour la classification des données au format tabulaire
	\end{itemize}
	Il est naturel de vouloir mettre les données à traiter sous forme tabulaire.
\end{frame}

% ███████ ████████  █████  ████████
% ██         ██    ██   ██    ██
% ███████    ██    ███████    ██
%      ██    ██    ██   ██    ██
% ███████    ██    ██   ██    ██

\subsection{Spécificités statistiques}
\begin{frame}{Loi de Zipf}
	Dans \textit{Ulysse} de James Joyce on\textsuperscript{1} observe\textsuperscript{2} que
	\begin{itemize}
		\item Le mot le plus courant apparaît \num{8000} fois
		\item Le dixième mot le plus courant apparaît \num{800} fois
		\item Le centième mot le plus courant apparaît \num{80} fois
		\item Le millième mot le plus courant apparaît \num{8} fois
	\end{itemize}

	\begin{block}{Hypothèse (Loi de Zipf)}
		La fréquence \(f(n)\) du \(n\)-ième mot le plus fréquent d'un corpus vérifie
		\begin{equation}
			f(n) = \frac{K}{n}
		\end{equation}
		où $K$ est une constante ne dépendant que du corpus.
	\end{block}
	\vskip0pt plus 1fill
	{\tiny 1. Georges Zipf (1902-1950) dans \parencite{zipf1949HumanBehaviorPrinciple}\\2. d'après la légende}
\end{frame}

% TODO: add Ulysse répartition
% TODO: add visualisation for fat tail
\begin{frame}{Loi de Zipf : observations}
	\begin{figure}
		\tikzset{external/export=true}
		\begin{tikzpicture}
			\begin{axis}[
				xmin=0, ymin=0, xmax=20,
				width=0.9\textwidth,
				height=0.8\textheight,
				xlabel={Rang},
				ylabel={Fréquence d'apparition},
				title={Loi de Zipf : $y=\frac{K}{x}$},
				scaled ticks=false,
				ticklabel style={/pgf/number format/.cd, 1000 sep={\,}}
			]
				\addplot[highlighta, domain=1:20, samples=1001] {max(8000/x, 1)};
			\end{axis}
		\end{tikzpicture}
		\caption{Courbe d'une loi de Zipf ($K=\num{8000}$)}
	\end{figure}
\end{frame}

% TODO: add cites
\begin{frame}{Loi de Zipf : observations}
	\begin{itemize}
		\item La loi de Zipf se vérifie plutôt bien expérimentalement
		\item En pratique, les répartitions observées sont de la forme $f(n)=\frac{k}{n^s}$ avec $s$ proche de $1$
		\item Ce type de distribution, dites \emph{loi de puissance} est omniprésent en statistiques (en en sciences en général)
		\item Elles font parties des distributions dites \emph{à longue traîne} (\textit{fat tail})
		\item Une autre approche due à Mandelbrot présente une généralisation de la loi de Zipf comme le résultat de contraintes d'optimalité d'utilisation d'un canal de communication.
	\end{itemize}
\end{frame}

\begin{frame}{Loi de Heaps}
	Si la loi de Zipf donne une estimation de la répartition du vocabulaire d'un corpus, la loi de Heaps est une estimation de sa diversité
	\begin{block}{Hypothèse : loi de Heaps}
		Le nombre \(V\) de mots distincts dans sous-corpus de taille \(M\) vérifie
		\begin{equation}
			V = K×M^β
		\end{equation}
		Où \(K\) et \(β\) sont des paramètres dépendant du corpus
	\end{block}
	\begin{itemize}
		\item[→] Ce \(K\) n'est pas le même que pour la loi de Zipf, et \(β\) n'a aucun lien avec la pondération de \(F_β\)
		\item[→] On parle souvent de relation \emph{types-tokens}
		\item[→] Pour l'anglais, on a \(K∈[30, 100]\) et \(β≈0.5\)
		\item[→] Cette loi peut-elle être valable quelle que soit la taille de la collection ?
	\end{itemize}
\end{frame}

\begin{frame}[label=heapslawgraph]{Loi de Heaps}
	\begin{figure}
		\tikzset{external/export=true}
		\begin{tikzpicture}
			\begin{axis}[
				xmin=0, xmax=800000, ymin=0,
				width=0.9\textwidth,
				height=0.8\textheight,
				xlabel={Taille de la collection},
				ylabel={Taille du vocabulaire},
				title={Loi de Heaps : \(y=K×x^β\)},
				scaled ticks=false,
				ticklabel style={/pgf/number format/1000 sep={\,}},
				xticklabel style={rotate=45, anchor=north east, yshift=-0.1em},
			]
				\addplot[highlighta, domain=0:800000, samples=1001] {40*x^0.5};
			\end{axis}
		\end{tikzpicture}
		\caption{Courbe d'une loi de Heaps}
	\end{figure}
\end{frame}
%
% ███████ ███████  █████  ████████ ███████
% ██      ██      ██   ██    ██    ██
% █████   █████   ███████    ██    ███████
% ██      ██      ██   ██    ██         ██
% ██      ███████ ██   ██    ██    ███████


\subsection{Choix des attributs}
\begin{frame}{Choix des attributs}
	Pour pouvoir faire de la classification, il faut pouvoir mettre les données au bon format
	\begin{description}[<+->][*]
		\item[Objectif] Transformer les documents à traiter en fichiers tabulaires pour les passer à un classifieur
		\item[Problème] Quel choix d'attributs (colonnes) ?
		\item[Solution] Ça dépend des cas !
			\begin{itemize}
				\item[→] Mais on peut identifier plusieurs stratégies élémentaires (air connu)
			\end{itemize}
	\end{description}
\end{frame}

\begin{frame}{Caractères}
	Puisqu'un texte brut est une séquence de caractères : utilisons comme attribut la fréquence de chaque caractère

	\begin{figure}
		\begin{tabular}{l*{4}{S[table-format = 4.0]}}
			\toprule
			{\textbf{Texte}}   & {\textbf{a}} & {\textbf{b}} & {\textbf{c}} & {\textbf{…}}\\
			\midrule
			Ulysse  & 90578 & 17848 & 27902 & {…}\\
			A hat full of sky  & {…}\\
			\(⋮\)\\
			\bottomrule
		\end{tabular}
	\end{figure}

	\pause

	\begin{itemize}
		\item Combien d'attributs par texte ?
		\item Est-ce suffisant ?
			\begin{itemize}
				\item<+-> Suffisant pour quoi ?
			\end{itemize}
	\end{itemize}
\end{frame}

\begin{frame}{\(n\)-grammes de caractères}
	% TODO: add corrections
	% TODO: add sources
	% TODO: add examples
	Une stratégie un peu plus raffinée est de compter non pas les caractères, mais les séquences de caractères de longueur \(n\).

	Exemple : « Le petit chat est content »

	\begin{itemize}
		\item Quels sont les unigrammes ?
		\item Quels sont les bigrammes ?
		\item Quels sont les trigrammes ?
		\item En général, pour un alphabet à \(k\) caractères, combien y a-t-il de \(n\)-grammes possibles ?
	\end{itemize}

	En pratique, les trigrammes peuvent suffire pour certaines applications !
	\begin{itemize}
		\item Détection du plagiat
	\end{itemize}

	Pour plus de richesse, on peut aussi utiliser des \emph{skip-grams}
\end{frame}


%  █████  ██████  ██████  ███████ ███    ██ ██████  ██ ██   ██
% ██   ██ ██   ██ ██   ██ ██      ████   ██ ██   ██ ██  ██ ██
% ███████ ██████  ██████  █████   ██ ██  ██ ██   ██ ██   ███
% ██   ██ ██      ██      ██      ██  ██ ██ ██   ██ ██  ██ ██
% ██   ██ ██      ██      ███████ ██   ████ ██████  ██ ██   ██

\ifSubfilesClassLoaded{
	\appendix
	\pgfkeys{/metropolis/inner/sectionpage=simple}  % Avoid random errors with section page progressbar
	\section{Annexes}
	\pdfbookmark[2]{Remerciements}{acknowledgements}
	\begin{frame}{Remerciements}
		Ce cours a été construit à partir du polycopié de cours \citetitle{tellier2017IntroductionFouilleTextes} \parencite{tellier2017IntroductionFouilleTextes} et des précieux conseils d'Isabelle Tellier que je ne saurais trop remercier pour sa confiance et son dévouement.
	\end{frame}

	\pdfbookmark[2]{Références}{references}
	\begin{frame}[allowframebreaks]{References}
		\printbibliography[heading=none]
	\end{frame}

	\pdfbookmark[2]{Licence}{licence}
	\begin{frame}{Licence}
		\begin{center}
			{\huge \ccby}
			\vfill
			This document is available under the terms of the Creative Commons Attribution 4.0 International License (CC BY 4.0) (\shorturl{creativecommons.org/licenses/by/4.0})

			Exceptions to the above statement are listed at {\small\shorturl{loicgrobol.github.io/intro-fouille-textes\#licences}}
			\vfill
			© 2019, Loïc Grobol <\shorturl[mailto][:]{loic.grobol@gmail.com}>

			\shorturl[http]{lattice.cnrs.fr/Grobol-Loic}
		\end{center}
	\end{frame}
}{}

\end{document}
