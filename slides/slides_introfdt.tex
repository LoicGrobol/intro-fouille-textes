% Copyright © 2018, Loïc Grobol <loic.grobol@gmail.com>
% This document is available under the terms of the Creative Commons Attribution 4.0 International License (CC BY 4.0) (https://creativecommons.org/licenses/by/4.0/)

\RequirePackage{xparse}
% Settings
\NewDocumentCommand\myname{}{Loïc Grobol}
\NewDocumentCommand\mylab{}{Lattice, ALMAnaCH}
\NewDocumentCommand\pdftitle{}{Introduction à la fouille de textes}
\NewDocumentCommand\mymail{}{loic.grobol@gmail.com}
\NewDocumentCommand\titlepagetitle{}{\pdftitle}
\NewDocumentCommand\docdate{}{2018}
\NewDocumentCommand\conference{}{M1 Plurital}

\documentclass[hyperref={unicode}, xcolor={svgnames}]{beamer}
\usetheme[sectionpage=none,
          subsectionpage=progressbar,
          progressbar=frametitle]{metropolis}
    \definecolor{accent}{RGB}{51, 34, 136}
    \setbeamercolor{alerted text}{fg=accent}
    \makeatletter
        \setlength{\metropolis@progressinheadfoot@linewidth}{1pt}
    \makeatother
\usepackage{appendixnumberbeamer}

% Use non-standard fonts
\usefonttheme{professionalfonts}
\setsansfont[BoldFont={Fira Sans SemiBold}, ItalicFont={Fira Sans Book Italic}]{Fira Sans Book}

% Notes on left screen
% \usepackage{pgfpages}
% \setbeameroption{show notes on second screen=left}


\usepackage{polyglossia}
	\setmainlanguage{french}
    \setotherlanguage{english}

    % PATCH: fix spacing in monospaced text
    \usepackage{xpatch}
    \makeatletter
        \xapptocmd\ttfamily{\nofrench@punctuation}{}{}
    \makeatother

\usepackage[math-style=french]{unicode-math}

\usepackage{tabu}
	\usepackage{booktabs}
\usepackage{siunitx}
\usepackage{multicol}
\usepackage{ccicons}
\usepackage{bookmark}

\usepackage{csquotes}
\usepackage{tikz}
	\NewDocumentCommand{\textnode}{O{}mm}{\tikz[remember picture, baseline=(#2.base), inner sep = 0pt]{\node[#1] (#2) {#3};}}
	\tikzset{
		invisible/.style={opacity=0},
		visible on/.style=,
		alt/.code args={<#1>#2#3}{%
		\alt<#1>{\pgfkeysalso{#2}}{\pgfkeysalso{#3}} % \pgfkeysalso doesn't change the path
		},
	}
	\usepackage{tikz-qtree}
    \usepackage{tkz-graph}
	\usetikzlibrary{positioning, calc, shapes, shapes.geometric}

\usepackage{listings}
\renewcommand{\lstlistingname}{Example}

    \makeatletter
    	\lstset{
    		basicstyle=\ttfamily\lst@ifdisplaystyle\tiny\fi,
    		columns=fullflexible,
    		showstringspaces=false,
    		commentstyle=\color{gray}\upshape
    	}
    \makeatother

	\lstdefinelanguage{XML}{
	    morestring=[s]{"}{"},
	    morecomment=[s]{?}{?},
	    morecomment=[s]{!--}{--},
	    moredelim=[s][\color{black}]{>}{<},
	    moredelim=[s][\color{red}]{\ }{=},
	    stringstyle=\color{DarkBlue},
	    identifierstyle=\color{Maroon}
	}

\usepackage[style=authoryear, block=ragged, doi=false, isbn=false]{biblatex}
    \AtEveryBibitem{
        \ifentrytype{online}
        {} {
            \iffieldequalstr{howpublished}{online}
            {
                \clearfield{howpublished}
            } {
                \clearfield{urlyear}\clearfield{urlmonth}\clearfield{urlday}
            }
        }
    }

	\addbibresource{biblio.bib}

% Compact bibliography style
\setbeamertemplate{bibliography item}[text]

\AtEveryBibitem{
    \clearfield{series}
    \clearfield{pages}
    \clearlist{publisher}
    \clearname{editor}
    \clearlist{location}
}
\renewcommand*{\bibfont}{\tiny}

\usepackage{hyperxmp}	% XMP metadata

\usepackage[type={CC},modifier={by},version={4.0}]{doclicense}

\usepackage{todonotes}
\let\todox\todo
\renewcommand\todo[1]{\todox[inline]{#1}}

\title{\titlepagetitle}
\author{\textbf{\myname}}
\institute{}
\date{\conference\\\docdate}

\titlegraphic{\ccby}

% Tikz styles

\tikzset{
    data/.style={draw, ellipse},
    program/.style={draw, rectangle}
}


% ██████   ██████   ██████ ██    ██ ███    ███ ███████ ███    ██ ████████
% ██   ██ ██    ██ ██      ██    ██ ████  ████ ██      ████   ██    ██
% ██   ██ ██    ██ ██      ██    ██ ██ ████ ██ █████   ██ ██  ██    ██
% ██   ██ ██    ██ ██      ██    ██ ██  ██  ██ ██      ██  ██ ██    ██
% ██████   ██████   ██████  ██████  ██      ██ ███████ ██   ████    ██


\begin{document}
\pdfbookmark[2]{Title}{title}

\begin{frame}[plain]
	\titlepage
\end{frame}

\lecture{Introduction à la fouille de textes}{2018-01-15}
\begin{frame}{Informations pratiques}
    \begin{description}
        \item[Où] Salle Benveniste, ILPGA, 9 Rue des Bernardins, 75005 Paris
        \item[Quand] Le lundi de 14h à 16h (voir le calendrier de Paris 3 pour les dates)
        \item[Email] \href{mailto:loic.grobol@gmail.com}{\texttt{loic.grobol@gmail.com}}
        \item[Web] \url{https://loicgrobol.github.io/intro-fouille-textes/}
    \end{description}
\end{frame}

\begin{frame}{Organisation}
    \begin{description}
        \item[Poly] À lire d'une fois sur l'autre \url{http://www.lattice.cnrs.fr/sites/itellier/poly_fouille_textes/fouille-textes.pdf}
        \item[Cours] Exemples, exercices, expérimentations, explications
        \item[Évaluation] en contrôle continu
            \begin{description}
                \item[Partiel] en semaine 13 (2018-04-23) → 50\%
                \item[Projet] → 50\%
            \end{description}
    \end{description}
\end{frame}

\begin{frame}{Projet final}
    « Utiliser des programmes d'apprentissage automatique pour un tâche de classification de textes »
    \begin{itemize}
        \item Définir le sujet
        \item Constituer le corpus de travail
        \item Transformation du texte en données utilisables par Weka
        \item Expérimentations
        \item Compte-rendu
    \end{itemize}
\end{frame}

\begin{frame}{Weka ?}
    \begin{itemize}
        \item \url{https://www.cs.waikato.ac.nz/ml/weka}
        \item À installer dès que possible sur votre machine
        \item Si possible, venir avec votre PC portable pour pouvoir suivre les expérimentations en cours
    \end{itemize}
\end{frame}

% ██ ███    ██ ████████ ██████   ██████  ██████  ██    ██  ██████ ████████ ██  ██████  ███    ██
% ██ ████   ██    ██    ██   ██ ██    ██ ██   ██ ██    ██ ██         ██    ██ ██    ██ ████   ██
% ██ ██ ██  ██    ██    ██████  ██    ██ ██   ██ ██    ██ ██         ██    ██ ██    ██ ██ ██  ██
% ██ ██  ██ ██    ██    ██   ██ ██    ██ ██   ██ ██    ██ ██         ██    ██ ██    ██ ██  ██ ██
% ██ ██   ████    ██    ██   ██  ██████  ██████   ██████   ██████    ██    ██  ██████  ██   ████


\section*{Introduction}

\begin{frame}{La fouille de données}
    Tout commença dans les années 90…
    \begin{itemize}
        \item Généralisation des ordinateurs personnels
        \item Augmentation de leurs capacités de mémorisation et de traitement
        \item[→] Il devient possible de traiter rapidement de grandes quantités d'informations
    \end{itemize}
    Le terme \alert{fouille de données} désigne l'ensemble des techniques permettant de prendre des décisions pertinentes à partir de l'analyse de données massives.
\end{frame}

\begin{frame}{La fouille de données}
    Intéressant et rentable
    \begin{itemize}
        \item Pour les banques et les assurances
        \item Pour la médecine
        \item Pour la vente et le marketing
        \item …
    \end{itemize}
\end{frame}

\begin{frame}{Particularités innovantes}
    Par rapport à l'IA traditionnelle, on privilégie
    \begin{itemize}
        \item \alert{La masse de données} plutôt que les compétences expertes
        \item \alert{Les méthodes numériques} plutôt que symboliques
        \item Une démarche \alert{inductive} plutôt que déductive
    \end{itemize}
    Et cette tendance se généralise depuis à l'ensemble des domaines de l'informatique.

    À lire : \citetitle{church2011pendulum} \parencite{church2011pendulum}
\end{frame}

\begin{frame}{La fouille de textes}
    Applications de la fouille de données au TAL et vice-versa pour le traitement des données linguistiques massives.

    \begin{itemize}
        \item Le web 2.0 se démocratise au même moment
        \item C'est une formidable source de données linguistiques
        \item Mais elles sont en général spontanées et potentiellement mal formées
    \end{itemize}

    Et le TAL traditionnel, et en particulier
    \begin{itemize}
        \item Automates
        \item Grammaires formelles
        \item Représentations logiques
    \end{itemize}
    ne sont pas adapté⋅e⋅s.
\end{frame}

\begin{frame}{Un changement de paradigme}
    En fouille de textes, par rapport au TAL traditionnel
    \begin{itemize}
        \item On privilégie les analyses de surface
        \item On s'appuie sur la quantité des données pour compenser leur hétérogénéité
        \item On réduit les ambitions : pas question d'accéder au sens profond des textes
    \end{itemize}

    L'objectif est de traiter efficacement quelques \alert{tâches} précises et limitées.
\end{frame}

\begin{frame}{Sommaire}
    La suite de ce cours est articulée en deux parties
    \begin{itemize}
        \item Les tâches élémentaires de la fouille de textes
         \begin{itemize}
             \item Qu'est-ce qu'une tâche ?
             \item Quelles sont les tâches élémentaires ?
             \item Combiner des tâches élémentaires
         \end{itemize}
        \item Une tâche particulière : la classification de textes
    \end{itemize}
\end{frame}

% ████████  █████   ██████ ██   ██ ███████ ███████
%    ██    ██   ██ ██      ██   ██ ██      ██
%    ██    ███████ ██      ███████ █████   ███████
%    ██    ██   ██ ██      ██   ██ ██           ██
%    ██    ██   ██  ██████ ██   ██ ███████ ███████

\section{Les tâches élémentaires de la fouille de textes}

\subsection{Notion de tâche}

\begin{frame}{Introduction}
    On traite du concept de \textit{tâche} au sens de l'informatique : spécification d'un programme qui mime une compétence humaine précise.

    \begin{itemize}
        \item Définition très large
            \begin{itemize}
                \item[→] « Tenir une conversation par écrit en se faisant passer pour un être humain » convient
            \end{itemize}
        \item En pratique on a des ambitions plus modestes
        \item Mais on est encore loin de les satisfaire !
         \begin{itemize}
             \item Notre mesure du succès est « s'approcher le plus possible d'une solution de référence validée par un humain »
         \end{itemize}
    \end{itemize}
 \end{frame}

\begin{frame}{Caractérisation}
    \begin{itemize}
        \item Chaque tâche a une visée applicative précise et autonome
        \item Elle est caractérisée par ses entrées, ses sorties et les ressources auxquelles elle fait appel
    \end{itemize}

    \begin{figure}
    \begin{tikzpicture}[>=stealth]
        \node[data] (in) {Entrée};
        \node[program, right=1cm of in, text width=10ex] (prog) {Programme réalisant la tâche};
        \node[data, right=1.5cm of prog] (out) {Résultat};
        \node[data, below=1cm of prog] (res) {Ressources};
        \draw[->] (in) -- (prog);
        \draw[->] (prog) -- (out);
        \draw[->] (res) -- (prog);
    \end{tikzpicture}
    \end{figure}

    On peut considérer chaque tâche comme une boîte noire : la façon dont le programme fonctionne ne nous intéresse pas ici.
\end{frame}

\begin{frame}[fragile]{Schéma général d'une tâche}
    \begin{figure}
    \begin{tikzpicture}[>=stealth]
        \tikzset{accent on/.style={alt={#1{draw=accent, text=accent, line width=2pt}{draw}}}}
        \node[ellipse, accent on=<2>] (in) {Entrée};
        \node[right=1cm of in, text width=10ex, accent on=<3>] (prog) {Programme réalisant la tâche};
        \node[ellipse, accent on=<2>, right=1.5cm of prog] (out) {Résultat};
        \node[ellipse, accent on=<2>, below=1cm of prog] (res) {Ressources};
        \draw[->] (in) -- (prog);
        \draw[->] (prog) -- (out);
        \draw[->] (res) -- (prog);
    \end{tikzpicture}
    \caption{Schéma général d'une tâche}
    \end{figure}
    \only<2->{
    Dans ce schéma
        \begin{itemize}
            \item<2-|alert@2> Les données sont figurées par des ellipses.
            \item<3-|alert@3> Les programmes sont figurés par des rectangles.
        \end{itemize}
    }
\end{frame}

%  █████  ██████  ██████  ███████ ███    ██ ██████  ██ ██   ██
% ██   ██ ██   ██ ██   ██ ██      ████   ██ ██   ██ ██  ██ ██
% ███████ ██████  ██████  █████   ██ ██  ██ ██   ██ ██   ███
% ██   ██ ██      ██      ██      ██  ██ ██ ██   ██ ██  ██ ██
% ██   ██ ██      ██      ███████ ██   ████ ██████  ██ ██   ██

\appendix
\section{Annexes}
\pdfbookmark[2]{Remerciements}{acknowledgements}
\begin{frame}{Remerciements}
    Ce cours a été construit à partir du polycopié de cours \citetitle{tellier2017fouille} \parencite{tellier2017fouille} et des précieux conseils d'Isabelle Tellier que je ne saurais trop remercier pour sa confiance et son dévouement.
\end{frame}
\pdfbookmark[2]{Références}{references}
\begin{frame}[allowframebreaks]{References}
    \printbibliography[heading=none]
\end{frame}

\pdfbookmark[2]{Licence}{licence}
\begin{frame}{Licence}
  \begin{center}
	  {\huge \ccby}
	  \vfill
	  This document is available under the terms of the Creative Commons Attribution 4.0 International License (CC BY 4.0) (\url{https://creativecommons.org/licenses/by/4.0/})
	  \vfill
	  Loïc Grobol

	  \url{http://lattice.cnrs.fr/Grobol-Loic}, \href{mailto:loic.grobol@gmail.com}{\texttt{loic.grobol@gmail.com}}
  \end{center}
\end{frame}

\end{document}
