% Copyright © 2018, Loïc Grobol <loic.grobol@gmail.com>
% This document is available under the terms of the Creative Commons Attribution 4.0 International License (CC BY 4.0) (https://creativecommons.org/licenses/by/4.0/)

\RequirePackage{xparse}
\RequirePackage{shellesc}
% Settings
\NewDocumentCommand\myname{}{Loïc Grobol}
\NewDocumentCommand\mylab{}{Lattice / ALMAnaCH}
\NewDocumentCommand\pdftitle{}{Introduction à la fouille de textes}
\NewDocumentCommand\mymail{}{loic.grobol@gmail.com}
\NewDocumentCommand\titlepagetitle{}{\pdftitle}
\NewDocumentCommand\titlepagesubtitle{}{Cours 8 : arbres de décision}
\NewDocumentCommand\docdate{}{2019-03-11}
\NewDocumentCommand\conference{}{M1 Plurital}

\documentclass[hyperref={unicode}, xcolor={svgnames}, french]{beamer}
% Alternative navy blue for ⩽4 palettes
\definecolor{highlighta}{RGB}{68, 118, 170}  % Navy blue

% Colour palette from [Paul Tol's technical note](https://personal.sron.nl/~pault/data/colourschemes.pdf) v3.1
% Bright scheme
\definecolor{sronbrighttblue}{RGB}{68, 119, 170}
\definecolor{sronbrightcyan}{RGB}{102, 204, 238}
\definecolor{sronbrightgreen}{RGB}{34, 136, 51}
\definecolor{sronbrightyellow}{RGB}{204, 187, 68}
\definecolor{sronbrightred}{RGB}{238, 102, 119}
\definecolor{sronbrightpurple}{RGB}{170, 51, 119}
\definecolor{sronbrightgrey}{RGB}{187, 187, 187}

\colorlet{highlight0}{sronbrighttblue}
\colorlet{highlight1}{sronbrightcyan}
\colorlet{highlight2}{sronbrightgreen}
\colorlet{highlight3}{sronbrightyellow}
\colorlet{highlight4}{sronbrightred}
\colorlet{highlight5}{sronbrightpurple}
\colorlet{highlight6}{sronbrightgrey}
% Legacy highlights
\definecolor{highlight7}{RGB}{136, 34, 85}    % Purple
\definecolor{highlight8}{RGB}{170, 68, 153}   % Violet


\usetheme[
	sectionpage=progressbar,
    subsectionpage=progressbar,
	progressbar=frametitle,
]{metropolis}
	\colorlet{accent}{sronbrightgreen}
	\setbeamercolor{frametitle}{bg=DarkSlateGrey}
	\setbeamercolor{alerted text}{fg=accent}
	\makeatletter
		\setlength{\metropolis@progressinheadfoot@linewidth}{2pt}
	\makeatother
	% Avoid ugly whitespace below figures
	% \makeatletter
	% 	\renewenvironment{figure}[1][]{%
	% 	\def\@captype{figure}%
	% 	\par\centering}%
	% 	{\par}
	% \makeatother
	% FIXME: not pretty and footnotes are still too big
	\let\footnoterule\relax  % No footnote rule, push down footnote


% Use non-standard fonts
\usefonttheme{professionalfonts}
\setsansfont[BoldFont={Fira Sans SemiBold}, ItalicFont={Fira Sans Book Italic}]{Fira Sans Book}
\setmonofont[Scale=0.9]{Fira Mono}

% Fix missing glyphs in Fira by delegating to polyglossia/babel
\usepackage{newunicodechar}
\newunicodechar{ }{~}   % U+202F NARROW NO-BREAK SPACE
\newunicodechar{ }{ }  % U+2009 THIN SPACE

% Notes on left screen
% \usepackage{pgfpages}
% \setbeameroption{show notes on second screen=left}


\usepackage[french, english]{babel}
\usepackage{amsfonts,amssymb}
\usepackage{amsmath,amsthm}
\usepackage{mathtools}	% AMS Maths service pack
	\newtagform{brackets}{[}{]}	% Pour des lignes d'équation numérotées entre crochets
	\mathtoolsset{showonlyrefs, showmanualtags, mathic}	% affiche les tags manuels (\tag et \tag*) et corrige le kerning des maths inline dans un bloc italique voir la doc de mathtools
	\usetagform{brackets}	% Utilise le style de tags défini plus haut
\usepackage{lualatex-math}

\usepackage[math-style=french]{unicode-math}
	\setmathfont{Libertinus Math}
\usepackage{newunicodechar}
	\newunicodechar{√}{\sqrt}
\usepackage{mleftright}

\usepackage{tabu}
\usepackage{booktabs}
\usepackage{siunitx}
\usepackage{multicol}
\usepackage{ccicons}
\usepackage{bookmark}
\usepackage{caption}
    \captionsetup{skip=1ex}
\usepackage[iso]{datetime}

\usepackage{tikz}
	\NewDocumentCommand{\textnode}{O{}mm}{\tikz[remember picture, baseline=(#2.base), inner sep=0pt]{\node[#1] (#2) {#3};}}
    \NewDocumentCommand{\mathnode}{O{}mm}{\tikz[remember picture, baseline=(#2.base), inner sep = 0pt]{\node[#1] (#2) {$\displaystyle #3$};}}
	\tikzset{
		alt/.code args={<#1>#2#3}{%
		\alt<#1>{\pgfkeysalso{#2}}{\pgfkeysalso{#3}} % \pgfkeysalso doesn't change the path
		},
        invisible/.style={opacity=0, fill opacity=0},
		visible on/.style={alt={<#1>{}{invisible}}}
	}
    \usepackage{forest}
    \usepackage{tkz-graph}
    \usepackage[beamer, markings]{hf-tikz}
    \usepackage{tikz-3dplot}
    \usepackage{pgfplots}
        % Due to pgfplots meddling with pgfkeys, we have to redefine alt here.
        \pgfplotsset{
    		alt/.code args={<#1>#2#3}{%
    		\alt<#1>{\pgfkeysalso{#2}}{\pgfkeysalso{#3}} % \pgfkeysalso doesn't change the path
    		},
    	}
        \pgfplotsset{compat=1.15}
        \pgfplotsset{colormap={SRON}{rgb255=(61,82,161) rgb255=(255,250,210) rgb255=(174,28,62)}}

    \usetikzlibrary{matrix}
    \usetikzlibrary{shapes, shapes.geometric}
    \usetikzlibrary{decorations.pathreplacing}
	\usetikzlibrary{positioning, calc, intersections}
    \usetikzlibrary{fit}
    \usetikzlibrary{backgrounds}

    % Do evil things with soft path
    % From <https://tex.stackexchange.com/a/301364/8547>
    \makeatletter
        \def\@appendnamedsoftpath#1{%
            \pgfsyssoftpath@getcurrentpath\@temppatha
            \expandafter\let\expandafter\@temppathb\csname tikz@intersect@path@name@#1\endcsname
            \expandafter\expandafter\expandafter\def\expandafter\expandafter\expandafter\@temppatha\expandafter\expandafter\expandafter{\expandafter\@temppatha\@temppathb}%
            \pgfsyssoftpath@setcurrentpath\@temppatha
        }
        \def\@appendnamedpathforactions#1{%
            \pgfsyssoftpath@getcurrentpath\@temppatha
            \expandafter\let\expandafter\@temppathb\csname tikz@intersect@path@name@#1\endcsname
            \expandafter\def\expandafter\@temppatha\expandafter{\csname @temppatha\expandafter\endcsname\@temppathb}%\usepackage{tikz-3dplot}

            \let\tikz@actions@path\@temppatha
        }

        \tikzset{
            use path for main/.code={%
                \tikz@addmode{%
                    \expandafter\pgfsyssoftpath@setcurrentpath\csname tikz@intersect@path@name@#1\endcsname
                }%
            },
            append path for main/.code={%
                \tikz@addmode{%
                    \@appendnamedsoftpath{#1}%
                }%
            },
            use path for actions/.code={%
                \expandafter\def\expandafter\tikz@preactions\expandafter{\tikz@preactions\expandafter\let\expandafter\tikz@actions@path\csname tikz@intersect@path@name@#1\endcsname}%
            },
            append path for actions/.code={%
                \expandafter\def\expandafter\tikz@preactions\expandafter{\tikz@preactions
                \@appendnamedpathforactions{#1}}%
            },
            use path/.style={%
                use path for main=#1,
                use path for actions=#1,
            },
            append path/.style={%
                append path for main=#1,
                append path for actions=#1
            }
        }
    \makeatother

    % TikZ externalisation
    \usetikzlibrary{external}
    % Create the `tikzpics/` folder if it does not exist
    \ShellEscape{mkdir -p tikzpics}
    % Only externalise pictures on demand (to avoid messing up with metropolis theme)
    \tikzset{
        external/export=false,
        external/prefix=tikzpics/
    }
    \tikzexternalize

\usepackage{minted}
	\usemintedstyle{lovelace}
	\setminted{autogobble, fontsize=\scriptsize, tabsize=2}
	\setmintedinline{fontsize=auto}

\usepackage{csquotes}

\usepackage[style=authoryear, block=ragged, doi=false, isbn=false]{biblatex}
    \AtEveryBibitem{
        \ifentrytype{online}
        {} {
            \iffieldequalstr{howpublished}{online}
            {
                \clearfield{howpublished}
            } {
                \clearfield{urlyear}\clearfield{urlmonth}\clearfield{urlday}
            }
        }
    }

	\addbibresource{biblio.bib}

% Compact bibliography style
\setbeamertemplate{bibliography item}[text]

\AtEveryBibitem{
    \clearfield{series}
    \clearfield{pages}
    \clearlist{publisher}
    \clearname{editor}
    \clearlist{location}
}
\renewcommand*{\bibfont}{\tiny}

\usepackage{hyperxmp}	% XMP metadata

\usepackage[type={CC}, modifier={by}, version={4.0}]{doclicense}

\usepackage{todonotes}
\let\todox\todo
\renewcommand\todo[1]{\todox[inline]{#1}}

\title{\titlepagetitle}
\subtitle{\titlepagesubtitle}
\author{\textbf{\myname} (\mylab)}
\institute{}
\date{\tiny Version {\yyyymmdddate\today}T\currenttime}

\titlegraphic{\ccby}

% Tikz styles

% Schémas de tâches
\tikzset{
    >=stealth,
    hair lines/.style={line width = 0.05pt, lightgray},
    data/.style={draw, ellipse},
    program/.style={draw, rectangle},
    accent on/.style={alt={<#1>{draw=accent, text=accent, thick}{draw}}},
    true scale/.style={scale=#1, every node/.style={transform shape}},
    highlight/.style={color=highlight#1}
}

% Styles des heatmap pour les moyennes
\pgfplotsset{
    meanheatmap/.style={
        colorbar, colormap name=SRON,
        view={0}{90},
        samples=100,
        domain=0:1,
        min=0, max=1,
        xlabel={$P$},
        ylabel={$R$},
    }
}

% Commands spécifiques
\NewDocumentCommand\shorturl{ O{https} O{://} m }{%
    \href{#1#2#3}{\nolinkurl{#3}}%
}

\DeclarePairedDelimiter\norm{\lVert}{\rVert}
\DeclarePairedDelimiter\abs{\lvert}{\rvert}
\DeclarePairedDelimiterX\compset[2]{\lbrace}{\rbrace}{#1\,\delimsize|\,#2}
\DeclarePairedDelimiterX\innprod[2]{\langle}{\rangle}{#1\,\delimsize|\,#2}

\DeclareMathOperator*\argmax{argmax}

% Easy column vectors \vcord{a,b,c} ou \vcord[;]{a;b;c}
% Here be black magic
\ExplSyntaxOn
	\NewDocumentCommand{\vcord}{O{,}m}{\vector_main:nnnn{p}{\\}{#1}{#2}}
	\NewDocumentCommand{\tvcord}{O{,}m}{\vector_main:nnnn{psmall}{\\}{#1}{#2}}
	\seq_new:N\l__vector_arg_seq
	\cs_new_protected:Npn\vector_main:nnnn #1 #2 #3 #4{
		\seq_set_split:Nnn\l__vector_arg_seq{#3}{#4}
		\begin{#1matrix}
			\seq_use:Nnnn\l__vector_arg_seq{#2}{#2}{#2}
		\end{#1matrix}
	}
\ExplSyntaxOff

\DeclareMathOperator{\TF}{TF}
\DeclareMathOperator{\IDF}{IDF}

\ExplSyntaxOn
    \DeclareExpandableDocumentCommand\eval{m}{\fp_eval:n{#1}}
\ExplSyntaxOff


% ██████   ██████   ██████ ██    ██ ███    ███ ███████ ███    ██ ████████
% ██   ██ ██    ██ ██      ██    ██ ████  ████ ██      ████   ██    ██
% ██   ██ ██    ██ ██      ██    ██ ██ ████ ██ █████   ██ ██  ██    ██
% ██   ██ ██    ██ ██      ██    ██ ██  ██  ██ ██      ██  ██ ██    ██
% ██████   ██████   ██████  ██████  ██      ██ ███████ ██   ████    ██


\begin{document}
\pdfbookmark[2]{Title}{title}

\begin{frame}[plain]
	\titlepage
\end{frame}

\begin{frame}{Exemples de travail}
	Mettre le corpus suivant sous forme tabulaire, en utilisant comme attributs les occurrences de noms ayant trait au \emph{cinéma} ou à l'\emph{économie} puis le représenter dans le plan
	{\small
		\begin{enumerate}
			\item « Le cinéma est un art, c’est aussi une industrie. »
			\item « Personne, quand il est petit, ne veut être critique de cinéma. Mais ensuite, en France, tout le monde a un deuxième métier : critique de cinéma ! »
			\item « Tout le monde a des rêves de Hollywood. »
			\item « Pendant la crise, l’usine à rêves Hollywood critique le cynisme de l’industrie. »
			\item « C’est la crise, l’économie de la France est menacée par la mondialisation. »
			\item « En temps de crise, reconstruire l’industrie : tout un art ! »
			\item « Quand une usine ferme, c’est que l’économie va mal. »
		\end{enumerate}
	}
\end{frame}


\section{Arbres de décision}
\begin{frame}{Arbres de décision}
    \textbf{Dans Weka} trees (> J48)

    \textbf{Espace de recherche} L'ensemble des arbres de recherche pour les attributs choisis

    % TODO: cite
    \textbf{Techniques de recherche} Plusieurs algorithmes, le plus connu étant C4.5, appelé J48 dans Weka
\end{frame}

\begin{frame}{Arbre de décision}
    Modèle de prise de décision déterministe par succession de choix.
    \begin{figure}
        \tikzset{external/export=true}
        \begin{forest}
            for tree={draw, l sep=2em, s sep=3em}
            [outlook,ellipse
                [humidity,ellipse,edge label={node[midway,left] {=sunny}}
                    [yes,edge label={node[midway,left] {$⩽75$}}]
                    [no,edge label={node[midway,right] {$>75$}}]
                ]
                [yes,edge label={node[midway] {=overcast}}]
                [windy,ellipse,edge label={node[midway,right] {=rainy}}
                    [yes,edge label={node[midway,left] {=true}}]
                    [no,edge label={node[midway,right] {=false}}]
                ]
            ]
        \end{forest}
        \caption{Arbre de décision pour \emph{weather}}
    \end{figure}
\end{frame}

\begin{frame}[fragile]{Arbre de décision}
    En Python
    \begin{figure}
        \begin{minted}{python}
            def classify(outlook, humidity, windy):
                if outlook == "sunny":
                    if humidity <= 75:
                        return "yes"
                    else:
                        return "no"
                elif outlook == "overcast":
                    return "yes"
                elif outlook == "rainy":
                    if windy:
                        return "yes"
                    else:
                        return "no"
        \end{minted}
        \caption{Code correspondant à l'arbre précédent}
    \end{figure}
\end{frame}

\begin{frame}{Technique d'apprentissage}
    C'est assez facile de construire un arbre parfait pour l'ensemble d'entraînement:

    \begin{itemize}
        \item<+-> Il suffit d'énumérer tous les attributs jusqu'à avoir généré toutes les combinaisons existantes
        \item<+->[→] Surapprentissage !
    \end{itemize}
    Comment faire pour avoir un arbre bon, mais pas trop profond ?
    \pause
    \begin{itemize}
        \item Faire en sorte de trier vite et bien
        \item[→] En choisissant les attributs les plus discriminants
    \end{itemize}
    L'idée est donc de construire l'arbre progressivement, prenant à chaque étape le test le plus \alert{discriminant}, reste à savoir comment le détermine.
\end{frame}

% FIXME: Review this whole part : start by growing the decision tree for the weather dataset by hand
% thus motivating both the Gini indice definition and the general procedure and only formalize
% things after. Plan on using real attribute names and values instead of abstract ones, use concrete
% examples EVERYWHERE and add a lot of figures (possibly animated) e.g. the change S→S_{a,v}
% graphically. Only introduce entropy after all that
\begin{frame}{Indice de diversité de Gini}
    \begin{block}{Définition}
        On appelle \emph{indice de diversité de Gini} d'une partition $S=⨆_{1⩽i⩽n}c_i$
        \begin{equation}
            Gini(S) = ∑_{1⩽i⩽n}p_i(1-p_i)
        \end{equation}
        avec $p_i=\frac{\#c_i}{∑_{j=1}^n\#c_j}$
    \end{block}
    Autrement dit, l'indice de diversité de Gini est la probabilité qu'un exemple choisi au hasard et classé au hasard soit mal classé.
\end{frame}

\begin{frame}{Indice de diversité de Gini}
    \begin{figure}
        \tikzset{external/export=true}
        \begin{tikzpicture}
            \begin{axis}[
				xmin=0, xmax=1,
				ymin=0,
				xlabel={$p_1$},
				ylabel={$Gini(S)$},
				width=0.9\textwidth,
				height=0.8\textheight,
				title={Indice de diversité de Gini : $y=x(1-x)+(1-x)x$},
				scaled ticks=false,
			]
                \addplot[highlighta, domain=0:1, samples=1001] {x*(1-x)+(1-x)*x};
            \end{axis}
        \end{tikzpicture}
        \caption{Indice de diversité de Gini pour un problème à deux classes}
    \end{figure}
\end{frame}

\begin{frame}{Entropie}
    \begin{block}{Définition}
        On appelle \emph{entropie} d'une partition $S=⨆_{1⩽i⩽n}c_i$
        \begin{equation}
            H(S) = -∑_{1⩽i⩽n}p_i\log₂(p_i)
        \end{equation}
        avec $p_i=\frac{\#c_i}{∑_{j=1}^n\#c_j}$
    \end{block}
    Intuitivement, si on choisit au hasard et de façon uniforme un exemple $x$ dans $S$
    \begin{itemize}
        \item $p_i$ est la probabilité de l'évènement « La classe de $x$ est $c_i$ »
        \item $-\log₂p_i$ mesure la surprise de l'évènement « La classe de $x$ est $c_i$ »
    \end{itemize}
    Finalement, $H$ est donc la surprise moyenne de $S$.
\end{frame}

\begin{frame}{Indice de diversité de Gini}
    \begin{figure}
        \tikzset{external/export=true}
        \begin{tikzpicture}
            \begin{axis}[xmin=0, xmax=1, ymin=0,
                         width=0.9\textwidth,
                         height=0.8\textheight,
                         xlabel={$p_1$},
                         ylabel={$H(S)$},
                         title={Entropie : $y=-x\log₂(x)-(1-x)\log₂(1-x)$},
                         scaled ticks=false,]
                \addplot[highlighta, domain=0:1, samples=1001] {-x*ln(x)/ln(2)-(1-x)*ln(1-x)/ln(2)};
            \end{axis}
        \end{tikzpicture}
        \caption{Entropie pour un problème à deux classes}
    \end{figure}
\end{frame}

\begin{frame}{Choix des attributs}
    On peut en déduire une valuation de « être un attribut discriminant » :
    \begin{itemize}
         \item Soit un attribut $a$ à valeurs discrètes
         \item Pour toute valeur $v$ prise par $a$, on note $S_{a=v}$ l'ensemble des éléments de $S$ pour lesquels $a$ vaut $v$.
    \end{itemize}
    On définit lors le gain associé à $a$ par
    \begin{equation}
        g(S, a) = Gini(S) - ∑_i\frac{\#S_{a=v}}{\#S}Gini(S_{a=v})
    \end{equation}
    \begin{itemize}
        \item Un attribut est d'autant plus discriminant que son gain est élevé.
        \item On peut procéder de même avec $H$ (ou un autre indice de diversité)
    \end{itemize}
\end{frame}

\begin{frame}{Attributs à valeurs scalaires}
    Pour des attributs numériques, on se ramène à un choix discret en utilisant des seuils
    \begin{itemize}
        \item Pour un attribut $a$ à valeur numérique $∈[α, β]$ et $r∈[α, β]$, on note $a_r$ l'attribut booléen $a(x)⩽r$
        \item On choisit $s$ tel que $g(S, a_r)$ soit maximal
    \end{itemize}
    On peut ensuite utiliser $a_r$ au lieu de $a$ dans notre choix d'attributs.
\end{frame}

\begin{frame}[fragile]{Représentation graphique des seuils}
    \begin{figure}
        \tikzset{external/export=true}
        \begin{tikzpicture}
            \draw [help lines, xstep=1cm, ystep=1cm] (0, 0) grid (5.25, 4.25);

            \draw[->] (-0.5,0) -- (5.25, 0);
            \foreach \x in {1,...,5}
                \draw[shift={(\x, 0)}] (0pt,2pt) -- (0pt,-2pt) node[below] {\footnotesize $\x$};
            \draw[->] (0, -0.5) -- (0, 4.25);

            \foreach \y in {1,...,4}
                \draw[shift={(0,\y)}] (2pt,0pt) -- (-2pt,0pt) node[left] {\footnotesize $\y$};

            \draw (0, 0) node[below left] {\footnotesize $0$};

            \foreach \x/\y/\c in {2/1/highlight4,4/0/highlight4,1/0/highlight4,2/3/highlight4,0/3/highlighta,1/2/highlighta,0/2/highlighta}{
                \path[fill=\c] (\x, \y) circle[radius=3pt];
            }

            \draw[highlight3, line width=2pt, alt={<2,4>{}{invisible}}] (-0.5, 1.5) -- (5.25, 1.5);
            \draw[highlight2, line width=2pt, visible on=3] (1.5, -0.5) -- (1.5, 4.25);
            \draw[highlight2, line width=2pt, visible on=4] (1.5, 1.5) -- (1.5, 4.25);
        \end{tikzpicture}
        \caption{Exemple de classification par seuils}
    \end{figure}
    \only<4>{}
\end{frame}

\begin{frame}{Propriétés}
    La principale qualité des arbres de décision tient dans leur simplicité
    \begin{itemize}
        \item \emph{White box} : on peut comprendre le résultat
        \item Le modèle est de petite taille et est efficace même avec peu de données
        \item La procédure d'apprentissage est assez proche du raisonnement humain conscient
        \item Facilement combinables
    \end{itemize}
    Leur principal défaut est leur instabilité
    \begin{itemize}
        \item[→] des petites variations dans l'ensemble de test peuvent conduire à des changements importants dans l'arbre
    \end{itemize}
    Des extensions plus performantes, mais moins simples existent (\emph{random forest}…)
\end{frame}


%  █████  ██████  ██████  ███████ ███    ██ ██████  ██ ██   ██
% ██   ██ ██   ██ ██   ██ ██      ████   ██ ██   ██ ██  ██ ██
% ███████ ██████  ██████  █████   ██ ██  ██ ██   ██ ██   ███
% ██   ██ ██      ██      ██      ██  ██ ██ ██   ██ ██  ██ ██
% ██   ██ ██      ██      ███████ ██   ████ ██████  ██ ██   ██

\appendix
\pgfkeys{/metropolis/inner/sectionpage=simple}  % Avoid random errors with section page progressbar
\section{Annexes}
\pdfbookmark[2]{Remerciements}{acknowledgements}
\begin{frame}{Remerciements}
    Ce cours a été construit à partir du polycopié de cours \citetitle{tellier2017fouille} \parencite{tellier2017fouille} et des précieux conseils d'Isabelle Tellier que je ne saurais trop remercier pour sa confiance et son dévouement.
\end{frame}

\pdfbookmark[2]{Références}{references}
\begin{frame}[allowframebreaks]{References}
    \printbibliography[heading=none]
\end{frame}

\pdfbookmark[2]{Licence}{licence}
\begin{frame}{Licence}
    \begin{center}
        {\huge \ccby}
        \vfill
        This document is available under the terms of the Creative Commons Attribution 4.0 International License (CC BY 4.0) (\shorturl{creativecommons.org/licenses/by/4.0})

        Exceptions to the above statement are listed at {\small\shorturl{loicgrobol.github.io/intro-fouille-textes\#licences}}
        \vfill
        © 2019, Loïc Grobol <\shorturl[mailto][:]{loic.grobol@gmail.com}>

        \shorturl[http]{lattice.cnrs.fr/Grobol-Loic}
    \end{center}
\end{frame}

\end{document}
