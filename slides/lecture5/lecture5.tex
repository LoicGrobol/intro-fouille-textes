% Copyright © 2018, Loïc Grobol <loic.grobol@gmail.com>
% This document is available under the terms of the Creative Commons Attribution 4.0 International License (CC BY 4.0) (https://creativecommons.org/licenses/by/4.0/)

\RequirePackage{xparse}
\RequirePackage{shellesc}
% Settings
\NewDocumentCommand\myname{}{Loïc Grobol}
\NewDocumentCommand\mylab{}{Lattice / ALMAnaCH}
\NewDocumentCommand\pdftitle{}{Introduction à la fouille de textes}
\NewDocumentCommand\mymail{}{loic.grobol@gmail.com}
\NewDocumentCommand\titlepagetitle{}{\pdftitle}
\NewDocumentCommand\titlepagesubtitle{}{Cours 5 : représentation des données}
\NewDocumentCommand\docdate{}{2019-02-18}
\NewDocumentCommand\conference{}{M1 Plurital}

\documentclass[hyperref={unicode}, xcolor={svgnames}, french]{beamer}
% Alternative navy blue for ⩽4 palettes
\definecolor{highlighta}{RGB}{68, 118, 170}  % Navy blue

% Colour palette from [Paul Tol's technical note](https://personal.sron.nl/~pault/data/colourschemes.pdf) v3.1
% Bright scheme
\definecolor{sronbrighttblue}{RGB}{68, 119, 170}
\definecolor{sronbrightcyan}{RGB}{102, 204, 238}
\definecolor{sronbrightgreen}{RGB}{34, 136, 51}
\definecolor{sronbrightyellow}{RGB}{204, 187, 68}
\definecolor{sronbrightred}{RGB}{238, 102, 119}
\definecolor{sronbrightpurple}{RGB}{170, 51, 119}
\definecolor{sronbrightgrey}{RGB}{187, 187, 187}

\colorlet{highlight0}{sronbrighttblue}
\colorlet{highlight1}{sronbrightcyan}
\colorlet{highlight2}{sronbrightgreen}
\colorlet{highlight3}{sronbrightyellow}
\colorlet{highlight4}{sronbrightred}
\colorlet{highlight5}{sronbrightpurple}
\colorlet{highlight6}{sronbrightgrey}
% Legacy highlights
\definecolor{highlight7}{RGB}{136, 34, 85}    % Purple
\definecolor{highlight8}{RGB}{170, 68, 153}   % Violet


\usetheme[
	sectionpage=progressbar,
    subsectionpage=progressbar,
	progressbar=frametitle,
]{metropolis}
	\colorlet{accent}{sronbrightgreen}
	\setbeamercolor{frametitle}{bg=DarkSlateGrey}
	\setbeamercolor{alerted text}{fg=accent}
	\makeatletter
		\setlength{\metropolis@progressinheadfoot@linewidth}{2pt}
	\makeatother
	% Avoid ugly whitespace below figures
	% \makeatletter
	% 	\renewenvironment{figure}[1][]{%
	% 	\def\@captype{figure}%
	% 	\par\centering}%
	% 	{\par}
	% \makeatother
	% FIXME: not pretty and footnotes are still too big
	\let\footnoterule\relax  % No footnote rule, push down footnote


% Use non-standard fonts
\usefonttheme{professionalfonts}
\setsansfont[BoldFont={Fira Sans SemiBold}, ItalicFont={Fira Sans Book Italic}]{Fira Sans Book}
\setmonofont[Scale=0.9]{Fira Mono}

% Fix missing glyphs in Fira by delegating to polyglossia/babel
\usepackage{newunicodechar}
\newunicodechar{ }{~}   % U+202F NARROW NO-BREAK SPACE
\newunicodechar{ }{ }  % U+2009 THIN SPACE

% Notes on left screen
% \usepackage{pgfpages}
% \setbeameroption{show notes on second screen=left}


\usepackage[french, english]{babel}
\usepackage{amsfonts,amssymb}
\usepackage{amsmath,amsthm}
\usepackage{mathtools}	% AMS Maths service pack
	\newtagform{brackets}{[}{]}	% Pour des lignes d'équation numérotées entre crochets
	\mathtoolsset{showonlyrefs, showmanualtags, mathic}	% affiche les tags manuels (\tag et \tag*) et corrige le kerning des maths inline dans un bloc italique voir la doc de mathtools
	\usetagform{brackets}	% Utilise le style de tags défini plus haut
\usepackage{lualatex-math}

\usepackage[math-style=french]{unicode-math}
	\setmathfont{Libertinus Math}
\usepackage{newunicodechar}
	\newunicodechar{√}{\sqrt}
\usepackage{mleftright}


\usepackage{tabu}
\usepackage{booktabs}
\usepackage{siunitx}
\usepackage{multicol}
\usepackage{ccicons}
\usepackage{bookmark}
\usepackage{caption}
    \captionsetup{skip=1ex}
\usepackage[iso]{datetime}

\usepackage{csquotes}
\usepackage{tikz}
	\NewDocumentCommand{\textnode}{O{}mm}{\tikz[remember picture, baseline=(#2.base), inner sep=0pt]{\node[#1] (#2) {#3};}}
    \NewDocumentCommand{\mathnode}{O{}mm}{\tikz[remember picture, baseline=(#2.base), inner sep = 0pt]{\node[#1] (#2) {$\displaystyle #3$};}}
	\tikzset{
		alt/.code args={<#1>#2#3}{%
		\alt<#1>{\pgfkeysalso{#2}}{\pgfkeysalso{#3}} % \pgfkeysalso doesn't change the path
		},
        invisible/.style={opacity=0, fill opacity=0},
		visible on/.style={alt={<#1>{}{invisible}}}
	}
    \usepackage{forest}
    \usepackage{tkz-graph}
    \usepackage[beamer, markings]{hf-tikz}
    \usepackage{tikz-3dplot}
    \usepackage{pgfplots}
        % Due to pgfplots meddling with pgfkeys, we have to redefine alt here.
        \pgfplotsset{
    		alt/.code args={<#1>#2#3}{%
    		\alt<#1>{\pgfkeysalso{#2}}{\pgfkeysalso{#3}} % \pgfkeysalso doesn't change the path
    		},
    	}
        \pgfplotsset{compat=1.15}
        \pgfplotsset{colormap={SRON}{rgb255=(61,82,161) rgb255=(255,250,210) rgb255=(174,28,62)}}

    \usetikzlibrary{matrix}
    \usetikzlibrary{shapes, shapes.geometric}
    \usetikzlibrary{decorations.pathreplacing}
	\usetikzlibrary{positioning, calc, intersections}
    \usetikzlibrary{fit}
    \usetikzlibrary{backgrounds}

    % Do evil things with soft path
    % From <https://tex.stackexchange.com/a/301364/8547>
    \makeatletter
        \def\@appendnamedsoftpath#1{%
            \pgfsyssoftpath@getcurrentpath\@temppatha
            \expandafter\let\expandafter\@temppathb\csname tikz@intersect@path@name@#1\endcsname
            \expandafter\expandafter\expandafter\def\expandafter\expandafter\expandafter\@temppatha\expandafter\expandafter\expandafter{\expandafter\@temppatha\@temppathb}%
            \pgfsyssoftpath@setcurrentpath\@temppatha
        }
        \def\@appendnamedpathforactions#1{%
            \pgfsyssoftpath@getcurrentpath\@temppatha
            \expandafter\let\expandafter\@temppathb\csname tikz@intersect@path@name@#1\endcsname
            \expandafter\def\expandafter\@temppatha\expandafter{\csname @temppatha\expandafter\endcsname\@temppathb}%\usepackage{tikz-3dplot}

            \let\tikz@actions@path\@temppatha
        }

        \tikzset{
            use path for main/.code={%
                \tikz@addmode{%
                    \expandafter\pgfsyssoftpath@setcurrentpath\csname tikz@intersect@path@name@#1\endcsname
                }%
            },
            append path for main/.code={%
                \tikz@addmode{%
                    \@appendnamedsoftpath{#1}%
                }%
            },
            use path for actions/.code={%
                \expandafter\def\expandafter\tikz@preactions\expandafter{\tikz@preactions\expandafter\let\expandafter\tikz@actions@path\csname tikz@intersect@path@name@#1\endcsname}%
            },
            append path for actions/.code={%
                \expandafter\def\expandafter\tikz@preactions\expandafter{\tikz@preactions
                \@appendnamedpathforactions{#1}}%
            },
            use path/.style={%
                use path for main=#1,
                use path for actions=#1,
            },
            append path/.style={%
                append path for main=#1,
                append path for actions=#1
            }
        }
    \makeatother

    % TikZ externalisation
    \usetikzlibrary{external}
    % Create the `tikzpics/` folder if it does not exist
    \ShellEscape{mkdir -p tikzpics}
    % Only externalise pictures on demand (to avoid messing up with metropolis theme)
    \tikzset{
        external/export=false,
        external/prefix=tikzpics/
    }
    \tikzexternalize

\usepackage{minted}
	\usemintedstyle{lovelace}
	\setminted{autogobble, fontsize=\scriptsize, tabsize=2}
	\setmintedinline{fontsize=auto}

\usepackage[style=authoryear, block=ragged, doi=false, isbn=false]{biblatex}
    \AtEveryBibitem{
        \ifentrytype{online}
        {} {
            \iffieldequalstr{howpublished}{online}
            {
                \clearfield{howpublished}
            } {
                \clearfield{urlyear}\clearfield{urlmonth}\clearfield{urlday}
            }
        }
    }

	\addbibresource{biblio.bib}

% Compact bibliography style
\setbeamertemplate{bibliography item}[text]

\AtEveryBibitem{
    \clearfield{series}
    \clearfield{pages}
    \clearlist{publisher}
    \clearname{editor}
    \clearlist{location}
}
\renewcommand*{\bibfont}{\tiny}

\usepackage{hyperxmp}	% XMP metadata

\usepackage[type={CC}, modifier={by}, version={4.0}]{doclicense}

\usepackage{todonotes}
\let\todox\todo
\renewcommand\todo[1]{\todox[inline]{#1}}

\title{\titlepagetitle}
\subtitle{\titlepagesubtitle}
\author{\textbf{\myname} (\mylab)}
\institute{}
\date{\tiny Version {\yyyymmdddate\today}T\currenttime}

\titlegraphic{\ccby}

% Tikz styles

% Schémas de tâches
\tikzset{
    >=stealth,
    hair lines/.style={line width = 0.05pt, lightgray},
    data/.style={draw, ellipse},
    program/.style={draw, rectangle},
    accent on/.style={alt={<#1>{draw=accent, text=accent, thick}{draw}}},
    true scale/.style={scale=#1, every node/.style={transform shape}},
    highlight/.style={color=highlight#1}
}

% Styles des heatmap pour les moyennes
\pgfplotsset{
    meanheatmap/.style={
        colorbar, colormap name=SRON,
        view={0}{90},
        samples=100,
        domain=0:1,
        min=0, max=1,
        xlabel={$P$},
        ylabel={$R$},
    }
}

% Commands spécifiques
\NewDocumentCommand\shorturl{ O{https} O{://} m }{%
    \href{#1#2#3}{\nolinkurl{#3}}%
}

\DeclarePairedDelimiter\norm{\lVert}{\rVert}
\DeclarePairedDelimiter\abs{\lvert}{\rvert}
\DeclarePairedDelimiterX\compset[2]{\lbrace}{\rbrace}{#1\,\delimsize|\,#2}
\DeclarePairedDelimiterX\innprod[2]{\langle}{\rangle}{#1\,\delimsize|\,#2}

\DeclareMathOperator*\argmax{argmax}

% Easy column vectors \vcord{a,b,c} ou \vcord[;]{a;b;c}
% Here be black magic
\ExplSyntaxOn
	\NewDocumentCommand{\vcord}{O{,}m}{\vector_main:nnnn{p}{\\}{#1}{#2}}
	\NewDocumentCommand{\tvcord}{O{,}m}{\vector_main:nnnn{psmall}{\\}{#1}{#2}}
	\seq_new:N\l__vector_arg_seq
	\cs_new_protected:Npn\vector_main:nnnn #1 #2 #3 #4{
		\seq_set_split:Nnn\l__vector_arg_seq{#3}{#4}
		\begin{#1matrix}
			\seq_use:Nnnn\l__vector_arg_seq{#2}{#2}{#2}
		\end{#1matrix}
	}
\ExplSyntaxOff

\DeclareMathOperator{\TF}{TF}
\DeclareMathOperator{\IDF}{IDF}

\ExplSyntaxOn
    \DeclareExpandableDocumentCommand\eval{m}{\fp_eval:n{#1}}
\ExplSyntaxOff


% ██████   ██████   ██████ ██    ██ ███    ███ ███████ ███    ██ ████████
% ██   ██ ██    ██ ██      ██    ██ ████  ████ ██      ████   ██    ██
% ██   ██ ██    ██ ██      ██    ██ ██ ████ ██ █████   ██ ██  ██    ██
% ██   ██ ██    ██ ██      ██    ██ ██  ██  ██ ██      ██  ██ ██    ██
% ██████   ██████   ██████  ██████  ██      ██ ███████ ██   ████    ██


\begin{document}
\pdfbookmark[2]{Title}{title}

\begin{frame}[plain]
	\titlepage
\end{frame}

\begin{frame}{Exercice}
    La tâche de \emph{résolution d'anaphores}\footnote{Ou en tout cas une version simplifiée} consiste à déterminer l'antécédent d'un pronom dans un texte.
	\vspace{\bigskipamount}

	\begin{overprint}
		\onslide<2-3> \alert<3>{Le père d'Aino} est heureux de vous annoncer que \alert{sa} fille se marie dimanche prochain
		\onslide<4-> Je n'ai pas pu faire rentrer \alert<5>{la poussette} dans \alert<6->{la valise} parce qu'\alert{elle} est trop \alt<4-5>{grande}{\textbf<6>{petite}}
	\end{overprint}

	\only<7>{
		Reformuler cette tâche comme une combinaison de tâches élémentaires.
	}
\end{frame}

\section{Attributs des documents}

\begin{frame}<-2>{Mots}
    Faisons \alt<1>{enfin de la}{un peu de} linguistique : comptons les \alert{mots} !
    \begin{itemize}
        \item Problème : qu'est-ce qu'un mot ?
            \begin{itemize}
                \item \emph{pomme de terre} ?
                \item \emph{y a-\textbf{t}-il} ?
                \item …
            \end{itemize}
        \item On a plutôt tendance à compter les \alert{tokens}
            \begin{itemize}
                \item « Séquences de caractères comprises entre deux séparateurs »
                \item[→] Problème : qu'est-ce qu'un séparateur ? \begin{itemize}
                    \item « E-mail » vs. « Alsace–Lorraine »
                    \item « […] kiwi. Hier […] » vs. « M. Martin »
                    \item « Yann Le Bourdonnec » vs. « Félix le chat »
                \end{itemize}
                \item On délègue à un segmenteur (\emph{tokenizer})
            \end{itemize}
    \end{itemize}
\end{frame}

\begin{frame}<1>[label=wordcols]{Des mots aux colonnes}
    Objectif : utiliser les \alert{fréquences} des mots comme attributs (colonnes) dans une représentation tabulaire.
    \begin{itemize}
        \item Combien de colonnes faut-il ?
\only<2->{
            \begin{itemize}
                \item[→] $K×M^β$ (loi de Heaps)
                \item Pour \emph{Ulysse}, environ \num{34000}
            \end{itemize}
        \item Il faut donc arriver à filtrer !
			% TODO: déroulé progressif avec exemples de colonnes
            \begin{itemize}[<+->]
                \item Supprimer les \alert{mots vides} (\emph{stop words}) : \emph{le}, \emph{à}, \emph{mais}…
                \item Ne conserver certaines catégories spécifiques à la tâche (noms, adjectifs…)
                \item Supprimer les \alert{hapax} (mots n'apparaissant qu'un seule fois)
                \item[→] De façon générale, ne conserver qu'une \alert{bande de fréquence}
            \end{itemize}
}
    \end{itemize}
\end{frame}

\begin{frame}{Loi de Heaps}
    \begin{figure}
        \tikzset{external/export=true}
        \begin{tikzpicture}
            \begin{axis}[
				title={Loi de Heaps : $y=K×x^β$},
				xlabel={Taille de la collection},
				ylabel={Taille du vocabulaire},
				xmin=0, xmax=800000, ymin=0,
                width=0.9\textwidth,
                height=0.8\textheight,
                scaled ticks=false,
                ticklabel style={/pgf/number format/.cd, 1000 sep={\,}},
			]
                \addplot[highlighta, domain=0:800000, samples=1001] {40*x^0.5};
            \end{axis}
        \end{tikzpicture}
        \caption{Courbe d'une loi de Heaps}
    \end{figure}
\end{frame}

\againframe<2>{wordcols}

\begin{frame}[fragile]{Filtrer les hapax}
    \begin{figure}
        \tikzset{external/export=true}
        \begin{tikzpicture}
            \begin{axis}[xmin=0, ymin=0, xmax=40, ymax=15,
                         xtick=\empty,
                         ytick=\empty,
                         alt=<2>{extra y ticks={1}}{},
                         width=0.9\textwidth,
                         height=0.8\textheight,
                         xlabel={Rang},
                         ylabel={Fréquence d'apparition},
                         title={$y=\frac{K}{x}$}]
                \addplot[highlighta, domain=1:20, samples=501] {20/x};
                \addplot[highlighta, alt=<2>{dotted}{}, domain=20:40, samples=500] {1};
                \only<2>{
                    \draw[highlight6] (axis cs:0,1) -- (current axis.east|-{axis cs:0,1});
                }
            \end{axis}
        \end{tikzpicture}
        \caption{\alt<2>{Filtrage des hapax sur la c}{C}ourbe de Zipf}
    \end{figure}
\end{frame}

\againframe<2>{wordcols}

\begin{frame}[fragile=singleslide]{Filtrer une bande de fréquences}
    \begin{figure}
        \tikzset{external/export=true}
        \begin{tikzpicture}
            \begin{axis}[xmin=0, ymin=0, xmax=40, ymax=20,
                         xtick=\empty,
                         ytick={2, 15},
                         yticklabels={$a$, $b$},
                         width=0.9\textwidth,
                         height=0.8\textheight,
                         xlabel={Rang},
                         ylabel={Fréquence d'apparition},
                         title={$y=\frac{K}{x}$}]
                \addplot[highlighta, dotted, domain=1:1.33, samples=333] {20/x};
                \addplot[highlighta, domain=1.33:10, samples=333] {20/x};
                \addplot[highlighta, dotted, domain=10:40, samples=333] {max(20/x,1)};
                \draw[highlight6] (axis cs:0,15) -- (current axis.east|-{axis cs:0,15});
                \draw[highlight6] (axis cs:0,2) -- (current axis.east|-{axis cs:0,2});
            \end{axis}
        \end{tikzpicture}
        \caption{Filtrage d'une bande de fréquences sur la courbe de Zipf}
    \end{figure}
\end{frame}

\begin{frame}{Mots}
    Pour réduire encore plus le nombre de colonnes
	% TODO: déroulé progressif avec exemples de colonnes
    \begin{itemize}[<+->]
        \item Lemmatiser : \{\emph{est}, \emph{était}, \emph{suis}, \emph{êtes}, \emph{furent}…\} → \emph{est}
        \item Radicaliser : \{\emph{homme}, \emph{hommes}\} → \emph{homme}
        \item Fixer un lexique \emph{a priori}
            \begin{itemize}
                \item[→] Avec les dangers que ça comporte
            \end{itemize}
    \end{itemize}

	\pause
    À l'inverse, pour plus de détails, on peut utiliser des $n$-grammes de mots
    \begin{itemize}
        \item[→] Représentation plus riche, mais amplifie le problème de quantité
    \end{itemize}
\end{frame}

\begin{frame}{Traits linguistiques avancés}
    Si les ressource le permettent, on peut envisager d'autres attributs
    \begin{itemize}[<+->]
        \item Avec un étiqueteur morphosyntaxique : POS, $n$-grammes de POS
        \item Avec un parser : largeur/profondeur/taille moyenne des arbres/constituants, paires gouverneur-gouverné, paires nœud-relation…
        \item Avec des analyseurs sémantiques : rôles
        \item Avec des bases de connaissances : présence de certains concepts
        \item …
    \end{itemize}
	\pause
    Mais on introduit alors des dépendances vis-à-vis des ressources, ce qui multiplie les causes d'erreurs.
\end{frame}


% ███    ███ ███████  █████  ███████ ██    ██ ██████  ███████
% ████  ████ ██      ██   ██ ██      ██    ██ ██   ██ ██
% ██ ████ ██ █████   ███████ ███████ ██    ██ ██████  █████
% ██  ██  ██ ██      ██   ██      ██ ██    ██ ██   ██ ██
% ██      ██ ███████ ██   ██ ███████  ██████  ██   ██ ███████

\section{Valuation des attributs}
\begin{frame}[label=attrval]{Valuation des attributs}
    Avoir choisi les attributs à utiliser pour une classification ne suffit pas : encore faut-il leur donner une \alert{valeur}
    \begin{description}[*]
        \item<+->[Booléenne] Vrai/Faux ($0$/$1$) suivant que l'attribut est ou n'est pas présent dans le document
        \item<+->[Occurrences] Le nombre $n_{i,j}$ de fois qu'apparaît l'attribut $a_j$ dans le document $t_i$
        \item<+->[Fréquences normalisées par ligne] (ici pour la norme $1$)
            \begin{equation}
                \frac{\mathnode{freqfracnum}{n_{i,j}}}{\mathnode{freqfracden}{\sum_k n_{i,k}}}
            \end{equation}
    \end{description}
	\pause
    \textbf{Note} : ici et dans la suite, on considère que toutes les colonnes sont de même nature
\end{frame}

\begin{frame}[fragile]{Notations}
	\begin{overprint}
		\onslide<1>
			\begin{figure}
		        \begin{tabu} to 0.9\linewidth {X[m, c]|*{4}{X[m, c]}}
		                    & je     & traque     & $…$ & kiwi\\
		            \hline
		            $t_1$   & $150$ & $12$ & $…$ & $0$\\
		            $t_2$   & $2713$ & $20$ & $…$ & $0$\\
		            $⋮$    & $⋮$      & $⋮$       &     & $⋮$\\
		            $t_m$   & $1$ & $1$ & $…$ & $1$\\
		         \end{tabu}
		    \end{figure}
		\onslide<2>
			\begin{figure}
		        \begin{tabu} to 0.9\linewidth {X[m, c]|*{4}{X[m, c]}}
		                    & $a_1$     & $a_2$     & $…$ & $a_n$\\
		            \hline
		            $t_1$   & $v_{1,1}$ & $v_{1,2}$ & $…$ & $v_{1,n}$\\
		            $t_2$   & $v_{2,1}$ & $v_{2,2}$ & $…$ & $v_{2,n}$\\
		            $⋮$    & $⋮$      & $⋮$       &     & $⋮$\\
		            $t_m$   & $v_{m,1}$ & $v_{m,2}$ & $…$ & $v_{m,n}$\\
		         \end{tabu}
		    \end{figure}
	\end{overprint}

	\only<2>{
		\begin{itemize}
	        \item Les textes sont notés $t_1$, $t_2$, …, $t_m$
	        \item Les attributs sont notés $a_1$, $a_2$, …, $a_n$
	        \item La valeur du $j$-ème attribut pour le $i$-ème texte est notée $v_{i,j}$ (ligne, colonne)
	    \end{itemize}
	}
\end{frame}

% TODO: add exemples

% TODO: ajouter une transition
% TODO: add cites
\begin{frame}{TF.IDF}
    \only<-2>{On veut pouvoir \alert{pondérer} l'importance d'un terme (mot, caractère, $n$-gramme…) en fonction de sa \alert{spécificité} (au sens large) pour un document.}

    \begin{itemize}
        \item \emph{\textbf{T}erm \textbf{F}requency $×$ \textbf{I}nverse \textbf{D}ocument \textbf{F}requency}
        \item $\TF(i,j) = n_{i,j}$ ou $\TF(i,j) = \frac{n_{i,j}}{\sum_k n_{i,j}}$
        \item
            \only<1>{
                \begin{equation}
                    \IDF(j) = \log\mleft(\frac{m}{\#\compset{i}{n_{i,j}>0}}\mright)
                \end{equation}
            }
            \only<2>{
                \begin{equation}
                    \IDF(j) = \log\frac{\overbrace{m}^{\mathclap{\text{nombre total de documents}}}}{\underbrace{\#\compset{i}{n_{i,j}>0}}_{\mathclap{\text{nombre de documents contenant $a_j$}}}}
                \end{equation}
            }
            \only<3->{$\IDF(j) = \log\frac{m}{\#\compset{i}{n_{i,j}>0}}$}
        \only<3->{
            \item Intuitivement $\IDF$ mesure la rareté d'un attribut, donc (on espère !) son informativité
            \item « Si $a_j$ est présent dans tous les textes, savoir qu'il est présent dans $t_i$ ne donne pas d'information sur $t_i$ »
            \item Permet un filtrage moins arbitraire des \emph{stop words}
            \item \alt<4>{$\log$ donne à $\IDF$ la forme d'une \emph{surprise} au sens de Shannon}{$\log$ est une façon pratique de modérer les effets de cette pondération : on devrait parler de $\log$-$\IDF$}
        }
    \end{itemize}
\end{frame}

% TODO: Slide sur la surprise, intro aux probas

\section{Représentations vectorielles et similarités}

\begin{frame}{Des sacs de mots aux vecteurs}
    Dans tout ce qui précède, on ne tient pas compte des positions relatives des termes, mais seulement de leurs nombres d'occurrences.
    \begin{itemize}
        \item Approche \emph{bag-of-words} (sac de mots)
        \item « Siobhan tue Eóghan » vs. « Eóghan tue Siobhan »…
        \begin{itemize}
            \item[→] On compte (comme d'habitude) sur la masse de données pour passer outre ce genre de problèmes
        \end{itemize}
    \end{itemize}
    En pratique on fait même plus fort
    \begin{itemize}
        \item Hypothèse d'\alert{indépendance} : si $k≠ℓ$, $a_k$ et $a_ℓ$ sont indépendants
        \item On peut alors considérer chaque ligne comme un vecteur (ou un point…) d'un espace vectoriel
        \item Et accéder à la puissance de l'algèbre linéaire et de la géométrie vectorielle !
    \end{itemize}
\end{frame}

\begin{frame}{Les maths en cinq secondes : vecteurs}
    On appelle \emph{vecteur de dimension $n$} tout $n$-uplet de nombres
    \begin{equation}
        v = \vcord{v_1, ⋮, v_n}
    \end{equation}
    Si $n=2$, on peut les représenter simplement dans le plan
    \begin{figure}
        \tikzset{external/export=true}
        \begin{tikzpicture}
            \draw [help lines, xstep=1cm, ystep=1cm] (0, 0) grid (5.25, 4.25);

            \draw[->] (-0.5,0) -- (5.25, 0);
            \foreach \x in {1,...,5}
                \draw[shift={(\x, 0)}] (0pt,2pt) -- (0pt,-2pt) node[below] {\footnotesize $\x$};
            \draw[->] (0, -0.5) -- (0, 4.25);

            \foreach \y in {1,...,4}
                \draw[shift={(0,\y)}] (2pt,0pt) -- (-2pt,0pt) node[left] {\footnotesize $\y$};

            \draw (0, 0) node[below left] {\footnotesize $0$};

            \draw [->, highlighta, line width=2pt] (0,0) -- (2,3);

            \draw (1.5, 1.5) node[anchor=west] {$\textcolor{highlighta}{v}=\vcord{2,3}$};
        \end{tikzpicture}
        \caption{Vecteur en dimension $2$}
    \end{figure}
\end{frame}

\begin{frame}{Les maths en cinq secondes : vecteurs}
    On définit la \emph{somme} et le \emph{produit par un nombre} par
    \begin{equation}
        \vcord{a_1, ⋮, a_n} + \vcord{b_1, ⋮, b_n} = \vcord{a_1+b_1, ⋮, a_n+b_n}
    \end{equation}
    et
    \begin{equation}
        λ⋅\vcord{a_1, ⋮, a_n} = \vcord{λa_1, ⋮, λa_n}
    \end{equation}
    par exemple
    \begin{equation}
        \vcord{2, 7} + \vcord{1, 3} = \vcord{3, 10}
    \end{equation}
    \begin{equation}
        5⋅\vcord{3.5, -3} = \vcord{17.5, -15}
    \end{equation}
\end{frame}

\begin{frame}[fragile]{Les maths en cinq secondes : vecteurs}
    Géométriquement
    \only<1-4>{
        \begin{figure}
            \tikzset{external/export=true}
            \begin{tikzpicture}
                \draw [help lines, xstep=1cm, ystep=1cm] (0, 0) grid (5.25, 4.25);

                \draw[->] (-0.5,0) -- (5.25, 0);
                \foreach \x in {1,...,5}
                    \draw[shift={(\x, 0)}] (0pt,2pt) -- (0pt,-2pt) node[below] {\footnotesize $\x$};
                \draw[->] (0, -0.5) -- (0, 4.25);

                \foreach \y in {1,...,4}
                    \draw[shift={(0,\y)}] (2pt,0pt) -- (-2pt,0pt) node[left] {\footnotesize $\y$};

                \draw (0, 0) node[below left] {\footnotesize $0$};

                \draw [->, highlighta, line width=2pt, alt=<2-3>{invisible}{}] (0,0) -- (3,1) node[midway, above] {$v$};
                \draw [->, highlighta, line width=2pt, alt=<2-3>{}{invisible}] (1,2) -- (4,3) node[midway, above] {$v$};

                \draw [->, highlight3, line width=2pt] (0,0) -- (1,2) node[midway, above left] {$u$};

                \draw[->, highlight6, line width=2pt, alt=<3->{}{invisible}] (0,0) -- (4,3) node[midway, right] {$u+v$};
            \end{tikzpicture}
            \caption{Somme de deux vecteurs}
        \end{figure}
    }
    \only<5-7>{
        \begin{figure}
            \tikzset{external/export=true}
            \begin{tikzpicture}
                \draw [help lines, xstep=1cm, ystep=1cm] (0, 0) grid (5.25, 4.25);

                \draw[->] (-0.5,0) -- (5.25, 0);
                \foreach \x in {1,...,5}
                    \draw[shift={(\x, 0)}] (0pt,2pt) -- (0pt,-2pt) node[below] {\footnotesize $\x$};
                \draw[->] (0, -0.5) -- (0, 4.25);

                \foreach \y in {1,...,4}
                    \draw[shift={(0,\y)}] (2pt,0pt) -- (-2pt,0pt) node[left] {\footnotesize $\y$};

                \draw (0, 0) node[below left] {\footnotesize $0$};

                \draw [->, highlighta, line width=2pt, alt=<5>{}{invisible}] (0,0) -- (2,1) node[midway, above] {$v$};

                \draw[->, highlighta, line width=2pt, alt=<6>{}{invisible}] (0,0) -- (4,2) node[midway, above] {$2v$};
                \draw[->, highlighta, line width=2pt, alt=<7>{}{invisible}] (0,0) -- (1,0.5) node[above] {$0.5v$};
            \end{tikzpicture}
            \caption{produit d'un vecteur par un nombre}
        \end{figure}
    }
\end{frame}

\begin{frame}{Exercice}
	Calculer
	\begin{enumerate}
		\item
			\begin{equation}
				\vcord{2, 1} + \vcord{3, 7}
			\end{equation}
		\item
			\begin{equation}
				3⋅\vcord{2, 1, -2, 0}
			\end{equation}
		\item
			\begin{equation}
				\frac{1}{2}⋅\left(\vcord{2, 4} + \vcord{3,6}\right)
			\end{equation}
		\item
			\begin{equation}
				0.2⋅\vcord{2, 4} + 0.8⋅\vcord{3,6}
			\end{equation}
	\end{enumerate}
\end{frame}

\begin{frame}{Interprétation}
    Bien entendu, l'hypothèse d'indépendance est fausse en général
    \begin{itemize}
        \item[→] Si un texte contient \emph{main}, il est plus probable qu'il contienne également \emph{doigt}
    \end{itemize}
    Et représenter les documents par des vecteurs a ses limites
    \begin{itemize}
        \item[→] Que signifie le fait des sommer les attributs de deux documents ?
    \end{itemize}
    Mais ces approximations permettent d'utiliser une machinerie mathématique qui marche suffisamment bien en pratique.

    On est bien dans la philosophie de la fouille de textes !
\end{frame}

\begin{frame}{Similarité : exemple de travail}
    En pratique, les documents considérés auront beaucoup de colonnes
    \begin{itemize}
        \item De quelques-unes à plusieurs centaines
        \item Donc les vecteurs considérés seront de grande dimension
        \item Mais pour l'intuition on va se retreindre à la dimension $2$ : le plan
    \end{itemize}
\end{frame}

\begin{frame}[standout]
    To deal with a $\mathbf{14}$-dimensional space, visualize a $3$-D space and say `fourteen' to yourself very loudly. Everyone does it.
	\vspace{\bigskipamount}

    Geoffrey Hinton
\end{frame}

\begin{frame}[fragile=singleslide]{Similarités : exemple de travail}
    On considère deux vecteurs $t_1=(v_{1,1}, … v_{1,n})$ et $t_2=(v_{2,1}, …, v_{2,n})$ représentant des documents

    Par exemple, si on utilise comme traits les fréquences de \emph{orange} et \emph{kiwi}, étant donné le fichier tabulaire suivant
    \begin{columns}
    \column{0.5\textwidth}
        \begin{figure}
            \begin{tabu} to .9\linewidth {*{3}{X[c]}}
                \texttt{t}  & \texttt{orange} & \texttt{kiwi}\\
                \hline
                \texttt{1}  & \texttt{3}      & \texttt{1}\\
                \texttt{2}  & \texttt{1}      & \texttt{2}\\
            \end{tabu}
        \end{figure}

        On aura $t_1=\tvcord{3, 1}$ et $t_2=\tvcord{1, 2}$
    \column{0.5\textwidth}
        \begin{figure}
            \tikzset{external/export=true}
            \begin{tikzpicture}[true scale=0.8]
                \draw [help lines, xstep=1cm, ystep=1cm] (0, 0) grid (5.25, 4.25);

                \draw[->] (-0.5,0) -- (5.25, 0);
                \foreach \x in {1,...,5}
                    \draw[shift={(\x, 0)}] (0pt,2pt) -- (0pt,-2pt) node[below] {\footnotesize $\x$};
                \draw[->] (0, -0.5) -- (0, 4.25);

                \foreach \y in {1,...,4}
                    \draw[shift={(0,\y)}] (2pt,0pt) -- (-2pt,0pt) node[left] {\footnotesize $\y$};

                \draw (0, 0) node[below left] {\footnotesize $0$};

                \draw [->, highlighta, line width=2pt] (0,0) -- (3,1) node[midway, above] {$t_1$};
                \draw [->, highlight3, line width=2pt] (0,0) -- (1,2) node[midway, above left] {$t_2$};
            \end{tikzpicture}
            \caption{Exemple de travail}
    \end{figure}
\end{columns}
\end{frame}

% TODO: overlay equation uncover ?
% TODO: distances comparisons (particularly minkovsky) : show distorted diagrams ?
% TODO: for all examples → connect to real-world examples with word frequences
\begin{frame}[fragile]{Distance de Manhattan}
    \begin{block}{Définition (distance de Manhattan)}
        On appelle \emph{distance de Manhattan} entre $t_1$ et $t_2$ le nombre
        \begin{equation}
            \norm{t_2-t_1}_1 = ∑_{k=1}^n\abs{v_{2,k}-v_{1,k}} \left(=\abs{v_{2,1}-v_{1,1}}+…+\abs{v_{2,n}-v_{1,n}}\right)
        \end{equation}
    \end{block}

    \only<1>{
    \begin{columns}
        \column{0.5\textwidth}
        On l'appelle aussi \emph{taxicab distance} : c'est la distance du chemin le plus court pour aller d'un point à un autre en se déplaçant sur une grille

        \column{0.5\textwidth}
        \begin{figure}
            \tikzset{external/export=true}
            \begin{tikzpicture}[background rectangle/.style={fill=gray}, show background rectangle, true scale=0.7]
                \draw [xstep=1, ystep=1, line width=5pt, white] (0, 0) grid (5.1, 5.1);
                \draw[highlighta, line width=2pt] (0, 0) -| (5, 5);
                \draw[highlight6, line width=2pt] (0, 0) |- (5, 5);
                \draw[highlight2, line width=2pt] (0, 0) -| (1, 1) -| (2, 2) -| (3, 3) -| (4, 4) -| (5, 5);
                \draw[highlight4, line width=2pt] (0, 0) |- (2, 3) -| (5, 5);
                \fill[highlight3] (0, 0) circle [radius=2pt];
                \fill[highlight3] (5, 5) circle [radius=2pt];
            \end{tikzpicture}
            \caption{\emph{Taxicab distance}}
        \end{figure}
    \end{columns}
    }
    \only<2>{
        \vspace{-1\bigskipamount}
        Pour notre exemple
        \vspace{-1\smallskipamount}
        \begin{columns}
        \column{.6\textwidth}
            \begin{align}
                \norm{\textcolor{highlight3}{t₂}-\textcolor{highlighta}{t₁}}
                    &= ∑_{k=1}^n\abs{v_{2,k}-v_{1,k}}\\
                    &= \textcolor{highlight2}{\abs{v_{2,1}-v_{1,1}}}+\textcolor{highlight5}{\abs{v_{2,2}-v_{1,2}}}\\
                    &= \textcolor{highlight2}{\abs{1-3}}+\textcolor{highlight5}{\abs{2-1}}\\
                    &= 2+1\\
                    &= \textcolor{highlight4}{3}
            \end{align}
        \column{.45\textwidth}
            \vspace{-1.5\bigskipamount}
            \begin{figure}
                \tikzset{external/export=true}
                \begin{tikzpicture}[true scale=0.8]
                    \draw [help lines, xstep=1cm, ystep=1cm] (0, 0) grid (5, 4);

                    \draw[->] (-0.5,0) -- (5.25, 0);
                    \foreach \x in {1,...,5}
                        \draw[shift={(\x, 0)}] (0pt,2pt) -- (0pt,-2pt) node[below] {\footnotesize $\x$};
                    \draw[->] (0, -0.5) -- (0, 4.25);

                    \foreach \y in {1,...,4}
                        \draw[shift={(0,\y)}] (2pt,0pt) -- (-2pt,0pt) node[left] {\footnotesize $\y$};

                    \draw (0, 0) node[below left] {\footnotesize $0$};

                    \draw [->, highlighta, line width=2pt] (0,0) -- (3,1) node[midway, below right] {$t₁$};
                    \draw [->, highlight3, line width=2pt] (0,0) -- (1,2) node[midway, above left] {$t₂$};
                    \draw [<->, highlight2] (3, 2) -- (1, 2) node [midway, above] {$\abs{1-3}$};
                    \draw[<->, highlight5] (3, 2) -- (3, 1)  node [midway, right] {$\abs{2-1}$};
                    \draw [<->, highlight4] (1, 2) --  (3, 1)  node [midway, below left] {$3$};
                \end{tikzpicture}
                \caption{Distance de Manhattan}
            \end{figure}
        \end{columns}
    }
\end{frame}

\begin{frame}{Distance euclidienne}
    \begin{block}{Définition (distance euclidienne)}
        On appelle \emph{distance euclidienne\textsuperscript{1}} entre $t_1$ et $t_2$ le nombre
        \begin{equation}
            \norm{t₂-t₁}₂ = √{∑_{k=1}^n(v_{2,k}-v_{1,k})²}
        \end{equation}
    \end{block}
    \vspace{-1.5\bigskipamount}
    C'est la distance usuelle dans le plan. Pour notre exemple :
    \vspace{-1\bigskipamount}
    \begin{columns}
    \column{.65\textwidth}
        \begin{align}
            \norm{\textcolor{highlight3}{t₂}-\textcolor{highlighta}{t₁}}₂
                &= √{∑_{k=1}^n(v_{2,k}-v_{1,k})²}\\
                &= √{\textcolor{highlight2}{(v_{2,1}-v_{1,1})}² +
                     \textcolor{highlight5}{(v_{2,2}-v_{1,2})}²}\\
                &= √{\textcolor{highlight2}{(1-3)}²+\textcolor{highlight5}{(2-1)}²}\\
                &= √{2²+1²}\\
                &= \textcolor{highlight4}{√{5}}
        \end{align}
    \column{.5\textwidth}
        \begin{figure}
            \tikzset{external/export=true}
            \begin{tikzpicture}[true scale=0.85]
                \draw [help lines, xstep=1cm, ystep=1cm] (0, 0) grid (5.25, 4.25);

                \draw[->] (-0.5,0) -- (5.25, 0);
                \foreach \x in {1,...,5}
                    \draw[shift={(\x, 0)}] (0pt,2pt) -- (0pt,-2pt) node[below] {\footnotesize $\x$};
                \draw[->] (0, -0.5) -- (0, 4.25);

                \foreach \y in {1,...,4}
                    \draw[shift={(0,\y)}] (2pt,0pt) -- (-2pt,0pt) node[left] {\footnotesize $\y$};

                \draw (0, 0) node[below left] {\footnotesize $0$};

                \draw [->, highlighta, line width=2pt] (0,0) -- (3,1) node[midway, below right] {$t₁$};
                \draw [->, highlight3, line width=2pt] (0,0) -- (1,2) node[midway, above left] {$t₂$};
                \draw [->, highlight2] (3, 2) -- (1, 2)  node [midway, above] {$(1-3)$};
                \draw[->, highlight5] (3, 2) -- (3, 1)  node [midway, right] {$(2-1)$};
                \draw [<->, highlight4] (1, 2) --  (3, 1)  node [midway, below left] {$√{5}$};
            \end{tikzpicture}
            \caption{Distance euclidienne}
        \end{figure}
    \end{columns}
    \vspace{-1\bigskipamount}
    {\tiny 1. Euclide d'Alexandrie, IIIè s AEC}
\end{frame}

\begin{frame}{Distance de Tchebychev}
    \begin{block}{Définition (distance de Tchebychev)}
        On appelle \emph{distance de Tchebychev\textsuperscript{1}} entre $t_1$ et $t_2$ le nombre
        \begin{equation}
            \norm{t₂-t₁}_∞ = \max_{1⩽k⩽n}\abs{v_{2,k}-v_{1,k}}
        \end{equation}
    \end{block}
    \vspace{-1\smallskipamount}
    \begin{columns}
    \column{.65\textwidth}
        \begin{align}
            \norm{\textcolor{highlight3}{t₂}-\textcolor{highlighta}{t₁}}_∞
                &= \max_{1⩽k⩽n}\abs{v_{2,k}-v_{1,k}}\\
                &= \max_{1⩽k⩽n}\mleft(
                    \textcolor{highlight2}{\abs{v_{2,1}-v_{1,1}}},
                    \textcolor{highlight5}{\abs{v_{2,2}-v_{1,2}}}\mright)\\
                &= \max_{1⩽k⩽n}\mleft(
                    \textcolor{highlight2}{\abs{1-3}},
                    \textcolor{highlight5}{\abs{2-1}}\mright)\\
                &= \max_{1⩽k⩽n}(2, 1)\\
                &= \textcolor{highlight4}{2}
        \end{align}
    \column{.5\textwidth}
        \begin{figure}
            \tikzset{external/export=true}
            \begin{tikzpicture}[true scale=0.85]
                \draw [help lines, xstep=1cm, ystep=1cm] (0, 0) grid (5.25, 4.25);

                \draw[->] (-0.5,0) -- (5.25, 0);
                \foreach \x in {1,...,5}
                    \draw[shift={(\x, 0)}] (0pt,2pt) -- (0pt,-2pt) node[below] {\footnotesize $\x$};
                \draw[->] (0, -0.5) -- (0, 4.25);

                \foreach \y in {1,...,4}
                    \draw[shift={(0,\y)}] (2pt,0pt) -- (-2pt,0pt) node[left] {\footnotesize $\y$};

                \draw (0, 0) node[below left] {\footnotesize $0$};

                \draw [->, highlighta, line width=2pt] (0,0) -- (3,1) node[midway, below right] {$t₁$};
                \draw [->, highlight3, line width=2pt] (0,0) -- (1,2) node[midway, above left] {$t₂$};
                \draw [->, highlight2] (3, 2) -- (1, 2)  node [midway, above] {$\abs{1-3}$};
                \draw[->, highlight5] (3, 2) -- (3, 1)  node [midway, right] {$\abs{2-1}$};
                \draw [<->, highlight4] (1, 2) --  (3, 1)  node [midway, below left] {$2$};
            \end{tikzpicture}
            \caption{Distance de Tchebychev}
        \end{figure}
    \end{columns}
    \vspace{-1\smallskipamount}
    {\tiny 1. Пафну́тий Льво́вич Чебышёв, 1821–1894}
\end{frame}

\begin{frame}{Distances de Minkowski}
    \begin{block}{Définition (distance de Minkowski)}
        Pour tout $p⩾1$, on appelle \emph{distance de Minkowski\textsuperscript{1} de paramètre $p$} entre $t_1$ et $t_2$ le nombre
        \begin{equation}
            \norm{t₂-t₁}_p ={\left(∑_{k=1}^n\abs{v_{2,k}-v_{1,k}}^p\right)}^{\frac{1}{p}}
        \end{equation}
    \end{block}
    \vspace{-1\bigskipamount}
    Pour $p=1$ on retrouve la distance de Manhattan, pour $p=2$ la distance euclidienne, et pour $p→+∞$, la distance de Minkowski tend vers la distance de Tchebychev.

    Intuitivement, pour $p=1$, elle traite également tous les écarts entre coordonnées, pour $p=∞$, elle ne conserve que le plus grand écart, et les autres $p$ donnent une interpolation entre ces deux extrêmes.
    \vskip0pt plus 1fill
    {\tiny 1. Hermann Minkowski, 1864–1909}
\end{frame}

\begin{frame}{Distances de Minkowski}
    \begin{figure}
        \tikzset{external/export=true}
        \begin{tikzpicture}
            \begin{axis}[xmin=1, xmax=6, ymin=0,
                         width=0.9\textwidth,
                         height=0.8\textheight,
                         xlabel={$p$},
                         ylabel={$\norm{t₂-t₁}_p$},
                         title={Distances de Minkowski : $y=(2^x+1^x)^{\frac{1}{x}}$},
                         scaled ticks=false,
                         ticklabel style={/pgf/number format/.cd, 1000 sep={\,}}]
                \addplot[highlighta, domain=1:6, samples=1001] {(2^x+1)^(1/x)};
            \end{axis}
        \end{tikzpicture}
        \caption{Distances de Minkowski pour notre exemple}
    \end{figure}
\end{frame}

%  █████  ██████  ██████  ███████ ███    ██ ██████  ██ ██   ██
% ██   ██ ██   ██ ██   ██ ██      ████   ██ ██   ██ ██  ██ ██
% ███████ ██████  ██████  █████   ██ ██  ██ ██   ██ ██   ███
% ██   ██ ██      ██      ██      ██  ██ ██ ██   ██ ██  ██ ██
% ██   ██ ██      ██      ███████ ██   ████ ██████  ██ ██   ██

\appendix
\pgfkeys{/metropolis/inner/sectionpage=simple}  % Avoid random errors with section page progressbar
\section{Annexes}
\pdfbookmark[2]{Remerciements}{acknowledgements}
\begin{frame}{Remerciements}
    Ce cours a été construit à partir du polycopié de cours \citetitle{tellier2017fouille} \parencite{tellier2017fouille} et des précieux conseils d'Isabelle Tellier que je ne saurais trop remercier pour sa confiance et son dévouement.
\end{frame}

\pdfbookmark[2]{Références}{references}
\begin{frame}[allowframebreaks]{References}
    \printbibliography[heading=none]
\end{frame}

\pdfbookmark[2]{Licence}{licence}
\begin{frame}{Licence}
    \begin{center}
        {\huge \ccby}
        \vfill
        This document is available under the terms of the Creative Commons Attribution 4.0 International License (CC BY 4.0) (\shorturl{creativecommons.org/licenses/by/4.0})

        Exceptions to the above statement are listed at {\small\shorturl{loicgrobol.github.io/intro-fouille-textes\#licences}}
        \vfill
        © 2019, Loïc Grobol <\shorturl[mailto][:]{loic.grobol@gmail.com}>

        \shorturl[http]{lattice.cnrs.fr/Grobol-Loic}
    \end{center}
\end{frame}

\end{document}
