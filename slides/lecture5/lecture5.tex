% Copyright © 2018, Loïc Grobol <loic.grobol@gmail.com>
% This document is available under the terms of the Creative Commons Attribution 4.0 International License (CC BY 4.0) (https://creativecommons.org/licenses/by/4.0/)

\RequirePackage{xparse}
\RequirePackage{shellesc}
% Settings
\NewDocumentCommand\myname{}{Loïc Grobol}
\NewDocumentCommand\mylab{}{Lattice / ALMAnaCH}
\NewDocumentCommand\pdftitle{}{Introduction à la fouille de textes}
\NewDocumentCommand\mymail{}{loic.grobol@gmail.com}
\NewDocumentCommand\titlepagetitle{}{\pdftitle}
\NewDocumentCommand\titlepagesubtitle{}{\docdate Cours 5 : représenation des données}
\NewDocumentCommand\docdate{}{2019-02-18}
\NewDocumentCommand\conference{}{M1 Plurital}

\documentclass[hyperref={unicode}, xcolor={svgnames}, french]{beamer}
% Alternative navy blue for ⩽4 palettes
\definecolor{highlighta}{RGB}{68, 118, 170}  % Navy blue

% Colour palette from [Paul Tol's technical note](https://personal.sron.nl/~pault/data/colourschemes.pdf) v3.1
% Bright scheme
\definecolor{sronbrighttblue}{RGB}{68, 119, 170}
\definecolor{sronbrightcyan}{RGB}{102, 204, 238}
\definecolor{sronbrightgreen}{RGB}{34, 136, 51}
\definecolor{sronbrightyellow}{RGB}{204, 187, 68}
\definecolor{sronbrightred}{RGB}{238, 102, 119}
\definecolor{sronbrightpurple}{RGB}{170, 51, 119}
\definecolor{sronbrightgrey}{RGB}{187, 187, 187}

\colorlet{highlight0}{sronbrighttblue}
\colorlet{highlight1}{sronbrightcyan}
\colorlet{highlight2}{sronbrightgreen}
\colorlet{highlight3}{sronbrightyellow}
\colorlet{highlight4}{sronbrightred}
\colorlet{highlight5}{sronbrightpurple}
\colorlet{highlight6}{sronbrightgrey}
% Legacy highlights
\definecolor{highlight7}{RGB}{136, 34, 85}    % Purple
\definecolor{highlight8}{RGB}{170, 68, 153}   % Violet


\usetheme[
	sectionpage=progressbar,
    subsectionpage=progressbar,
	progressbar=frametitle,
]{metropolis}
	\colorlet{accent}{sronbrightgreen}
	\setbeamercolor{frametitle}{bg=DarkSlateGrey}
	\setbeamercolor{alerted text}{fg=accent}
	\makeatletter
		\setlength{\metropolis@progressinheadfoot@linewidth}{2pt}
	\makeatother
	% Avoid ugly whitespace below figures
	% \makeatletter
	% 	\renewenvironment{figure}[1][]{%
	% 	\def\@captype{figure}%
	% 	\par\centering}%
	% 	{\par}
	% \makeatother
	% FIXME: not pretty and footnotes are still too big
	\let\footnoterule\relax  % No footnote rule, push down footnote


% Use non-standard fonts
\usefonttheme{professionalfonts}
\setsansfont[BoldFont={Fira Sans SemiBold}, ItalicFont={Fira Sans Book Italic}]{Fira Sans Book}
\setmonofont[Scale=0.9]{Fira Mono}

% Fix missing glyphs in Fira by delegating to polyglossia/babel
\usepackage{newunicodechar}
\newunicodechar{ }{~}   % U+202F NARROW NO-BREAK SPACE
\newunicodechar{ }{ }  % U+2009 THIN SPACE

% Notes on left screen
% \usepackage{pgfpages}
% \setbeameroption{show notes on second screen=left}


\usepackage[french, english]{babel}
\usepackage{amsfonts,amssymb}
\usepackage{amsmath,amsthm}
\usepackage{mathtools}	% AMS Maths service pack
	\newtagform{brackets}{[}{]}	% Pour des lignes d'équation numérotées entre crochets
	\mathtoolsset{showonlyrefs, showmanualtags, mathic}	% affiche les tags manuels (\tag et \tag*) et corrige le kerning des maths inline dans un bloc italique voir la doc de mathtools
	\usetagform{brackets}	% Utilise le style de tags défini plus haut
\usepackage{lualatex-math}

\usepackage[math-style=french]{unicode-math}
	\setmathfont{Libertinus Math}
\usepackage{newunicodechar}
	\newunicodechar{√}{\sqrt}
\usepackage{mleftright}


\usepackage{tabu}
\usepackage{booktabs}
\usepackage{siunitx}
\usepackage{multicol}
\usepackage{ccicons}
\usepackage{bookmark}
\usepackage{caption}
    \captionsetup{skip=1ex}
\usepackage[iso]{datetime}

\usepackage{csquotes}
\usepackage{tikz}
	\NewDocumentCommand{\textnode}{O{}mm}{\tikz[remember picture, baseline=(#2.base), inner sep=0pt]{\node[#1] (#2) {#3};}}
    \NewDocumentCommand{\mathnode}{O{}mm}{\tikz[remember picture, baseline=(#2.base), inner sep = 0pt]{\node[#1] (#2) {$\displaystyle #3$};}}
	\tikzset{
		alt/.code args={<#1>#2#3}{%
		\alt<#1>{\pgfkeysalso{#2}}{\pgfkeysalso{#3}} % \pgfkeysalso doesn't change the path
		},
        invisible/.style={opacity=0, fill opacity=0},
		visible on/.style={alt={<#1>{}{invisible}}}
	}
    \usepackage{forest}
    \usepackage{tkz-graph}
    \usepackage[beamer, markings]{hf-tikz}
    \usepackage{tikz-3dplot}
    \usepackage{pgfplots}
        % Due to pgfplots meddling with pgfkeys, we have to redefine alt here.
        \pgfplotsset{
    		alt/.code args={<#1>#2#3}{%
    		\alt<#1>{\pgfkeysalso{#2}}{\pgfkeysalso{#3}} % \pgfkeysalso doesn't change the path
    		},
    	}
        \pgfplotsset{compat=1.15}
        \pgfplotsset{colormap={SRON}{rgb255=(61,82,161) rgb255=(255,250,210) rgb255=(174,28,62)}}

    \usetikzlibrary{matrix}
    \usetikzlibrary{shapes, shapes.geometric}
    \usetikzlibrary{decorations.pathreplacing}
	\usetikzlibrary{positioning, calc, intersections}
    \usetikzlibrary{fit}
    \usetikzlibrary{backgrounds}

    % Do evil things with soft path
    % From <https://tex.stackexchange.com/a/301364/8547>
    \makeatletter
        \def\@appendnamedsoftpath#1{%
            \pgfsyssoftpath@getcurrentpath\@temppatha
            \expandafter\let\expandafter\@temppathb\csname tikz@intersect@path@name@#1\endcsname
            \expandafter\expandafter\expandafter\def\expandafter\expandafter\expandafter\@temppatha\expandafter\expandafter\expandafter{\expandafter\@temppatha\@temppathb}%
            \pgfsyssoftpath@setcurrentpath\@temppatha
        }
        \def\@appendnamedpathforactions#1{%
            \pgfsyssoftpath@getcurrentpath\@temppatha
            \expandafter\let\expandafter\@temppathb\csname tikz@intersect@path@name@#1\endcsname
            \expandafter\def\expandafter\@temppatha\expandafter{\csname @temppatha\expandafter\endcsname\@temppathb}%\usepackage{tikz-3dplot}

            \let\tikz@actions@path\@temppatha
        }

        \tikzset{
            use path for main/.code={%
                \tikz@addmode{%
                    \expandafter\pgfsyssoftpath@setcurrentpath\csname tikz@intersect@path@name@#1\endcsname
                }%
            },
            append path for main/.code={%
                \tikz@addmode{%
                    \@appendnamedsoftpath{#1}%
                }%
            },
            use path for actions/.code={%
                \expandafter\def\expandafter\tikz@preactions\expandafter{\tikz@preactions\expandafter\let\expandafter\tikz@actions@path\csname tikz@intersect@path@name@#1\endcsname}%
            },
            append path for actions/.code={%
                \expandafter\def\expandafter\tikz@preactions\expandafter{\tikz@preactions
                \@appendnamedpathforactions{#1}}%
            },
            use path/.style={%
                use path for main=#1,
                use path for actions=#1,
            },
            append path/.style={%
                append path for main=#1,
                append path for actions=#1
            }
        }
    \makeatother

    % TikZ externalisation
    \usetikzlibrary{external}
    % Create the `tikzpics/` folder if it does not exist
    \ShellEscape{mkdir -p tikzpics}
    % Only externalise pictures on demand (to avoid messing up with metropolis theme)
    \tikzset{
        external/export=false,
        external/prefix=tikzpics/
    }
    \tikzexternalize

\usepackage{minted}
	\usemintedstyle{lovelace}
	\setminted{autogobble, fontsize=\scriptsize, tabsize=2}
	\setmintedinline{fontsize=auto}

\usepackage[style=authoryear, block=ragged, doi=false, isbn=false]{biblatex}
    \AtEveryBibitem{
        \ifentrytype{online}
        {} {
            \iffieldequalstr{howpublished}{online}
            {
                \clearfield{howpublished}
            } {
                \clearfield{urlyear}\clearfield{urlmonth}\clearfield{urlday}
            }
        }
    }

	\addbibresource{biblio.bib}

% Compact bibliography style
\setbeamertemplate{bibliography item}[text]

\AtEveryBibitem{
    \clearfield{series}
    \clearfield{pages}
    \clearlist{publisher}
    \clearname{editor}
    \clearlist{location}
}
\renewcommand*{\bibfont}{\tiny}

\usepackage{hyperxmp}	% XMP metadata

\usepackage[type={CC}, modifier={by}, version={4.0}]{doclicense}

\usepackage{todonotes}
\let\todox\todo
\renewcommand\todo[1]{\todox[inline]{#1}}

\title{\titlepagetitle}
\subtitle{\titlepagesubtitle}
\author{\textbf{\myname} (\mylab)}
\institute{}
\date{\tiny Version {\yyyymmdddate\today}T\currenttime}

\titlegraphic{\ccby}

% Tikz styles

% Schémas de tâches
\tikzset{
    >=stealth,
    hair lines/.style={line width = 0.05pt, lightgray},
    data/.style={draw, ellipse},
    program/.style={draw, rectangle},
    accent on/.style={alt={<#1>{draw=accent, text=accent, thick}{draw}}},
    true scale/.style={scale=#1, every node/.style={transform shape}},
    highlight/.style={color=highlight#1}
}

% Styles des heatmap pour les moyennes
\pgfplotsset{
    meanheatmap/.style={
        colorbar, colormap name=SRON,
        view={0}{90},
        samples=100,
        domain=0:1,
        min=0, max=1,
        xlabel={$P$},
        ylabel={$R$},
    }
}

% Commands spécifiques
\NewDocumentCommand\shorturl{ O{https} O{://} m }{%
    \href{#1#2#3}{\nolinkurl{#3}}%
}

\DeclarePairedDelimiter\norm{\lVert}{\rVert}
\DeclarePairedDelimiter\abs{\lvert}{\rvert}
\DeclarePairedDelimiterX\compset[2]{\lbrace}{\rbrace}{#1\,\delimsize|\,#2}
\DeclarePairedDelimiterX\innprod[2]{\langle}{\rangle}{#1\,\delimsize|\,#2}

\DeclareMathOperator*\argmax{argmax}

% Easy column vectors \vcord{a,b,c} ou \vcord[;]{a;b;c}
% Here be black magic
\ExplSyntaxOn
	\NewDocumentCommand{\vcord}{O{,}m}{\vector_main:nnnn{p}{\\}{#1}{#2}}
	\NewDocumentCommand{\tvcord}{O{,}m}{\vector_main:nnnn{psmall}{\\}{#1}{#2}}
	\seq_new:N\l__vector_arg_seq
	\cs_new_protected:Npn\vector_main:nnnn #1 #2 #3 #4{
		\seq_set_split:Nnn\l__vector_arg_seq{#3}{#4}
		\begin{#1matrix}
			\seq_use:Nnnn\l__vector_arg_seq{#2}{#2}{#2}
		\end{#1matrix}
	}
\ExplSyntaxOff

\DeclareMathOperator{\TF}{TF}
\DeclareMathOperator{\IDF}{IDF}

\ExplSyntaxOn
    \DeclareExpandableDocumentCommand\eval{m}{\fp_eval:n{#1}}
\ExplSyntaxOff


% ██████   ██████   ██████ ██    ██ ███    ███ ███████ ███    ██ ████████
% ██   ██ ██    ██ ██      ██    ██ ████  ████ ██      ████   ██    ██
% ██   ██ ██    ██ ██      ██    ██ ██ ████ ██ █████   ██ ██  ██    ██
% ██   ██ ██    ██ ██      ██    ██ ██  ██  ██ ██      ██  ██ ██    ██
% ██████   ██████   ██████  ██████  ██      ██ ███████ ██   ████    ██


\begin{document}
\pdfbookmark[2]{Title}{title}

\begin{frame}[plain]
	\titlepage
\end{frame}

\begin{frame}{Exercice}
    Une requête dans un moteur de recherche renvoie \num{27} documents. Parmi ceux-ci, \num{13} sont pertinents pour cette requête. Une étude du corpus montre que parmi ses \num{1024} documents, \num{15} étaient pertinents pour cette requête.
    \begin{itemize}
        \item Déterminer les nombres de vrais positifs, vrais négatifs, faux positifs et faux négatifs
        \item Déterminer la précision, le rappel et la $F$-mesure associées
        \item Mêmes questions pour un corpus de taille \num{2713}
    \end{itemize}
\end{frame}

\begin{frame}<-2>{Mots}
    Faisons \alt<1>{enfin de la}{un peu de} linguistique : comptons les \alert{mots} !
    \begin{itemize}
        \item Problème : qu'est-ce qu'un mot ?
            \begin{itemize}
                \item \emph{pomme de terre} ?
                \item \emph{y a-\textbf{t}-il} ?
                \item …
            \end{itemize}
        \item On a plutôt tendance à compter les \alert{tokens}
            \begin{itemize}
                \item « Séquences de caractères comprises entre deux séparateurs »
                \item[→] Problème : qu'est-ce qu'un séparateur ? \begin{itemize}
                    \item « E-mail » vs. « Alsace–Lorraine »
                    \item « […] kiwi. Hier […] » vs. « M. Martin »
                    \item « Yann Le Bourdonnec » vs. « Félix le chat »
                \end{itemize}
                \item On délègue à un segmenteur (\emph{tokenizer})
            \end{itemize}
    \end{itemize}
\end{frame}

\begin{frame}<1>[label=wordcols]{Des mots aux colonnes}
    Objectif : utiliser les fréquences des mots comme attributs (colonnes) dans une représentation tabulaire.
    \begin{itemize}
        \item Combien de colonnes faut-il ?
\only<2->{
            \begin{itemize}
                \item[→] $K×M^β$ (loi de Heaps)
                \item Pour \emph{Ulysse}, environ \num{34000}
            \end{itemize}
        \item Il faut donc arriver à filtrer !
			% TODO: déroulé progressif avec exemples de colonnes
            \begin{itemize}
                \item Supprimer les mots vides (\emph{stop words}) : \emph{le}, \emph{à}, \emph{mais}…
                \item Ne conserver certaines catégories spécifiques à la tâche (noms, adjectifs…)
                \item Supprimer les hapax (mots n'apparaissant qu'un seule fois)
                \item[→] De façon générale, ne conserver qu'une bande de fréquence
            \end{itemize}
}
    \end{itemize}
\end{frame}

\begin{frame}{Loi de Heaps}
    \begin{figure}
        \tikzset{external/export=true}
        \begin{tikzpicture}
            \begin{axis}[
				title={Loi de Heaps : $y=K×x^β$},
				xlabel={Taille de la collection},
				ylabel={Taille du vocabulaire},
				xmin=0, xmax=800000, ymin=0,
                width=0.9\textwidth,
                height=0.8\textheight,
                scaled ticks=false,
                ticklabel style={/pgf/number format/.cd, 1000 sep={\,}},
			]
                \addplot[highlighta, domain=0:800000, samples=1001] {40*x^0.5};
            \end{axis}
        \end{tikzpicture}
        \caption{Courbe d'une loi de Heaps}
    \end{figure}
\end{frame}

\againframe<2>{wordcols}

\begin{frame}[fragile]{Filtrer les hapax}
    \begin{figure}
        \tikzset{external/export=true}
        \begin{tikzpicture}
            \begin{axis}[xmin=0, ymin=0, xmax=40, ymax=15,
                         xtick=\empty,
                         ytick=\empty,
                         alt=<2>{extra y ticks={1}}{},
                         width=0.9\textwidth,
                         height=0.8\textheight,
                         xlabel={Rang},
                         ylabel={Fréquence d'apparition},
                         title={$y=\frac{K}{x}$}]
                \addplot[highlighta, domain=1:20, samples=501] {20/x};
                \addplot[highlighta, alt=<2>{dotted}{}, domain=20:40, samples=500] {1};
                \only<2>{
                    \draw[highlight6] (axis cs:0,1) -- (current axis.east|-{axis cs:0,1});
                }
            \end{axis}
        \end{tikzpicture}
        \caption{\alt<2>{Filtrage des hapax sur la c}{C}ourbe de Zipf}
    \end{figure}
\end{frame}

\againframe<2>{wordcols}

\begin{frame}[fragile=singleslide]{Filtrer une bande de fréquences}
    \begin{figure}
        \tikzset{external/export=true}
        \begin{tikzpicture}
            \begin{axis}[xmin=0, ymin=0, xmax=40, ymax=20,
                         xtick=\empty,
                         ytick={2, 15},
                         yticklabels={$a$, $b$},
                         width=0.9\textwidth,
                         height=0.8\textheight,
                         xlabel={Rang},
                         ylabel={Fréquence d'apparition},
                         title={$y=\frac{K}{x}$}]
                \addplot[highlighta, dotted, domain=1:1.33, samples=333] {20/x};
                \addplot[highlighta, domain=1.33:10, samples=333] {20/x};
                \addplot[highlighta, dotted, domain=10:40, samples=333] {max(20/x,1)};
                \draw[highlight6] (axis cs:0,15) -- (current axis.east|-{axis cs:0,15});
                \draw[highlight6] (axis cs:0,2) -- (current axis.east|-{axis cs:0,2});
            \end{axis}
        \end{tikzpicture}
        \caption{Filtrage d'une bande de fréquences sur la courbe de Zipf}
    \end{figure}
\end{frame}

\begin{frame}{Mots}
    Pour réduire encore plus le nombre de colonnes
	% TODO: déroulé progressif avec exemples de colonnes
    \begin{itemize}
        \item Lemmatiser : \{\emph{est}, \emph{était}, \emph{suis}, \emph{êtes}, \emph{furent}…\} → \emph{est}
        \item Radicaliser : \{\emph{homme}, \emph{hommes}\} → \emph{homme}
        \item Fixer un lexique \emph{a priori}
            \begin{itemize}
                \item[→] Avec les dangers que ça comporte
            \end{itemize}
    \end{itemize}

    À l'inverse, pour plus de détails, on peut utiliser des $n$-grammes de mots
    \begin{itemize}
        \item[→] Représentation plus riche, mais amplifie le problème de quantité
    \end{itemize}
\end{frame}

\begin{frame}{Traits linguistiques avancés}
    Si les ressource le permettent, on peut envisager d'autres attributs
    \begin{itemize}
        \item Avec un étiqueteur morphosyntaxique : POS, $n$-grammes de POS
        \item Avec un parser : largeur/profondeur/taille moyenne des arbres/constituants, paires gouverneur-gouverné, paires nœud-relation…
        \item Avec des analyseurs sémantiques : rôles
        \item Avec des bases de connaissances : présence de certains concepts
        \item …
    \end{itemize}
    Mais on introduit alors des dépendances vis-à-vis des ressources, ce qui multiplie les causes d'erreurs.
\end{frame}



%  █████  ██████  ██████  ███████ ███    ██ ██████  ██ ██   ██
% ██   ██ ██   ██ ██   ██ ██      ████   ██ ██   ██ ██  ██ ██
% ███████ ██████  ██████  █████   ██ ██  ██ ██   ██ ██   ███
% ██   ██ ██      ██      ██      ██  ██ ██ ██   ██ ██  ██ ██
% ██   ██ ██      ██      ███████ ██   ████ ██████  ██ ██   ██

\appendix
\pgfkeys{/metropolis/inner/sectionpage=simple}  % Avoid random errors with section page progressbar
\section{Annexes}
\pdfbookmark[2]{Remerciements}{acknowledgements}
\begin{frame}{Remerciements}
    Ce cours a été construit à partir du polycopié de cours \citetitle{tellier2017fouille} \parencite{tellier2017fouille} et des précieux conseils d'Isabelle Tellier que je ne saurais trop remercier pour sa confiance et son dévouement.
\end{frame}

\pdfbookmark[2]{Références}{references}
\begin{frame}[allowframebreaks]{References}
    \printbibliography[heading=none]
\end{frame}

\pdfbookmark[2]{Licence}{licence}
\begin{frame}{Licence}
    \begin{center}
        {\huge \ccby}
        \vfill
        This document is available under the terms of the Creative Commons Attribution 4.0 International License (CC BY 4.0) (\shorturl{creativecommons.org/licenses/by/4.0})

        Exceptions to the above statement are listed at {\small\shorturl{loicgrobol.github.io/intro-fouille-textes\#licences}}
        \vfill
        © 2019, Loïc Grobol <\shorturl[mailto][:]{loic.grobol@gmail.com}>

        \shorturl[http]{lattice.cnrs.fr/Grobol-Loic}
    \end{center}
\end{frame}

\end{document}
