% This document is available under the terms of the Creative Commons Attribution 4.0 International License (CC BY 4.0) (https://creativecommons.org/licenses/by/4.0/)

\RequirePackage{xparse}
\RequirePackage{shellesc}
% Settings
\newcommand\myname{Loïc Grobol}
\newcommand\mylab{Lattice / LLF}
\newcommand\pdftitle{Introduction à la fouille de textes}
\newcommand\mymail{loic.grobol@gmail.com}
\newcommand\titlepagetitle{\pdftitle}
\newcommand\titlepagesubtitle{Cours 10 : SVM}
\newcommand\docdate{2020-03-12}  % chktex 8
\newcommand\conference{M1 Plurital}

% \RequirePackage{hyperxmp}  % Results in a complex failure with beamer + tikzexternal (2020-01-17)
\documentclass[xcolor={svgnames}, french]{beamer}
\hypersetup{
    pdftitle={\pdftitle},
    pdfauthor={\myname},
    % pdfcontactemail={\mymail},  % hyperxmpAnnotation
    breaklinks,
    hypertexnames=true,
    pdfdisplaydoctitle,
    pdflang={fr},
    % keeppdfinfo,  % hyperxmp
}

% Colour palette from [Paul Tol's technical note](https://personal.sron.nl/~pault/data/colourschemes.pdf) v3.1
% Bright scheme
\definecolor{sronbrightblue}{RGB}{68, 119, 170}
\definecolor{sronbrightcyan}{RGB}{102, 204, 238}
\definecolor{sronbrightgreen}{RGB}{34, 136, 51}
\definecolor{sronbrightyellow}{RGB}{204, 187, 68}
\definecolor{sronbrightred}{RGB}{238, 102, 119}
\definecolor{sronbrightpurple}{RGB}{170, 51, 119}
\definecolor{sronbrightgrey}{RGB}{187, 187, 187}


\colorlet{highlight0}{sronbrightblue}
\colorlet{highlight1}{sronbrightred}
\colorlet{highlight2}{sronbrightgreen}
\colorlet{highlight3}{sronbrightyellow}
\colorlet{highlight4}{sronbrightcyan}
\colorlet{highlight5}{sronbrightpurple}
\colorlet{highlight6}{sronbrightgrey}
% Legacy highlights
\definecolor{highlighta}{RGB}{68, 118, 170}  % Navy blue
\definecolor{highlight7}{RGB}{136, 34, 85}    % Purple
\definecolor{highlight8}{RGB}{170, 68, 153}   % Violet


\usetheme[
	sectionpage=progressbar,
    subsectionpage=progressbar,
	progressbar=frametitle,
]{metropolis}
	\colorlet{accent}{sronbrightgreen}
	\setbeamercolor{frametitle}{bg=DarkSlateGrey}
	\setbeamercolor{alerted text}{fg=accent}
	\makeatletter
		\setlength{\metropolis@progressinheadfoot@linewidth}{2pt}
	\makeatother
	% Avoid ugly whitespace below figures
	% \makeatletter
	% 	\renewenvironment{figure}[1][]{%
	% 	\def\@captype{figure}%
	% 	\par\centering}%
	% 	{\par}
	% \makeatother
	% FIXME: not pretty and footnotes are still too big
	\let\footnoterule\relax  % No footnote rule, push down footnote


% Use non-standard fonts
\usefonttheme{professionalfonts}
    \setsansfont[BoldFont={* Semibold}]{Fira Sans}
\setmonofont[Scale=0.9]{Fira Mono}

% Fix missing glyphs in Fira by delegating to polyglossia/babel
\usepackage{newunicodechar}
\newunicodechar{ }{~}   % U+202F NARROW NO-BREAK SPACE
\newunicodechar{ }{ }  % U+2009 THIN SPACE

% Notes on left screen
% \usepackage{pgfpages}
% \setbeameroption{show notes on second screen=left}


\usepackage[french, english]{babel}
\usepackage{amsfonts,amssymb}
\usepackage{amsmath,amsthm}
\usepackage{mathtools}	% AMS Maths service pack
    % Pour des lignes d'équation numérotées entre crochets
	\newtagform{brackets}{[}{]} % chktex 9
	\mathtoolsset{showonlyrefs, showmanualtags, mathic}	% affiche les tags manuels (\tag et \tag*) et corrige le kerning des maths inline dans un bloc italique voir la doc de mathtools
	\usetagform{brackets}	% Utilise le style de tags défini plus haut
\usepackage{lualatex-math}

\usepackage[math-style=french]{unicode-math}
	\setmathfont[Scale=1.1]{Libertinus Math}
\usepackage{newunicodechar}
	\newunicodechar{√}{\sqrt}
\usepackage{mleftright}

\usepackage{tabu}
\usepackage{booktabs}
\usepackage{siunitx}
\usepackage{multicol}
\usepackage{ccicons}
\usepackage{bookmark}
\usepackage{caption}
    \captionsetup{skip=1ex}
\usepackage[iso]{datetime}

\usepackage{tikz}
	\NewDocumentCommand{\textnode}{O{}mm}{\tikz[remember picture, baseline=(#2.base), inner sep=0pt]{\node[#1] (#2) {#3};}} % chktex 36
    \NewDocumentCommand{\mathnode}{O{}mm}{\tikz[remember picture, baseline=(#2.base), inner sep=0pt]{\node[#1] (#2) {\(\displaystyle #3\)};}} % chktex 36
	\tikzset{
		alt/.code args={<#1>#2#3}{%
		\alt<#1>{\pgfkeysalso{#2}}{\pgfkeysalso{#3}} % \pgfkeysalso doesn't change the path
		},
        invisible/.style={opacity=0, fill opacity=0},
		visible on/.style={alt={<#1>{}{invisible}}}
	}
    \usepackage{forest}
    \usepackage{tkz-graph}
    \usepackage[beamer, markings]{hf-tikz}
    \usepackage{tikz-3dplot}
    \usepackage{pgfplots}
        % Due to pgfplots meddling with pgfkeys, we have to redefine alt here.
        \pgfplotsset{
    		alt/.code args={<#1>#2#3}{%
    		\alt<#1>{\pgfkeysalso{#2}}{\pgfkeysalso{#3}} % \pgfkeysalso doesn't change the path
    		},
    	}
        \pgfplotsset{compat=1.16}
        \pgfplotsset{colormap={SRON}{rgb255=(61,82,161) rgb255=(255,250,210) rgb255=(174,28,62)}} % chktex 36
    \usetikzlibrary{matrix}
    \usetikzlibrary{shapes, shapes.geometric}
    \usetikzlibrary{decorations.pathreplacing}
	\usetikzlibrary{positioning, calc, intersections}
    \usetikzlibrary{fit}
    \usetikzlibrary{backgrounds}

    % Do evil things with soft path
    \makeatletter
        \def\@appendnamedsoftpath#1{% chktex 21
            \pgfsyssoftpath@getcurrentpath\@temppatha % chktex 21 chktex 1
            \expandafter\let\expandafter\@temppathb\csname tikz@intersect@path@name@#1\endcsname % chktex 41 chktex 21
            \expandafter\expandafter\expandafter\def\expandafter\expandafter\expandafter\@temppatha\expandafter\expandafter\expandafter{\expandafter\@temppatha\@temppathb} % chktex 41 chktex 21
            \pgfsyssoftpath@setcurrentpath\@temppatha % chktex 21 chktex 1
        }
        \def\@appendnamedpathforactions#1{% chktex 21
            \pgfsyssoftpath@getcurrentpath\@temppatha % chktex 1 chktex 21
            \expandafter\let\expandafter\@temppathb\csname tikz@intersect@path@name@#1\endcsname % chktex 21 chktex 41
            \expandafter\def\expandafter\@temppatha\expandafter{\csname @temppatha\expandafter\endcsname\@temppathb} % chktex 21 chktex 41

            \let\tikz@actions@path\@temppatha  % chktex 21 chktex 1
        }

        \tikzset{
            use path for main/.code={%
                \tikz@addmode{%
                    \expandafter\pgfsyssoftpath@setcurrentpath\csname tikz@intersect@path@name@#1\endcsname % chktex 21 chktex 41
                }%
            },
            append path for main/.code={%
                \tikz@addmode{%
                    \@appendnamedsoftpath{#1}%
                }%
            },
            use path for actions/.code={%
                \expandafter\def\expandafter\tikz@preactions\expandafter {\tikz@preactions\expandafter\let\expandafter\tikz@actions@path\csname tikz@intersect@path@name@#1\endcsname}% chktex 21 chktex 41
            },
            append path for actions/.code={%
                \expandafter\def\expandafter\tikz@preactions\expandafter{\tikz@preactions % chktex 21 chktex 41 chktex 1
                \@appendnamedpathforactions{#1}}%
            },
            use path/.style={%
                use path for main=#1,
                use path for actions=#1,
            },
            append path/.style={%
                append path for main=#1,
                append path for actions=#1
            }
        }
    \makeatother

    % TikZ externalisation
    \usetikzlibrary{external}
    % Create the `tikzpics/` folder if it does not exist
    \ShellEscape{mkdir -p tikzpics}
    % Only externalise pictures on demand (to avoid messing up with metropolis theme)
    \tikzset{
        external/export=false,
        external/prefix=tikzpics/
    }
    \tikzexternalize % chktex 1

\usepackage{minted}
	\usemintedstyle{lovelace}
	\setminted{autogobble, tabsize=2}
	\setmintedinline{fontsize=auto}

\usepackage{csquotes}

\usepackage[style=authoryear, block=ragged, doi=false, isbn=false]{biblatex}
    \AtEveryBibitem{
        \ifentrytype{online}
        {} {
            \iffieldequalstr{howpublished}{online}
            {
                \clearfield{howpublished}
            } {
                \clearfield{urlyear}\clearfield{urlmonth}\clearfield{urlday}
            }
        }
    }

	\addbibresource{../biblio.bib}

% Compact bibliography style
\setbeamertemplate{bibliography item}[text]

\AtEveryBibitem{
    \clearfield{series}
    \clearfield{pages}
    \clearlist{publisher}
    \clearname{editor}
    \clearlist{location}
}
\renewcommand*{\bibfont}{\tiny}

\usepackage{hyperxmp}	% XMP metadata

\usepackage[type={CC}, modifier={by}, version={4.0}]{doclicense}

\usepackage{todonotes}
\let\todox\todo % chktex 1
\renewcommand\todo[1]{\todox[inline]{#1}}

\title{\titlepagetitle}
\subtitle{\titlepagesubtitle}
\author{\textbf{\myname} (\mylab)}
\institute{}
\date{\tiny Version {\yyyymmdddate\today}T\currenttime}

\titlegraphic{\ccby}

% Tikz styles

% Schémas de tâches
\tikzset{
    >=stealth,
    hair lines/.style={line width = 0.05pt, lightgray},
    data/.style={draw, ellipse},
    program/.style={draw, rectangle},
    accent on/.style={alt={<#1>{draw=accent, text=accent, thick}{draw}}},
    true scale/.style={scale=#1, every node/.style={transform shape}},
    highlight/.style={color=highlight#1}
}

% Styles des heatmap pour les moyennes
\pgfplotsset{
    meanheatmap/.style={
        colorbar, colormap name=SRON,
        view={0}{90},
        samples=100,
        domain=0:1,
        min=0, max=1,
        xlabel={\(P\)},
        ylabel={\(R\)},
    }
}

% Commands spécifiques
\NewDocumentCommand\shorturl{ O{https} O{://} m }{%
    \href{#1#2#3}{\nolinkurl{#3}}%
}

\DeclarePairedDelimiter\norm{\lVert}{\rVert}
\DeclarePairedDelimiter\abs{\lvert}{\rvert}
\DeclarePairedDelimiterX\compset[2]{\lbrace}{\rbrace}{#1\,\delimsize|\,#2}
\DeclarePairedDelimiterX\innprod[2]{\langle}{\rangle}{#1\,\delimsize|\,#2}

\DeclareMathOperator*\argmax{argmax}

% Easy column vectors \vcord{a,b,c} ou \vcord[;]{a;b;c}
% Here be black magic
\ExplSyntaxOn % chktex 1
	\NewDocumentCommand{\vcord}{O{,}m}{\vector_main:nnnn{p}{\\}{#1}{#2}}
	\NewDocumentCommand{\tvcord}{O{,}m}{\vector_main:nnnn{psmall}{\\}{#1}{#2}}
	\seq_new:N\l__vector_arg_seq
	\cs_new_protected:Npn\vector_main:nnnn #1 #2 #3 #4{
		\seq_set_split:Nnn\l__vector_arg_seq{#3}{#4}
		\begin{#1matrix}
			\seq_use:Nnnn\l__vector_arg_seq{#2}{#2}{#2}
		\end{#1matrix}
	}
\ExplSyntaxOff % chktex 1

\DeclareMathOperator{\TF}{TF}
\DeclareMathOperator{\IDF}{IDF}

\ExplSyntaxOn % chktex 1
    \DeclareExpandableDocumentCommand\eval{m}{\fp_eval:n{#1}}
\ExplSyntaxOff % chktex 1


% Fixes for pauses after an unveiled list
\newcommand{\itpause}{%
	\addtocounter{beamerpauses}{-1}%
	\pause % chktex 1
}


% ██████   ██████   ██████ ██    ██ ███    ███ ███████ ███    ██ ████████
% ██   ██ ██    ██ ██      ██    ██ ████  ████ ██      ████   ██    ██
% ██   ██ ██    ██ ██      ██    ██ ██ ████ ██ █████   ██ ██  ██    ██
% ██   ██ ██    ██ ██      ██    ██ ██  ██  ██ ██      ██  ██ ██    ██
% ██████   ██████   ██████  ██████  ██      ██ ███████ ██   ████    ██


\begin{document}
\pdfbookmark[2]{Title}{title}

\begin{frame}[plain]
	\titlepage % chktex 1
\end{frame}

% ███████ ██    ██ ███    ███
% ██      ██    ██ ████  ████
% ███████ ██    ██ ██ ████ ██
%      ██  ██  ██  ██  ██  ██
% ███████   ████   ██      ██

% TODO: exooooooos!

\section{Les maths en cinq secondes : hyperplans}
\begin{frame}[fragile]{Hyperplans}
    Dans un espace de dimension $n$, un \alert<1>{hyperplan} est un espace de dimension $n-1$
    \begin{overprint}
        \onslide<1>
            \begin{itemize}
                \item En dimension $3$, les hyperplans sont les \alert{plans}
                    \begin{figure}
                        \tikzset{external/export=true}
                        \tdplotsetmaincoords{70}{110}
                        \begin{tikzpicture}[tdplot_main_coords]
                            \node[coordinate] at (0, 0, 0) (O) {};
                            \draw[->] (O) -- (2, 0, 0);
                            \draw[->] (O) -- (0, 2, 0);
                            \draw (O) -- (0, 0, 1) node[coordinate] (passthrough) {};
                            \filldraw[draw=highlighta, fill=highlighta!30, fill opacity=0.75] (3, -3, 1) -- (3, 3, 1) -- (-3, 3, 1)  -- (-3, -3, 1) --cycle;
                            \draw[->] (passthrough) -- + (0, 0, 1);
                        \end{tikzpicture}
                        \caption{Hyperplan en dimension $3$}
                    \end{figure}
                \end{itemize}
        \onslide<2>
            \begin{itemize}
                \item En dimension $2$, les hyperplans sont les \alert{droites}
                    \begin{figure}
                        \tikzset{external/export=true}
                        \begin{tikzpicture}
                            \node[coordinate] at (0, 0) (O) {};
                            \draw[->] (-3, 0) -- (3, 0);
                            \draw[->] (0, -2) -- (0, 2);

                            \draw[highlighta, line width=1.5pt] (-3, -2) -- (3, 1);
                        \end{tikzpicture}
                        \caption{Hyperplan en dimension $2$}
                    \end{figure}
            \end{itemize}
        \onslide<3>
            \begin{itemize}
                \item En dimension $1$, les hyperplans sont les \alert{points}
                    \begin{figure}
                        \tikzset{external/export=true}
                        \begin{tikzpicture}
                            \node[coordinate] at (0, 0) (O) {};
                            \draw[line width=1.2pt] (-3, 0) -- (3, 0);

                            \fill[highlighta] (1, 0) circle[radius=2.2pt];
                        \end{tikzpicture}
                        \caption{Hyperplan en dimension $1$}
                    \end{figure}
            \end{itemize}
    \end{overprint}
\end{frame}

\begin{frame}[fragile]{Propriétés des hyperplans}
    Les hyperplans séparent les espaces en demi-espaces
    \begin{overprint}
        \onslide<1-3>
            \begin{itemize}
                \item En dimension $3$ : demi-espaces
                    \begin{figure}
                        \tikzset{external/export=true}
                        \tdplotsetmaincoords{70}{110}
                        \begin{tikzpicture}[tdplot_main_coords]
                            \node[coordinate] at (0, 0, 0) (O) {};
                            \draw[->] (O) -- (2, 0, 0);
                            \draw[->] (O) -- (0, 2, 0);
                            \draw (O) -- (0, 0, 1) node[coordinate] (passthrough) {};

                            % Lower half-space
                            \begin{scope}[draw=highlight5, fill=highlight5!30, fill opacity=0.6, visible on=3]
                                \filldraw (3, -3, -1) -- (3, 3, -1) -- (-3, 3, -1)  -- (-3, -3, -1) -- cycle;
                                \filldraw (-3, 3, -1) -- (-3, -3, -1) -- (-3, -3, 1) -- (-3, 3, 1) -- cycle;
                                \filldraw (3, -3, -1) -- (-3, -3, -1) -- (-3, -3, 1) -- (3, -3, 1) -- cycle;
                                \filldraw (3, -3, -1) -- (3, 3, -1) -- (3, 3, 1) -- (3, -3, 1) -- cycle;
                                \filldraw  (3, 3, -1) -- (-3, 3, -1) --  (-3, 3, 1) -- (3, 3, 1) -- cycle;
                            \end{scope}

                            \path[draw=highlighta, fill=highlighta!30, fill opacity=0.6] (3, -3, 1) -- (3, 3, 1) -- (-3, 3, 1)  -- (-3, -3, 1) -- cycle;
                            \draw[->] (passthrough) -- + (0, 0, 1);

                            % Upper half-space
                            \begin{scope}[draw=highlight5, fill=highlight5!30, fill opacity=0.6, visible on=2]
                                \filldraw (3, -3, 2) -- (3, 3, 2) -- (-3, 3, 2)  -- (-3, -3, 2) -- cycle;
                                \filldraw (-3, 3, 2) -- (-3, -3, 2) -- (-3, -3, 1) -- (-3, 3, 1) -- cycle;
                                \filldraw (3, -3, 2) -- (-3, -3, 2) -- (-3, -3, 1) -- (3, -3, 1) -- cycle;
                                \filldraw (3, -3, 2) -- (3, 3, 2) -- (3, 3, 1) -- (3, -3, 1) -- cycle;
                                \filldraw  (3, 3, 2) -- (-3, 3, 2) --  (-3, 3, 1) -- (3, 3, 1) -- cycle;
                            \end{scope}
                        \end{tikzpicture}
                        \caption{Demi-espaces en dimension $3$}
                    \end{figure}
                \end{itemize}
        \onslide<4-6>
            \begin{itemize}
                \item En dimension $2$ : demi-plans
                    \begin{figure}
                        \tikzset{external/export=true}
                        \begin{tikzpicture}
                            \node[coordinate] at (0, 0) (O) {};
                            \draw[->] (-3, 0) -- (3, 0);
                            \draw[->] (0, -2) -- (0, 2);

                            \path[draw=highlight5, fill=highlight5!30, fill opacity=0.75, visible on=5] (-3, -2) -- (3, 1) -- (3, 2) -- (-3, 2) -- cycle;
                            \path[draw=highlight5, fill=highlight5!30, fill opacity=0.75, visible on=6] (-3, -2) -- (3, 1) -- (3, -2) -- (-3, -2) -- cycle;

                            \draw[highlighta] (-3, -2) -- (3, 1);
                        \end{tikzpicture}
                        \caption{Demi-espaces en dimension $2$}
                    \end{figure}
            \end{itemize}
        \onslide<7-9>
            \begin{itemize}
                \item En dimension $1$ : demi-droites
                    \begin{figure}
                        \tikzset{external/export=true}
                        \begin{tikzpicture}
                            \node[coordinate] at (0, 0) (O) {};
                            \draw[line width=1.2pt] (-3, 0) -- (3, 0);
                            \draw[highlight5, line width=1.2pt, visible on=8] (1,0) -- (3, 0);
                            \draw[highlight5, line width=1.2pt, visible on=9] (1,0) -- (-3, 0);
                            \fill[highlighta] (1, 0) circle[radius=2pt];
                        \end{tikzpicture}
                        \caption{Demi-espaces en dimension $1$}
                    \end{figure}
            \end{itemize}
    \end{overprint}
\end{frame}

\section{Classifieurs linéaires}

\begin{frame}{Classifieurs linéaires}
	La propriété de séparation des hyperplans permet de construire des classifieurs
	\begin{figure}
	    \tikzset{external/export=true}
	    \begin{tikzpicture}
	        \draw[->] (-0.5,0) -- (5.25, 0);
	        \draw[->] (0, -0.5) -- (0, 4.25);

	        \foreach \x/\y/\c in {2/1/1,4/0/1,1/0/1,2/3/0,0/3/0,1/2/0,0/2/0}{
	            \path[alt=<1-2>{fill=highlight3}{fill=highlight\c}] (\x, \y) circle[radius=3pt];
	        }

	        \path[fill=highlight0!30, fill opacity=0.5, visible on={3-4}] (-0.5, -0.5)  -- (4.25, 4.25) -- (-0.5, 4.25) -- cycle;
	        \path[fill=highlight1!30, fill opacity=0.5, visible on={3-4}] (-0.5, -0.5)  -- (5.25, -0.5) -- (5.25, 4.25) -- (4.25, 4.25) -- cycle;
	        \draw[highlight2, visible on={2-}] (-0.5, -0.5)  -- (4.25, 4.25);
	    \end{tikzpicture}
	    \caption{Classification \only<5>{linéaire} en dimension $2$}
	\end{figure}
	\only<5>{On parle de \alert{classifieur linéaire}.}
\end{frame}
% TODO: figs
\begin{frame}{Classifieurs linéaires}
    Le principe des classifieurs linéaire est donc, étant donné un problème à deux classes
    \begin{itemize}
        \item Trouver un hyperplan qui sépare l'espace des attributs en demi-espaces contenant chacun tous les exemples d'entraînement d'une seule classe
        \item On classe les exemples de test dans la classe correspondant à leur demi-espace
    \end{itemize}

    Cela suppose de résoudre deux problèmes
    \begin{itemize}
        \item Existe-t-il un tel hyperplan ? (on dit dans ce cas que le problème est \emph{linéairement séparable})
        \item S'il en existe un, il en existe en général une infinité
            \begin{itemize}
                \item[→] Lequel choisir ?
            \end{itemize}
    \end{itemize}
\end{frame}

\begin{frame}[fragile]{Notion de marge}
    \only<1-7>{
        Intuitivement, pour espérer qu'un classifieur linéaire généralise bien, il faut qu'il sépare les classes le plus largement possible.
    }
    \begin{overprint}
        \onslide<1-3>
            \begin{figure}
                \tikzset{external/export=true}
                \begin{tikzpicture}
                    \path[clip] (-0.5, -0.5) rectangle (5.25, 4.25);
                    \draw[->] (-0.5,0) -- (5.25, 0);
                    \draw[->] (0, -0.5) -- (0, 4.25);

                    \foreach \x/\y/\c in {2/1/1,4/0/1,1/0/1,2/3/0,0/3/0,1/2/0,0/2/0}{
                        \path[fill=highlight\c] (\x, \y) circle[radius=3pt];
                    }

                    \begin{scope}[domain=-0.5:5.25]
                        \draw[highlight2, visible on=1] plot (\x, 1+0.75*\x);
                        \draw[highlight2, visible on=2]  plot (\x, \x-1);
                        \draw[highlight2, visible on=3] plot (\x, \x);
                    \end{scope}
                \end{tikzpicture}
                \caption{Largeurs de séparation d'hyperplans}
            \end{figure}
        \onslide<4-7>
            \vspace{-1\bigskipamount}
            \begin{columns}
                \column{0.45\textwidth}
                    \begin{figure}
                        \tikzset{external/export=true}
                        \begin{tikzpicture}[true scale=0.7]
                            \path[clip] (-0.5, -0.5) rectangle (5.25, 4.25);
                            \draw[->] (-0.5,0) -- (5.25, 0);
                            \draw[->] (0, -0.5) -- (0, 4.25);

                            \foreach \x/\y/\c in {2/1/1,4/0/1,1/0/1,2/3/0,0/3/0,1/2/0,0/2/0}{
                                \path[fill=highlight\c] (\x, \y) circle[radius=3pt];
                            }

                            \begin{scope}[domain=-0.5:5.25]
                                \draw[dashed, visible on={5-}] plot (\x, \x+1);
                                \draw[dashed, visible on={5-}]  plot (\x, \x-1);
                                \draw[highlight2, alt=<{4-5,7}>{}{opacity=0}] plot (\x, \x) node[right] {$H$};
                            \end{scope}

                            \draw[<->, visible on={6-}] (1, 2) -- (2, 1) node[midway, above right] {$M(H)$};
                        \end{tikzpicture}
                        \caption{Marge}
                    \end{figure}
                \column{0.45\textwidth}
                    \begin{block}{Définition (marge)}
                        Soit deux classes, $C$ et $C'$, on appelle \emph{marge} d'un hyperplan séparateur $H$ le nombre
                        \begin{equation}
                            M(H) = \min_{x∈C}d(x, H) + \min_{x'∈C'}d(x', H)
                        \end{equation}
                    \end{block}
            \end{columns}
			\uncover<6->{
	            On choisit $H$ pour lequel $M(H)$ est maximale sur l'ensemble d'entraînement.
	            \begin{itemize}
	                \item[→] D'où \emph{séparateur à \alert{vaste marge}}.
	            \end{itemize}
			}
    \end{overprint}
\end{frame}

\begin{frame}{Vecteurs de support}
    Les points $x$ et $x'$ pour lesquels les minima sont atteints sont appelés \textcolor{highlight3}{vecteurs de support}
    \begin{figure}
        \tikzset{external/export=true}
        \begin{tikzpicture}[true scale=0.7]
            \path[clip] (-0.5, -0.5) rectangle (5.25, 4.25);
            \draw[->] (-0.5,0) -- (5.25, 0);
            \draw[->] (0, -0.5) -- (0, 4.25);

            \foreach \x/\y/\c in {4/0/1,1/0/1,0/3/0,0/2/0}{
                \path[fill=highlight\c] (\x, \y) circle[radius=3pt];
            }

            \foreach \x/\y/\c in {2/1/1,1/0/1,2/3/0,1/2/0}{
                \path[fill=highlight\c] (\x, \y) circle[radius=3pt];
                \draw[highlight3] (\x, \y) circle[radius=5pt];
            }

            \begin{scope}[domain=-0.5:5.25]
                \draw[dashed] plot (\x, \x+1);
                \draw[dashed]  plot (\x, \x-1);
                \draw[highlight2] plot (\x, \x) node[right] {$H$};
            \end{scope}
        \end{tikzpicture}
        \caption{Vecteurs de support}
    \end{figure}
    On peut montrer que $H$ ne dépend que de la position des vecteurs de support.
\end{frame}

% TODO: cite
\begin{frame}{SVM}
    \emph{Support Vector Machines} : \emph{machines à vecteurs de support} ou \emph{Séparateur à Vaste Marge}

    \textbf{Dans Weka} classifiers > functions > SMO
    \pause

    \textbf{Espace de recherche} Ensemble des hyperplans affines de $ℝ^n$, où $n$ est le nombre d'attributs.
    \pause

    \textbf{Technique de recherche}
    \begin{itemize}
        \item Traditionnellement : problème d'optimisation quadratique
        \item Plus récemment : descente de sous-gradient, descente de gradient par coordonnées
        \begin{itemize}
            \item[→] Linéaire ou quasi-linéaire (au moins en pratique)
        \end{itemize}
    \end{itemize}
\end{frame}


\subsection{Raffinements}
% TODO: figs
\begin{frame}{Problèmes non-séparables}
    En général, les problèmes réels ne sont pas linéairement séparables.
    Plusieurs méthodes existent pour y adapter les SVM.
    \begin{description}
        \item[Marge douce] Modifier la définition de \emph{marge} pour tenir compte des exemples mal classés
        \item[\emph{Kernel trick}] Transformer le problème en un problème équivalent mais linéairement séparable
    \end{description}
\end{frame}

\begin{frame}{Marge douce}
    On remplace la fonction de marge $M$ par une fonction de la forme
    \begin{equation}
        L(H) = M(H) - C(H)
    \end{equation}
    où $C(H)$ est une pénalité pour les exemples mal classés par $H$.
    \begin{figure}
        \tikzset{external/export=true}
        \begin{tikzpicture}[true scale=0.7]
            \path[clip] (-0.5, -0.5) rectangle (5.25, 4.25);
            \draw[->] (-0.5,0) -- (5.25, 0);
            \draw[->] (0, -0.5) -- (0, 4.25);

            \foreach \x/\y/\c in {2/1/1,4/0/1,1/0/1,2/3/0,0/3/0,1/2/0,0/2/0}{
                \path[fill=highlight\c] (\x, \y) circle[radius=3pt];
            }
            \path[fill=highlight1, visible on={2-}] (1, 3) circle[radius=3pt];

            \begin{scope}[domain=-0.5:5.25]
                \draw[highlight2, visible on=3] plot (\x, \x+1) node[right] {$H$};
            \end{scope}
        \end{tikzpicture}
        \caption{Hyperplan à marge douce maximale}
    \end{figure}
    \only<3>{
        \vspace{-\bigskipamount}
        On choisit comme précédemment l'hyperplan $H$ qui maximise $L(H)$.
    }
\end{frame}

\begin{frame}[fragile]{\emph{Kernel trick}}
    L'\emph{astuce du noyau} consiste à transformer un problème non-linéairement séparable en un problème linéairement séparable dans un espace de dimension supérieure.

    \begin{figure}
        \tikzset{external/export=true}
        \begin{tikzpicture}[x=0.4cm, y=0.2cm]
            \draw[->] (65, 0) -- (90, 0);
            \draw[->, visible on={2-}] (65, 0) -- (65, 13);

            \begin{scope}[domain=65:90]
                \path[clip] (64, -0.5) rectangle (92, 12);
                \draw[highlight6, visible on={3-5}] plot (\x, {0.1*(\x-78)^2}) node [pos=0, right] {$y=ax²+bx+c$};
                \draw[highlight2, visible on={6-}] plot (\x, {-0.1*\x+11.61});
            \end{scope}

            \foreach \x/\c [count=\i] in {68/1,69/1,70/1,73/0,74/0,76/0,80/0,82/0,85/1,86/1} {
                \path[fill=highlight\c, visible on={1-4}] (\x, 0) node (base\i) {} circle[radius=3pt];
                \path[fill=highlight\c, visible on={4-}] (\x, {0.1*(\x-78)^2})  node (raised\i) {} circle[radius=3pt];
                \draw[->, visible on=4] (base\i) -- (raised\i);
            }
        \end{tikzpicture}
        \caption{Hyperplan séparateur en dimension supérieure}
    \end{figure}
    \vspace{-1\bigskipamount}
    \begin{overprint}
        \onslide<6>
            \begin{itemize}
                \item Les \emph{noyaux} en question sont des fonctions qui permettent d'éviter d'avoir à calculer explicitement les transformations pour passer en dimension supérieure.
            \end{itemize}
        \onslide<7>
            \begin{itemize}
                \item En réalité, le problème en dimension supérieure est souvent lui aussi non-linéairement séparable, et on combine en général le \emph{kernel trick} avec une marge douce.
            \end{itemize}
    \end{overprint}
\end{frame}

\begin{frame}{Propriétés}
    Les SVM sont des modèles très efficaces, qui jusqu'à récemment étaient systématiquement les plus performants.
    \begin{itemize}
        \item Une fois l'apprentissage fait, la classification est très rapide
        \item Les modèles sont peu gourmands en mémoire
        \item Les modèles marchent mieux que les autres modèles présentés (en général)
    \end{itemize}
    Avec quelques inconvénients
    \begin{itemize}
        \item Les modèles sont difficilement interprétables
        \item Le choix d'un noyau tient plus de l'artisanat que de la science
        \item Les algorithmes classiques ne permettent pas d'apprentissage incrémental.
    \end{itemize}
\end{frame}

%  █████  ██████  ██████  ███████ ███    ██ ██████  ██ ██   ██
% ██   ██ ██   ██ ██   ██ ██      ████   ██ ██   ██ ██  ██ ██
% ███████ ██████  ██████  █████   ██ ██  ██ ██   ██ ██   ███
% ██   ██ ██      ██      ██      ██  ██ ██ ██   ██ ██  ██ ██
% ██   ██ ██      ██      ███████ ██   ████ ██████  ██ ██   ██

\appendix
\pgfkeys{/metropolis/inner/sectionpage=simple}  % Avoid random errors with section page progressbar
\section{Annexes}
\pdfbookmark[2]{Remerciements}{acknowledgements}
\begin{frame}{Remerciements}
    Ce cours a été construit à partir du polycopié de cours \citetitle{tellier2017IntroductionFouilleTextes} \parencite{tellier2017IntroductionFouilleTextes} et des précieux conseils d'Isabelle Tellier que je ne saurais trop remercier pour sa confiance et son dévouement.
\end{frame}

\pdfbookmark[2]{Références}{references}
\begin{frame}[allowframebreaks]{References}
    \printbibliography[heading=none]
\end{frame}

\pdfbookmark[2]{Licence}{licence}
\begin{frame}{Licence}
    \begin{center}
        {\huge \ccby}
        \vfill
        This document is available under the terms of the Creative Commons Attribution 4.0 International License (CC BY 4.0) (\shorturl{creativecommons.org/licenses/by/4.0})

        Exceptions to the above statement are listed at {\small\shorturl{loicgrobol.github.io/intro-fouille-textes\#licences}}
        \vfill
        © 2020, Loïc Grobol <\shorturl[mailto][:]{loic.grobol@gmail.com}>

        \shorturl[http]{lattice.cnrs.fr/Grobol-Loic}
    \end{center}
\end{frame}

\end{document}
