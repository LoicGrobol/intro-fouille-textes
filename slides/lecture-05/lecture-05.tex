% LTeX: language=fr
% Copyright © 2019, Loïc Grobol <loic.grobol@gmail.com>
% This document is available under the terms of the Creative Commons Attribution 4.0 International License (CC BY 4.0) (https://creativecommons.org/licenses/by/4.0/)

\documentclass[../allslides.tex]{subfiles}

\begin{document}

\renewcommand\docdate{2021-02-18}  % chktex 8

\lecture{Représentation des données}{Cours 5}


\begin{frame}{Exercice}
	La tâche de \emph{résolution d'anaphores}\footnote{Ou en tout cas une version simplifiée} consiste à déterminer l'antécédent d'un pronom dans un texte.
	\vspace{\bigskipamount}

	\begin{overprint}
		\onslide<2-3> \alert<3>{Le père d'Aino} est heureux de vous annoncer que \alert{sa} fille se marie dimanche prochain
		\onslide<4-> Je n'ai pas pu faire rentrer \alert<5>{la poussette} dans \alert<6->{la valise} parce qu'\alert{elle} est trop \alt<4-5>{grande}{\textbf<6>{petite}}
	\end{overprint}

	\only<7>{
		Reformuler cette tâche comme une combinaison de tâches élémentaires.
	}
\end{frame}

\section{Attributs des documents}

\begin{frame}<-2>{Mots}
	Faisons \alt<1>{enfin de la}{un peu de} linguistique : comptons les \alert{mots} !
	\begin{itemize}
		\item Problème : qu'est-ce qu'un mot ?
			\begin{itemize}
				\item \emph{pomme de terre} ?
				\item \emph{y a-\textbf{t}-il} ?
				\item …
			\end{itemize}
		\item On a plutôt tendance à compter les \alert{tokens}
			\begin{itemize}
				\item « Séquences de caractères comprises entre deux séparateurs »
				\item[→] Problème : qu'est-ce qu'un séparateur ? \begin{itemize}
					\item « E-mail » vs. « Alsace–Lorraine »
					\item « […] kiwi. Hier […] » vs. « M. Martin »
					\item « Yann Le Bourdonnec » vs. « Félix le chat »
				\end{itemize}
				\item On délègue à un segmenteur (\emph{tokenizer})
			\end{itemize}
	\end{itemize}
\end{frame}

\begin{frame}<-2>[label=wordcols]{Des mots aux colonnes}
	Objectif : utiliser les \alert{fréquences} des mots comme attributs (colonnes) dans une représentation tabulaire.
	\pause
	\begin{itemize}
		\item<1-> Combien de colonnes faut-il ?
			\pause
				\begin{itemize}
					\item[→] \(K×M^β\) (loi de Heaps)
					\item Pour \emph{Ulysse}, environ \num{34000}
				\end{itemize}
		\item<3-> Il faut donc arriver à filtrer !
			% TODO: déroulé progressif avec exemples de colonnes
			\begin{itemize}
				\item<4-> Supprimer les \alert{mots vides} (\emph{stop words}) : \emph{le}, \emph{à}, \emph{mais}…
				\item<5-> Ne conserver certaines catégories spécifiques à la tâche (noms, adjectifs…)
				\item<6-> Supprimer les \alert{hapax}\footnote{Du grec \textit{ἅπαξ λεγόμενον}.} (les mots n'apparaissant qu'une seule fois)
				\item<7->[→] De façon générale, ne conserver qu'une \alert{bande de fréquences}
			\end{itemize}
	\end{itemize}
	\mbox{}  % For the footnote
\end{frame}

\begin{frame}[label=heapslawgraph]{Loi de Heaps}
	\begin{figure}
		\tikzset{external/export=true}
		\begin{tikzpicture}
			\begin{axis}[
				xmin=0, xmax=800000, ymin=0,
				width=0.9\textwidth,
				height=0.8\textheight,
				xlabel={Taille du corpus en nombre de mots},
				ylabel={Taille du vocabulaire},
				title={Loi de Heaps : \(y=K×x^β\)},
				scaled ticks=false,
				ticklabel style={/pgf/number format/1000 sep={\,}},
				xticklabel style={rotate=45, anchor=north east, yshift=-0.1em},
			]
				\addplot[accent, domain=0:800000, samples=1001] {40*x^0.5};
			\end{axis}
		\end{tikzpicture}
		\caption{Courbe d'une loi de Heaps}
	\end{figure}
\end{frame}

\againframe<4-7>{wordcols}

\begin{frame}[fragile]{Filtrer les hapax}
	\begin{figure}
		\tikzset{external/export=true}
		\begin{tikzpicture}
			\begin{axis}[
				title={\(y=\frac{K}{x}\)},
				xlabel={Rang},
				ylabel={Fréquence d'apparition},
				xmin=0, ymin=0, xmax=40, ymax=15,
				xtick=\empty,
				ytick=\empty,
				alt=<2>{extra y ticks={1}}{},
				width=0.9\textwidth,
				height=0.8\textheight,
			]
				\addplot[accent, domain=1:20, samples=501] {20/x};
				\addplot[accent, alt=<2>{dotted}{}, domain=20:40, samples=500] {1};
				\draw<2>[highlight6] (axis cs:0,1) -- (current axis.east|-{axis cs:0,1});
			\end{axis}
		\end{tikzpicture}
		\caption{\alt<2>{Filtrage des hapax sur la c}{C}ourbe de Zipf}
	\end{figure}
\end{frame}

\againframe<8->{wordcols}

\begin{frame}[fragile=singleslide]{Filtrer une bande de fréquences}
	\begin{figure}
		\tikzset{external/export=true}
		\begin{tikzpicture}
			\begin{axis}[
				title={\(y=\frac{K}{x}\)},
				xlabel={Rang},
				ylabel={Fréquence d'apparition},
				xmin=0, ymin=0, xmax=40, ymax=20,
				xtick=\empty,
				ytick={2, 15},
				yticklabels={\(a\), \(b\)},
				width=0.9\textwidth,
				height=0.8\textheight,
			]			
				\addplot[accent, dotted, domain=1:1.33, samples=333] {20/x};
				\addplot[accent, domain=1.33:10, samples=333] {20/x};
				\addplot[accent, dotted, domain=10:40, samples=333] {max(20/x,1)};
				\draw[highlight6] (axis cs:0,15) -- (current axis.east|-{axis cs:0,15});
				\draw[highlight6] (axis cs:0,2) -- (current axis.east|-{axis cs:0,2});
			\end{axis}
		\end{tikzpicture}
		\caption{Filtrage d'une bande de fréquences sur la courbe de Zipf}
	\end{figure}
\end{frame}

\begin{frame}{Mots}
	Pour réduire encore plus le nombre de colonnes
	\pause
	% TODO: déroulé progressif avec exemples de colonnes
	\begin{itemize}[<+->]
		\item Lemmatiser : \{\emph{est}, \emph{était}, \emph{suis}, \emph{êtes}, \emph{furent}…\} → \emph{est}
		\item Radicaliser : \{\emph{homme}, \emph{hommes}\} → \emph{homme}
		\item Fixer un lexique \emph{a priori}
			\begin{itemize}
				\item[→] Avec les dangers que ça comporte
			\end{itemize}
	\end{itemize}

	\pause
	À l'inverse, pour plus de détails, on peut utiliser des \(n\)-grammes de mots
	\begin{itemize}
		\item[→] Représentation plus riche, mais amplifie le problème de quantité
	\end{itemize}
\end{frame}

\begin{frame}{Traits linguistiques avancés}
	Si les ressource le permettent, on peut envisager d'autres attributs
	\pause
	\begin{itemize}[<+->]
		\item Avec un étiqueteur morphosyntaxique : POS, \(n\)-grammes de POS
		\item Avec un parser : largeur/profondeur/taille moyenne des arbres/constituants, paires gouverneur-gouverné, paires nœud-relation…
		\item Avec des analyseurs sémantiques : rôles
		\item Avec des bases de connaissances : présence de certains concepts
		\item …
	\end{itemize}
	\itpause
	Mais on introduit alors des dépendances vis-à-vis des ressources, ce qui multiplie les causes d'erreurs.
\end{frame}


% ███    ███ ███████  █████  ███████ ██    ██ ██████  ███████
% ████  ████ ██      ██   ██ ██      ██    ██ ██   ██ ██
% ██ ████ ██ █████   ███████ ███████ ██    ██ ██████  █████
% ██  ██  ██ ██      ██   ██      ██ ██    ██ ██   ██ ██
% ██      ██ ███████ ██   ██ ███████  ██████  ██   ██ ███████

\section{Valuation des attributs}
\begin{frame}<1>[label=attrval]{Valuation des attributs}
	Avoir choisi les attributs à utiliser pour une classification ne suffit pas : encore faut-il leur donner une \alert{valeur}.

	\textbf{Note} : ici et dans la suite, on considère que toutes les colonnes sont de même nature

	\begin{description}[*]
		\item<2->[Booléenne] Vrai/Faux (\(0\)/\(1\)) suivant que l'attribut est ou n'est pas présent dans le document
		\item<3->[Occurrences] Le nombre \(n_{i,j}\) de fois qu'apparaît l'attribut \(a_j\) dans le document \(t_i\)
		% FIXME: Commencer par faire un exemple, puis écrire avec …, puis seulement passer à Σ
		\item<4->[Fréquences normalisées par ligne] (ici pour la norme \(1\))
			\begin{equation}
				\frac{\mathnode{freqfracnum}{n_{i,j}}}{\mathnode{freqfracden}{\sum_k n_{i,k}}}
			\end{equation}
	\end{description}
\end{frame}

\begin{frame}<1>[label=reprexpls]{Représentations}
	\begin{overprint}
		\onslide<1>
			\begin{figure}
				\begin{tabular}{c*{5}{c}c}
					\toprule
					\textbf{Texte}	& {\textbf{\#chaton}}	& {\textbf{\#traque}}	& {\textbf{\#kiwi}}	& {\textbf{\#volcan}}	& {\textbf{\#marron}}	& {…}\\
					\midrule
					\(t_1\)	& ?	& ?	& ?	& ?	& ?	& …\\
					\(t_2\)	& ?	& ?	& ?	& ?	& ?	& …\\
					\(⋮\)	& \(⋮\)	& \(⋮\)	& \(⋮\)	& \(⋮\)	& \(⋮\)	& …\\
					\(t_m\)	& ?	& ?	& ?	& ?	& ?	& …\\
					\bottomrule
				\end{tabular}
				\caption{Quelle valeurs donner si on choisit les mots comme attributs ?}
			\end{figure}
		\onslide<2>
			\begin{figure}
				\begin{tabular}{c*{5}{c}c}
					\toprule
					\textbf{Texte}	& {\textbf{\#chaton}}	& {\textbf{\#traque}}	& {\textbf{\#kiwi}}	& {\textbf{\#volcan}}	& {\textbf{\#marron}}	& {…}\\
					\midrule
					\(t_1\)	& 1	& 1	& 0	& 1	& 0	& …\\
					\(t_2\)	& 1	& 1	& 0	& 0	& 1	& …\\
					\(⋮\)	& \multicolumn{1}{c}{\(⋮\)}	& \multicolumn{1}{c}{\(⋮\)}	& \multicolumn{1}{c}{\(⋮\)}	& \multicolumn{1}{c}{\(⋮\)}	& \multicolumn{1}{c}{\(⋮\)}	& …\\
					\(t_m\)	& 1	& 1	& 1	& 1	& 0	& …\\
					\bottomrule
				\end{tabular}
				\caption{Représentation pour des valeurs booléennes}
			\end{figure}
		\onslide<3>
			\begin{figure}
				\begin{tabular}{c*{5}{S[table-format=4.0]}c}
					\toprule
					\textbf{Texte}	& {\textbf{\#chaton}}	& {\textbf{\#traque}}	& {\textbf{\#kiwi}}	& {\textbf{\#volcan}}	& {\textbf{\#marron}}	& {…}\\
					\midrule
					\(t_1\)	& 150	& 12	& 0	& 1	& 0	& …\\
					\(t_2\)	& 2713	& 20	& 0	& 0	& 5	& …\\
					\(⋮\)	& \multicolumn{1}{c}{\(⋮\)}	& \multicolumn{1}{c}{\(⋮\)}	& \multicolumn{1}{c}{\(⋮\)}	& \multicolumn{1}{c}{\(⋮\)}	& \multicolumn{1}{c}{\(⋮\)}	& …\\
					\(t_m\)	& 1	& 2	& 1	& 12	& 0	& …\\
					\bottomrule
				\end{tabular}
				\caption{Représentation pour des nombres d'occurrences}
			\end{figure}
		\onslide<4>
			\begin{figure}
				\begin{tabular}{c*{5}{S}c}
					\toprule
					\textbf{Texte}	& {\textbf{\#chaton}}	& {\textbf{\#traque}}	& {\textbf{\#kiwi}}	& {\textbf{\#volcan}}	& {\textbf{\#marron}}	& {…}\\
					\midrule
					\(t_1\)	& 0.150	& 0.012	& 0	& 0.001	& 0	& …\\
					\(t_2\)	& 0.217	& 0.001	& 0	& 0	& 0.000	& …\\
					\(⋮\)	& \multicolumn{1}{c}{\(⋮\)}	& \multicolumn{1}{c}{\(⋮\)}	& \multicolumn{1}{c}{\(⋮\)}	& \multicolumn{1}{c}{\(⋮\)}	& \multicolumn{1}{c}{\(⋮\)}	& …\\
					\(t_m\)	& 0.001	& 0.002	& 0.001	& 0.012	& 0	& …\\
					\bottomrule
				\end{tabular}
				\caption{Représentation pour des fréquences relatives}
			\end{figure}
	\end{overprint}
\end{frame}

\againframe<1-2>{attrval}
\againframe<2>{reprexpls}
\againframe<2-3>{attrval}
\againframe<3>{reprexpls}
\againframe<3-4>{attrval}
\againframe<4>{reprexpls}


% TODO: add exemples

\begin{frame}{Le problème des occurrences}
	Il y a un problème avec les comptes d'occurrences et les fréquences relatives.

	\pause

	Même avec les meilleurs filtres du monde, il va rester des mots dont la présence n'est pas très \alert{informative}.

	\pause

	Par exemple, dans un corpus de recette, savoir que le mot \emph{ingrédients} apparaît dans un texte n'apporte aucune information : il est \alert{présent dans tous les textes}.

	\pause

	Ça nous fait une colonne inutile, et pire : une colonne dont la valeur va nous distraire des valeurs des autres colonnes.

	\pause

	On va donc chercher à supprimer les colonnes de ce type, ou au pire à \alert{réduire fortement leurs poids}.
\end{frame}

% TODO: add cites
\begin{frame}<1>[label=tfidfdef]{TF.IDF}
	\only<-2>{On veut pouvoir \alert{pondérer} l'importance d'un terme (mot, caractère, \(n\)-gramme…) en fonction de sa \alert{spécificité} (au sens large) pour un document.}
	
	\pause
	\begin{itemize}
		\item \emph{\textbf{T}erm \textbf{F}requency × \textbf{I}nverse \textbf{D}ocument \textbf{F}requency}
		\item \(\TF(i,j) = v_{i,j}\) ou \(\TF(i,j) = \frac{v_{i,j}}{\sum_j v_{i,j}}\)
		\item
			\only<1>{
				\begin{equation}
					\IDF(j) = \log\mleft(\frac{m}{\#\compset{i}{v_{i,j}>0}}\mright)
				\end{equation}
			}
			\only<2>{
				\begin{equation}
					\IDF(j) = \log\frac{\overbrace{m}^{\mathclap{\text{nombre total de documents}}}}{\underbrace{\#\compset{i}{v_{i,j}>0}}_{\mathclap{\text{nombre de documents contenant \(a_j\)}}}}
				\end{equation}
			}
			\only<3->{\(\IDF(j) = \log\frac{m}{\#\compset{i}{v_{i,j}>0}}\)}
		\only<3->{
			\item Intuitivement \(\IDF\) mesure la rareté d'un attribut, donc (on espère !) son informativité
			\item « Si \(a_j\) est présent dans tous les textes, savoir qu'il est présent dans \(t_i\) ne donne pas d'information sur \(t_i\) »
			\item Permet un filtrage moins arbitraire des \emph{stop words}
			\item \alt<4>{\(\log\) donne à \(\IDF\) la forme d'une \emph{surprise} au sens de Shannon}{\(\log\) est une façon pratique de modérer les effets de cette pondération : on devrait parler de \(\log\)-\(\IDF\)}
		}
	\end{itemize}
\end{frame}

\begin{frame}[label=textvalnotations]{Notations}
	\begin{overprint}
		\onslide<1>
			\begin{figure}
				\begin{tabular}{c*{5}{S[table-format=4.0]}c}
					\toprule
					\textbf{Texte}	& {\textbf{\#chaton}}	& {\textbf{\#traque}}	& {\textbf{\#kiwi}}	& {\textbf{\#volcan}}	& {\textbf{\#marron}}	& {…}\\
					\midrule
					\(t_1\)	& 150	& 12	& 0	& 1	& 0	& …\\
					\(t_2\)	& 2713	& 20	& 0	& 0	& 5	& …\\
					\(⋮\)	& \multicolumn{1}{c}{\(⋮\)}	& \multicolumn{1}{c}{\(⋮\)}	& \multicolumn{1}{c}{\(⋮\)}	& \multicolumn{1}{c}{\(⋮\)}	& \multicolumn{1}{c}{\(⋮\)}	& …\\
					\(t_m\)	& 1	& 2	& 1	& 12	& 0	& …\\
					\bottomrule
				 \end{tabular}
			\end{figure}
		\onslide<2>
			\begin{figure}
				\begin{tabular}{c*{4}{c}}
					\toprule
					\textbf{Texte}	& \(a_1\)	& \(a_2\)	& \(…\)	& \(a_n\)\\
					\midrule
					\(t_1\)	& \(v_{1,1}\)	& \(v_{1,2}\)	& \(…\) & \(v_{1,n}\)\\
					\(t_2\)	& \(v_{2,1}\)	& \(v_{2,2}\)	& \(…\) & \(v_{2,n}\)\\
					\(⋮\)	& \(⋮\)	& \(⋮\)	& 	& \(⋮\)\\
					\(t_m\)	& \(v_{m,1}\)	& \(v_{m,2}\)	& \(…\) & \(v_{m,n}\)\\
					\bottomrule
				 \end{tabular}
			\end{figure}
	\end{overprint}

	\only<2>{
		\begin{itemize}
			\item Les textes sont notés \(t_1\), \(t_2\), …, \(t_m\)
			\item Les attributs (par exemple les mots du vocabulaire) sont notés \(a_1\), \(a_2\), …, \(a_n\)
			\item La valeur du \(j\)-ème attribut pour le \(i\)-ème texte est notée \(v_{i,j}\) (ligne, colonne). Ainsi pour des fréquences absolues, \(v_{i,j}\) est le nombre d'occurrences du \(j\)-ème mot dans le \(i\)-ème texte.
		\end{itemize}
	}
\end{frame}

\againframe<1->{tfidfdef}

\begin{frame}{IDF, probabilités et surprise}
	Dans un corpus de \num{1000} textes, \num{55} contiennent le mot \emph{kiwi}.
	\pause

	On choisit un texte au hasard sans en favoriser aucun. Serait-il surprenant que le texte choisi contienne le mot \emph{kiwi} ?
\end{frame}

\begin{frame}<-2>[label=fivesecproba]{Les maths en cinq secondes : probabilités}
	Objectif : on veut une \alert{mesure} des chances qu'un évènement se produise, qu'on appelle \alert{probabilité} de l'évènement.
	
	\pause
	On impose les contraintes suivantes :
	\begin{itemize}
		\item La probabilité d'un évènement est un nombre compris entre \num{0} (pour un évènement impossible) et \num{1} (pour un évènement certain)
		\item Si deux évènements sont \alert{incompatibles}, la probabilité qu'un des deux se produise est la somme de leurs probabilités respectives
	\end{itemize}
	\itpause
	Alors, soit un corpus de \(n\) documents et un ensemble de \(k\) documents dans ce corpus.
	Si on choisit au hasard et de façon non-biaisée un texte du corpus, la probabilité que le texte choisi soit un de ces \(k\) textes est \(\frac{k}{n}\).
\end{frame}

\begin{frame}{IDF, probabilités et surprise}
	Dans un corpus de \num{1000} textes, \num{55} contiennent le mot \emph{kiwi}.
	
	On considère les évènements \(Aᵢ\) « avoir choisi le texte \(i\) ».
	\pause
	\begin{enumerate}
		\item Combien y a-t-il de \(Aᵢ\) ?  \only<3->{→ \num{1000}}
		\item Quelle est la probabilité qu'au moins un des \(Aᵢ\) se réalise ?  \only<4->{→ \num{1}}
		\item Pour un \(i\) donné, quelle est la probabilité (notée \(P(Aᵢ)\)) de \(Aᵢ\) ?  \only<5->{→ \(\frac{1}{1000}\)}
		\item En déduire la probabilité que le texte choisi contienne le mot \emph{kiwi}  \only<6->{→ \(\frac{55}{1000}\)}
	\end{enumerate}
\end{frame}

\againframe<2->{fivesecproba}

\begin{frame}{IDF, probabilités et surprise}
	La notion de \alert{surprise} vient de la théorie de l'information. La surprise d'un évènement \(A\) mesure le caractère inattendu de la réalisation d'un évènement.

	On la définit par
	\begin{equation}
		I(A) = -\log P(A)
	\end{equation}

	Intuitivement, plus la probabilité d'un évènement est faible, plus sa surprise est grande.
\end{frame}

\begin{frame}{Surprise}
	\begin{figure}
		\tikzset{external/export=true}
		\begin{tikzpicture}
			\begin{axis}[
				title={Surprise : \(y=-\log x\)},
				xlabel={\(P(A)\)},
				ylabel={\(I(A)\)},
				xmin=0, xmax=1, ymin=0,
				width=0.9\textwidth,
				height=0.8\textheight,
			]
				\addplot[highlighta, domain=0:1, samples=1001] {-ln(x)};
			\end{axis}
		\end{tikzpicture}
		\caption{Suprise d'un évènement en fonction de sa probabilité}
	\end{figure}
\end{frame}

\begin{frame}{IDF}
	On a pour le mot \(m_j\)
	\begin{equation}
		\IDF(m_j) = \log\mleft(\frac{m}{\#\compset{i}{n_{i,j}>0}}\mright) = -\log\mleft(\frac{\#\compset{i}{n_{i,j}>0}}{m}\mright) = I(…)
	\end{equation}

	Autrement dit : l'\(\IDF\) du mot \emph{kiwi} est la surprise de l'événement \enquote{un texte choisi au hasard dans le corpus contient le mot \emph{kiwi}}.
\end{frame}


%  █████  ██████  ██████  ███████ ███    ██ ██████  ██ ██   ██
% ██   ██ ██   ██ ██   ██ ██      ████   ██ ██   ██ ██  ██ ██
% ███████ ██████  ██████  █████   ██ ██  ██ ██   ██ ██   ███
% ██   ██ ██      ██      ██      ██  ██ ██ ██   ██ ██  ██ ██
% ██   ██ ██      ██      ███████ ██   ████ ██████  ██ ██   ██

\ifSubfilesClassLoaded{
	\appendix
	\pgfkeys{/metropolis/inner/sectionpage=simple}  % Avoid random errors with section page progressbar
	\section{Annexes}
	\pdfbookmark[2]{Remerciements}{acknowledgements}
	\begin{frame}{Remerciements}
		Ce cours a été construit à partir du polycopié de cours \citetitle{tellier2017IntroductionFouilleTextes} \parencite{tellier2017IntroductionFouilleTextes} et des précieux conseils d'Isabelle Tellier que je ne saurais trop remercier pour sa confiance et son dévouement.
	\end{frame}

	\pdfbookmark[2]{Références}{references}
	\begin{frame}[allowframebreaks]{References}
		\printbibliography[heading=none]
	\end{frame}

	\pdfbookmark[2]{Licence}{licence}
	\begin{frame}{Licence}
		\begin{center}
			{\huge \ccby}
			\vfill
			This document is available under the terms of the Creative Commons Attribution 4.0 International License (CC BY 4.0) (\shorturl{creativecommons.org/licenses/by/4.0})

			Exceptions to the above statement are listed at {\small\shorturl{loicgrobol.github.io/intro-fouille-textes\#licences}}
			\vfill
			© 2019, Loïc Grobol <\shorturl[mailto][:]{loic.grobol@gmail.com}>

			\shorturl[http]{lattice.cnrs.fr/Grobol-Loic}
		\end{center}
	\end{frame}
}{}

\end{document}
