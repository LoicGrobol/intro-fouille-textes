% LTeX: language=fr
% Copyright © 2019, Loïc Grobol <loic.grobol@gmail.com>
% This document is available under the terms of the Creative Commons Attribution 4.0 International License (CC BY 4.0) (https://creativecommons.org/licenses/by/4.0/)

\documentclass[../allslides.tex]{subfiles}

\begin{document}

\renewcommand\docdate{2021-02-18}  % chktex 8

\lecture{Représentation des données}{Cours 5}


\begin{frame}{Exercice}
	La tâche de \emph{résolution d'anaphores}\footnote{Ou en tout cas une version simplifiée} consiste à déterminer l'antécédent d'un pronom dans un texte.
	\vspace{\bigskipamount}

	\begin{overprint}
		\onslide<2-3> \alert<3>{Le père d'Aino} est heureux de vous annoncer que \alert{sa} fille se marie dimanche prochain
		\onslide<4-> Je n'ai pas pu faire rentrer \alert<5>{la poussette} dans \alert<6->{la valise} parce qu'\alert{elle} est trop \alt<4-5>{grande}{\textbf<6>{petite}}
	\end{overprint}

	\only<7>{
		Reformuler cette tâche comme une combinaison de tâches élémentaires.
	}
\end{frame}

\section{Attributs des documents}

\begin{frame}<-2>{Mots}
	Faisons \alt<1>{enfin de la}{un peu de} linguistique : comptons les \alert{mots} !
	\begin{itemize}
		\item Problème : qu'est-ce qu'un mot ?
			\begin{itemize}
				\item \emph{pomme de terre} ?
				\item \emph{y a-\textbf{t}-il} ?
				\item …
			\end{itemize}
		\item On a plutôt tendance à compter les \alert{tokens}
			\begin{itemize}
				\item « Séquences de caractères comprises entre deux séparateurs »
				\item[→] Problème : qu'est-ce qu'un séparateur ? \begin{itemize}
					\item « E-mail » vs. « Alsace–Lorraine »
					\item « […] kiwi. Hier […] » vs. « M. Martin »
					\item « Yann Le Bourdonnec » vs. « Félix le chat »
				\end{itemize}
				\item On délègue à un segmenteur (\emph{tokenizer})
			\end{itemize}
	\end{itemize}
\end{frame}

\begin{frame}<-2>[label=wordcols]{Des mots aux colonnes}
	Objectif : utiliser les \alert{fréquences} des mots comme attributs (colonnes) dans une représentation tabulaire.
	\pause
	\begin{itemize}
		\item<+-> Combien de colonnes faut-il ?
			\pause
			\begin{itemize}
				\item[→] $K×M^β$ (loi de Heaps)
				\item Pour \emph{Ulysse}, environ \num{34000}
			\end{itemize}
		\item<+-> Il faut donc arriver à filtrer !
			% TODO: déroulé progressif avec exemples de colonnes
			\begin{itemize}[<+->]
				\item Supprimer les \alert{mots vides} (\emph{stop words}) : \emph{le}, \emph{à}, \emph{mais}…
				\item Ne conserver certaines catégories spécifiques à la tâche (noms, adjectifs…)
				\item Supprimer les \alert{hapax} (mots n'apparaissant qu'un seule fois)
				\item[→] De façon générale, ne conserver qu'une \alert{bande de fréquences}
			\end{itemize}
	\end{itemize}
\end{frame}

\begin{frame}{Loi de Heaps}
	\begin{figure}
		\tikzset{external/export=true}
		\begin{tikzpicture}
			\begin{axis}[
				title={Loi de Heaps : $y=K×x^β$},
				xlabel={Taille de la collection},
				ylabel={Taille du vocabulaire},
				xmin=0, xmax=800000, ymin=0,
				width=0.9\textwidth,
				height=0.8\textheight,
				scaled ticks=false,
				ticklabel style={/pgf/number format/.cd, 1000 sep={\,}},
			]
				\addplot[highlighta, domain=0:800000, samples=1001] {40*x^0.5};
			\end{axis}
		\end{tikzpicture}
		\caption{Courbe d'une loi de Heaps}
	\end{figure}
\end{frame}

\againframe<4-7>{wordcols}

\begin{frame}[fragile]{Filtrer les hapax}
	\begin{figure}
		\tikzset{external/export=true}
		\begin{tikzpicture}
			\begin{axis}[
				title={$y=\frac{K}{x}$},
				xlabel={Rang},
				ylabel={Fréquence d'apparition},
				xmin=0, ymin=0, xmax=40, ymax=15,
				xtick=\empty,
				ytick=\empty,
				alt=<2>{extra y ticks={1}}{},
				width=0.9\textwidth,
				height=0.8\textheight,
			]
				\addplot[highlighta, domain=1:20, samples=501] {20/x};
				\addplot[highlighta, alt=<2>{dotted}{}, domain=20:40, samples=500] {1};
				\draw<2>[highlight6] (axis cs:0,1) -- (current axis.east|-{axis cs:0,1});
			\end{axis}
		\end{tikzpicture}
		\caption{\alt<2>{Filtrage des hapax sur la c}{C}ourbe de Zipf}
	\end{figure}
\end{frame}

\againframe<8->{wordcols}

\begin{frame}[fragile=singleslide]{Filtrer une bande de fréquences}
	\begin{figure}
		\tikzset{external/export=true}
		\begin{tikzpicture}
			\begin{axis}[
				title={$y=\frac{K}{x}$}
				xlabel={Rang},
				ylabel={Fréquence d'apparition},
				xmin=0, ymin=0, xmax=40, ymax=20,
				 xtick=\empty,
				 ytick={2, 15},
				 yticklabels={$a$, $b$},
				 width=0.9\textwidth,
				 height=0.8\textheight,
			]
				\addplot[highlighta, dotted, domain=1:1.33, samples=333] {20/x};
				\addplot[highlighta, domain=1.33:10, samples=333] {20/x};
				\addplot[highlighta, dotted, domain=10:40, samples=333] {max(20/x,1)};
				\draw[highlight6] (axis cs:0,15) -- (current axis.east|-{axis cs:0,15});
				\draw[highlight6] (axis cs:0,2) -- (current axis.east|-{axis cs:0,2});
			\end{axis}
		\end{tikzpicture}
		\caption{Filtrage d'une bande de fréquences sur la courbe de Zipf}
	\end{figure}
\end{frame}

\begin{frame}{Mots}
	Pour réduire encore plus le nombre de colonnes
	\pause
	% TODO: déroulé progressif avec exemples de colonnes
	\begin{itemize}[<+->]
		\item Lemmatiser : \{\emph{est}, \emph{était}, \emph{suis}, \emph{êtes}, \emph{furent}…\} → \emph{est}
		\item Radicaliser : \{\emph{homme}, \emph{hommes}\} → \emph{homme}
		\item Fixer un lexique \emph{a priori}
			\begin{itemize}
				\item[→] Avec les dangers que ça comporte
			\end{itemize}
	\end{itemize}

	\pause
	À l'inverse, pour plus de détails, on peut utiliser des $n$-grammes de mots
	\begin{itemize}
		\item[→] Représentation plus riche, mais amplifie le problème de quantité
	\end{itemize}
\end{frame}

\begin{frame}{Traits linguistiques avancés}
	Si les ressource le permettent, on peut envisager d'autres attributs
	\pause
	\begin{itemize}[<+->]
		\item Avec un étiqueteur morphosyntaxique : POS, $n$-grammes de POS
		\item Avec un parser : largeur/profondeur/taille moyenne des arbres/constituants, paires gouverneur-gouverné, paires nœud-relation…
		\item Avec des analyseurs sémantiques : rôles
		\item Avec des bases de connaissances : présence de certains concepts
		\item …
	\end{itemize}
	\pause
	Mais on introduit alors des dépendances vis-à-vis des ressources, ce qui multiplie les causes d'erreurs.
\end{frame}


% ███    ███ ███████  █████  ███████ ██    ██ ██████  ███████
% ████  ████ ██      ██   ██ ██      ██    ██ ██   ██ ██
% ██ ████ ██ █████   ███████ ███████ ██    ██ ██████  █████
% ██  ██  ██ ██      ██   ██      ██ ██    ██ ██   ██ ██
% ██      ██ ███████ ██   ██ ███████  ██████  ██   ██ ███████

\section{Valuation des attributs}
\begin{frame}[label=attrval]{Valuation des attributs}
	Avoir choisi les attributs à utiliser pour une classification ne suffit pas : encore faut-il leur donner une \alert{valeur}
	\pause
	\begin{description}[*]
		\item<+->[Booléenne] Vrai/Faux ($0$/$1$) suivant que l'attribut est ou n'est pas présent dans le document
		\item<+->[Occurrences] Le nombre $n_{i,j}$ de fois qu'apparaît l'attribut $a_j$ dans le document $t_i$
		% FIXME: Commencer par faire un exemple, puis écrire avec …, puis seulement passer à Σ
		\item<+->[Fréquences normalisées par ligne] (ici pour la norme $1$)
			\begin{equation}
				\frac{\mathnode{freqfracnum}{n_{i,j}}}{\mathnode{freqfracden}{\sum_k n_{i,k}}}
			\end{equation}
	\end{description}
	\pause
	\textbf{Note} : ici et dans la suite, on considère que toutes les colonnes sont de même nature
\end{frame}

% TODO: peut-être pas nécessaire à ce stade d'introduir les notations et le tableau
% d'exemple pourrait être remonté
\begin{frame}[fragile]{Notations}
	\begin{overprint}
		\onslide<1>
			\begin{figure}
				\begin{tabu} to 0.9\linewidth {X[m, c]|*{4}{X[m, c]}}
							& je     & traque     & $…$ & kiwi\\
					\hline
					$t_1$   & $150$ & $12$ & $…$ & $0$\\
					$t_2$   & $2713$ & $20$ & $…$ & $0$\\
					$⋮$    & $⋮$      & $⋮$       &     & $⋮$\\
					$t_m$   & $1$ & $1$ & $…$ & $1$\\
				 \end{tabu}
			\end{figure}
		\onslide<2>
			\begin{figure}
				\begin{tabu} to 0.9\linewidth {X[m, c]|*{4}{X[m, c]}}
							& $a_1$     & $a_2$     & $…$ & $a_n$\\
					\hline
					$t_1$   & $v_{1,1}$ & $v_{1,2}$ & $…$ & $v_{1,n}$\\
					$t_2$   & $v_{2,1}$ & $v_{2,2}$ & $…$ & $v_{2,n}$\\
					$⋮$    & $⋮$      & $⋮$       &     & $⋮$\\
					$t_m$   & $v_{m,1}$ & $v_{m,2}$ & $…$ & $v_{m,n}$\\
				 \end{tabu}
			\end{figure}
	\end{overprint}

	\only<2>{
		\begin{itemize}
			\item Les textes sont notés $t_1$, $t_2$, …, $t_m$
			\item Les attributs sont notés $a_1$, $a_2$, …, $a_n$
			\item La valeur du $j$-ème attribut pour le $i$-ème texte est notée $v_{i,j}$ (ligne, colonne)
		\end{itemize}
	}
\end{frame}

% TODO: add exemples

% TODO: ajouter une transition
% TODO: add cites
\begin{frame}{TF.IDF}
	\only<-2>{On veut pouvoir \alert{pondérer} l'importance d'un terme (mot, caractère, $n$-gramme…) en fonction de sa \alert{spécificité} (au sens large) pour un document.}

	\begin{itemize}
		\item \emph{\textbf{T}erm \textbf{F}requency $×$ \textbf{I}nverse \textbf{D}ocument \textbf{F}requency}
		\item $\TF(i,j) = n_{i,j}$ ou $\TF(i,j) = \frac{n_{i,j}}{\sum_k n_{i,j}}$
		\item
			\only<1>{
				\begin{equation}
					\IDF(j) = \log\mleft(\frac{m}{\#\compset{i}{n_{i,j}>0}}\mright)
				\end{equation}
			}
			\only<2>{
				\begin{equation}
					\IDF(j) = \log\frac{\overbrace{m}^{\mathclap{\text{nombre total de documents}}}}{\underbrace{\#\compset{i}{n_{i,j}>0}}_{\mathclap{\text{nombre de documents contenant $a_j$}}}}
				\end{equation}
			}
			\only<3->{$\IDF(j) = \log\frac{m}{\#\compset{i}{n_{i,j}>0}}$}
		\only<3->{
			\item Intuitivement $\IDF$ mesure la rareté d'un attribut, donc (on espère !) son informativité
			\item « Si $a_j$ est présent dans tous les textes, savoir qu'il est présent dans $t_i$ ne donne pas d'information sur $t_i$ »
			\item Permet un filtrage moins arbitraire des \emph{stop words}
			\item \alt<4>{$\log$ donne à $\IDF$ la forme d'une \emph{surprise} au sens de Shannon}{$\log$ est une façon pratique de modérer les effets de cette pondération : on devrait parler de $\log$-$\IDF$}
		}
	\end{itemize}
\end{frame}


%  █████  ██████  ██████  ███████ ███    ██ ██████  ██ ██   ██
% ██   ██ ██   ██ ██   ██ ██      ████   ██ ██   ██ ██  ██ ██
% ███████ ██████  ██████  █████   ██ ██  ██ ██   ██ ██   ███
% ██   ██ ██      ██      ██      ██  ██ ██ ██   ██ ██  ██ ██
% ██   ██ ██      ██      ███████ ██   ████ ██████  ██ ██   ██

\ifSubfilesClassLoaded{
	\appendix
	\pgfkeys{/metropolis/inner/sectionpage=simple}  % Avoid random errors with section page progressbar
	\section{Annexes}
	\pdfbookmark[2]{Remerciements}{acknowledgements}
	\begin{frame}{Remerciements}
		Ce cours a été construit à partir du polycopié de cours \citetitle{tellier2017IntroductionFouilleTextes} \parencite{tellier2017IntroductionFouilleTextes} et des précieux conseils d'Isabelle Tellier que je ne saurais trop remercier pour sa confiance et son dévouement.
	\end{frame}

	\pdfbookmark[2]{Références}{references}
	\begin{frame}[allowframebreaks]{References}
		\printbibliography[heading=none]
	\end{frame}

	\pdfbookmark[2]{Licence}{licence}
	\begin{frame}{Licence}
		\begin{center}
			{\huge \ccby}
			\vfill
			This document is available under the terms of the Creative Commons Attribution 4.0 International License (CC BY 4.0) (\shorturl{creativecommons.org/licenses/by/4.0})

			Exceptions to the above statement are listed at {\small\shorturl{loicgrobol.github.io/intro-fouille-textes\#licences}}
			\vfill
			© 2019, Loïc Grobol <\shorturl[mailto][:]{loic.grobol@gmail.com}>

			\shorturl[http]{lattice.cnrs.fr/Grobol-Loic}
		\end{center}
	\end{frame}
}{}

\end{document}
