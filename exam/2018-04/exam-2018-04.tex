\RequirePackage{xparse}
% Settings
\NewDocumentCommand\myname{}{Loïc Grobol}
\NewDocumentCommand\mylab{}{Lattice, ALMAnaCH}
\NewDocumentCommand\pdftitle{}{Introduction à la fouille de textes : examen final}
\NewDocumentCommand\mymail{}{loic.grobol@gmail.com}
\NewDocumentCommand\titlepagetitle{}{\pdftitle}
\NewDocumentCommand\docdate{}{23 avril 2018}
\NewDocumentCommand\pageheadertitle{}{\pdftitle}


\documentclass[a4paper, 11pt]{article}
\usepackage[hmargin=2cm, vmargin=1.5cm]{geometry}
\usepackage[subtle]{savetrees}
% \setlength\parindent{0pt}
\setlength\parskip{.2em}

\usepackage{polyglossia}
    \setmainlanguage{french}
    \setotherlanguage{english}

    % PATCH: fix spacing in monospaced text
    \usepackage{xpatch}
    \makeatletter
        \xapptocmd\ttfamily{\nofrench@punctuation}{}{}
    \makeatother


\usepackage{fontspec}
    \setmainfont{Libertinus Serif}
    \setsansfont{Libertinus Sans}

\usepackage{booktabs}
\usepackage{csquotes}
\usepackage{titleps}
\usepackage{titling}
\usepackage{tabu}
\usepackage[totpages,user]{zref}

% ## Packages with a specific loading order

% ### Load after math packages
\usepackage[math-style=french]{unicode-math}
	\setmathfont{Libertinus Math}

% Generally load after everything else
\usepackage{hyperxmp}
\usepackage[unicode]{hyperref}
\hypersetup{
    pdftitle={\pdftitle},	% informations apparaissant dans
    pdfauthor={\myname <\mymail>},	% dans les informations du document
    pdfcopyright={Copyright © \the\year, \myname},
    pdfcontactemail={\mymail},
}


% ## Style and metadata
% ### Page style

\newpagestyle{main}[\small]{
	\sethead{}{\ifnum\thepage>1 {\pageheadertitle}\fi}{}
	\setfoot{}{\ifnum\ztotpages>1 {\thepage\ /\ztotpages}\fi}{}
}
\pagestyle{main}

% ### Metadata

\title{\titlepagetitle}
\author{\myname <\mymail>}
\date{\docdate}

\begin{document}
{
   \begin{center}
      \large{M1 PluriTAL}\par
      \vspace{1em}
      \huge\textbf{Introduction à la fouille de textes}\par
      \vspace{0.5em}
      \LARGE\textit{Examen final}\par
      \vspace{0.5em}
      \Large\thedate
   \end{center}
}

Durée de l'examen : deux heures

Tous les documents non-électroniques sont autorisés.

\section{Classification par apprentissage automatique}
On dispose d'un ensemble de textes représentés par deux attributs que l'on souhaite ranger dans deux classes distinctes notées $+$ et $-$.
Des exemples déjà classés sont fournis dans le tableau suivant :

\begin{center}
\begin{tabu} to 0.6\linewidth {*{4}{X[c]}}
    \toprule
    \textbf{Identifiant} & \textbf{Attribut 1} & \textbf{Attribut 2} & \textbf{Classe}\\
    \midrule
    $1$ & $1$ & $4$ & $+$\\
    $2$ & $2$ & $1$ & $-$\\
    $3$ & $3$ & $5$ & $+$\\
    $4$ & $4$ & $4$ & $+$\\
    $5$ & $4$ & $1$ & $-$\\
    $6$ & $6$ & $2$ & $-$\\
    $7$ & $7$ & $5$ & $-$\\
    \bottomrule
\end{tabu}
\end{center}

\begin{enumerate}
    \item Dessiner dans un repère cartésien à $2$ dimensions les points représentant chacun de ces textes (en mettant des symboles $+$ et $-$ sur votre figure).
    \item Expliquer ce que donnera sur ces exemples l'application de chacun des algorithmes d'apprentissage automatique de classifieurs suivants :
        \begin{enumerate}
            \item L'algorithme de la classe majoritaire
            \item Un algorithme de construction d'un arbre de décision
                (construire un arbre qui classe correctement tous les exemples, plusieurs solutions sont possibles, deux niveaux dans l'arbre suffisent)
            \item Un algorithme de SVM (dessiner un séparateur crédible sur la figure)
        \end{enumerate}
    \item Classer à l'aide des trois classifieurs construits précédemment les nouvelles donné!:es suivantes
        \begin{center}
        \begin{tabu} to 0.5\linewidth {*{3}{X[c]}}
            \toprule
            \textbf{Identifiant} & \textbf{Attribut 1} & \textbf{Attribut 2}\\
            \midrule
            $8$ & $2$ & $3$\\
            $9$ & $2$ & $2$\\
            $10$ & $5$ & $1$\\
            $11$ & $6$ & $5$\\
            $12$ & $7$ & $1$\\
            \bottomrule
        \end{tabu}
        \end{center}

    \item Sachant que parmi ces nouvelles données, $8$, $9$ et $10$ appartiennent en fait à la classe $+$ et $11$ et $12$ à la classe $-$
        \begin{enumerate}
            \item Donner les matrices de confusion obtenues par les classifieurs construits précédemment pour ces nouvelles données
            \item En déduire la précision, le rappel et la F-mesure de ces algorithmes pour la classe $+$ sur ces données
        \end{enumerate}
\end{enumerate}

\section{Identification de tâches}
La tâche d'\emph{annotation sémantique} est une extension de la tâche de détection des entité nommées qui consiste à lier des entités d'une base de connaissances avec leurs mentions dans un texte.
Par exemple, un système d'annotation sémantique doit être capable de lier la phrase
\begin{displayquote}
    Madame Curie est la seule personne à avoir été récompensée d'un prix Nobel dans deux domaines scientifiques distincts.
\end{displayquote}
avec les entrées « Marie Skłodowska-Curie » et « prix Nobel » de sa base de connaissances mais pas avec les entrées « Pierre Curie » ou « Irène Joliot-Curie ».

\begin{enumerate}
    \item Reformuler la tâche d'annotation sémantique comme un enchaînement de tâches élémentaires de la fouille de textes vues en cours (plusieurs solutions sont possibles, deux tâches distinctes suffisent)
    \item Suggérer des ressources pertinentes pour un système d'annotation sémantique
\end{enumerate}


\section{Extraction d'information}
On désire tester différentes méthodes pour apprendre à extraire automatiquement des noms complets de réalisateurs de cinéma, en distinguant leur nom et leur prénom, afin de les ranger dans deux champs distincts d'une base de données.
Pour cela, on dispose des exemples suivants, dans lesquels les mots \underline{soulignés} sont
les \emph{prénoms} et les mots \textbf{en gras} sont les \emph{noms de famille} à extraire

\begin{enumerate}
    \item\label{ex|extr|expl1} Le film de \underline{Lars} \textbf{von Trier}, « Melancolia », a été bien accueilli au Festival de Cannes
    \item Tout le monde attend le prochain film de \textbf{Wong} \underline{Kar Wai}.
    \item \underline{Francis Ford} \textbf{Coppola} est un des rares réalisateurs à avoir obtenu deux palmes d'or
    \item L'intégralité des films de \underline{Jacques} \textbf{Demy} sont sortis en DVD.
\end{enumerate}

Pour extraire ces noms et prénoms, on propose d'appliquer une méthode de classification aux \emph{séparateurs entre mots}, c'est-à-dire les espaces et les ponctuations (qui sont supprimées du texte initial) séparant deux mots ainsi que le début et la fin d'un texte.
Par exemple, dans la phrase \ref{ex|extr|expl1}, en représentant les séparateurs entre mots par des barres verticales, on obtient :

{
    | Le | film | de | \underline{Lars} | \textbf{von | Trier} | Melancolia | a | été | bien | accueilli | au | Festival | de | Cannes |
}

\begin{enumerate}
    \item Proposer un ensemble de classes dans lesquelles ranger les séparateurs, qui permettront d'extraire correctement les informations souhaitées.

    \item Si on veut utiliser un programme d'apprentissage automatique pour distinguer ces séparateurs, il faut les transformer en vecteurs.
        Pour chaque séparateur, on suppose que les attributs choisis sont : la catégorie du mot deux places avant, la catégorie du mot précédent, la catégorie du mot suivant et la catégorie du mot deux places après.
        Les catégories possibles des mots sont : \texttt{DET}, \texttt{NC}, \texttt{NP} (pour « nom propre », identifiés uniquement parce ce qu'ils sont inconnus d'un dictionnaire et débutent par  une majuscule), \texttt{V}, \texttt{PREP}, \texttt{ADJ}…
        Donner, pour chacune de vos classes de  séparateurs, un exemple d'une donnée dans cette représentation.

    \item D'après la régularité des données, quelle classe sera la plus facile à apprendre ?

    \item Une autre solution pour extraire ces noms et prénoms est d'\emph{annoter} les textes, en associant à chacun des mots une étiquette.
        Définir un jeu d'étiquettes permettant de résoudre la tâche  qui nous préoccupe et annoter chacune des phrases fournies en exemples avec ce jeu d'étiquettes.
\end{enumerate}

\section{Questions diverses}
Les questions suivantes sont indépendantes
\begin{enumerate}
    \item Si on ne se préoccupe pas de la précision d'un système de recherche d'information, comment peut-on s'assurer simplement que son rappel soit maximal ?
    \item Quand un classifieur utilise l'algorithme de la classe majoritaire, quel est son rappel pour les classes non-majoritaires ? Peut-on donner sa précision pour ces classes ? Justifier.
    \item On considère un corpus de deux textes, $t₁$ et $t₂$
        \begin{enumerate}
            \item Dans quelle situation la forme \emph{mange} a-t-elle un indice TF⋅IDF non-nul pour $t₁$ et nul pour $t₂$ ?
            \item On suppose que l'on se trouve dans la situation de la question précédente. Est-il possible qu'en lemmatisant le corpus, le lemme \emph{manger} ait un indice TF⋅IDF nul pour $t₁$ \textbf{et} pour $t₂$ ? Justifier.
        \end{enumerate}
\end{enumerate}

\end{document}
