\RequirePackage{xparse}
% Settings
\NewDocumentCommand\myname{}{Loïc Grobol}
\NewDocumentCommand\mylab{}{Lattice, ALMAnaCH}
\NewDocumentCommand\pdftitle{}{Introduction à la fouille de textes : examen final}
\NewDocumentCommand\mymail{}{loic.grobol@gmail.com}
\NewDocumentCommand\titlepagetitle{}{\pdftitle}
\NewDocumentCommand\docdate{}{2 mai 2019}
\NewDocumentCommand\pageheadertitle{}{\pdftitle}


\documentclass[a4paper, 11pt]{article}
\usepackage[hmargin=2cm, vmargin=1.5cm]{geometry}
\usepackage[subtle]{savetrees}
% \setlength\parindent{0pt}
\setlength\parskip{.2em}

\usepackage{polyglossia}
    \setmainlanguage{french}
    \setotherlanguage{english}

\usepackage{fontspec}
    \setmainfont{Libertinus Serif}
    \setsansfont{Libertinus Sans}

\usepackage{amsmath,amsthm}
\usepackage{mathtools}	% AMS Maths service pack
	\newtagform{brackets}{[}{]}	% Pour des lignes d'équation numérotées entre crochets
	\mathtoolsset{showonlyrefs, showmanualtags, mathic}	% affiche les tags manuels (\tag et \tag*) et corrige le kerning des maths inline dans un bloc italique voir la doc de mathtools
	\usetagform{brackets}	% Utilise le style de tags défini plus haut
\usepackage{lualatex-math}
\usepackage[math-style=french]{unicode-math}
	\setmathfont{Libertinus Math}
\usepackage{newunicodechar}
	\newunicodechar{√}{\sqrt}
\usepackage{mleftright}

\usepackage{enumitem}
\usepackage{booktabs}
\usepackage{csquotes}
\usepackage{titleps}
\usepackage{titling}
\usepackage{tabu}
\usepackage[totpages,user]{zref}
\usepackage{verse}

\usepackage{tikz}
    \usetikzlibrary{matrix}
    \usetikzlibrary{shapes, shapes.geometric}
    \usetikzlibrary{decorations.pathreplacing}
    \usetikzlibrary{positioning, calc, intersections}
    \tikzset{
        >=stealth,
        hair lines/.style={line width = 0.05pt, lightgray},
        data/.style={draw, ellipse},
        program/.style={draw, rectangle},
    }

% ## Packages with a specific loading order

% Generally load after everything else
\usepackage{hyperxmp}
\usepackage[unicode]{hyperref}
\hypersetup{
    colorlinks,
    breaklinks,
    pdfdisplaydoctitle,
    unicode,
    pdftitle={\pdftitle},
    pdfauthor={\myname},
    pdflang={fr},
    pdfcontactemail={\mymail},
}


% ## Style and metadata
% ### Page style

\newpagestyle{main}[\small]{
	\sethead{}{\ifnum\thepage>1 {\pageheadertitle}\fi}{}
	\setfoot{}{\ifnum\ztotpages>1 {\thepage\ /\ztotpages}\fi}{}
}
\pagestyle{main}

% ### License
\usepackage[type={CC}, modifier={by}, version={4.0}]{doclicense}

% ### Metadata

\title{\titlepagetitle}
\author{\myname <\mymail>}
\date{\docdate}

\begin{document}
{
   \begin{center}
      \large{M1 PluriTAL}\par
      \vspace{1em}
      \huge\textbf{Introduction à la fouille de textes}\par
      \vspace{0.5em}
      \LARGE\textit{Examen final}\par
      \vspace{0.5em}
      \Large\thedate
   \end{center}
}

Durée de l'examen : deux heures


\section{Classification automatique des vainqueurs}
On veut analyser automatiquement des textes journalistiques portant
sur les résultats de l'Euro de football. On considère les phrases suivantes :

\begin{enumerate}[label=\alph*.]
    \item Le Real Madrid bat sur son terrain le Borussia Dortmund mais échoue à se qualifier pour la finale.
    \item Malgré ses vedettes et ses supporters, Barcelona se fait  écraser par le Bayern de Munich sur son terrain.
    \item Lille, battue par Toulouse, est distancée.
    \item L'équipe du Paris St-Germain doit attendre attendre encore une journée de championnat avant de dépasser définitivement Marseille.
    \item L'Olympique de Marseille domine Bastia 2-1.
\end{enumerate}

Les questions suivantes portent sur l'ensemble des phrases précédentes.
\begin{enumerate}
    \item On souhaite tout d'abord extraire automatiquement les noms des équipes concernées.
        Pour cela, on associe à chaque mot de chaque phrase une étiquette.
        Choisir des noms pour ces étiquettes et étiqueter une phrase de votre choix avec elles.
    \item Pour chaque phrase contenant deux noms distincts d'équipe, on souhaite déterminer
        automatiquement si c'est la première équipe citée qui a gagné ou si c'est la deuxième.
        Quelle tâche doit-on réaliser ?
    \item Si on souhaite réaliser la tâche précédente par apprentissage automatique, une
        représentation en sac de mots habituelle en utilisant les lemmes des mots présents dans les
        phrases risque de ne pas être efficace du tout : pourquoi ?
        Proposer des attributs (ou propriétés à mettre en colonne) à prendre en compte pour aider au
        mieux les techniques d'apprentissage automatique à réaliser cette tâche.
    \item On imagine que les deux seuls attributs que l'on garde (même si c'est sûrement
        insuffisant en réalité) sont les positions en nombre de mots du début des deux noms
        d'équipes dans la phrase, auquel on ajoute une étiquette qui signifie soit qu'on est dans
        le cas où c'est la première équipe qui gagne, soit dans l'autre cas.
        Donner la représentation de chaque phrase dans cet espace à deux dimensions (plus celle de
        la classe), telle qu'on la fournirait à un logiciel comme Weka.
    \item Dessiner sur un repère cartésien à deux dimensions la position des points qui
        représentent les textes dans la représentation de la question précédente, en prenant une
        couleur ou un symbole différent pour chaque classe.
    \item Donner un arbre de décision simple qui classe correctement les exemples, en justifiant
        la position des séparateurs sur le dessin.
    \item Dessiner un séparateur SVM qui classe correctement les exemples.
\end{enumerate}

\section{Identification de tâches}
Les systèmes « Question/Réponse » (« \emph{Question Answering} » en anglais) sont des programmes
capables de répondre à une question quelconque en langage naturel (LN), en fournissant une réponse
prise dans un corpus de textes de référence ou sur le Web, comme sur le schéma suivant

\begin{center}
    \begin{tikzpicture}
        \node[data] (in) {Question en LN};
        \node[program, right=1cm of in, text width=10ex] (prog) {Système QR};
        \node[data, right=1cm of prog] (out) {Réponse en LN};
        \node[data, below=1cm of prog] (res) {Corpus de textes};
        \draw[->] (in) -- (prog);
        \draw[->] (prog) -- (out);
        \draw[->] (res) -- (prog);
    \end{tikzpicture}
\end{center}

Par exemple, à la question : « Qui a tué Lee Harvey Oswald ? » (le tueur de Kennedy, qui a été
lui-même assassiné avant son procès), le système doit répondre « Jack Ruby ».

De tels systèmes ne sont capables de répondre qu'à des questions « factuelles », c'est-à-dire
correspondant à un champ unique d'un formulaire, mais le domaine de la question peut en principe
être quelconque.
La plupart des systèmes actuels procèdent de la manière suivante : analyse de la question pour
savoir sur quel domaine elle porte et quels sont ses mots importants, puis utilisation d'un moteur
de recherche pour récupérer un ensemble de documents pertinents et recherche de la réponse dans ces
documents (en tenant compte aussi du résultat de l'analyse de la question).

En général, on obtient ainsi un grand nombre de réponses possibles, (par exemple « Ruby », « J. Ruby
», « Kennedy », « JFK », « J.F. Kennedy » etc.), on les regroupe entre elles en paquets et on
choisit le plus représenté.

\begin{enumerate}
    \item Reformuler l'explication précédente en expliquant la succession des tâches mises en œuvre
        dans un tel système.
    \item Expliquer pourquoi, avec un tel système, on risque d'obtenir la réponse « Kennedy » comme
        réponse à la question « Qui a tué Lee Harvey Oswald ? ».
\end{enumerate}

Quelques systèmes (par exemple celui du portail Orange) procèdent d'une autre manière, en tenant à
jour une base de données encyclopédique (par exemple au format RDF du web sémantique), et en
transformant la question en une requête dans un langage formel adapté au format de stockage (par
exemple SPARQL).

\begin{enumerate}[resume*]
    \item Quelle tâche doit être mise en œuvre pour alimenter la base de données encyclopédique (à
        partir, par exemple, du site Wikipedia). Aura-t-on ainsi la même réponse à la question ?
\end{enumerate}

\section{Précision et/ou rappel…}
\begin{enumerate}
    \item Vous devez écrire rapidement un programme pour extraire les noms propres de personnes dans
        des textes bruts provenant d'articles de journaux d'actualité, \emph{sans faire appel à de
        l'apprentissage automatique} mais en ayant le droit de faire appel à des ressources externes
        simples (listes trouvées sur Internet, par exemple). Expliquer comment vous écririez un tel
        programme (très simple) dans les deux cas suivants :
        \begin{enumerate}
            \item On veut que le programme ait une excellente précision, quel
              que soit son rappel
            \item On veut que le programme ait un excellent rappel, quelle que
              soit sa précision
        \end{enumerate}
        Dans les deux cas, expliquer pourquoi votre programme sera efficace
        pour l'une des deux mesures et pas pour l'autre.
    \item Pour un programme d'apprentissage automatique quelconque sur un problème à trois classes
        notées $A$, $B$ et $C$, on obtient la matrice de confusion suivante

        \begin{center}
            \begin{tikzpicture}[
                table/.style={
                    matrix of nodes,
                    row sep=-\pgflinewidth,
                    column sep=-\pgflinewidth,
                    nodes={draw, text width=3ex, align=center},
                    text depth=0.25ex,
                    text height=1em,
                    nodes in empty cells
                },
            ]
                \matrix (conf) [table]{
                        & A    & B    &  C\\
                    A   & $5$  & $1$  & $0$\\
                    B   & $3$  & $7$  & $1$\\
                    C   & $0$  & $2$  & $9$\\
                };

                \draw [decorate, decoration={brace, amplitude=1em}, yshift=-3ex]
                    (conf-1-2.north west) -- (conf-1-4.north east)
                    node [midway, above=2ex] {Classes prédites};
                \draw [decorate, decoration={brace, amplitude=1em, mirror}, xshift=-3ex]
                    (conf-2-1.north west) -- (conf-4-1.south west)
                    node [midway, left=2ex, rotate=90, anchor=south] {Vraies classes};
            \end{tikzpicture}
        \end{center}

        Calculer la précision et le rappel de chaque classe A, B et C sur cet exemple.
\end{enumerate}

\section{Transformation de textes en vecteurs}
On donne en annexe le texte d'un poème célèbre de Charles Baudelaire.
On suppose qu'on ne garde comme mots servant à représenter un texte en vecteur que quelques noms
communs courants (présents ou non dans le texte) à savoir, par ordre alphabétique : \emph{ciel},
\emph{cœur}, \emph{fleur}, \emph{jour}, \emph{nuit}, \emph{soleil}, \emph{temps}, \emph{valse}.
\begin{enumerate}
    \item On suppose que le texte est segmenté, mis en minuscules et lemmatisé. Donner la
        représentation du poème dans l'espace défini par ces mots dans les cas suivants :
        \begin{itemize}
            \item la représentation est booléenne
            \item la réprésentation est en nombre d'occurrences
        \end{itemize}
    \item On donne dans le tableau ci-dessous la représentation de 4 autres textes dans ce même
        espace, où chaque case du tableau compte le nombre d'occurrences du mot dans le texte :

        \begin{center}
        \begin{tabu}{|c|c|c|c|c|c|c|c|c|}
            \hline
            \textbf{texte} & ciel & cœur & fleur & jour & nuit & soleil & temps & valse\\
            \hline
            texte 1 & 3 & 0 & 0 & 4 & 6 & 5 & 2 & 0\\
            texte 2 & 2 & 2 & 1 & 3 & 0 & 1 & 1 & 0 \\
            texte 3 &  0 & 2 & 1 & 1 & 2 & 0 & 0 & 0\\
            texte 4 & 1 & 1 & 1 & 2 & 3 & 2 & 1 & 3 \\
            \hline
        \end{tabu}
        \end{center}

        \begin{enumerate}
            \item Donner la représentation TF⋅IDF du poème, sachant que la valeur d'IDF n'est calculée qu'à partir des textes 1 à 4.
            \item Les textes 1 et 2 appartiennent à une classe C1 tandis que les textes 3 et 4 appartiennent à une classe C2. A quelle classe sera rattaché le poème par l'algorithme Naive Bayes fondé sur la représentation en nombre d'occurrences ?
        \end{enumerate}
\end{enumerate}

\section*{Annexe}
\begin{verse}
Harmonie du soir\\!

Voici venir les temps où vibrant sur sa tige\\
Chaque fleur s'évapore ainsi qu'un encensoir ;\\
Les sons et les parfums tournent dans l'air du soir ;\\
Valse mélancolique et langoureux vertige !\\!

Chaque fleur s'évapore ainsi qu'un encensoir ;\\
Le violon frémit comme un cœur qu'on afflige ;\\
Valse mélancolique et langoureux vertige !\\
Le ciel est triste et beau comme un grand reposoir.\\!

Le violon frémit comme un cœur qu'on afflige,\\
Un cœur tendre, qui hait le néant vaste et noir !\\
Le ciel est triste et beau comme un grand reposoir ;\\
Le soleil s'est noyé dans son sang qui se fige.\\!

Un cœur tendre, qui hait le néant vaste et noir,\\
Du passé lumineux recueille tout vestige !\\
Le soleil s'est noyé dans son sang qui se fige…\\
Ton souvenir en moi luit comme un ostensoir !\\!
\end{verse}

\begin{align}
    \log(1) &= 0\\
    \log\mleft(\frac{4}{3}\mright) &≈ 0.29\\
    \log(2) &≈ 0.69\\
    \log(4) &≈ 1.39
\end{align}
\end{document}
