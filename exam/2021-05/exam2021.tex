% LTeX: language=fr
% Copyright © 2021, Loïc Grobol <loic.grobol@gmail.com>
% This document is available under the terms of the Creative Commons Attribution 4.0 International License (CC BY 4.0) (https://creativecommons.org/licenses/by/4.0/)

\newcommand\myname{Loïc Grobol}
\newcommand\mylab{LLF, Lattice}
\newcommand\pdftitle{Introduction à la fouille de textes : examen final}
\newcommand\mymail{loic.grobol@gmail.com}
\newcommand\titlepagetitle{\pdftitle}
\newcommand\docdate{10 mai 2021}
\newcommand\pageheadertitle{\pdftitle}

\documentclass[a4paper, 11pt]{article}
\usepackage{luatexbase}
\usepackage[hmargin=2cm]{geometry}
\usepackage{savetrees}
\setlength\parindent{0pt}
\setlength\parskip{.2em}

\usepackage{polyglossia}
		\setmainlanguage{french}
		\setotherlanguage{english}

\usepackage{csquotes}

\usepackage{fontspec}
	\directlua{
		luaotfload.add_fallback(
			"emojifallback",
			{"NotoColorEmoji:mode=harf;"}
		)
	}
	\setmainfont[
			UprightFont={* Regular},
			ItalicFont={* Italic},
			BoldFont={* Bold},
			BoldItalicFont={* Bold Italic},
	]{Libertinus Serif}[Renderer=Harfbuzz]
	\setsansfont{Libertinus Sans}[Renderer=Harfbuzz]
	\setmonofont{Fira Mono}[Scale=MatchLowercase, Renderer=Harfbuzz]
	\newfontfamily\emojifont{Noto Color Emoji}[Renderer=HarfBuzz]
	\frenchspacing

\usepackage{xurl}
\usepackage{hyperref}

% ## Style and metadata
% ### Page style

\usepackage[totpages]{zref}
\newpagestyle{main}[\small]{
	\sethead{}{\ifnum\thepage>1 {\pageheadertitle}\fi}{}
	\setfoot{}{\ifnum\ztotpages>1 {\thepage\ /\ztotpages}\fi}{}
}
\pagestyle{main}

% ### License
\usepackage[type={CC}, modifier={by}, version={4.0}]{doclicense}

% ### Metadata

\title{\titlepagetitle}
\author{\myname <\mymail>}
\date{\docdate}

\begin{document}
{
	 \begin{center}
			\large{M1 PluriTAL}\par
			\vspace{1em}
			\huge\textbf{Introduction à la fouille de textes}\par
			\vspace{0.5em}
			\LARGE\textit{Examen final}\par
			\vspace{0.5em}
			\Large{2021-05-10}
	 \end{center}
}

\section{Recherche d'information}

On considère les textes suivants, constitués chacun d'une unique phrase :
\begin{description}
	\item[texte 1] L'entraîneur de l'équipe de France a donné sa liste de joueurs pour le match.
	\item[texte 2] Les matchs de Rolland-Garros ont fini dimanche.
	\item[texte 3] En préparation, les français ont joué un match amical.
	\item[texte 4] Les joueurs français se font éliminer à Rolland-Garros.
\end{description}

\begin{enumerate}
	\item Les textes sont segmentés et mis en minuscule, la ponctuation est éliminée. On fournit en
		annexe la liste des mots vides. Donner la liste des mots qui constituent l'espace de
		représentation de ces textes et donner la représentation vectorielle booléenne (qui, ici, sera
		identique à la représentation en termes de nombre d'occurrences) de chacun d'eux dans cet
		espace.
	\item Quels mots de l'espace ont le plus faible TF⋅IDF quand on considère ces 4 textes (on demande
		de justifier la réponse mais pas de calculer la valeur exacte) ?
	\item Qu'est-ce qui changerait dans les représentations précédentes :
		\begin{itemize}
		\item si on disposait d'un lemmatiseur et que l'espace de représentation était constitué des
			\emph{lemmes} des mots non vides
		\item si on disposait d'un ``raciniseur'' (qui découpe les mots en éliminant leur terminaison)
			et que l'espace de représentation était constitué des \emph{racines} des mots non vides
		\end{itemize}
	\item On garde la représentation de la question 1 (sans lemmatiser ni raciniser). On suppose qu'on
		soumet à un moteur de recherche vectoriel la requête constituée de la nouvelle phrase suivante :
		\enquote{Les joueurs français ont peur de se faire éliminer lors du match.} Donner sa
		représentation dans l'espace défini question 1. En prenant comme mesure de proximité la mesure
		de Dice, classer les quatre textes par ordre de similarité croissant.
	\item En fait, les textes 1 et 3 parlent de football tandis que les textes 2 et 4 parlent de
		tennis. On imagine que la classe \enquote{football} est ajoutée à la représentation des textes 1
		et 3 tandis que la classe \enquote{tennis} est ajoutée à la représentation des textes 2 et 4, et
		que l'ensemble de ces 4 exemples est soumis à un programme d'apprentissage automatique
		NaiveBayes. Dans quelle classe ce programme va-t-il classer la requête ?
\end{enumerate}

\section{Classification par apprentissage automatique}
On suppose disposer d'un ensemble de textes représentés pour simplifier par des vecteurs à deux
dimensions seulement, que l'on souhaite ranger dans trois classes distinctes notées 1, 2 et 3.
L'objectif de cet exercice est de montrer comment une technique comme celle des SVM, initialement
prévue pour distinguer uniquement deux classes, peut s'adapter pour distinguer trois classes
distinctes. Des exemples de données avec leur classe sont fournies dans le tableau suivant :

\begin{center}
	\begin{tabular}{|c|c|c|c|}
		\hline
		\textbf{Identifiant} & \textbf{attribut 1} & \textbf{attribut 2} & \textbf{classe}\\
		\hline
		1 & 1 & 3 & 1\\
		2 & 2 & 4 & 1\\
		3 & 2 & 2 & 2\\
		4 & 4 & 1 & 2\\
		5 & 5 & 1 & 2\\
		6 & 3 & 4 & 3\\
		7 & 4 & 5 & 3\\
		8 & 5 & 3 & 3\\
		\hline
	\end{tabular}
\end{center}

\begin{enumerate}
	\item Dessiner dans un repère cartésien à deux dimensions les points représentant chacun de ces
		textes. On fournit une grille pour cela : prendre une échelle de 4 petits carreaux pour une
		unité et mettre un symbole ou une couleur différente pour différencier chaque classe. 
	\item Dessiner (sans faire de calcul) à partir des données de la question 1 un arbre de décision
		permettant de ranger correctement les données dans les trois classes. Représenter à la fois les
		séparateurs sur le repère cartésien et le dessin de l'arbre obtenu (plusieurs solutions
		différentes mais tout aussi correctes sont possibles).
	\item Proposer les coordonnées de trois nouvelles données choisies de façon à ce que l'algorithme
		des 3 plus proches voisins avec la distance euclidienne basé sur les données du tableau classe
		chacune de ces nouvelles données dans une des trois classes : une donnée dans la classe 1, une
		autre dans la classe 2, la troisième dans la classe 3.
	\item Puisqu'un SVM ne permet de séparer l'espace qu'en deux sous-espaces, on va considérer chaque
		classe indépendamment et on va construire plusieurs SVM : chacun séparera une classe
		particulière des deux autres. Comme il y a trois classes, on va construire trois SVM :

		\begin{itemize}
			\item le premier sépare la classe 1 des classes 2 et 3 ;
			\item le deuxième sépare la classe 2 des classes 1 et 3 ;
			\item le troisième sépare la classe 3 des classes 1 et 2.
		\end{itemize}

		Dessiner trois séparateurs de type SVM ``crédibles'' sur la figure à partir des données du
		tableau (c'est-à-dire sans tenir compte des données ajoutées en question 3).
	\item On fournit les nouvelles données suivantes, dont la classe est inconnue :

		\begin{center}
			\begin{tabular}{|c|c|c|}
				\hline
				\textbf{Identifiant} & \textbf{attribut 1} & \textbf{attribut 2} \\
				\hline
				9 & 3 & 1 \\
				10 & 1 & 4 \\
				11 & 6 & 3 \\
				\hline
			\end{tabular}
		\end{center}

		Donner la réponse (classe) fournie par votre arbre de décision de la question 2 à chacune de ces
		données.
	\item Pour chacune de ces données, on va aussi consulter les trois SVM et on va combiner leurs
		classifications sous la forme de "votes" ou de "points". Par exemple, pour une nouvelle donnée,
		suivant de quel côté du séparateur elle se trouve, le premier SVM (celui qui sépare la classe 1
		des classes 2 et 3) va soit voter (donner 1 point) pour la classe 1 soit voter pour la classe "2
		et 3", ce qui revient à donner à la fois œ point pour la classe 2 et œ point pour la classe 3.
		On cumule ensuite tous les points accordés par les différents SVM et on associe à la donnée la
		classe ayant recueilli le plus de points. Pour les trois nouvelles données 9 à 11 du tableau,
		compter les points accordés ainsi par chacun des trois SVM et en déduire dans quelle classe ces
		données vont être rangées par la combinaison de ces trois SVM.
	\item Sachant que parmi ces nouvelles données, 9 et 11 appartiennent en fait à la classe 2 et la
		donnée 10 à la classe 1, donner la matrice de confusion sur ces 3 données de l'arbre de décision
		construit en question 2. Donner la précision, le rappel et la F-mesure de la classe 2 pour cet
		algorithme.
	\item Donner aussi la matrice de confusion pour l'algorithme qui combine les votes des trois SVM. 
\end{enumerate}

\section{Questions diverses}

\begin{enumerate}
	\item Expliquer pourquoi, si on effectue sur un moteur de recherche généraliste fonctionnant
		exclusivement en "sacs de mots" une requête portant sur les mots clés "père" "de" "Gérard"
		"Depardieu", on risque d'obtenir surtout des pages traitant de Guillaume et Julie Depardieu, qui
		sont les enfants (et non les parents) de Gérard.
	\item On suppose réaliser une expérience d'apprentissage automatique supervisée de classification
		(avec Weka par exemple). On dispose de 50 données annotées. 90\% des données sont utilisées en
		apprentissage, le reste en test. Combien cela fait-il de données de test ? Quelles sont toutes
		les valeurs possibles d'exactitude (c'est-à-dire de bonne classification, quelle que soit le
		nombre de classes possibles) qu'on peut atteindre avec ces données de test (envisager tous les
		cas : 1 erreur parmi les données de test, 2 erreurs, ..., autant d'erreurs que données).
	\item Calculer la précision et le rappel de chaque classe A, B et C sur cet exemple de matrice de
		confusion, ainsi que l'exactitude globale :

		\begin{center}
			\begin{tabular}{|c|c|c|c|}
				\hline
				vraie classe $\backslash$ classé en & A & B & C \\
				\hline
				A & 8 & 1 & 0\\
				\hline
				B & 2 & 10 & 3\\
				\hline
				C & 0 & 2 & 9\\
				\hline
			\end{tabular}
		\end{center}
\end{enumerate}

\newpage
\noindent\underline{ANNEXE : liste des mots vides à prendre en compte
	pour l'exercice 1}
\begin{verbatim}
a
à
amical
après
cet
de
des
dimanche
donné
en
entraîneur
est
faire
fini
font
l'
le
les
liste
ont
pour
préparation
qui
s'
sa
se
un
\end{verbatim}
\end{document}
