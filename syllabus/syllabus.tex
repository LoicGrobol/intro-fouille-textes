\documentclass[a4paper, 11pt]{article}
\usepackage{luatexbase}
\usepackage[hmargin=2cm]{geometry}
\usepackage{savetrees}
\setlength\parindent{0pt}
\setlength\parskip{.2em}

\usepackage{polyglossia}
    \setmainlanguage{french}
    \setotherlanguage{english}

\usepackage{csquotes}

\usepackage{fontspec}
    \setmainfont[
        UprightFont={* Regular},
        ItalicFont={* Italic},
        BoldFont={* Bold},
        BoldItalicFont={* Bold Italic},
    ]{Libertinus Serif}[Renderer=Harfbuzz]
    \setsansfont{Libertinus Sans}[Renderer=Harfbuzz]
    \setmonofont{Fira Mono}[Scale=MatchLowercase,Renderer=Harfbuzz]
    \newfontfamily\fallbackfont{Deja Vu Sans}[Renderer=Harfbuzz]
    % \newfontfamily\emojifont{Noto Color Emoji}[Renderer=Harfbuzz]

\usepackage{xurl}
\usepackage{hyperref}

\usepackage[style=authoryear]{biblatex}
\AtEveryBibitem{
    \ifentrytype{online}
    {} {
        \iffieldequalstr{howpublished}{online}
        {
            \clearfield{howpublished}
        } {
            \clearfield{urlyear}\clearfield{urlmonth}\clearfield{urlday}
        }
    }
}

\usepackage[iso]{datetime}

\usepackage{titleps}
	\newpagestyle{main}[\small]{
		\sethead{}{}{}
		\setfoot{}{}{\footnotesize{Version {\yyyymmdddate\today}T\currenttime}}
	}
	\pagestyle{main}

\title{Introduction à la fouille de textes\\M1 Plurital S2}
\author{Loïc Grobol}
\date{2020–2021}

\addbibresource{biblio.bib}
\nocite{*}

\begin{document}

\vspace*{-7em}
{\let\newpage\relax\maketitle}
\thispagestyle{main}

\begin{description}
    \item[Où] Sur le serveur Discord du master PluriTAL
    \item[Quand] Le jeudi de 13h00 à 15h (voir le calendrier de Paris 3 pour les dates)
    \item[Email] \href{mailto:loic.grobol@gmail.com}{\texttt{loic.grobol@gmail.com}}
    \item[Web] \url{https://loicgrobol.github.io/intro-fouille-textes/}
\end{description}

\section*{Contenu}
Ce cours propose une introduction aux grandes tâches d'ingénierie linguistique qui constituent aujourd'hui ce que l'on désigne par le terme de \emph{fouille de textes}.
Dans un premier temps nous présenterons ainsi la recherche d'information, la classification, l'annotation et l'extraction d'information.
Le  cours se  concentrera ensuite plus particulièrement sur la tâche de classification de textes et sur les différentes techniques actuelles de classification de textes par apprentissage artificiel supervisé (\emph{Naive Bayes}, arbres de décision, SVM, réseaux de neurones).
Cette deuxième partie sera concrétisée par la construction pratique d'un système de classification de textes utilisant le logiciel Weka.

\section*{Objectifs}

\begin{itemize}
    \item Expliciter dans le contexte de la fouille de texte la notion théorique de \emph{tâche}
    \item Identifier les différentes tâches composant la fouille de texte et leurs champs d'application
    \item Décrire le fonctionnement général et les mesures usuelles de qualité d'un système de classification de textes
    \item Décrire quelques algorithmes usuels de classification et leurs applications à la classification de textes
    \item Modéliser des problèmes concrets comme des tâches de fouille de textes
    \item Reformuler des tâches complexes de fouille de textes comme des combinaisons de tâches élémentaires
    \item Construire avec Weka un système de classification de textes par apprentissage artificiel
\end{itemize}

\section*{Évaluation}

\textbf{Modalités données à titre indicatif et ammenées à changer suivant l'évolution de la situation sanitaire}

En contrôle continu :

\begin{description}
    \item[Projet] Développement d'un système de classification de texte par apprentissage artificiel (50\%)
    \item[Examen final] Examen écrit de deux heures en fin de semestre (50\%)
\end{description}

\section*{Bibliographie}
\printbibliography[heading=none]

\section*{Liens utiles}
\begin{itemize}
    \item Le polycopié du cours (indispensable) :  \url{https://github.com/LoicGrobol/intro-fouille-textes/releases/download/stable/poly.pdf}
    \item Le site du projet Weka : \url{https://www.cs.waikato.ac.nz/ml/weka/}
    \item La page web de l'an dernier : \url{https://loicgrobol.github.io/intro-fouille-textes/archives/2020}
\end{itemize}
\end{document}
