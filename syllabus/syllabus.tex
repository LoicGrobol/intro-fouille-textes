\documentclass[a4paper, 11pt]{article}
\usepackage[hmargin=2cm]{geometry}
\usepackage{savetrees}
\setlength\parindent{0pt}
\setlength\parskip{.2em}

\usepackage{polyglossia}
    \setmainlanguage{french}
    \setotherlanguage{english}

    % PATCH: fix spacing in monospaced text
    \usepackage{xpatch}
    \makeatletter
        \xapptocmd\ttfamily{\nofrench@punctuation}{}{}
    \makeatother

\usepackage{csquotes}

\usepackage{fontspec}
    \setmainfont{Libertinus Serif}

\usepackage{xurl}
\usepackage{hyperref}

\usepackage[style=authoryear]{biblatex}
\AtEveryBibitem{
    \ifentrytype{online}
    {} {
        \iffieldequalstr{howpublished}{online}
        {
            \clearfield{howpublished}
        } {
            \clearfield{urlyear}\clearfield{urlmonth}\clearfield{urlday}
        }
    }
}

\usepackage{titleps}
	\newpagestyle{main}[\small]{
		\sethead{}{}{}
		\setfoot{}{}{}
	}
	\pagestyle{main}

\title{Introduction à la fouille de textes\\M1 Plurital S2}
\author{Loïc Grobol}
\date{2017–2018}

\addbibresource{biblio.bib}
\nocite{*}

\begin{document}

\vspace*{-7em}
{\let\newpage\relax\maketitle}
\thispagestyle{main}

\begin{description}
    \item[Où] Salle Benveniste, ILPGA, 9 Rue des Bernardins, 75005 Paris
    \item[Quand] Le lundi de 14h à 16h (voir le calendrier de Paris 3 pour les dates)
    \item[Email] \href{mailto:loic.grobol@gmail.com}{\texttt{loic.grobol@gmail.com}}
    \item[Web] \url{https://loicgrobol.github.io/intro-fouille-textes/}
\end{description}

\section*{Contenu}
Ce cours propose une introduction aux grandes tâches d'ingénierie linguistique qui constituent aujourd'hui ce que l'on désigne par le terme de \emph{fouille de textes}.
Dans un premier temps nous présenterons ainsi la recherche d'information, la classification, l'annotation et l'extraction d'information.
Le  cours se  concentrera ensuite plus particulièrement sur la tâche de classification de textes et sur les différentes techniques actuelles de classification de textes par apprentissage artificiel supervisé (\emph{Naive Bayes}, arbres de décision, SVM, réseaux de neurones).
Cette deuxième partie sera concrétisée par la construction pratique d'un système de classification de textes.

\section*{Objectifs}

\begin{itemize}
    \item Expliciter dans le contexte de la fouille de texte la notion théorique de \emph{tâche}
    \item Identifier les différentes tâches composant la fouille de texte et leurs champs d'application
    \item Décrire le fonctionnement général et les mesures usuelles de qualité d'un système de classification de textes
    \item Décrire quelques algorithmes usuels de classification et leurs applications à la classification de textes
    \item Construire un système de classification de textes par apprentissage artificiel
\end{itemize}

\section*{Évaluation}
En contrôle continu :

\begin{description}
    \item[Projet] Développement d'un système de classification de texte par apprentissage artificiel (50\%)
    \item[Examen final] Examen écrit de deux heures en semaine 13 (50\%)
\end{description}

\section*{Bibliographie}
\printbibliography[heading=none]

\section*{Liens utiles}
\begin{itemize}
    \item La page web des années précédentes : \url{http://www.lattice.cnrs.fr/sites/itellier/fouille_textes.html}
    \item Le polycopié du cours (indispensable) :  \url{http://www.lattice.cnrs.fr/sites/itellier/poly_fouille_textes/fouille-textes.pdf}
\end{itemize}
\end{document}
